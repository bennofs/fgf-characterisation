\documentclass[aspectratio=169]{beamer}

\usetheme{auriga}
\usepackage{ifxetex}

\ifxetex
  \usepackage{polyglossia}
  \usepackage{fontspec}

  \setmainlanguage{english}
\else
  \usepackage[english]{babel}
\fi

\usepackage{marvosym}
\usepackage{hyperref}
\usepackage{blindtext}
\usepackage{graphicx}
\usepackage{tabularx}
\usepackage{listings}
\usepackage{tikz}
\usepackage{pifont}
\usepackage{fontawesome}
\usepackage{tikzpeople}
\usepackage{upquote}
\usepackage{tikzsymbols}
\usepackage{bookmark}
\usepackage{printlen}
\usepackage{xcolor}
\usepackage[fleqn]{mathtools}
\usepackage{amssymb}
\usepackage[fleqn]{amsmath}
\usepackage[h]{esvect}
\usepackage{multicol}

\usetikzlibrary{arrows, arrows.meta, decorations.markings, shapes, calc, positioning, fit, graphs, trees, matrix, backgrounds, bending, decorations.pathmorphing, decorations.pathreplacing, decorations.shapes, fadings, shadings, patterns, graphs.standard, patterns.meta}

\usepackage{appendixnumberbeamer}

%\includeonlyframes{current}

\lstset{upquote=true}
%\setsansfont[BoldFont={Fira Sans SemiBold}]{Fira Sans Book}
\makeatletter
\newlength\beamerleftmargin
\setlength\beamerleftmargin{\Gm@lmargin}
\makeatother

\hypersetup{bookmarks=true,colorlinks=true,allcolors=blue}

\title{\textcolor{tolhighcontrastBlue}{A Constructive Proof of a Van~Benthem Theorem for the Forward Guarded Fragment of First~Order~Logic}}
\author{\textcolor{toldarkDarkblue}{Benno \textcolor{tolbrightYellowDarkest}{Fünfstück}}}
\institute{\textcolor{toldarkDarkcyan}{Master's defense}}
\date{24.06.2024}

\definecolor{tolbrightBlue}{HTML}{4477AA}
\definecolor{tolbrightRed}{HTML}{EE6677}
\definecolor{tolbrightGreen}{HTML}{228833}
\definecolor{tolbrightYellow}{HTML}{CCBB44}
\colorlet{tolbrightYellowDarker}{tolbrightYellow!70!black}
\definecolor{tolbrightCyan}{HTML}{66CCEE}
\colorlet{tolbrightCyanDarker}{tolbrightCyan!70!black}
\definecolor{tolbrightPurple}{HTML}{AA3377}
\definecolor{tolbrightGrey}{HTML}{BBBBBB}
\definecolor{tolhighcontrastBlue}{HTML}{004488}
\definecolor{tolhighcontrastYellow}{HTML}{DDAA33}
\definecolor{tolhighcontrastRed}{HTML}{BB5566}
\definecolor{tolvibrantOrange}{HTML}{EE7733}
\definecolor{tolvibrantBlue}{HTML}{0077BB}
\definecolor{tolvibrantCyan}{HTML}{33BBEE}
\definecolor{tolvibrantMagenta}{HTML}{EE3377}
\definecolor{tolvibrantRed}{HTML}{CC3311}
\definecolor{tolvibrantTeal}{HTML}{009988}
\definecolor{tolvibrantGrey}{HTML}{BBBBBB}
\definecolor{tolmutedRose}{HTML}{CC6677}
\definecolor{tolmutedIndigo}{HTML}{332288}
\definecolor{tolmutedSand}{HTML}{DDCC77}
\definecolor{tolmutedGreen}{HTML}{117733}
\definecolor{tolmutedCyan}{HTML}{88CCEE}
\definecolor{tolmutedWine}{HTML}{882255}
\definecolor{tolmutedTeal}{HTML}{44AA99}
\definecolor{tolmutedOlive}{HTML}{999933}
\definecolor{tolmutedPurple}{HTML}{AA4499}
\definecolor{tolmutedPalegrey}{HTML}{DDDDDD}
\definecolor{tolmediumcontrastLightblue}{HTML}{6699CC}
\definecolor{tolmediumcontrastDarkblue}{HTML}{004488}
\definecolor{tolmediumcontrastLightyellow}{HTML}{EECC66}
\definecolor{tolmediumcontrastDarkred}{HTML}{994455}
\definecolor{tolmediumcontrastDarkyellow}{HTML}{997700}
\definecolor{tolmediumcontrastLightred}{HTML}{EE99AA}
\definecolor{tolpalePaleblue}{HTML}{BBCCEE}
\definecolor{tolpalePalered}{HTML}{FFCCCC}
\definecolor{tolpalePalegreen}{HTML}{CCDDAA}
\definecolor{tolpalePaleyellow}{HTML}{EEEEBB}
\definecolor{tolpalePalecyan}{HTML}{CCEEFF}
\definecolor{tolpalePalegrey}{HTML}{DDDDDD}
\definecolor{toldarkDarkblue}{HTML}{222255}
\definecolor{toldarkDarkred}{HTML}{663333}
\definecolor{toldarkDarkgreen}{HTML}{225522}
\definecolor{toldarkDarkyellow}{HTML}{666633}
\definecolor{toldarkDarkcyan}{HTML}{225555}
\definecolor{toldarkDarkgrey}{HTML}{555555}
\definecolor{tollightLightblue}{HTML}{77AADD}
\definecolor{tollightOrange}{HTML}{EE8866}
\definecolor{tollightLightyellow}{HTML}{EEDD88}
\definecolor{tollightPink}{HTML}{FFAABB}
\definecolor{tollightLightcyan}{HTML}{99DDFF}
\definecolor{tollightMint}{HTML}{44BB99}
\definecolor{tollightPear}{HTML}{BBCC33}
\definecolor{tollightOlive}{HTML}{AAAA00}
\definecolor{tollightPalegrey}{HTML}{DDDDDD}

% Inline comments
\definecolor{ao(english)}{rgb}{0.0, 0.5, 0.0}
\definecolor{brickred}{rgb}{0.8, 0.25, 0.33}
\newcommand{\myundef}[1]{\textcolor{brickred}{\textbf{#1}}}
\newcommand{\possiblelie}[1]{\textcolor{brickred}{\textbf{#1}}}
\newcommand{\becareful}[1]{\textcolor{brickred}{\textbf{#1}}}
\newcommand{\bennof}[1]{\textbf{\textcolor{blue}{\textbf{#1}}}}
\newcommand{\bbe}[1]{\textbf{\textcolor{ao(english)}{#1}}}
\newcommand{\bbebox}[1]{\todo[inline,color=green!30]{\textbf{BB\@: }#1}\xspace}
\newcommand{\bbeside}[1]{\todo[color=green!30,size=\scriptsize,fancyline]{\textbf{BB\@: }#1}\xspace}
\newcommand{\bfbox}[1]{\todo[inline,color=blue!30]{\textbf{BF\@: }#1}\xspace}
\newcommand{\bfside}[1]{\todo[color=blue!30,size=\scriptsize,fancyline]{\textbf{BF\@: }#1}\xspace}

% logics:
\newcommand{\Logic}[1]{\ensuremath{\mathsf{#1}}} % a Logic
\newcommand{\logicL}{\Logic{L}} % some logic L
\newcommand{\Laffix}{\Logic{L}_{\mathsf{affix}}}   % L_affix
\newcommand{\Linfix}{\Logic{L}_{\mathsf{inf}}}   % L_infix
\newcommand{\Lsuffix}{\Logic{L}_{\mathsf{suf}}} % L_suffix
\newcommand{\Lprefix}{\Logic{L}_{\mathsf{pre}}} % L_prefix

\newcommand{\Gaffix}{\Logic{G}_{\mathsf{affix}}}   % G_affix
\newcommand{\Ginfix}{\Logic{G}_{\mathsf{inf}}}     % G_infix
\newcommand{\Gsuffix}{\Logic{G}_{\mathsf{suf}}}    % G_suffix
\newcommand{\Gprefix}{\Logic{G}_{\mathsf{pre}}}    % G_prefix

\newcommand{\GF}{\Logic{GF}}   % Guarded Fragment
\newcommand{\FGF}{\Logic{FGF}}   % Forward Guarded Fragment
\newcommand{\FO}{\Logic{FO}}   % First-Order Logic

% Complexity classses:
\newcommand{\complexityclass}[1]{\textsc{#1}} % any complexity class
\newcommand{\ExpTime}{\complexityclass{ExpTime}} % exponential time
\hyphenation{Exp-Time} % prevent "Ex-PTime" (see, e.g. Tobies'01, Glimm'07 ;-)
\newcommand{\NExpTime}{\complexityclass{NExpTime}} % nondeterministic exponential time
\hyphenation{NExp-Time} % prevent "Ex-PTime" (see, e.g. Tobies'01, Glimm'07 ;-)
\newcommand{\TwoExpTime}{\complexityclass{2ExpTime}} % doubly-exponential time
\newcommand{\TwoNExpTime}{\complexityclass{2NExpTime}} % doubly-nondeterministic-exponential time
\newcommand{\coNExpTime}{\complexityclass{coNExpTime}} % co nondeterministic exponential time
\hyphenation{coNExp-Time} 
\newcommand{\Tower}{\complexityclass{Tower}}
\newcommand{\LogSpace}{\complexityclass{LogSpace}}
\newcommand{\PSpace}{\complexityclass{PSpace}}
\newcommand{\PTime}{\complexityclass{PTime}}

% Others
\newcommand{\str}[1]{{\mathfrak{#1}}}
\DeclareRobustCommand{\unravel}[1]{\vv{\mathfrak{#1}}}
\newcommand{\deff}{\coloneqq}
\newcommand{\arity}{\mathsf{ar}}
\renewcommand{\iff}{\leftrightarrow}

\newcommand{\N}{{\mathbb{N}}}
\newcommand{\Z}{{\mathbb{Z}}} 
\newcommand{\Q}{{\mathbb{Q}}}
\newcommand{\V}{\mathbf{V}} 
\newcommand{\R}{\mathbf{R}} 
\newcommand{\Var}{\mathrm{Var}}
\newcommand{\sigSigma}{\Sigma}

\newcommand{\sqin}{%
  \mathrel{\vphantom{\sqsubset}\text{%
    \mathsurround=0pt
    \ooalign{$\sqsubset$\cr$-$\cr}%
  }}%
}

% Rel symbols
\newcommand{\rel}[1]{\mathrm{#1}}
\newcommand{\relP}{\rel{P}}
\newcommand{\relR}{\rel{R}}
\newcommand{\relQ}{\rel{Q}}
\newcommand{\relS}{\rel{S}}
\newcommand{\relT}{\rel{T}}
\newcommand{\relU}{\rel{U}}
\newcommand{\relA}{\rel{A}}
\newcommand{\relB}{\rel{B}}
\newcommand{\relC}{\rel{C}}
\newcommand{\relD}{\rel{D}}
\newcommand{\relE}{\rel{E}}
\newcommand{\relH}{\rel{H}}
\newcommand{\sig}{\mathsf{sig}}

% Tuples 
\newcommand{\emptytupl}{\varepsilon}
\newcommand{\set}{\mathsf{set}}

% Domain elements
\newcommand{\elem}[1]{\mathrm{#1}}                           % domain element
\newcommand{\elema}{\elem{a}}                             % domain element a
\newcommand{\elemb}{\elem{b}}                             % domain element b
\newcommand{\elemc}{\elem{c}}                             % domain element c
\newcommand{\elemd}{\elem{d}}                             % domain element d
\newcommand{\eleme}{\elem{e}}                             % domain element e
\newcommand{\elemf}{\elem{f}}                               % domain element f
\newcommand{\elemg}{\elem{g}}                             % domain element g
\newcommand{\elemh}{\elem{h}}                             % domain element h
\newcommand{\elemi}{\elem{i}}                             % domain element i
\newcommand{\elemj}{\elem{j}}                             % domain element j
\newcommand{\elemo}{\elem{o}}                             % domain element o
\newcommand{\elemp}{\elem{p}}                             % domain element p
\newcommand{\elemq}{\elem{q}}                             % domain element q
\newcommand{\elemr}{\elem{r}}                             % domain element r
\newcommand{\elems}{\elem{s}}                             % domain element s
\newcommand{\elemt}{\elem{t}}                             % domain element t
\newcommand{\elemw}{\elem{w}}                             % domain element w
\newcommand{\elemv}{\elem{v}}                             % domain element v
\newcommand{\elemu}{\elem{u}}                             % domain element u
\newcommand{\elemx}{\elem{x}}                             % domain element u
\newcommand{\elemtuplea}{\overline{\elema}}                         % tuple of domain element a
\newcommand{\elemtupleb}{\overline{\elemb}}                         % tuple of domain element b
\newcommand{\elemtuplec}{\overline{\elemc}}                         % tuple of domain element c
\newcommand{\elemtupled}{\overline{\elemd}}                         % tuple of domain element d
\newcommand{\elemtuplee}{\overline{\eleme}}                         % tuple of domain element e
\newcommand{\elemtuplef}{\overline{\elemf}}                           % tuple of domain element f
\newcommand{\elemtupleg}{\overline{\elemg}}                         % tuple of domain element g
\newcommand{\elemtupleh}{\overline{\elemh}}                         % tuple of domain element h
\newcommand{\elemtuplei}{\overline{\elemi}}                         % tuple of domain element i
\newcommand{\elemtupleo}{\overline{\elemo}}                         % tuple of domain element o
\newcommand{\elemtuplep}{\overline{\elemp}}                         % tuple of domain element p
\newcommand{\elemtupleq}{\overline{\elemq}}                         % tuple of domain element q
\newcommand{\elemtupler}{\overline{\elemr}}                         % tuple of domain element r
\newcommand{\elemtuples}{\overline{\elems}}                         % tuple of domain element s
\newcommand{\elemtuplet}{\overline{\elemt}}                         % tuple of domain element t
\newcommand{\elemtuplew}{\overline{\elemw}}                         % tuple of domain element w
\newcommand{\elemtupleu}{\overline{\elemu}}                         % tuple of domain element u
\newcommand{\elemtuplev}{\overline{\elemv}}                         % tuple of domain element v
\newcommand{\elemtuplex}{\overline{\elemx}}                         % tuple of domain element v

\newcommand{\elemtupledfromto}[2]{\overline{\elemd}_{#1\ldots#2}}  % tuple of domelements d from #1 to #2
\newcommand{\elemtupleefromto}[2]{\overline{\eleme}_{#1\ldots#2}}  % tuple of domelements e from #1 to #2
\newcommand{\elemtuplecfromto}[2]{\overline{\elemc}_{#1\ldots#2}}  % tuple of domelements c from #1 to #2
\newcommand{\elemtuplebfromto}[2]{\overline{\elemb}_{#1\ldots#2}}  % tuple of domelements b from #1 to #2
\newcommand{\elemtupleafromto}[2]{\overline{\elema}_{#1\ldots#2}}  % tuple of domelements a from #1 to #2
\newcommand{\elemtupleffromto}[2]{{\elemtuplef}_{#1\dots#2}} % tuple of domelements f from #1 to #2
\newcommand{\elemtuplegfromto}[2]{{\elemtupleg}_{#1\dots#2}} % tuple of domelements g from #1 to #2

\DeclareRobustCommand{\elemtuptuplea}{\vv{\elema}}                         % tuple of domain element a
\DeclareRobustCommand{\elemtuptupleb}{\vv{\elemb}}                         % tuple of domain element b
\DeclareRobustCommand{\elemtuptuplec}{\vv{\elemc}}                         % tuple of domain element c
\DeclareRobustCommand{\elemtuptupled}{\vv{\elemd}}                         % tuple of domain element d
\DeclareRobustCommand{\elemtuptuplee}{\vv{\eleme}}                         % tuple of domain element e
\DeclareRobustCommand{\elemtuptuplef}{\vv{\elemf}}                           % tuple of domain element f
\DeclareRobustCommand{\elemtuptupleg}{\vv{\elemg}}                         % tuple of domain element g
\DeclareRobustCommand{\elemtuptupleh}{\vv{\elemh}}                         % tuple of domain element h
\DeclareRobustCommand{\elemtuptuplei}{\vv{\elemi}}                         % tuple of domain element i
\DeclareRobustCommand{\elemtuptuplej}{\vv{\elemj}}                         % tuple of domain element j
\DeclareRobustCommand{\elemtuptuplep}{\vv{\elemp}}                         % tuple of domain element p
\DeclareRobustCommand{\elemtuptupleq}{\vv{\elemq}}                         % tuple of domain element q
\DeclareRobustCommand{\elemtuptupler}{\vv{\elemr}}                         % tuple of domain element r
\DeclareRobustCommand{\elemtuptuples}{\vv{\elems}}                         % tuple of domain element s
\DeclareRobustCommand{\elemtuptuplet}{\vv{\elemt}}                         % tuple of domain element t
\DeclareRobustCommand{\elemtuptuplew}{\vv{\elemw}}                         % tuple of domain element w
\DeclareRobustCommand{\elemtuptupleu}{\vv{\elemu}}                         % tuple of domain element u
\DeclareRobustCommand{\elemtuptuplev}{\vv{\elemv}}                         % tuple of domain element v
\DeclareRobustCommand{\elemtuptuplex}{\vv{\elemx}}                         % tuple of domain element v

% Variables:
\newcommand{\var}[1]{\mathit{#1}}       % variable
\newcommand{\varx}{\var{x}}             % variable x
\newcommand{\vary}{\var{y}}             % variable y
\newcommand{\varz}{\var{z}}             % variable z
\newcommand{\varv}{\var{v}}             % variable v
\newcommand{\varu}{\var{u}}             % variable u
\newcommand{\varw}{\var{w}}             % variable w
\newcommand{\varh}{\var{h}}             % variable h
\newcommand{\vartuplex}{\overline{\varx}}    % tuple of variables x
\newcommand{\vartuplexomega}{\overline{\varx_{\omega}}}      % tuple of variables x_omega
\newcommand{\vartupley}{\overline{\vary}}                    % tuple of variables y
\newcommand{\vartupleyone}{\overline{\vary_1}}                    % tuple of variables y_1
\newcommand{\vartupleytwo}{\overline{\vary_2}}                    % tuple of variables y_2
\newcommand{\vartuplez}{\overline{\varz}}                    % tuple of variables z
\newcommand{\vartuplev}{\overline{\varv}}                    % tuple of variables v
\newcommand{\vartupleu}{\overline{\varu}}                    % tuple of variables u
\newcommand{\vartuplew}{\overline{\varw}}                    % tuple of variables w
\newcommand{\vartupleh}{\overline{\varh}}                    % tuple of variables h
\newcommand{\vartuplexfromto}[2]{\overline{\varx}_{#1\ldots#2}}  % tuple of variables x from #1 to #2
\newcommand{\vartupleyfromto}[2]{\overline{\vary}_{#1\ldots#2}}  % tuple of variables y from #1 to #2
\newcommand{\vartupleufromto}[2]{\overline{\varu}_{#1\ldots#2}}  % tuple of variables u from #1 to #2
\newcommand{\vartuplevfromto}[2]{\overline{\varv}_{#1\ldots#2}}  % tuple of variables v from #1 to #2
\newcommand{\vartuplewfromto}[2]{\overline{\varw}_{#1\ldots#2}}  % tuple of variables v from #1 to #2

% Theory
\newcommand{\theory}[1]{\mathcal{#1}}   % theory
\newcommand{\theoryT}{\theory{T}}       % theory T

% Types
\newcommand{\atp}[3]{\mathsf{atp}^{#1}_{#2}(#3)}
\newcommand{\tp}[3]{\mathsf{tp}^{#1}_{#2}(#3)}

% Tree unravelings
\newcommand{\seq}[1]{\mathsf{seq}(#1)}
\newcommand{\ctr}[1]{\mathsf{ctr}(#1)}
\newcommand{\bound}[1]{\mathsf{bnd}(#1)}
\newcommand{\relNext}{\rel{Next}}
\newcommand{\unraveldom}[1]{\vv{#1}}
\newcommand{\Seq}[1]{\mathsf{Seq}(\str{#1})}
\newcommand{\hist}[2]{\mathsf{hist}_{#1}(#2)}

% Restrictions
\renewcommand{\restriction}{\mathord{\upharpoonright}}
\newcommand{\restr}[2]{#1\restriction_{#2}} % the restriction of #1 to #2

% morphisms
\newcommand{\homo}[1]{\mathfrak{#1}}    % homomorphism
\newcommand{\homof}{\homo{f}}           % homomorphism f
\newcommand{\homog}{\homo{g}}           % homomorphism g
\newcommand{\homoh}{\homo{h}}           % homomorphism h
\newcommand{\homop}{\homo{p}}           % homomorphism p
\newcommand{\homoe}{\homo{e}}           % homomorphism e
\newcommand{\ishomoto}{\vartriangleleft} % is homomorhic to
\newcommand{\homeq}{\rightleftarrows} % homomorphically equivalent
\newcommand{\isoeq}{\cong} % isomorphic
\newcommand{\elemext}{\preceq} % elementary extension
\newcommand{\omegasat}[1]{\widehat{#1}}
\newcommand{\partisof}{\homo{f}}           % partial isomorphism f
\newcommand{\partisog}{\homo{g}}           % partial isomorphism g
\newcommand{\partisoh}{\homo{h}}           % partial isomorphism h

% bisimulations
\newcommand{\bisimulation}[1]{\mathcal{#1}} % a bisimulation
\newcommand{\bisimY}{\bisimulation{Y}}
\newcommand{\bisimZ}{\bisimulation{Z}}
\newcommand{\bisimto}{\sim} % bisimilarity relation
\newcommand{\strbisimto}{\approx} % strong bisimilarity relation
\newcommand{\PartIso}[2]{\mathsf{Part}(#1,#2)}

% tikz helpers
\newcommand{\tikzdbg}{%
  \draw[step=5em,color=lightgray]%
    (current bounding box.south west) grid (0,0)%
    (current bounding box.north west) grid (0,0)%
    (current bounding box.south east) grid (0,0)%
    (current bounding box.north east) grid (0,0);%
  \fill[red] (0,0) circle (0.1);%
}

% Proof sketchs
\let\realproof\proof
\let\realendproof\endproof
% \newenvironment{proofsketch}{%
%   \renewcommand{\proofname}{\normalfont\emph{Proof Sketch}}\realproof}{\realendproof}

% hide proofs
\let\proof\appendixproof
\let\endproof\endappendixproof


\newenvironment{checkenv}{%
  \only{%
    \setbeamertemplate{enumerate item}{\checkmark}%
    \setbeamertemplate{itemize item}{\checkmark}%
  }%
}{}

\catcode`§=\active


\begin{document}

\begin{frame}
  \titlepage
\end{frame}

\begin{frame}{What is the ``Forward Guarded Fragment ($\FGF$)''?}
  \textbf{guarded}: quantification follows pattern ``$\exists{\elemtuplex} (\relR(\vartuplex, \vartupley) \land \varphi(\vartuplex, \vartupley))$''
  \vspace{0.5em}
  \begin{overprint}
    \onslide<-2>
      \begin{exampleblock}{Example 1}
        $\varphi_{1} = \exists{x_{1}x_{2}}\; (\relE(x_{1}, x_{2}) \land (\exists{x_{3}}(\relE(x_{2}, x_{3}) \land \relP(x_{3}))$
        \begin{center}
        \begin{tikzpicture}
  \tikzset{el/.style={circle,draw,fill=white,minimum width=2em,minimum height=2em}};
  \tikzset{relP/.style={tolhighcontrastBlue, line width=0.4em, -{Stealth[round]}, shorten >= 0.4em}};
  \tikzset{relE/.style={tolhighcontrastYellow, line width=0.25em, -{Stealth[round]}, shorten >= 0.2em}};
  \tikzset{relS/.style={tolhighcontrastRed, line width=0.15em, -{Stealth[round]}, shorten >= 0.2em}};
  \def\radiusP{0.7em};
  \def\radiusE{0.4em};
  \def\radiusS{0.2em};

  \matrix[matrix of nodes, every node/.style={el},name=n,column sep=4em, row sep=4em]  {
    |(n1)| $1$ & |(n2)| 2 & |(n3) [fill=tollightLightcyan]| 3 \\
  };
  \begin{scope}[on background layer]
    \fill[relE] (n1.east) circle [radius=\radiusE] coordinate (p1);
    \fill[relE] (n2.west) circle [radius=\radiusE] coordinate (p2);
    \draw[relE] (p1) to node[above]{E} (p2);
    \fill[relE] (n2.east) circle [radius=\radiusE] coordinate (p1);
    \fill[relE] (n3.west) circle [radius=\radiusE] coordinate (p2);
    \draw[relE] (p1) to node[above]{E} (p2);
  \end{scope}

  \node[above right=0em of n3, color=tollightLightcyan] { P };
\end{tikzpicture}

        \end{center}
      \end{exampleblock}
      \uncover<2>{
      \begin{alertblock}{Counterexample 1}
        $\varphi_{2} = \exists{x_{1}x_{2}}\; (\relE(x_{1}, x_{2}) \land (\exists{x_{3}}(\relE(x_{2}, x_{3}) \land \relE(x_{3}, x_{1}))$
        \begin{center}
        \begin{tikzpicture}
  \tikzset{el/.style={circle,draw,fill=white,minimum width=2em,minimum height=2em}};
  \tikzset{relP/.style={tolhighcontrastBlue, line width=0.4em, -{Stealth[round]}, shorten >= 0.4em}};
  \tikzset{relE/.style={tolhighcontrastYellow, line width=0.25em, -{Stealth[round]}, shorten >= 0.2em}};
  \tikzset{relS/.style={tolhighcontrastRed, line width=0.15em, -{Stealth[round]}, shorten >= 0.2em}};
  \def\radiusP{0.7em};
  \def\radiusE{0.4em};
  \def\radiusS{0.2em};

  \matrix[matrix of nodes, every node/.style={el},name=n,column sep=4em, row sep=4em]  {
    |(n1)| $1$ & |(n2)| 2 & |(n3)| 3 \\
  };
  \begin{scope}[on background layer]
    \fill[relE] (n1.east) circle [radius=\radiusE] coordinate (p1);
    \fill[relE] (n2.west) circle [radius=\radiusE] coordinate (p2);
    \draw[relE] (p1) to node[above]{E} (p2);
    \fill[relE] (n2.east) circle [radius=\radiusE] coordinate (p1);
    \fill[relE] (n3.west) circle [radius=\radiusE] coordinate (p2);
    \draw[relE] (p1) to node[above]{E} (p2);
    \fill[relE] (n3.south) circle [radius=\radiusE] coordinate (p1);
    \fill[relE] (n1.south) circle [radius=\radiusE] coordinate (p2);
    \draw[relE,dashed] (p1) to[out=-160,in=-20] node[below]{E} (p2);
  \end{scope}
\end{tikzpicture}

        \end{center}
      \end{alertblock}
      }
    \onslide<3->
      \begin{exampleblock}{More examples}
        \vspace{1em}
        \hspace{1em}
        \begin{minipage}{20em}
        $\varphi_{3} = \lnot \exists{x_{1}x_{2}}\; (\relG(x_{1}, x_{2}) \land \lnot \relE(x_{2}, x_{1}))$ \\[1em]
        $\varphi_{3} = \forall{x_{1}x_{2}}\; (\relG(x_{1}, x_{2}) \to \relE(x_{2}, x_{1}))$ \\[1em]
        $\varphi_{4} = \forall{x_{1}}\; (\relG(x_{1}, x_{1}) \to \exists{x_{2}} (\relP(x_{1},x_{2}, x_{1})))$
        \end{minipage}
      \end{exampleblock}
  \end{overprint}
\end{frame}

\begin{frame}{What is the ``Forward Guarded Fragment ($\FGF$)''?}
  \textbf{forward}: variables must appear in order $x_{1}$, $x_2$, \ldots
  \vspace{0.5em}
  \begin{alertblock}{Counterexamples}
    $\varphi_{3} = \forall{x_{1}x_{2}}\; (\relG(x_{1}, x_{2}) \to \relE(x_{2}, x_{1}))$ \\[1em]
    $\varphi_{5} = \exists{x_{1}}\; (\relE(x_{1}, x_{1}))$ \\[1em]
    $\varphi_{6} = \exists{x_{1}x_{2}x_{3}}\; (\relP(x_{1}, x_{2}, x_{3}) \land \relE(x_{1}, x_{3}))$ \\[1em]
    $\varphi_{7} = \exists{x_{2}x_{1}} \relE(x_{2}, x_{1})$
  \end{alertblock}

  \begin{exampleblock}{Examples}
    $\varphi_{1} = \exists{x_{1}x_{2}}\; (\relE(x_{1}, x_{2}) \land (\exists{x_{3}}(\relE(x_{2}, x_{3}) \land \relP(x_{3}))$ \\[1em]

    $\varphi_{8} = \exists{x_{1}x_{2}x_{3}}\; (\relP(x_{1}x_{2}x_{3}) \land \exists{x_{3}}(\relE(x_{2}, x_{3})))$ \\[1em]
    $\varphi_{9} = \exists{x_{1}x_{2}x_{3}}\; (\relP(x_{1}x_{2}x_{3}) \land \relE(x_{2}, x_{3}))$ \\[1em]
  \end{exampleblock}

  \vskip 0pt plus 1filll
\end{frame}

\begin{frame}{How expressive is $\FGF$?}
  \begin{center}
    \begin{tikzpicture}
  \tikzset{faded edge/.style={dash pattern=on 1pt off 1pt on 1pt off 1pt on 2pt off 1pt on 2pt off 1pt on 100pt}};
  \tikzset{faded edge reverse/.style={tips=false,dash pattern=on 5pt off 1pt on 5pt off 1pt on 2pt off 1pt on 2pt off 1pt on 1pt off 1pt on 1pt off 1pt on 1pt off 1pt on 1pt off 1pt on 1pt off 100pt}};

  \begin{scope}
    \graph[empty nodes, nodes={draw,circle}] {
        subgraph C_n [n=7,clockwise,radius=1.5cm];
    };
    \node[font=\LARGE] (strA) at (0,0) { $\str{A}$ };
    \draw[faded edge reverse] (1) -- (strA);
    \draw[faded edge reverse] (7) -- (2);
    \draw[faded edge reverse] (7) -- (4);
    \draw[faded edge reverse] (2) -- (6);
    \draw[faded edge reverse] (4) -- (1);
    \draw[faded edge reverse] (5) -- (3);
    \draw[faded edge reverse] (6) -- (3);
  \end{scope}

  \begin{scope}[xshift=15em]
    \graph[empty nodes, nodes={draw,circle}] {
        subgraph C_n [n=7,clockwise,radius=1.5cm];
    };
    \node[font=\LARGE] (strB) at (0,0) { $\str{B}$ };
    \draw[faded edge reverse] (1) -- (4);
    \draw[faded edge reverse] (1) -- (5);
    \draw[faded edge reverse] (7) -- (3);
    \draw[faded edge reverse] (2) -- (6);
    \draw[faded edge reverse] (4) -- (1);
    \draw[faded edge reverse] (5) -- (3);
    \draw[faded edge reverse] (6) -- (3);
    \draw[faded edge reverse] (3) -- (strB);
  \end{scope}

  \node[font=\LARGE] at ($(strA)!.5!(strB)$) { $\bisimto_{\FGF}$ };

\end{tikzpicture}

  \end{center}
  \begin{itemize}
    \item \emph{FGF-bisimulation}: equivalence relation on structures
    \item $\FGF$-formulae are $\FGF$-bisimulation-invariant
    \item \textbf{Can we express all $\FGF$-bisimulation-invariant $\FO$-definable properties in $\FGF$?}
  \end{itemize}
\end{frame}

\begin{frame}\frametitle<1>{Van Benthem's theorem}\frametitle<2>{Van Benthem's theorem (finitary version)}
  \begin{theorem}[Van Benthem's theorem for $\FGF$]
    A first-order formula $\varphi$ is equivalent to some $\FGF$-formula\only<2>{ \textbf{over finite structures}} if and only if it is invariant under $\FGF$-bisimulation\only<2>{ \textbf{over finite structures}}.
  \end{theorem}
\end{frame}

\begin{frame}<-2>[label=outline]{Outline}%
  \begin{enumerate}
    \item<alert@2| check@3-> Definitions: $\GF$-, $\FGF$- and $\ell$-bisimulation
    \item<check@3-> Observation: $\ell$-$\FGF$-invariance implies expressibility in $\FGF$
    \item<alert@4>
          Upgrading: $\FGF$-invariance implies $\ell$-$\FGF$-invariance
          \only<3->{\uncover<4->{\\\textcolor{black}{\small(Martin Otto has proved the same for $\GF$)}}}
  \end{enumerate}

  \onslide*<3->{
  \uncover<4->{
    \begin{itemize}
      \item[]
            \begin{center}
              \begin{tikzpicture}
\matrix[row sep=2em, column sep=3em]
{
    \node (a) {$(\str{A}, \elemtuplea)$}; &
        \node[font=\large] (sim_ab) {$\bisimto_{\FGF}^{\homop(\mathit{g})}$}; &
    \node (b) {$(\str{B}, \elemtupleb)$}; &
    \node[anchor=west] {}; \\

    \node[font=\large] (sim_a_unravel) {$\bisimto_{\FGF}$}; &
    \node {}; &
    \node[font=\large] (sim_b_unravel) {$\bisimto_{\FGF}$}; &
    \node[anchor=west] {}; \\

    \node (a_unravel) {$(\str{A}', \elemtuplea')$}; &
    \node[font=\large] (sim_unravel) {$\bisimto_{\GF}^{\mathit{g}}$}; &
    \node (b_unravel) {$(\str{B}', \elemtupleb')$};
    \node[anchor=west] {}; \\
};

\draw[->] (a) -- (sim_ab) -> (b);
\draw[dashed,->] (a) -- (sim_a_unravel) -> (a_unravel);
\draw[->] (a_unravel) -- (sim_unravel) -> (b_unravel);
\draw[dashed,->] (b_unravel) -- (sim_b_unravel) -> (b);

\node[below=1.5em of sim_unravel, font=\bfseries] (companion_label) { finite companions };
\draw[->] (companion_label) -- (a_unravel);
\draw[->] (companion_label) -- (b_unravel);
\end{tikzpicture}

            \end{center}
    \end{itemize}
  }
  }
\end{frame}

\begin{frame}{$\GF$-bisimulation: partial isomorphism}
  A $\GF$-bisimulation is a set $\mathcal{Z}$ of \emph{partial isomorphisms} between two structures, satisfying \emph{back-and-forth conditions} (next slide)
  \vfill
  \begin{center}
  \begin{tikzpicture}

  \tikzset{faded edge/.style={dash pattern=on 1pt off 1pt on 1pt off 1pt on 2pt off 1pt on 2pt off 1pt on 100pt}};
  \tikzset{faded edge reverse/.style={tips=false,dash pattern=on 5pt off 1pt on 5pt off 1pt on 2pt off 1pt on 2pt off 1pt on 1pt off 1pt on 1pt off 1pt on 1pt off 1pt on 1pt off 1pt on 1pt off 100pt}};

  \tikzset{relP/.style={tolhighcontrastBlue, line width=0.25em, -{Stealth[round]}, shorten >= 0.4em}};
  \tikzset{relE/.style={tolhighcontrastYellow, line width=0.15em, -{Stealth[round]}, shorten >= 0.2em}};
  \tikzset{relS/.style={tolhighcontrastRed, line width=0.10em, -{Stealth[round]}, shorten >= 0.2em}};
  \def\radiusE{0.2em};
  \def\radiusP{0.4em};
  \def\radiusS{0.2em};

  \def\r{1.3cm};
  \def\rnode{0.6cm};

  \begin{scope}[name prefix=a]
    \foreach \i in {0, ..., 6} {
      \draw[gray] (\i*360/7:\r) -- ({(\i+1)*360/7}:\r);
    };

    \node at (0, 1.3*\r) { $\str{A}$ };

    \begin{scope}[every node/.style={draw,circle,inner sep=0.5ex,fill=white,font=\small}]
      \node (1) at (4*360/14:1.2*\rnode) {2};
      \node (2) at (9*360/14:1.1*\rnode) {3};
      \node (3) at (0*360/14:\rnode) {1};
    \end{scope}

    \begin{scope}[on background layer]
      \fill[relP] (1.west) circle [radius=\radiusP] coordinate (p1);
      \fill[relP] (2.south west) circle [radius=\radiusP] coordinate (p2);
      \fill[relP] (3.south) circle [radius=\radiusP] coordinate (p3);
      \draw[relP] (p1) to[out=-120,in=120] (p2) to[out=-60,in=-100] (p3);

      \fill[relS] (1.south east) circle [radius=\radiusS] coordinate (p1);
      \fill[relS] (3.north west) circle [radius=\radiusS] coordinate (p2);
      \draw[relS] (p1) to (p2);

      \fill[relE] (2.north) circle [radius=\radiusE] coordinate (p1);
      \fill[relE] (1.240) circle [radius=\radiusE] coordinate (p2);
      \draw[relE] (p1) to (p2);
    \end{scope}
  \end{scope}

  \begin{scope}[xshift=15em, name prefix=b]
    \foreach \i in {0, ..., 7} {
      \draw[gray] (\i*360/8:\r) -- ({(\i+1)*360/8}:\r);
    };
    \node at (0, 1.3*\r) { $\str{B}$ };

    \begin{scope}[every node/.style={draw,rectangle,inner sep=0.7ex,font=\large,fill=white}]
      \node (1) at (3*360/16:\rnode) {b};
      \node (2) at (-3*360/16:\rnode) {c};
      \node (3) at (180:\rnode) {a};
    \end{scope}

    \begin{scope}[on background layer]
      \fill[relP] (1.east) circle [radius=\radiusP] coordinate (p1);
      \fill[relP] (2.south east) circle [radius=\radiusP] coordinate (p2);
      \fill[relP] (3.south) circle [radius=\radiusP] coordinate (p3);
      \draw[relP] (p1) to[out=-60,in=60] (p2) to[out=-120,in=-90] (p3);

      \fill[relS] (1.west) circle [radius=\radiusS] coordinate (p1);
      \fill[relS] (3.north) circle [radius=\radiusS] coordinate (p2);
      \draw[relS] (p1) to (p2);

      \fill[relE] (2.north) circle [radius=\radiusE] coordinate (p1);
      \fill[relE] (1.south) circle [radius=\radiusE] coordinate (p2);
      \draw[relE] (p1) to (p2);
    \end{scope}
  \end{scope}

  \coordinate (mid) at ($(a3)!0.5!(b3)$);
  \node[below=3em of mid] { \emph{partial isomorphism} $\homop$ };

  \begin{scope}[dotted, very thick]
    \draw (a1) -- (b1);
    \draw (a2) -- (b2);
    \draw (a3) -- (b3);
  \end{scope}[dashed]
\end{tikzpicture}

  \end{center}
\end{frame}

\begin{frame}
  \frametitle{$\GF$-bisimulation: back-and-forth}
  \begin{center}
    \tikzset{
  /backandforth/.cd,
  default/.style = {
    edges1/.style = {hidden},
    step1/.style = {},
    step2/.style = {},
  },
  ante/.style = {
    /backandforth/default,
    step2/.style = {hidden}
  },
  conseq/.style = {
    /backandforth/default,
    step1/.style = {hidden},
  },
}

\def\gfbackandforth#1{
\begin{tikzpicture}[/backandforth/#1]
  \tikzset{faded edge/.style={dash pattern=on 1pt off 1pt on 1pt off 1pt on 2pt off 1pt on 2pt off 1pt on 100pt}};
  \tikzset{faded edge reverse/.style={tips=false,dash pattern=on 5pt off 1pt on 5pt off 1pt on 2pt off 1pt on 2pt off 1pt on 1pt off 1pt on 1pt off 1pt on 1pt off 1pt on 1pt off 1pt on 1pt off 100pt}};

  \tikzset{relP/.style={tolhighcontrastBlue, line width=0.2em, -{Stealth[round]}, shorten >= 0.4em}};
  \tikzset{relE/.style={tolhighcontrastYellow, line width=0.15em, -{Stealth[round]}, shorten >= 0.2em}};
  \tikzset{relS/.style={tolhighcontrastRed, line width=0.10em, -{Stealth[round]}, shorten >= 0.2em}};
  \tikzset{relQ/.style={black, line width=0.25em, -{Stealth[round]}, shorten >= 0.4em}};
  \def\radiusE{0.2em};
  \def\radiusP{0.4em};
  \def\radiusS{0.2em};
  \def\radiusQ{0.4em};

  \def\r{1.9cm};
  \def\rnode{0.5cm};

  \tikzset{egroup1/.style={draw,tolhighcontrastBlue, dashed, very thick}};
  \tikzset{egroup2/.style={fill,tolpalePaleyellow}};

  \begin{scope}[name prefix=a]
    \foreach \i in {0, ..., 6} {
      \draw[gray] (\i*360/7:\r) -- ({(\i+1)*360/7}:\r);
    };

    \begin{scope}[every node/.style={draw,circle,inner sep=0.5ex,fill=white,font=\small}, yshift=0.5cm, xshift=-0.2cm]
      \path {
        (3*360/14:1.2*\rnode) node (1) { 2 }
        ++(-110:1.8*\rnode) node (2) { 3 }
        ++(0:1.8*\rnode) node (3) { 1 }
        ++(-80:2*\rnode) node (4) { 4 }
        ++(-180:2*\rnode) node (5) { 5 }
      };
    \end{scope}

    \begin{scope}[on background layer]
      \node[fit={(2) (3) (4) (5)}, circle, inner sep=0pt, xshift=-0.1em, yshift=-0.1em, egroup2] (eg2) {};
      \node[fit={(1) (2) (3)}, ellipse, inner sep=0ex, yshift=-0.05*\rnode, egroup1] (eg1) {};
    \end{scope}

    \begin{scope}[on background layer, every path/.style=edges1]
      \fill[relP] (1.west) circle [radius=\radiusP] coordinate (p1);
      \fill[relP] (2.south west) circle [radius=\radiusP] coordinate (p2);
      \fill[relP] (3.south) circle [radius=\radiusP] coordinate (p3);
      \draw[relP] (p1) to[out=-120,in=120] (p2) to[out=-60,in=-100] (p3);

      \fill[relS] (1.south east) circle [radius=\radiusS] coordinate (p1);
      \fill[relS] (3.north west) circle [radius=\radiusS] coordinate (p2);
      \draw[relS] (p1) to (p2);

      \fill[relE] (2.60) circle [radius=\radiusE] coordinate (p1);
      \fill[relE] (1.240) circle [radius=\radiusE] coordinate (p2);
      \draw[relE] (p1) to (p2);
    \end{scope}
  \end{scope}


  \begin{scope}[xshift=15em, name prefix=b]
    \foreach \i in {0, ..., 7} {
      \draw[gray] (\i*360/8:\r) -- ({(\i+1)*360/8}:\r);
    };

    \begin{scope}[every node/.style={draw,rectangle,inner sep=0.7ex,font=\large,fill=white}, yshift=0.5cm]
      \path {
        (4*360/14:1.2*\rnode) node (1) { b }
        ++(-110:1.8*\rnode) node (3) { a }
        ++(0:1.8*\rnode) node (2) { c }
        ++(-80:2*\rnode) node[step2] (5) { e }
        ++(-180:2*\rnode) node[step2] (4) { d }
      };
    \end{scope}

    \begin{scope}[on background layer]
      \node[fit={(2) (3) (4) (5)}, circle, inner sep=0pt, xshift=-0.1em, yshift=-0.1em, egroup2, step2] (eg2) {};
      \node[fit={(1) (2) (3)}, circle, inner sep=0ex, yshift=-0.05*\rnode, egroup1] (eg1) {};
    \end{scope}

    \begin{scope}[on background layer, every path/.style={edges1}]
      \fill[relP] (1.east) circle [radius=\radiusP] coordinate (p1);
      \fill[relP] (2.south east) circle [radius=\radiusP] coordinate (p2);
      \fill[relP] (3.south) circle [radius=\radiusP] coordinate (p3);
      \draw[relP] (p1) to[out=-20,in=45] (p2) to[out=-120,in=-90] (p3);

      \fill[relS] (1.west) circle [radius=\radiusS] coordinate (p1);
      \fill[relS] (3.north) circle [radius=\radiusS] coordinate (p2);
      \draw[relS] (p1) to (p2);

      \fill[relE] (2.north) +(-0.2em,0) circle [radius=\radiusE] coordinate (p1);
      \fill[relE] (1.south) circle [radius=\radiusE] coordinate (p2);
      \draw[relE] (p1) to (p2);
    \end{scope}
  \end{scope}

  \path[egroup1,solid,step1] (aeg1) to[bend left] node[midway, above] {$\homop \in \mathcal{Z}$} (beg1);
  \path[egroup2,solid,draw,fill=none,tolbrightYellowDarker, step2] (aeg2) to[bend left=10] node[midway, above] {$\homof \in \mathcal{Z}$} (beg2);

  \coordinate (center) at (7.5em, 0);
  \node[step1, right=1.5em of a4, yshift=-0.5em, anchor=north west, color=tolbrightYellowDarker!80!black] (gset) {guarded set};
  \draw[step1, solid, tolbrightYellowDarker] (aeg2) -- (gset);
\end{tikzpicture}
}

    \gfbackandforth{ante}
    \pause\vfill
    \gfbackandforth{conseq}
    \begin{tikzpicture}[remember picture,overlay]
      \node[yshift=-2.5em, xshift=-1em, rotate=-90, font=\LARGE\bfseries] (implies) at (current page.center) { $\implies$ };
      \node[right=0.5em of implies.center] { (forth) };
    \end{tikzpicture}
  \end{center}
\end{frame}

\begin{frame}{$\GF$-bisimulation: example}
\end{frame}

\begin{frame}{$\FGF$-bisimulation: forward partial map}
  An $\FGF$-bisimulation is a set $\mathcal{Z}$ of \emph{forward partial maps} between two structures, satisfying \emph{back-and-forth conditions} (next slide)
  \begin{center}
    \def\forwardmap#1{%
\begin{tikzpicture}
  \tikzset{nonforwardedges/.style={opacity=0.2}}
  \tikzset{tupleedges/.style={}}
  \tikzset{tuples/.style={}}
  \tikzset{caption/.style={}}
  \tikzset{#1}

  \tikzset{faded edge/.style={dash pattern=on 1pt off 1pt on 1pt off 1pt on 2pt off 1pt on 2pt off 1pt on 100pt}};
  \tikzset{faded edge reverse/.style={tips=false,dash pattern=on 5pt off 1pt on 5pt off 1pt on 2pt off 1pt on 2pt off 1pt on 1pt off 1pt on 1pt off 1pt on 1pt off 1pt on 1pt off 1pt on 1pt off 100pt}};

  \tikzset{relP/.style={tolhighcontrastBlue, line width=0.25em, -{Stealth[round]}, shorten >= 0.4em}};
  \tikzset{relE/.style={tolhighcontrastYellow, line width=0.15em, -{Stealth[round]}, shorten >= 0.2em}};
  \tikzset{relS/.style={tolhighcontrastRed, line width=0.10em, -{Stealth[round]}, shorten >= 0.2em}};
  \def\radiusE{0.2em};
  \def\radiusP{0.4em};
  \def\radiusS{0.2em};

  \def\r{1.3cm};
  \def\rnode{0.6cm};

  \begin{scope}[name prefix=a]
    \foreach \i in {0, ..., 6} {
      \draw[gray] (\i*360/7:\r) -- ({(\i+1)*360/7}:\r);
    };

    \node at (0, 1.3*\r) { $\str{A}$ };

    \begin{scope}[every node/.style={draw,circle,inner sep=0.5ex,fill=white,font=\small}]
      \node (1) at (4*360/14:1.2*\rnode) {1};
      \node (2) at (9*360/14:1.1*\rnode) {2};
      \node (3) at (0*360/14:\rnode) {3};
    \end{scope}

    \begin{scope}[on background layer]
      \fill[relP] (1.west) circle [radius=\radiusP] coordinate (p1);
      \fill[relP] (2.south west) circle [radius=\radiusP] coordinate (p2);
      \fill[relP] (3.south) circle [radius=\radiusP] coordinate (p3);
      \draw[relP] (p1) to[out=-120,in=120] (p2) to[out=-60,in=-100] (p3);

      \fill[relS] (1.south east) circle [radius=\radiusS] coordinate (p1);
      \fill[relS] (3.north west) circle [radius=\radiusS] coordinate (p2);
      \draw[relS] (p1) to (p2);

      \fill[relE] (1.240) circle [radius=\radiusE] coordinate (p1);
      \fill[relE] (2.north) circle [radius=\radiusE] coordinate (p2);
      \draw[relE] (p1) to (p2);

      \fill[relE] (3.200) circle [radius=\radiusE] coordinate (p1);
      \fill[relE] (2.30) circle [radius=\radiusE] coordinate (p2);
      \draw[relE] (p1) to (p2);
    \end{scope}
  \end{scope}

  \begin{scope}[xshift=15em, name prefix=b]
    \foreach \i in {0, ..., 7} {
      \draw[gray] (\i*360/8:\r) -- ({(\i+1)*360/8}:\r);
    };
    \node at (0, 1.3*\r) { $\str{B}$ };

    \begin{scope}[every node/.style={draw,rectangle,inner sep=0.7ex,font=\large,fill=white}]
      \node (1) at (7*360/16:\rnode) {a};
      \node (2) at (15*360/16:\rnode) {b};
    \end{scope}

    \begin{scope}[on background layer]
      \fill[relP] (1.south) circle [radius=\radiusP] coordinate (p1);
      \fill[relP] (2.south west) circle [radius=\radiusP] coordinate (p2);
      \fill[relP] (2.south east) circle [radius=\radiusP] coordinate (p3);
      \draw[relP] (p1) to[out=-60,in=150] (p2) to[out=-90,in=-120, looseness=3] (p3);

      \fill[relE] (1.east) circle [radius=\radiusE] coordinate (p1);
      \fill[relE] (2.west) circle [radius=\radiusE] coordinate (p2);
      \draw[relE] (p1) to (p2);

      \fill[relS] (2.north) circle [radius=\radiusS] coordinate (p1);
      \fill[relS] (1.north) circle [radius=\radiusS] coordinate (p2);
      \draw[relS] (p1) to[out=90,in=90] (p2);

    \end{scope}
  \end{scope}

  \begin{scope}[yshift=-2.5cm, name prefix=s, tuples]
    \node (lab) { $\elemtuples = (1,2,3)$ };
    \matrix[
    matrix of nodes,
    ampersand replacement=\&,
    nodes={draw,circle,fill=white},
    column sep=1.5em,
    below=0em of lab,
    ] (n) {
      |(1)| 1 \& |(2)| 2 \& |(3)| 3 \\
    };

    \begin{scope}[on background layer, tupleedges]
      \fill[relE] (1.east) circle [radius=\radiusE] coordinate (p1);
      \fill[relE] (2.west) circle [radius=\radiusE] coordinate (p2);
      \draw[relE] (p1) to (p2);

      \fill[relP] (1.south) circle [radius=\radiusP] coordinate (p1);
      \fill[relP] (2.south) circle [radius=\radiusP] coordinate (p2);
      \fill[relP] (3.south) circle [radius=\radiusP] coordinate (p3);
      \draw[relP] (p1) to[out=-60,in=-150] (p2) to[out=-60,in=-150] (p3);

      \begin{scope}[transparency group, nonforwardedges]
        \fill[relE] (3.west) circle [radius=\radiusE] coordinate (p1);
        \fill[relE] (2.east) circle [radius=\radiusE] coordinate (p2);
        \draw[relE] (p1) to (p2);

        \fill[relS] (1.60) circle [radius=\radiusS] coordinate (p1);
        \fill[relS] (3.120) circle [radius=\radiusS] coordinate (ps2);
        \draw[relS] (p1) to[bend left=20] (ps2);
      \end{scope}
  \end{scope}

  \end{scope}

  \begin{scope}[yshift=-2.5cm, xshift=15em, name prefix=t, tuples]
    \node (lab) { $\elemtuplet = (a,b,b)$ };
    \matrix[
    matrix of nodes,
    ampersand replacement=\&,
    nodes={draw,rectangle,fill=white,minimum height=3ex,anchor=center},
    column sep=1.5em,
    below=0em of lab,
    ] (n) {
      |(1)| a \& |(2)| b \& |(3)| b \\
    };

    \begin{scope}[on background layer, tupleedges]
      \fill[relE] (1.east) circle [radius=\radiusE] coordinate (p1);
      \fill[relE] (2.west) circle [radius=\radiusE] coordinate (p2);
      \draw[relE] (p1) to (p2);

      \fill[relP] (1.south) circle [radius=\radiusP] coordinate (p1);
      \fill[relP] (2.south) circle [radius=\radiusP] coordinate (p2);
      \fill[relP] (3.south) circle [radius=\radiusP] coordinate (p3);
      \draw[relP] (p1) to[out=-60,in=-150] (p2) to[out=-60,in=-150] (p3);

      \begin{scope}[transparency group, nonforwardedges]
        \fill[relE] (3.west) circle [radius=\radiusE] coordinate (p1);
        \fill[relE] (2.east) circle [radius=\radiusE] coordinate (p2);
        \draw[relE] (p1) to (p2);

        \fill[relS] (2.120) circle [radius=\radiusS] coordinate (p1);
        \fill[relS] (1.60) circle [radius=\radiusS] coordinate (ps2);
        \draw[relS] (p1) to[bend right=45] (ps2);
      \end{scope}
    \end{scope}

  \end{scope}

  \coordinate (mid) at ($(s3)!0.5!(t1)$);
  \node[below=2.5em of mid, caption] { \emph{forward partial map $(\elemtuples, \elemtuplet)$} };
\end{tikzpicture}
}

    \forwardmap{
      nonforwardedges/.style = {
        onslide=<3>{opacity=0.2},
      },
      tuples/.style = {
        onslide=<1>{hidden, every path/.style=hidden},
      },
      caption/.style={
        onslide=<-2>{hidden},
      }
    }
  \end{center}
\end{frame}

\begin{frame}{$\FGF$-bisimulation: back and forth}
  \def\fgfbackandforth#1{
\begin{tikzpicture}
  \tikzset{tuples/.style={}}
  \tikzset{ante/.style={}}
  \tikzset{conseq/.style={}}
  \tikzset{tupler/.style={}}
  \tikzset{infix/.style={hidden}}
  \tikzset{infixnodes/.style={}}
  \tikzset{#1}

  \tikzset{pmap1/.style={tolhighcontrastBlue, dashed}};

  \tikzset{faded edge/.style={dash pattern=on 1pt off 1pt on 1pt off 1pt on 2pt off 1pt on 2pt off 1pt on 100pt}};
  \tikzset{faded edge reverse/.style={tips=false,dash pattern=on 5pt off 1pt on 5pt off 1pt on 2pt off 1pt on 2pt off 1pt on 1pt off 1pt on 1pt off 1pt on 1pt off 1pt on 1pt off 1pt on 1pt off 100pt}};

  \tikzset{relP/.style={tolhighcontrastBlue, line width=0.25em, -{Stealth[round]}, shorten >= 0.4em}};
  \tikzset{relE/.style={tolhighcontrastYellow, line width=0.15em, -{Stealth[round]}, shorten >= 0.2em}};
  \tikzset{relS/.style={tolhighcontrastRed, line width=0.10em, -{Stealth[round]}, shorten >= 0.2em}};
  \def\radiusE{0.2em};
  \def\radiusP{0.4em};
  \def\radiusS{0.2em};


  \begin{scope}[name prefix=s, tuples]
    \matrix[
    matrix of nodes,
    ampersand replacement=\&,
    nodes={draw,circle,outer sep=0.5ex,fill=white,minimum height=4ex},
    column sep=1em,
    row sep=1.5em,
    matrix anchor=s2.center,
    ] (n) at (0,0) {
      |(1)| 1 \& |(2) [infixnodes]| 2 \& |(3) [infixnodes]| 3 \\
      \& \& |(4) [tupler]| 4 \\
    };

    \node[draw, rectangle, rounded corners=3pt, pmap1, fit={(1) (2) (3)}] (tuple) {};
    \node[above left=0em of tuple, anchor=south west] { tuple $\elemtuples$ };

    \begin{scope}[on background layer]
      \path[fill=tolpalePaleyellow, rounded corners, tupler] {
        (2.north -| 2.west)
        -- (3.north -| 3.east)
        -- (4.south -| 4.east)
        -- (4.south -| 4.west)
        -- (3.south -| 3.west)
        -- (2.south -| 2.west)
        -- cycle
      };
    \end{scope}

    \begin{scope}[on background layer]
      \path[fill=tolpalePaleyellow, rounded corners, infix] {
        (2.north -| 2.west)
        -- (3.north -| 3.east)
        -- (3.south -| 3.east)
        -- (2.south -| 2.west)
        -- cycle
      };
    \end{scope}


    \node[tolbrightYellowDarkest, left=0.5ex of 4, text width=3em, align=right, tupler] (nextlab) { tuple $\elemtupler$ \\ (live) };
  \end{scope}

  \begin{scope}[xshift=18em, name prefix=t, tuples]
    \matrix[
    matrix of nodes,
    ampersand replacement=\&,
    nodes={draw,rectangle,fill=white,minimum height=4ex,minimum width=3ex, anchor=center,outer sep=0.5ex},
    column sep=1em,
    row sep=1.5em,
    matrix anchor=t2.center,
    ] (n) at (0,0) {
      |(1)| a \& |(2)| b \& |(3)| b \\
      \& \& |(4) [conseq]| c \\
    };

    \node[draw, rectangle, rounded corners=3pt, pmap1, fit={(1) (2) (3)}] (tuple) {};
    \node[above left=0em of tuple, anchor=south west] { tuple $\elemtuplet$ };

    \begin{scope}[on background layer]
      \path[fill=tolpalePaleyellow, rounded corners, conseq] {
        (2.north -| 2.west)
        -- (3.north -| 3.east)
        -- (4.south -| 4.east)
        -- (4.south -| 4.west)
        -- (3.south -| 3.west)
        -- (2.south -| 2.west)
        -- cycle
      };
    \end{scope}

    \node[tolbrightYellowDarkest, left=0.5ex of 4, text width=3.5em, align=right, conseq] (nextlab) { tuple $\elemtupleq$ \\ (live) };
  \end{scope}

  \draw[thick, pmap1, solid, ante] (stuple) to node[above, midway, text width=7em] {forward partial map $(\elemtuples, \elemtuplet)$} (ttuple);

  \draw[tolbrightYellowDarkest, thick, conseq] (s4.east) to node[above, midway, text width=7em] {forward partial map $(\elemtupleq, \elemtupler)$} (tnextlab.west);

\end{tikzpicture}
}

  \begin{center}
    \fgfbackandforth{
      conseq/.style={hidden, overlay},
      tupler/.style={uncover=<3->},
      infix/.style={uncover=<2>}
    }
  \uncover<4->{
  \vfill
  \fgfbackandforth{ante/.style={hidden, overlay}}
  \begin{tikzpicture}[remember picture,overlay]
    \node[xshift=-1em, rotate=-90, font=\LARGE\bfseries] (implies) at (current page.center) { $\implies$ };
    \node[right=0.5em of implies.center] { (forth) };
  \end{tikzpicture}
  }
  \end{center}
\end{frame}

\begin{frame}{$\FGF$-bisimulation: example}
  \begin{center}
    \begin{tikzpicture}[x=1em, y=1em]
  \tikzset{el/.style={minimum width=3ex, minimum height=3ex, circle, draw}};
  \tikzset{rel2/.style={dashed}}
  \tikzset{p1/.style={color=white, fill=tolmutedGreen}}
  \tikzset{p2/.style={color=white, pattern color=tolvibrantRed, pattern={bricks}, preaction={fill, tolmutedWine}}};
  \tikzset{p3/.style={pattern color=tolbrightYellow, pattern={north west lines}}};
  \tikzset{p4/.style={color=white, pattern color=tolbrightBlue, pattern={checkerboard}, preaction={fill, black}}};

  \matrix [
  nodes={el}, matrix of nodes, row sep=2em, column sep=1em, anchor=north east
  ] (left) at (-1.5,0) {
        & {1} & {2} &     \\
    {6} &     & |[p3]| {3} & |[p1]| {4} \\
    {7} & |[p2]| {8} &     & |[p1]| {5} \\
        & {9} & |[p4]| {10} &     \\
  };

  \matrix [
  nodes={el}, matrix of nodes, row sep=1.5em, column sep=1.5em, anchor=north west
  ] (right) at (1.5,0) {
        & {a} & {b} &     \\
     {c} & |[p3]| {o}   & |[p3]| {d} & |[p1]| {e} \\
    |[p2]| {f} & |[p2]| {p}   & |[p2]| {g} & |[p1]| {h} \\
        & {i} & |[p4]| {j} &     \\
        & |[p1]| {k} & |[p1]| {l} & |[p1]| {m} \\
  };

  \path[->, very thick]{
    (left-1-3) edge (left-1-2)
    (left-1-3) edge (left-2-4)
    (left-2-4) edge[loop below, looseness=6, in=-110, out=-70] (left-2-4)
    (left-1-2) edge (left-2-1)
    (left-1-2) edge (left-2-3)
    (left-2-1) edge (left-3-2)
    (left-2-3) edge (left-3-2)
    (left-4-2) edge[bend left] (left-4-3)
    (left-4-2) edge[bend right, rel2] (left-4-3)
    (left-3-4) edge[loop below, looseness=6, in=-110, out=-70] (left-3-4)
  };

  \path[->, very thick] {
    (right-1-3) edge (right-1-2)
    (right-1-3) edge (right-2-4)
    (right-2-4) edge[bend left] (right-3-4)
    (right-3-4) edge[bend left] (right-2-4)
    (right-1-2) edge (right-2-1)
    (right-1-2) edge (right-2-3)
    (right-2-1) edge (right-3-1)
    (right-2-3) edge (right-3-3)
    (right-4-2) edge[bend left] (right-4-3)
    (right-4-2) edge[bend right, rel2] (right-4-3)
    (right-5-2) edge (right-5-3)
    (right-5-3) edge (right-5-4)
    (right-5-4) edge[bend left] (right-5-2)
    (right-1-2) edge (right-2-2)
    (right-2-2) edge (right-3-2)
  };

  \coordinate (bot) at (current bounding box.south -| 0,0);
  \draw[dashed] (bot) -- (current bounding box.north -| 0,0);
  % \draw ($(current bounding box.south east) + (2em, 0)$) rectangle (current bounding box.north west);

  \begin{scope}[font=\Large]
    \path (bot -| left.center) +(0,-1.5) node {$\str{A}$};
    \path (bot -| right.center) +(0,-1.5) node {$\str{B}$};
    \path (bot) +(0,-1.5) node {$\bisimto_{\FGF}$};
  \end{scope}

\end{tikzpicture}

  \end{center}
\end{frame}

\begin{frame}{$\ell$-bisimulation}
  The sets $\mathcal{Z}_{0} \supseteq \cdots \supseteq \mathcal{Z}_{\ell}$ form an \textbf{$\ell$-bisimulation} ($\bisimto^{\ell}$) if $\mathcal{Z}_{i}$ contains the witnesses for the back-and-forth conditions of $\mathcal{Z}_{i+1}$.

  \vspace{0.5em}

  \begin{center}
    \tikz {
      \node (p) { $\homop \in \mathcal{Z}_{i}$ };
      \node[right=5em of p] (f) { $\homof \in \mathcal{Z}_{i-1}$ };
      \draw[->,thick] ($(p.east) + (0.5em,0em)$) -- node[above,midway] {(forth)} ($(f.west) - (0.5em,0em)$);
    }
  \end{center}

  \vspace{1em}

  \pause
  The \textbf{quantifier rank $\qr(\varphi)$} is the maximum degree of nesting of quantifiers in $\varphi$. \\[1em]
  $\FGF_{\ell}$ denotes the set of all $\FGF$ formulae with quantifier rank at most $\ell$.\\[0.5em]
  \begin{example}
    $\varphi(x) = \forall{y}\; \textcolor{tolbrightRed}{\textbf{(}}\relR(x,y) \to \exists{z} \relR(y,z)\textcolor{tolbrightRed}{\textbf{)}} \lor \exists{y}\textcolor{tolbrightRed}{\textbf{(}}\relR(x,y) \land \relP(y)\textcolor{tolbrightRed}{\textbf{)}}$ has quantifier rank 2, hence $\varphi \in \FGF_{2}$.
  \end{example}
\end{frame}

\begin{frame}{$\FGF_{\ell}$-invariance $\cong$ $\FGF$-expressible}
  \textbf{every $\FGF_{\textcolor{tolhighcontrastRed}{\ell}}$-invariant $\varphi$ is expressible in $\FGF$}:
  \begin{enumerate}
    \item each $\bisimto_{\FGF}^{\ell}$ equivalence class is characterized by some $\FGF_{\ell}$ formula
    \item the number of distinct $\FGF_{\ell}$ formulae is finite
    \item let $\mathcal{C}$ be the set of charateristic formulae of those $\bisimto_{\FGF}^{\ell}$ equivalence classes in which $\varphi$ is satisfied
    \item now, $\varphi \equiv \bigvee C$
  \end{enumerate}
  \vspace{0.5em}
  \begin{center}
    \begin{tikzpicture}
    \def\relem{0.15em};
    \begin{scope}
      \draw circle[radius=3em];
      \node [anchor=north, text width=6em, align=center] at (0,-3.5em) {all structures \\ (infinite)};

      \fill[radius=\relem, x=0.1em, y=0.1em, tolbrightGrey] {
        (5,2) circle []
        (10, 12) circle []
        (-12, 4) circle []
        (-14, 17) circle []
        (-23, -14) circle []
        (23, -14) circle []
        (13, -5) circle []
        (14, -13) circle []
        (-5, -13) circle []
        (2, -25) circle []
        (-4, -24) circle []
        (5, 18) circle []
        (15, 8) circle []
      };
      \fill[radius=0.5*\relem, x=0.1em, y=0.1em, tolbrightGrey] {
        (-7,-5) circle []
        (-10,-3) circle []
        (-15,-6) circle []
        (-3,27) circle []
        (-8,24) circle []
        (-10,22) circle []
      };
      \fill[radius=\relem, x=0.1em, y=0.1em, tolbrightRed] {
        (13, -5) circle []
        (14, -13) circle []
        (2, -25) circle []
        (-12, 4) circle []
        (-14, 17) circle []
        (-4, -24) circle []
      };
      \fill[radius=0.5*\relem, x=0.1em, y=0.1em, tolbrightRed] {
        (5, -3) circle []
        (7, -8) circle []
        (3, -13) circle []
      };
    \end{scope}

    \begin{scope}[xshift=10em]
      \node [anchor=north] at (0,-3.5em) {$\bisimto_{\FGF}^{\ell}$ classes};
      \draw[thick] circle[radius=3em];
      \draw[thick] {
        (135:3em) rectangle (0,0)
        (-135:3em) rectangle (0,0)
        (45:3em) rectangle (0,0)
        (-45:3em) rectangle (0,0)
      };

      \fill[radius=\relem, x=0.1em, y=0.1em, tolbrightGrey] {
        (5,2) circle []
        (10, 12) circle []
        (-12, 4) circle []
        (-14, 17) circle []
        (-23, -14) circle []
        (23, -14) circle []
        (13, -5) circle []
        (14, -13) circle []
        (-5, -13) circle []
        (2, -25) circle []
        (5, 18) circle []
        (15, 8) circle []
      };
      \fill[radius=0.5*\relem, x=0.1em, y=0.1em, tolbrightGrey] {
        (-7,-5) circle []
        (-10,-3) circle []
        (-15,-6) circle []
        (-3,27) circle []
        (-8,24) circle []
        (-10,22) circle []
      };
      \fill[radius=\relem, x=0.1em, y=0.1em, tolbrightRed] {
        (13, -5) circle []
        (14, -13) circle []
        (2, -25) circle []
        (-12, 4) circle []
        (-14, 17) circle []
        (-4, -24) circle []
      };
      \fill[radius=0.5*\relem, x=0.1em, y=0.1em, tolbrightRed] {
        (5, -3) circle []
        (7, -8) circle []
        (3, -13) circle []
      };
    \end{scope}

    \begin{scope}[xshift=20em]
      \draw[thick] circle[radius=3em];
      \draw[thick] {
        (135:3em) rectangle (0,0)
        (-135:3em) rectangle (0,0)
        (45:3em) rectangle (0,0)
        (-45:3em) rectangle (0,0)
      };

      \begin{scope}[font=\small]
        \node[tolhighcontrastRed] at (135:1.5em) {$\chi_{1}$};
        \node[tolhighcontrastRed] at (45:1.5em) {$\chi_{2}$};
        \node at (-45:1.5em) {$\chi_{3}$};
        \node at (-135:1.5em) {$\chi_{4}$};
        \node at (0:2.6em) {$\chi_{5}$};
        \node at (90:2.6em) {$\chi_{6}$};
        \node at (180:2.6em) {$\chi_{7}$};
        \node[tolhighcontrastRed] at (270:2.6em) {$\chi_{8}$};
      \end{scope}

      \node [anchor=north] (setc) at (0,-3.5em) {$\mathcal{C} = \{\chi_{1}, \chi_{3}, \chi_{8}\}$};
    \end{scope}

    \node[anchor=north, below=0.2em of setc.south] { $\implies \varphi = \bigvee \mathcal{C}$ };
 \end{tikzpicture}

  \end{center}
\end{frame}

\againframe<3->{outline}

\begin{frame}{Tree unraveling}
  \begin{center}
    \begin{tikzpicture}
  \matrix[
  ampersand replacement=§,
  matrix of nodes,
  column sep=1.5em,
  row sep=1em,
  nodes={
    draw,
    circle,
    fill=white,
  },
  ] (struct) {
            § |(2)| 2 §         \\[1.5em]
            § |(3)| 3 §         \\
    |(1)| 1 §         § |(4)| 4 \\
            § |(5)| 5 §         \\
  };

  \tikzset{hl/.style={tolhighcontrastRed}};
  \graph[use existing nodes, edges={very thick, -{Stealth[round]}}] {
    [edges={white}]
    2 -> 3,
    1 -> 3,
    3 -> 4,
    1 -> 5 -> 4,
  };

  \begin{scope}[on background layer, every path/.style={line width=1em}]
    \draw[tolhighcontrastBlue] plot[smooth] coordinates { ($(1)+(0.5em,0em)$) ($(5)+(0em,0.5em)$) ($(4)+(-0.5em,0em)$) };
    \draw[tolhighcontrastYellow] plot[smooth] coordinates { ($(2)+(0em,0em)$) ($(3)+(0.3em,0.7em)$) ($(4)+(0.0em,0.5em)$) };
    \draw[tolhighcontrastRed] plot[smooth] coordinates { ($(1)+(0.5em,0em)$) ($(3)+(-0.0em,-0.5em)$) ($(4)+(-0.5em,0em)$) };
  \end{scope}

  \begin{scope}[xshift=10em]
    \matrix[
    ampersand replacement=§,
    matrix of nodes,
    column sep=0.1em,
    row sep=1.5em,
    matrix anchor=west,
    nodes={
      draw,
      rectangle,
      rounded corners=5pt,
      fill=white,
    },
    ] (tree) {
                  § |(1)| 1 §             §[1em] |(2)| 2 §[1em] |(3)| 3 §[1em] |(4)| 4 §[1em] |(5)| 5 \\
      |(15)| 15   §         § |(13)| 13   § |(23)| 23      § |(34)| 34    §              § |(54)| 54    \\
      |(154)| 154 §         § |(134)| 134 § |(234)| 234    § \\
    };
    \graph[use existing nodes,edges={very thick, -{Stealth[round]}, white}] {
      1 -> 15 -> 154,
      1 -> 13 -> 134,
      2 -> 23 -> 234,
      3 -> 34,
      5 -> 54,
    };

  \end{scope}

  \begin{scope}[on background layer, every path/.style={line width=1em}]
    \draw[tolhighcontrastBlue] plot[smooth] coordinates { (1) (15) (154) };
    \draw[tolhighcontrastRed] plot[smooth] coordinates { (1) (13) (134) };
    \draw[tolhighcontrastYellow] plot[smooth] coordinates { (2) (23) (234) };
    \draw[tolhighcontrastYellow] plot[smooth] coordinates { ($(3)+(0.2em,0em)$) ($(34)+(0.3em,0em)$) };
    \draw[tolhighcontrastRed] plot[smooth] coordinates { ($(3)+(-0.2em,0em)$) ($(34)+(-0.3em,0em)$) };
    \draw[tolhighcontrastBlue] plot[smooth] coordinates { (5) (54) };
  \end{scope}

  \coordinate (mid) at ($(struct.east)!.5!(tree.west)$);
  \node[font=\Large] at (mid) {$\bisimto_{\FGF}$};
\end{tikzpicture}

  \end{center}
\end{frame}


\begin{frame}{Higher-arity relations: HAF unraveling}
  \def\exunravelhat#1{
\begin{tikzpicture}[baseline]
  \tikzset{reledges/.style={}}
  \tikzset{gaifman/.style={}}
  \tikzset{#1}
  \tikzset{el/.style={ellipse,draw,fill=white,minimum width=2.5em,minimum height=2em}};
  \tikzset{adj/.style={densely dashed,thick,draw,opacity=0.8, gaifman}};
  \tikzset{relP/.style={tolhighcontrastBlue, line width=0.4em, -{Stealth[round]}, shorten >= 0.4em}};
  \tikzset{relE/.style={tolhighcontrastYellow, line width=0.25em, -{Stealth[round]}, shorten >= 0.2em}};
  \tikzset{relS/.style={tolhighcontrastRed, line width=0.15em, -{Stealth[round]}, shorten >= 0.2em}};
  \tikzset{ellided/.style={fill=tolpalePalegrey, draw=none}};
  \def\radiusP{0.7em};
  \def\radiusE{0.4em};
  \def\radiusS{0.2em};

  \matrix[matrix of nodes, every node/.style={el}, column sep=2em, row sep=2em, ampersand replacement=§] {
    |(n1)| \textbf{1}                 § |(n2)| 1\textbf{2}  § |(n3)| 12\textbf{3}       § |(n4)| 123\textbf{4}         § |(n5) [ellided]| $\cdots{}$ § \\
                                      § |(nx3)| 1\textbf{3} § |(nx4)| {13\textbf{4}}    § |(nx5)| 134\textbf{3}        § |(nx6)| 1343\textbf{2}      § \\
    |(nx32) [overlay]| {13\textbf{2}} §                     § |(nx54)| {1343\textbf{4}} § |[ellided] (nx8)| $\cdots{}$ § |(nx7)| 13432\textbf{3}     § \\
  };
  % extra out of grid nodes
  \begin{scope}[every node/.style={el}]
    \node[above right=2em of n3,xshift=-1em] (n32) { 123\textbf{2} };

    \node[below right=1em of nx32,ellided] (ex32) {$\cdots{}$};
    \node[left=1.5em of nx54,ellided] (ex54) {$\cdots{}$};
    \node[right=1em of n32,ellided] (e32) {$\cdots{}$};
  \end{scope}

  \begin{scope}[on background layer]
    \graph[use existing nodes, edges=adj] {
      (n1) -> (n2) -> (n3) -> (n4) -> (n5),
      (n1) -> (nx3) -> (nx4) -> (nx5) -> (nx6) -> (nx7) -> (nx8),
      (n3) -> (n32) -> (e32),
      (nx3) -> (nx32) -> (ex32),
      (nx5) -> (nx54) -> (ex54),
    };
  \end{scope}

  \begin{scope}[on background layer, every path/.style={reledges}]

    \fill[relP] (n1.north) circle[radius=\radiusP] coordinate (p1);
    \fill[relP] (n2.north) circle[radius=\radiusP] coordinate (p2);
    \fill[relP] (n3.north west) circle[radius=\radiusP] coordinate (p3);
    \draw[relP] (p1) to[out=30,in=150] (p2) to[out=30,in=150] (p3);

    \fill[relP] (nx4.north) circle[radius=\radiusP] coordinate (p1);
    \fill[relP] (nx5.north) circle[radius=\radiusP] coordinate (p2);
    \fill[relP] (nx6.north) circle[radius=\radiusP] coordinate (p3);
    \draw[relP] (p1) to[out=30,in=150] (p2) to[out=30,in=150] (p3);

    \fill[relP] (nx54.south) circle[radius=\radiusP] coordinate (p1);
    \draw[relP,tips=never] (p1) to[out=south west, in=south east] (ex54);

    \fill[relP] (n4.north east) circle[radius=\radiusP] coordinate (p1);
    \draw[relP,tips=never] (p1) to[out=30, in=150] (n5);

    \fill[relE] (n3.east) circle[radius=\radiusE] coordinate (p1);
    \fill[relE] (n4.west) circle[radius=\radiusE] coordinate (p2);
    \draw[relE] (p1) to[out=30,in=150] (p2);

    \fill[relE] (n1.south) circle[radius=\radiusE] coordinate (p1);
    \fill[relE] (nx3.west) circle[radius=\radiusE] coordinate (p2);
    \draw[relE] (p1) to[out=-80,in=170] (p2);

    \fill[relE] (nx3.east) +(0em,-0.4em) circle[radius=\radiusE] coordinate (p1);
    \fill[relE] (nx4.west) +(0em,-0.4em) circle[radius=\radiusE] coordinate (p2);
    \draw[relE] (p1) to[out=-30,in=-150] (p2);

    \fill[relE] (nx5.south) +(-0.5em,0em) circle[radius=\radiusE] coordinate (p1);
    \fill[relE] (nx54.east) +(-0.2em,0.5em) circle[radius=\radiusE] coordinate (p2);
    \draw[relE] (p1) to (p2);

    \fill[relE] (nx7.north west) circle[radius=\radiusE] coordinate (p1);
    \draw[relE] (p1) to[out=north west,in=north east] (nx8.north east);

    \fill[relS] (n1.east) +(-0.1em,-0.4em) circle[radius=\radiusS] coordinate (p1);
    \fill[relS] (n2.west) +(0.1em,-0.4em) circle[radius=\radiusS] coordinate (p2);
    \draw[relS] (p1) to[out=-20,in=-160] (p2);

    \fill[relS] (n2.east) +(-0.1em,-0.4em) circle[radius=\radiusS] coordinate (p1);
    \fill[relS] (n3.west) +(0.1em,-0.4em) circle[radius=\radiusS] coordinate (p2);
    \draw[relS] (p1) to[out=-20,in=-160] (p2);

    \fill[relS] (nx6.south) +(0.5em,0em) circle[radius=\radiusS] coordinate (p1);
    \fill[relS] (nx7.north) +(0.5em,0em) circle[radius=\radiusS] coordinate (p2);
    \draw[relS] (p1) to[out=-60,in=60] (p2);

  \end{scope}

\end{tikzpicture}
}

  \begin{tikzpicture}[baseline]
  \tikzset{el/.style={circle,draw,fill=white,minimum width=2em,minimum height=2em}};
  \tikzset{relP/.style={tolhighcontrastBlue, line width=0.4em, -{Stealth[round]}, shorten >= 0.4em}};
  \tikzset{relE/.style={tolhighcontrastYellow, line width=0.25em, -{Stealth[round]}, shorten >= 0.2em}};
  \tikzset{relS/.style={tolhighcontrastRed, line width=0.15em, -{Stealth[round]}, shorten >= 0.2em}};
  \def\radiusP{0.7em};
  \def\radiusE{0.4em};
  \def\radiusS{0.2em};

  \matrix[matrix of nodes, every node/.style={el}, column sep=4em] {
    |(n1)| 1 & |(n2)| 2 & |(n3)| 3 & |(n4)| 4  \\
  };

  \begin{scope}[on background layer]
    % 1 2 3
    \fill[relP] (n1.north) circle[radius=\radiusP] coordinate (p1);
    \fill[relP] (n2.north) circle[radius=\radiusP] coordinate (p2);
    \fill[relP] (n3.north west) circle[radius=\radiusP] coordinate (p3);
    \draw[relP] (p1) to[out=30,in=150] (p2) to[out=30,in=150] (p3);

    % 4 3 2
    \fill[relP] (n4.south) circle[radius=\radiusP] coordinate (p1);
    \fill[relP] (n3.south) circle[radius=\radiusP] coordinate (p2);
    \fill[relP] (n2.south) circle[radius=\radiusP] coordinate (p3);
    \draw[relP] (p1) to[in=-30,out=-150] (p2) to[in=-30,out=-150] (p3);

    % 3 4
    \fill[relE] ($(n3.east) + (0,-0.2em)$) circle[radius=\radiusE] coordinate (p1);
    \fill[relE] (n4.west) circle[radius=\radiusE] coordinate (p2);
    \draw[relE] (p1) to (p2);

    % 1 3
    \fill[relE] (n1.west) circle[radius=\radiusE] coordinate (p1);
    \fill[relE] (n3.north east) circle[radius=\radiusE] coordinate (p2);
    \draw[relE] (p1) to[out=140,in=north] (p2);

    % 1 2
    \fill[relS] (n1.east) circle[radius=\radiusS] coordinate (p1);
    \fill[relS] (n2.west) circle[radius=\radiusS] coordinate (p2);
    \draw[relS] (p1) to[out=east,in=west] (p2);

    \fill[relS] (n2.east) circle[radius=\radiusS] coordinate (p1);
    \fill[relS] ($(n3.west) + (0,-0.2em)$) circle[radius=\radiusS] coordinate (p2);
    \draw[relS] (p1) to[out=east,in=west] (p2);
  \end{scope}

  \begin{scope}[on background layer]
    %\tikzdbg
  \end{scope}
\end{tikzpicture}


  \begin{columns}
    \begin{column}{0.2\textwidth}
      \begin{overprint}
        \onslide<1>

        \onslide<2-3>
        \exunravelstruct{reledges/.style={hidden}, gaifman/.style={}}

        \onslide<4>
        \exunravelstruct{gaifman/.style={}}
      \end{overprint}
    \end{column}
    \begin{column}{0.75\textwidth}
      \begin{overprint}
        \onslide<1-2>
        \exunravelstruct{}

        \onslide<3>
        \exunravelhat{reledges/.style={hidden}}

        \onslide<4>
        \exunravelhat{}
      \end{overprint}
    \end{column}
  \end{columns}
\end{frame}

\begin{frame}{HAF-unraveling does not upgrade to $\GF$}
  \begin{tikzpicture}
  \tikzset{el/.style={rectangle,rounded corners=10pt,draw,fill=white,minimum width=2em,minimum height=2em,font=\small}};
  \tikzset{relP/.style={tolhighcontrastBlue, line width=0.4em, -{Stealth[round]}, shorten >= 0.4em}};
  \tikzset{relE/.style={tolhighcontrastYellow, line width=0.25em, -{Stealth[round]}, shorten >= 0.2em}};
  \tikzset{relS/.style={tolhighcontrastRed, line width=0.15em, -{Stealth[round]}, shorten >= 0.2em}};
  \def\radiusP{0.7em};
  \def\radiusE{0.4em};
  \def\radiusS{0.2em};

  \matrix[matrix of nodes, every node/.style={el},name=n,column sep=2em, row sep=2em]  {
    1 & 12 & 123 &[8em] a & ab & abc & b & bc \\
    3 & 2 & 23 & c & abd &  & bd & d \\
  };

  \begin{scope}[on background layer]
    \fill[relP] (n-1-1.north) circle [radius=\radiusP] coordinate (p1);
    \fill[relP] (n-1-2.north) circle [radius=\radiusP] coordinate (p2);
    \fill[relP] (n-1-3.north) circle [radius=\radiusP] coordinate (p3);
    \draw[relP] (p1) to[out=30,in=150] (p2) to[out=30,in=150] (p3);

    \fill[relP] (n-1-4.north) circle [radius=\radiusP] coordinate (p1);
    \fill[relP] (n-1-5.north) circle [radius=\radiusP] coordinate (p2);
    \fill[relP] (n-1-6.north) circle [radius=\radiusP] coordinate (p3);
    \draw[relP] (p1) to[out=30,in=150] (p2) to[out=30,in=150] (p3);

    \fill[relE] (n-1-2.east) circle [radius=\radiusE] coordinate (p1);
    \fill[relE] (n-1-3.west) circle [radius=\radiusE] coordinate (p2);
    \draw[relE] (p1) to (p2);

    \fill[relE] (n-2-2.east) circle [radius=\radiusE] coordinate (p1);
    \fill[relE] (n-2-3.west) circle [radius=\radiusE] coordinate (p2);
    \draw[relE] (p1) to (p2);

    \fill[relE] (n-1-5.east) circle [radius=\radiusE] coordinate (p1);
    \fill[relE] (n-1-6.west) circle [radius=\radiusE] coordinate (p2);
    \draw[relE] (p1) to (p2);

    \fill[relE] (n-1-5.south) circle [radius=\radiusE] coordinate (p1);
    \fill[relE] (n-2-5.north) circle [radius=\radiusE] coordinate (p2);
    \draw[relE] (p1) to (p2);

    \fill[relE] (n-1-7.east) circle [radius=\radiusE] coordinate (p1);
    \fill[relE] (n-1-8.west) circle [radius=\radiusE] coordinate (p2);
    \draw[relE] (p1) to (p2);

    \fill[relE] (n-1-7.south) circle [radius=\radiusE] coordinate (p1);
    \fill[relE] (n-2-7.north) circle [radius=\radiusE] coordinate (p2);
    \draw[relE] (p1) to (p2);
  \end{scope}

  \node[above=3ex of n-1-2] {\huge{$\hatunravel{A}$}};
  \node[above=3ex of n-1-6] {\huge{$\hatunravel{B}$}};
  \node at ($(n-1-3.east)!.5!(n-1-4.west)$) {\huge{$\bisimto_{\FGF}$}};
\end{tikzpicture}

\end{frame}

\begin{frame}{Biseqs and Bipoints}
  \begin{tikzpicture}
  \matrix[
  matrix of nodes,
  name=n,
  ampersand replacement=§,
  column sep=2em,
  row sep=2em,
  nodes={draw, circle, fill=white, minimum width=4ex, minimum height=4ex, outer sep=1ex, anchor=center},
  row 1/.style={every node/.style={draw=none}},
  matrix anchor=n-1-4.base east,
  ] {
    1       § 2       § 3       § 4       \\[-1.2em]
    |(a)| a § |(b)| b § |(c)| c § |(d)| d \\
            §         § |(e)| e §         \\
            §         § |(f)| f §         \\
  };

  \begin{scope}[on background layer]
    \node[fit={(a) (b) (c) (d)}, fill, tolhighcontrastBlue, rounded corners=5pt] {};
    \coordinate (bnw) at (b.north -| b.west);
    \coordinate (bsw) at (b.south -| b.west);
    \coordinate (cne) at (c.north -| c.east);
    \coordinate (cse) at (c.south -| c.east);
    \coordinate (csw) at (c.south -| c.west);
    \coordinate (fsw) at (f.south -| f.west);
    \coordinate (fse) at (f.south -| f.east);
    \fill[tolhighcontrastYellow, rounded corners=5pt] {
      (bnw) -- (cne) -- (fse) -- (fsw) -- (csw) -- (bsw) -- cycle
    };
  \end{scope}

  \node[left=0.5em of a, color=tolhighcontrastBlue, font=\huge] { $\elemtuples$ };
  \node[below=0.5em of f, color=tolbrightYellowDarkest, font=\huge] { $\elemtupler$ };

  \node[right=5em of n-1-4.base east, font=\large, anchor=base west] (title) { bisimulation sequence (biseq) };

  \def\tups{\textcolor{tolhighcontrastBlue}{\elemtuples}};
  \def\tupr{\textcolor{tolhighcontrastYellow}{\elemtupler}};
  \node[below=3em of title.base west, anchor=base west, font=\Huge, uncover=<-3>] (biseq) {
    $\tups(2,3)\tupr$
  };
  \node[below=3em of title.base west, anchor=base west, font=\Huge, uncover=<4->] (biseq) {
    $\tups(2,2)\tupr$
  };
  \begin{scope}[uncover=<2>]
  \node[below=3em of biseq.base west, anchor=base west, font=\Huge] (biseq2) {
    $\tups(2,2)\tupr$
  };
  \node[below=3em of biseq2.base west, anchor=base west, font=\Huge] (biseq3) {
    $\tups$
  };
  \end{scope}

  \node[uncover=<3->, below=4em of biseq.base west, font=\large, anchor=base west] (bipointdesc) { bisimulation points (bipoints) };
  \begin{scope}[uncover=<3>]
    \node[below=3em of bipointdesc.base west, anchor=base west, font=\Huge] (bipoint1) {
      $\langle\tups(2,3)\tupr / 3\rangle$
    };
    \node[below=3.5em of bipoint1.base west, anchor=base west, font=\Huge] (bipoint2) {
      $\langle\tups(2,3)\tupr / 4\rangle$
    };
  \end{scope}
  \begin{scope}[uncover=<4>]
    \node[below=3em of bipointdesc.base west, anchor=base west, font=\Huge] (bipoint1) {
      $\langle\tups(2,2)\tupr / 3\rangle$
    };
    \node[below=3.5em of bipoint1.base west, anchor=base west, font=\Huge] (bipoint2) {
      $\langle\tups(2,2)\tupr / 4\rangle$
    };
    \node[below=3.5em of bipoint2.base west, anchor=base west, font=\Huge] (bipoint3) {
      $\langle\tups(2,2)\tupr / 2\rangle$
    };
  \end{scope}

  \node[below=3em of a] (tupsdef) { $\tups = (a,b,c,d)$ };
  \node[below=1em of tupsdef] { $\tupr = (b,c,e,f)$ };
\end{tikzpicture}

\end{frame}

\begin{frame}{The forward unraveling $\unravel{A}$}
  \begin{tikzpicture}
  \tikzset{el/.style={circle,draw,fill=white,minimum width=2em,minimum height=2em}};
  \def\radiusP{0.7em};
  \def\radiusE{0.4em};
  \def\radiusS{0.2em};
  \tikzset{relP/.style={tolhighcontrastBlue, line width=0.4em, -{Stealth[round]}, shorten >= 0.4em, radius=\radiusP}};
  \tikzset{relE/.style={tolhighcontrastYellow, line width=0.25em, -{Stealth[round]}, shorten >= 0.2em, radius=\radiusE}};
  \tikzset{relS/.style={tolhighcontrastRed, line width=0.15em, -{Stealth[round]}, shorten >= 0.2em, radius=\rediusS}};

  \def\tupp{{\color{tolhighcontrastBlue}\elemtuplep}}%
  \def\tupe{{\color{tolbrightYellowDarkest}\elemtuplee}}%

  \matrix[matrix of nodes, every node/.style={el},name=n,column sep=2em, row sep=2em, ampersand replacement=§, matrix anchor=1.base]  {
    |(1)| a \\ |(2)| b \\ |(3)| c \\
  };

  \begin{scope}[on background layer]
    \fill[relP] (1.west) circle [radius=\radiusP] coordinate (p1);
    \fill[relP] (2.west) circle [radius=\radiusP] coordinate (p2);
    \fill[relP] (3.west) circle [radius=\radiusP] coordinate (p3);
    \draw[relP] (p1) to[out=-150,in=150] (p2) to[out=-150,in=150] (p3);

    \fill[relE] (2.south) circle [radius=\radiusE] coordinate (p1);
    \fill[relE] (3.north) circle [radius=\radiusE] coordinate (p2);
    \draw[relE] (p1) to (p2);
  \end{scope}

  \node[above=1.5ex of 1] (laba) {\huge{$\str{A}$}};

  \begin{scope}[uncover=<2->]
    \tikzset{element/.style = {
        inner sep=0.25em, fill=white, rounded corners=0.7em, draw=black
      }};
    \tikzset{edge from parent/.style = {draw=none}};
    \node[font=\Large, right=14em of laba.center] (heading) { domain $\unraveldom{A}$ of the unraveling };
    \node[element, anchor=base, xshift=-2em, below=1em of heading] (p1) {$\langle\tupp/1\rangle$}
      [level distance=4em, grow=down, sibling distance=8em]
      child[missing]
      child[grow=down, sibling distance=5em] { node[element] (p2) {$\langle\tupp/2\rangle$}
        child[uncover=<3->] { node[element] (p22e2) {$\langle\tupp(2,2)\tupe/2\rangle$} }
        child [grow=down] { node[element] (p3) {$\langle\tupp/3\rangle$}
        }
        child [missing]
      }
      child [uncover=<3->] { node[element] (p11p2) {$\langle\tupp(1,1)\tupp/2\rangle$}
        child { node[element] (p11p3) {$\langle\tupp(1,1)\tupp/3\rangle$} }
      }
    ;

    \tikzset{link/.style={-{Stealth[round]}}}
    \begin{scope}[on background layer, every path/.style={uncover=<2->, link}]
      \draw (p1) -- (p2);
      \draw (p2) -- (p3);
    \end{scope}
    \begin{scope}[on background layer, every path/.style={uncover=<3->, link}]
      \draw (p2) -- (p22e2);
      \draw (p1) -- (p11p2);
      \draw (p11p2) -- (p11p3);
    \end{scope}

    \begin{scope}[uncover=<3>]
      \node[anchor=center] (d2) at ($(p3.east)!0.45!(p11p3.west)$) { \ldots };
      \node[anchor=center] (dside) at ($(p11p2.south east)+(1.5em,-1em)$) { \ldots };

      \draw[-{Straight Barb[]}, loosely dotted, thick] (p2) -- (d2);
      \draw[-{Straight Barb[]}, loosely dotted, thick] (p11p2) -- (dside);
    \end{scope}
  \end{scope}

  \begin{scope}[font=\Large]
  \def\rP{\textcolor{tolhighcontrastBlue}{\mathbf{P}}};
  \def\rE{\textcolor{tolhighcontrastYellow}{\mathbf{E}}};
    \matrix [right=3em of laba.center, matrix anchor=head.west, anchor=base west, row sep=0.25em] {
      \node (head) { tuples };     \\
      \node { $\tupp = (a,b,c)$ }; \\
      \node { $\tupe = (b,c)$ };   \\[0.25em]
      \node { relations }; \\
      \node { $\rP^{\str{A}} = \{ \tupp \}$ }; \\
      \node { $\rE^{\str{A}} = \{ \tupe \}$ }; \\
    };
  \end{scope}


  \coordinate (bot) at ($(3.west)!.3!(p22e2.east)$);
  \begin{scope}[uncover=<-3>, font=\Large]
    % \node[below=3.5em of bot] (biseqhead) { biseqs };
    \matrix[anchor=base west, matrix anchor=headbiseqs.north west, ampersand replacement=§, column sep=1.5em, row sep=0.1em, yshift=-2em] at (bot -| head.west) {
      \node (headbiseqs) { biseqs }; §                              §                                        \\
      \node { $\tupp$ };             § \node { $\tupp(1,1)\tupp$};  § \node { $\tupp(1,1)\tupp(1,2)\tupp$ }; \\
      \node { $\tupp(1,2)\tupp$ };   § \node { $\tupp(2,3)\tupe$ }; § \node { $\tupp(2,2)\tupe$ };           \\
    };
  \end{scope}

  \begin{scope}[uncover=<4->]
    \node[below=3.5em of bot, font=\huge] (defif) { $\elemtuptupler \in \relR^{\unravel{A}}$ if: };
    \node[text width=25em, right=1em of defif, font=\Large] {
      \begin{enumerate}
        \item<alert@6> $\pi[\elemtuptupler] \in \relR^{\str{A}}$,
        \item<alert@7> $\elemtuptupler$ is a path in the domain $\unraveldom{A}$, and
        \item<alert@8> $|\elemtuptupler| \le b$, for $b$ s.t.\ $\mathtt{last}(\elemtuptupler) = \langle\ldots/b\rangle$.
      \end{enumerate}
    };
    % \node at (botmid) {test};
  \end{scope}

  \tikzset{relNotP/.style={tolhighcontrastRed, line width=0.4em, -{Stealth[round]}, shorten >= 0.4em, radius=\radiusP}};
  \begin{scope}[on background layer, every path/.style={uncover=<6>}]
    \fill[relNotP] (p1.south east) circle[] coordinate (c1);
    \fill[relNotP] (p2.east) circle[] coordinate (c2);
    \fill[relNotP] (p11p2.west) circle[] coordinate (c3);
    \draw[relNotP, dashed] (c1) to[out=-60, in=60] (c2) to[out=-30,in=-150] (c3);
  \end{scope}
  \begin{scope}[on background layer, every path/.style={uncover=<7>}]
    \fill[relNotP] (p1.south east) circle[] coordinate (c1);
    \fill[relNotP] (p2.east) circle[] coordinate (c2);
    \fill[relNotP] (p11p3.160) circle[] coordinate (c3);
    \draw[relNotP, dashed] (c1) to[out=-60, in=60] (c2) to[out=-45, in=145] (c3);
  \end{scope}
  \begin{scope}[on background layer, every path/.style={uncover=<8>}]
    \fill[relNotP] (p1.south west) circle[] coordinate (c1);
    \fill[relNotP] (p2.west) circle[] coordinate (c2);
    \fill[relNotP] (p22e2.north) circle[] coordinate (c3);
    \draw[relNotP, dashed] (c1) to[out=-120, in=120] (c2) to[out=-120, in=80] (c3);
  \end{scope}
  \begin{scope}[on background layer, every path/.style={uncover=<9->}]
    \fill[relP] (p1.south east) circle[] coordinate (c1);
    \fill[relP] (p2.east) circle[] coordinate (c2);
    \fill[relP] (p3.north east) circle[] coordinate (c3);
    \draw[relP] (c1) to[out=-60, in=60] (c2) to[out=-60, in=60] (c3);
  \end{scope}
  \begin{scope}[on background layer, every path/.style={uncover=<6->}]
    \fill[relP] (p1.north east) circle[] coordinate (c1);
    \fill[relP] (p11p2.east) circle[] coordinate (c2);
    \fill[relP] (p11p3.east) circle[] coordinate (c3);
    \draw[relP] (c1) to[out=-5, in=110] (c2) to[out=-80, in=80] (c3);
  \end{scope}
  \begin{scope}[on background layer, every path/.style={uncover=<9->}]
    \fill[relE] (p2.south west) circle[] coordinate (c1);
    \fill[relE] (p22e2.60) circle[] coordinate (c2);
    \draw[relE] (c1) to (c2);

    \fill[relE] (p2.south) circle[] coordinate (c1);
    \fill[relE] (p3.north) circle[] coordinate (c2);
    \draw[relE] (c1) to (c2);

    \fill[relE] (p11p2.south) circle[] coordinate (c1);
    \fill[relE] (p11p3.north) circle[] coordinate (c2);
    \draw[relE] (c1) to (c2);
  \end{scope}
\end{tikzpicture}

\end{frame}

\begin{frame}{Cutting for a finite unraveling}
  \begin{center}
    \begin{tikzpicture}[x=1em,y=-1em]
  \tikzset{el/.style={fill=black,circle,inner sep=2.5}};
  \tikzset{missingchild/.style={el,fill=none,draw=black,circle,double}};
  \tikzset{excluded/.style={el,fill=none,draw=black,circle}};
  \tikzset{virtual/.style={draw=none,inner sep=0.2,red}};
  \tikzset{faded edge/.style={dash pattern=on 1pt off 1pt on 1pt off 1pt on 2pt off 1pt on 2pt off 1pt on 100pt}};
  \tikzset{faded edge reverse/.style={tips=false,dash pattern=on 5pt off 1pt on 2pt off 1pt on 2pt off 1pt on 1pt off 1pt on 1pt off 1pt on 1pt off 1pt on 1pt off 1pt on 1pt off 1pt}};
  \tikzset{index/.style={font=\bfseries\boldmath,color=toldarkDarkgreen}};

  % level <2l
  \node[el] (a1) at (1.5,0) {};
  \node[el] (a2) at (4.5,0) {};

  % level 2l
  \node[el] (b1) at (0,5) {};
  \node[el] (b2) at (2,3) {};
  \node[el] (b3) at (6,4) {};
  \node[el] (b4) at (7,7) {};
  \node[el] (b5) at (2,7) {};

  % level 2l+1
  \node[missingchild,label={[index]below:1}] (c1) at (-3,11) {};
  \node[missingchild,label={[index]right:2}] (c2) at (0.5,10) {};
  \node[excluded]     (c3) at (0.5,12) {};

  \node[missingchild,label={[index]left:$M{-}1$}] (c4) at (7,10) {};
  \node[missingchild,label={[index]-90:$M$}] (c5) at (10,11) {};
  \node[excluded]     (c6) at (6.5,12) {};

  % virtual nodes
  \node[virtual] (v1) at (2.5,-2) {};
  \node[virtual] (v2) at (3.5,-2) {};
  \node[virtual] (v3) at (3,9) {};
  \node[virtual] (v5) at (4.5,5.5) {};

  % tree container
  \begin{scope}
    \clip[overlay] (-20,-3) rectangle (20,20);
    \draw[densely dotted] (3,-5) coordinate (root) -- (-6,13) coordinate (left);
    \draw[densely dotted] (3,-5) -- (12,13) coordinate (right);
  \end{scope}

  % edges
  \graph[use existing nodes] {
    (v1) ->[faded edge] (a1) -> {
      (b1) -> { (c1), (c2) -> (c3), (b5) ->[faded edge reverse] (v3) },
      (b2)
    },
    (v2) ->[faded edge] (a2) -> {
      (b3) -> (b4) -> {
        (c4) -> (c6),
        (c5)
      },
      (b3) ->[faded edge reverse] (v5)
    }
  };

  % cut lines
  \path
    ($(root)!.35!(left) - (1,0)$) coordinate (l1left)
    ($(root)!.35!(right) + (1,0)$) coordinate (l1right)
    ($(root)!.72!(left) - (1,0)$) coordinate (l2left)
    ($(root)!.78!(right) + (1,0)$) coordinate (l2right);

  \draw[opacity=0.8] (l1left) -- ($(l1left)!0.35!(l1right)$) ($(l1left)!0.65!(l1right)$) -- (l1right);
  \draw[opacity=0.8,dashed] ($(l1left)!0.35!(l1right)$) -- ($(l1left)!0.65!(l1right)$);
  \draw[thick,tolmutedWine] (l2left) -- ($(l2left)!0.35!(l2right)$) ($(l2left)!0.65!(l2right)$) -- (l2right);
  \draw[thick,tolmutedWine,dashed] ($(l2left)!0.35!(l2right)$) -- ($(l2left)!0.65!(l2right)$);

  \node[left=0.5 of l2left,font=\large,anchor=east,yshift=1,color=tolmutedWine] {cut here\hspace{0.5em}{\raisebox{-0.8ex}{\Huge\Rightscissors}}};

  % text
  \node[anchor=base east] at ($(root)!.20!(left) - (1,0)$) {$\mathsf{seq}$ level $< 2 * \ell$ };
  \node[anchor=base east] at ($(root)!.50!(left) - (1,0)$) {$\mathsf{seq}$ level $= 2 * \ell$ };
  \node[anchor=base east] at ($(root)!.90!(left) - (1,0)$) {$\mathsf{seq}$ level $= 2 * \ell + 1$ };

  % legend
  \node[el] (leg1) at (-12,14) {};
  \node[right=0.5 of leg1.south, anchor=south west, inner sep=1] { element };

  \node[excluded] (leg2) at (-6,14) {};
  \node[right=0.5 of leg2.south, anchor=south west, inner sep=1] { excluded element };

  \node[missingchild] (leg3) at (4,14) {};
  \node[right=0 of leg3, anchor=west] (leg3t) { missing child (labelled with \textcolor{toldarkDarkgreen}{index}) };

\end{tikzpicture}

  \end{center}
\end{frame}

\begin{frame}{The finite forward unraveling $\unravel{A}_{\ell}$}
  \begin{tikzpicture}
  \tikzset{el/.style={fill=black,circle,inner sep=2.5}};
  \tikzset{smallel/.style={el,inner sep=1.5}};
  \tikzset{missingchild/.style={el,fill=white,draw=black,circle,double}};

  % detailed tree
  \draw[pattern color=tollightPalegrey,pattern=bricks] (0,0) coordinate (t1)  -- +(4em,-6em) coordinate (r1) -- +(-4em,-6em) coordinate (l1) -- (t1);
  \node[el] at (t1) {};

  % nodes on path under focus
  \node[el] (n) at (0.2em, -5em) {};
  \draw (0.2em, -8em) node[missingchild] (missing) {};
  \draw[uncover=<2->] (missing) -- +(.8em,-2em) +(-.8em,-2em) -- (missing);
  \draw[densely dashed,->] (n) -- (missing);
  \node[inner sep=0,outer sep=0,xshift=0.3em,below right=0 of missing, toldarkDarkgreen,font=\boldmath] {$i$};
  % hinted at nodes
  \begin{scope}[transparency group,onslide=<2->{opacity=0.4}]
    \path[onslide=<2->draw] (-4.5em, -8em) node[missingchild] (missing0) {} -- +(.8em,-2em) +(-.8em,-2em) -- (missing0);
    \path[onslide=<2->draw] (-1.8em, -8em) node[missingchild] (missing1) {} -- +(.8em,-2em) +(-.8em,-2em) -- (missing1);
    \path[onslide=<2->draw] (2em, -8em) node[missingchild] (missing2) {} -- +(.8em,-2em) +(-.8em,-2em) -- (missing2);
    \path[onslide=<2->draw] (4.5em, -8em) node[missingchild] (missing3) {} -- +(.8em,-2em) +(-.8em,-2em) -- (missing3);

    \begin{scope}[color=toldarkDarkgreen, font=\small\boldmath]
      \node[inner sep=0,outer sep=0,xshift=-0.2em,below left=0 of missing0] {$1$};
      \node[inner sep=0,outer sep=0,xshift=-0.1em,below left=0 of missing1] {$i{-}1$};
      \node[inner sep=0,outer sep=0,xshift=0.2em,below right=0 of missing2] {$i{+}1$};
      \node[inner sep=0,outer sep=0,xshift=0.2em,below right=0 of missing3] {$M$};
    \end{scope}
  \end{scope}

  \draw[densely dashed,->,path fading=north] (-2em,-5em) -- (missing0);
  \draw[densely dashed,->,path fading=north] (-1em, -4.5em) -- (missing1);
  \draw[densely dashed,->,opacity=0.4] (n) -- (missing2);
  \draw[densely dashed,->,path fading=north] (2em,-5em) -- (missing3);

  % fillings
  \begin{scope}[uncover=<2->]
    \path[pattern color=tollightPalegrey,pattern=checkerboard] (missing) -- +(.8em,-2em) -- +(-.8em,-2em) -- (missing);
    \path[pattern color=tollightPalegrey,pattern=checkerboard] (missing0) -- +(.8em,-2em) -- +(-.8em,-2em) -- (missing0);
    \path[pattern color=tollightPalegrey,pattern=bricks] (missing1) -- +(.8em,-2em) -- +(-.8em,-2em) -- (missing1);
    \path[pattern color=tollightPalegrey,pattern=checkerboard] (missing2) -- +(.8em,-2em) -- +(-.8em,-2em) -- (missing2);
    \path[pattern color=tollightPalegrey,pattern=grid] (missing3) -- +(.8em,-2em) -- +(-.8em,-2em) -- (missing3);
  \end{scope}

  \node[gray,font=\large] at ($(missing0)!.55!(missing1)$) {$\cdots$};
  \node[gray,font=\large] at ($(missing2)!.55!(missing3)$) {$\cdots$};

  % paths
  \draw[decorate,decoration=zigzag] (t1) -- (n);

  % labels
  \node[above=2ex of t1,xshift=0.5em,font=\Large] { $\mathcal{T}_{r,\ell}$ };
  \begin{scope}[font=\small]
    \node[anchor=base east,text width=5em] at ($(t1)!-.2!(l1) - (0.5em,0)$) { $\mathsf{seq}$ level $\le \ell$ };
    \node[anchor=base east,text width=5em] at ($(t1)!.45!(l1) - (0.5em,0)$) { $\mathsf{seq}$ level $\le 2 * \ell$ };
    \node[anchor=base east,text width=5em] at ($(t1)!1.2!(l1) - (0.5em,0)$) { $\mathsf{seq}$ level $= 2 * \ell + 1$ };
  \end{scope}

  % copies of trees
  \begin{scope}[uncover=<3->]
    \draw[pattern color=tollightPalegrey,pattern=bricks] (5em,0) node[smallel] (cop1) {}  -- +(1em,-1.5em) -- +(-1em,-1.5em) -- cycle;
    \draw[pattern color=tollightPalegrey,pattern=bricks] (6em,-3em) node[smallel] (cop2) {}  -- +(1em,-1.5em) -- +(-1em,-1.5em) -- cycle;

    \draw[pattern color=tollightPalegrey,pattern=checkerboard] (15em,-2em) node[el] (t2) {}  -- +(2em,-3em) coordinate (r2) -- +(-2em,-3em) coordinate (l2) -- cycle;
    \coordinate (b2) at ($(l2)!0.5!(r2)$);
    \node[below=0.6ex of b2] { $i$-th copy };

    \draw[pattern color=tollightPalegrey,pattern=checkerboard] (12em,0em) node[smallel] (cop3) {} -- +(1em,-1.5em) -- +(-1em,-1.5em) -- cycle;
    \draw[pattern color=tollightPalegrey,pattern=checkerboard] (18em,0) node[smallel] (cop4) {} -- +(1em,-1.5em) -- +(-1em,-1.5em) -- cycle;
    \draw[pattern color=tollightPalegrey,pattern=checkerboard] (20em,-3em) node[smallel] (cop5) {} -- +(1em,-1.5em) -- +(-1em,-1.5em) -- cycle;

    \draw[pattern color=tollightPalegrey,pattern=grid] (23em,-1em) node[smallel] (cop6) {} -- +(1em,-1.5em) -- +(-1em,-1.5em) -- cycle;
    \draw[pattern color=tollightPalegrey,pattern=grid] (24em,-4em) node[smallel] (cop7) {} -- +(1em,-1.5em) -- +(-1em,-1.5em) -- cycle;

    % ellipses
    \node[gray,font=\huge,rotate=30] at (8.7em, -2em) {$\cdots$};

    \node at (17em,-8.5em) { $M$ copies of each tree };
  \end{scope}

  % replacement link
  \begin{scope}[uncover=<4->]
    \draw[->,shorten >=0.30em] plot [smooth] coordinates {
      (n)
      (3.5em, -7em)
      ($(t2) + (-2em, 0em)$)
      (t2)
    };
  \end{scope}

  % hint at some connections between copies as well
  \begin{scope}[uncover=<5->]
    \begin{scope}[onslide=<5->{opacity=.4}]
      \draw ($(cop4) + (0.2em,-1em)$) edge[out=-90,in=180,->] (cop6);
      \draw ($(cop6) + (0.2em,-1.3em)$) edge[out=-90,in=90,->] (cop5);
      \draw ($(cop7) + (0.2em,-1.3em)$) edge[out=-90,in=45,->,bend right=150,looseness=5] (cop7);
      \draw ($(cop5) + (0.2em,-1.3em)$) edge[out=-90,in=100,->,looseness=1.5] (cop7);
      \draw[->,shorten >=0.30em] plot [smooth] coordinates {
        ($(cop1) + (0em,-1em)$)
        ($(cop1) + (1em,-2em)$)
        ($(cop3) + (1em, 1em)$)
        (t2)
      };
    \end{scope}
    \path ($(cop3) + (-0.2em,-1em)$) edge[out=-90,in=45,->,looseness=1.5,path fading=south] (10em,-3em);
    \path ($(cop2) + (0.2em,-1.3em)$) edge[out=-90,in=-120,->,looseness=1.5,path fading=north] (8em,-4em);
    \path ($(t2) + (-0.5em,-2.5em)$) edge[out=-90,in=-45,->,looseness=1.5,path fading=west,looseness=0.8] (11em,-4.5em);
  \end{scope}
\end{tikzpicture}

\end{frame}

\begin{frame}{Finite forward unraveling upgrades from $\FGF$ to $\GF$}
  \textbf{Claim:} If $\str{A} \bisimto_{\FGF}^{\homof_{1}(g)} \str{B}$, then $\unravel{A}_{\homof_{2}(g)} \bisimto_{\GF}^{g} \unravel{B}_{\homof_{2}(g)}$ (for some functions $\homof_{1}$ and $\homof_{2}$)\\[1.5em]

  \textbf{Properties of the unraveling:}\\[0.5em]
  \begin{enumerate}
    \item If $(\elemtuples, \elemtuplet)$ is a forward partial map, then $\homop: s_{i} \mapsto t_{i}$ is a partial isomorphism
          \vspace{0.5em}

          \begin{tikzpicture}
  \tikzset{faded edge/.style={dash pattern=on 1pt off 1pt on 1pt off 1pt on 2pt off 1pt on 2pt off 1pt on 100pt}};
  \tikzset{faded edge reverse/.style={tips=false,dash pattern=on 5pt off 1pt on 5pt off 1pt on 2pt off 1pt on 2pt off 1pt on 1pt off 1pt on 1pt off 1pt on 1pt off 1pt on 1pt off 1pt on 1pt off 100pt}};

  \tikzset{relP/.style={tolhighcontrastBlue, line width=0.20em, -{Stealth[round]}, shorten >= 0.3em, radius=0.3em}};
  \tikzset{relE/.style={tolhighcontrastYellow, line width=0.15em, -{Stealth[round]}, shorten >= 0.2em, radius=0.2em}};
  \tikzset{relS/.style={tolhighcontrastRed, line width=0.10em, -{Stealth[round]}, shorten >= 0.2em, radius=0.2em}};
  \tikzset{relNotP/.style={tolhighcontrastRed, line width=0.2em, -{Stealth[round]}, shorten >= 0.3em, radius=0.3em}};
  \tikzset{relNotE/.style={tolhighcontrastRed, line width=0.15em, -{Stealth[round]}, shorten >= 0.2em, radius=0.2em}};

  \begin{scope}
    \draw[tolbrightGreen, line width=1ex, rounded corners=2pt] (80:2em) -- (80:-2em) -- +(160:2em);

    \matrix[
    name=s,
    ampersand replacement=§,
    column sep=1em,
    matrix of nodes,
    matrix anchor=west,
    nodes={
      draw, circle, minimum height=2.3ex, font=\small, inner sep=0.2ex, fill=white
    },
    ] at (1.5em,0em) {
        {1} § {2} § {3} § {4} \\
    };
    \graph[use existing nodes, edges={-{Stealth[round]}}] { (s-1-1) -> (s-1-2) -> (s-1-3) -> (s-1-4) };
  \end{scope}
  \begin{scope}[on background layer]
    \fill[relP] (s-1-2.north) circle [] coordinate (p1);
    \fill[relP] (s-1-3.north) circle [] coordinate (p2);
    \fill[relP] (s-1-4.north) circle [] coordinate (p3);
    \draw[relP, bend left] (p1) to (p2) to (p3);

    \fill[relE] (s-1-1.south) circle [] coordinate (p1);
    \fill[relE] (s-1-2.south) circle [] coordinate (p2);
    \draw[relE, bend right] (p1) to (p2);

    \fill[relE] (s-1-3.south) circle [] coordinate (p1);
    \fill[relE] (s-1-4.south) circle [] coordinate (p2);
    \draw[relE, bend right] (p1) to (p2);
  \end{scope}

  \begin{scope}[xshift=17em]
    \draw[tolbrightRed, line width=1ex, rounded corners=2pt] (135:2.5em) -- (135:-2.5em) (45:2.5em) -- (45:-2.5em);

    \matrix[
    name=s,
    ampersand replacement=§,
    column sep=1em,
    matrix of nodes,
    matrix anchor=west,
    nodes={
      draw, circle, minimum height=2.3ex, font=\small, inner sep=0.2ex, fill=white
    },
    ] at (3em,0em) {
        {1} § {2} § {3} § {4} \\
    };
    \graph[use existing nodes, edges={-{Stealth[round]}}] { (s-1-1) -> (s-1-2) -> (s-1-3) -> (s-1-4) };

    \begin{scope}[on background layer]
      \fill[relNotP] (s-1-1.north) circle [] coordinate (p1);
      \fill[relNotP] (s-1-3.north) circle [] coordinate (p2);
      \fill[relNotP] (s-1-4.north) circle [] coordinate (p3);
      \draw[dashed, relNotP, bend left] (p1) to (p2) to (p3);

      \fill[relNotE] (s-1-2.south west) circle [] coordinate (p1);
      \fill[relNotE] (s-1-2.south east) circle [] coordinate (p2);
      \draw[relNotE, dashed, bend right=130, looseness=8] (p1) to (p2);
    \end{scope}
  \end{scope}
\end{tikzpicture}

    \item<2> The intersection of a guarded set $\mathcal{S}$ and a tuple $\elemtupler$ is an infix
          \vspace{0.5em}

          \begin{tikzpicture}
  \tikzset{faded edge/.style={dash pattern=on 1pt off 1pt on 1pt off 1pt on 2pt off 1pt on 2pt off 1pt on 100pt}};
  \tikzset{faded edge reverse/.style={tips=false,dash pattern=on 5pt off 1pt on 5pt off 1pt on 2pt off 1pt on 2pt off 1pt on 1pt off 1pt on 1pt off 1pt on 1pt off 1pt on 1pt off 1pt on 1pt off 100pt}};

  \tikzset{relP/.style={tolhighcontrastBlue, line width=0.25em, -{Stealth[round]}, shorten >= 0.4em, radius=0.4em}};
  \tikzset{relE/.style={tolhighcontrastYellow, line width=0.15em, -{Stealth[round]}, shorten >= 0.2em, radius=0.2em}};
  \tikzset{relS/.style={tolhighcontrastRed, line width=0.10em, -{Stealth[round]}, shorten >= 0.2em, radius=0.2em}};

  \tikzset{el/.style={draw, circle, minimum height=2.3ex, font=\small, inner sep=0.2ex, fill=white}};

  \begin{scope}
    \draw[tolbrightGreen, line width=1ex, rounded corners=2pt] (80:2em) -- (80:-2em) -- +(160:2em);

    \def\denom{3}
    \path[every node/.style=el] (5.5em, 0)
       +(15:-3em) -- node[pos=0] (1) {1} node[pos=1/\denom] (2) {2} node[pos=2/\denom] (3) {3} node[pos=1] (4) {4} +(15:3em);

    \node [el,above=0.7em of 2] (5) { 5 };
    \node [el,below=0.7em of 3] (6) { 6 };
    \node [el,below right=0.7em of 6] (7) { 7 };

    \graph[use existing nodes, edges={white, -{Stealth[round]}, very thick}] {
      (1) -> (2) -> (3) -> (4),
      (5) -> (2),
      (3) -> (6) -> (7),
    };

    \begin{scope}[on background layer]
      \def\dx{2ex};
      \def\dy{1.9ex};
      \path {
        (1) ++(15:-\dx) ++(105:\dy) coordinate (c1)
        (1) ++(15:-\dx) ++(105:-\dy) coordinate (c4)
        (4) ++(15:\dx) ++(105:\dy) coordinate (c2)
        (4) ++(15:\dx) ++(105:-\dy) coordinate (c3)
      };
      \fill[tolhighcontrastBlue, rounded corners=5pt] (c1) -- (c2) -- (c3) -- (c4) -- cycle;

      \def\dx{1.7ex};
      \def\dy{1.6ex};
      \path {
        (5) ++(90:\dy) ++(0:-\dx) coordinate (c1)
        (2) ++(15:-\dx) ++(105:-0.9*\dy) coordinate (c2)
        (3) ++(15:-\dx) ++(105:-0.9*\dy) coordinate (c3)
        (6) ++(-\dx,-\dy) coordinate (c4)
        (7) ++(-\dx,-\dy) coordinate (c5)
        (7) ++(\dx,-\dy) coordinate (c6)
        (7) ++(\dx,\dy) coordinate (c7)
        (6) ++(\dx,0) coordinate (c8)
        (3) ++(15:\dx) ++(105:-0.8*\dy) coordinate (c9)
        (3) ++(15:\dx) ++(105:0.9*\dy) coordinate (c10)
        (3) ++(15:-\dx) ++(105:0.9*\dy) coordinate (c11)
        (5) ++(90:\dy) ++(0:\dx) coordinate (c12)
      };
      \fill[tolhighcontrastYellow, rounded corners=5pt] (c1) -- (c2) -- (c3) -- (c4) -- (c5) -- (c6) -- (c7) -- (c8) -- (c9) -- (c10) -- (c11) -- (c12) -- cycle;
    \end{scope}
  \end{scope}

  \begin{scope}[xshift=17em]
    \draw[tolbrightRed, line width=1ex, rounded corners=2pt] (135:2.5em) -- (135:-2.5em) (45:2.5em) -- (45:-2.5em);

    \def\denom{3}
    \path[every node/.style=el] (7.5em, 0)
    +(15:-3em) -- node[pos=0] (1) {1} node[pos=1/\denom] (2) {2} node[pos=2/\denom] (3) {3} node[pos=1] (4) {4} +(15:3em);

    \node [el, below=1em of 2, xshift=0.8em] (5) { 5 };
    \node [el, below=1em of 3, xshift=0.8em] (6) { 6 };

    \graph[use existing nodes, edges={white, -{Stealth[round]}, very thick}] {
      (1) -> (2) -> (3) -> (4),
      (1) -> (5) -> (6) -> (4),
    };

    \begin{scope}[on background layer]
      \def\dx{2ex};
      \def\dy{1.9ex};
      \path {
        (1) ++(15:-\dx) ++(105:\dy) coordinate (c1)
        (1) ++(15:-\dx) ++(105:-\dy) coordinate (c4)
        (4) ++(15:\dx) ++(105:\dy) coordinate (c2)
        (4) ++(15:\dx) ++(105:-\dy) coordinate (c3)
      };
      \fill[tolhighcontrastBlue, rounded corners=5pt] (c1) -- (c2) -- (c3) -- (c4) -- cycle;

      \def\dx{1.7ex};
      \def\dy{1.6ex};
      \path {
        (1) ++(15:-\dx) ++(105:\dy) coordinate (c1)
        (1) ++(15:-\dx) ++(105:-0.7*\dy) coordinate (c2)
        (5) ++(15:-1.2*\dx) ++(105:0*\dy) coordinate (c2p)
        (5) ++(15:-\dx) ++(105:-\dy) coordinate (c3)
        (6) ++(15:\dx) ++(105:-\dy) coordinate (c4)
        (6) ++(15:\dx) ++(105:-0*\dy) coordinate (c4a)
        (4) ++(15:\dx) ++(105:-0.7*\dy) coordinate (c4b)
        (4) ++(15:\dx) ++(105:\dy) coordinate (c5)
        (4) ++(15:-\dx) ++(105:\dy) coordinate (c6)
        (4) ++(15:-\dx) ++(105:-0.7*\dy) coordinate (c7)
        (6) ++(15:0*\dx) ++(105:\dy) coordinate (c8)
        (5) ++(15:0) ++(105:\dy) coordinate (c9)
        (1) ++(15:\dx) ++(105:-0*\dy) coordinate (c10)
        (1) ++(15:\dx) ++(105:\dy) coordinate (c11)
      };
      \fill[tolhighcontrastYellow, rounded corners=5pt] (c1) -- (c2) -- (c2p) -- (c3) -- (c4) -- (c4a) -- (c4b) -- (c5) -- (c6) -- (c7) -- (c8) -- (c9) -- (c10) -- (c11) -- cycle;
    \end{scope}
  \end{scope}
\end{tikzpicture}

  \end{enumerate}
\end{frame}

\def\tups{\textcolor{tolmutedTeal}{\elemtuples}}
\def\tupt{\textcolor{tolmutedTeal}{\elemtuplet}}
\def\ij#1{\textcolor{tolmutedWine}{(i_{#1},j_{#1})}}
\def\op#1{\textcolor{tolmutedWine}{(o_{#1},p_{#1})}}
\def\picelemOneA{%
  \tikz[baseline] {
    \tikzset{el/.style={draw, circle, minimum height=2.3ex, font=\small, inner sep=0.2ex, fill=white}};
    \node[el,anchor=base] {1};
  }%
}
\def\picelemOneB{%
  \tikz[baseline] {%
    \tikzset{el/.style={draw, circle, minimum height=2.3ex, font=\small, inner sep=0.2ex, fill=white}};
    \node[el,anchor=base] {a};
  }%
}
\def\picelemTwoA{%
  \tikz[baseline] {
    \tikzset{el/.style={draw, circle, minimum height=2.3ex, font=\small, inner sep=0.2ex, fill=white}};
    \node[el,anchor=base] {2};
  }%
}
\def\picelemTwoB{%
  \tikz[baseline] {
    \tikzset{el/.style={draw, circle, minimum height=2.3ex, font=\small, inner sep=0.2ex, fill=white}};
    \node[el,anchor=base] {b};
  }%
}
\def\ni{\textcolor{tolbrightYellowDarker}{i}}
\def\nj{\textcolor{tolbrightYellowDarker}{j}}
\begin{frame}{$z$-similarity}
  \begin{tikzpicture}
  \tikzset{relP/.style={tolhighcontrastBlue, line width=0.25em, -{Stealth[round]}, shorten >= 0.4em, radius=0.4em}};
  \tikzset{relE/.style={tolhighcontrastYellow, line width=0.15em, -{Stealth[round]}, shorten >= 0.2em, radius=0.2em}};
  \tikzset{relS/.style={tolhighcontrastRed, line width=0.10em, -{Stealth[round]}, shorten >= 0.2em, radius=0.2em}};
  \tikzset{el/.style={draw, circle, minimum height=2.3ex, font=\small, inner sep=0.2ex, fill=white}};

  \begin{scope}
    \def\denom{3}
    \path[every node/.style=el] (0, 0)
       +(15:-3em) -- node[pos=0] (1) {1} node[pos=1/\denom] (2) {2} node[pos=2/\denom] (3) {3} node[pos=1] (4) {4} +(15:3em);

    \node [el,above=0.7em of 2] (5) { 5 };
    \node [el,below=0.7em of 3] (6) { 6 };
    \node [el,below right=0.7em of 6] (7) { 7 };

    \graph[use existing nodes, edges={white, -{Stealth[round]}, very thick}] {
      (1) -> (2) -> (3) -> (4),
      (5) -> (2),
      (3) -> (6) -> (7),
    };

    \begin{scope}[on background layer]
      \def\dx{2ex};
      \def\dy{1.9ex};
      \path {
        (1) ++(15:-\dx) ++(105:\dy) coordinate (c1)
        (1) ++(15:-\dx) ++(105:-\dy) coordinate (c4)
        (4) ++(15:\dx) ++(105:\dy) coordinate (c2)
        (4) ++(15:\dx) ++(105:-\dy) coordinate (c3)
      };
      \fill[tolhighcontrastBlue, rounded corners=5pt] (c1) -- (c2) -- (c3) -- (c4) -- cycle;

      \def\dx{1.7ex};
      \def\dy{1.6ex};
      \path {
        (5) ++(90:\dy) ++(0:-\dx) coordinate (c1)
        (2) ++(15:-\dx) ++(105:-0.9*\dy) coordinate (c2)
        (3) ++(15:-\dx) ++(105:-0.9*\dy) coordinate (c3)
        (6) ++(-\dx,-\dy) coordinate (c4)
        (7) ++(-\dx,-\dy) coordinate (c5)
        (7) ++(\dx,-\dy) coordinate (c6)
        (7) ++(\dx,\dy) coordinate (c7)
        (6) ++(\dx,0) coordinate (c8)
        (3) ++(15:\dx) ++(105:-0.8*\dy) coordinate (c9)
        (3) ++(15:\dx) ++(105:0.9*\dy) coordinate (c10)
        (3) ++(15:-\dx) ++(105:0.9*\dy) coordinate (c11)
        (5) ++(90:\dy) ++(0:\dx) coordinate (c12)
      };
      \fill[tolhighcontrastYellow, rounded corners=5pt] (c1) -- (c2) -- (c3) -- (c4) -- (c5) -- (c6) -- (c7) -- (c8) -- (c9) -- (c10) -- (c11) -- (c12) -- cycle;
    \end{scope}

    \node[font=\Large] at (-1em,4em) { $\str{A}$ };
  \end{scope}

  \begin{scope}[xshift=12em]
    \def\denom{3}
    \path[every node/.style=el] (0, 0)
       +(15:-3em) -- node[pos=0] (1) {a} node[pos=1/\denom] (2) {b} node[pos=2/\denom] (3) {c} node[pos=1] (4) {d} +(15:2.5em);

    \graph[use existing nodes, edges={white, -{Stealth[round]}, very thick}] {
      (1) -> (2) -> (3) -> (4),
    };

    \begin{scope}[on background layer]
      \def\dx{2ex};
      \def\dy{1.9ex};
      \path {
        (1) ++(15:-\dx) ++(105:\dy) coordinate (c1)
        (1) ++(15:-\dx) ++(105:-\dy) coordinate (c4)
        (4) ++(15:\dx) ++(105:\dy) coordinate (c2)
        (4) ++(15:\dx) ++(105:-\dy) coordinate (c3)
      };
      \fill[tolhighcontrastBlue, rounded corners=5pt] (c1) -- (c2) -- (c3) -- (c4) -- cycle;

      \def\dx{1.7ex};
      \def\dy{1.6ex};
      \path {
        (2) ++(15:-\dx) ++(105:0.9*\dy) coordinate (c1)
        (2) ++(15:-\dx) ++(105:-0.9*\dy) coordinate (c2)
        (3) ++(15:\dx) ++(105:-0.9*\dy) coordinate (c3)
        (3) ++(15:\dx) ++(105:0.9*\dy) coordinate (c4)
      };
      \fill[tolhighcontrastYellow, rounded corners=5pt] (c1) -- (c2) -- (c3) -- (c4) -- cycle;
    \end{scope}
    \node[font=\Large] at (-1em,5em) { $\str{B}$ };
  \end{scope}

  \begin{scope}[xshift=21em]
    \def\denom{3}
    \path[every node/.style=el] (0, 0)
       +(15:-3em) -- node[pos=0] (1) {a} node[pos=1/\denom] (2) {b} node[pos=2/\denom] (3) {c} node[pos=1] (4) {d} +(15:2.5em);

    \node [el,below=0.7em of 3] (6) { f };
    \node [el,below right=0.7em of 6] (7) { g };

    \graph[use existing nodes, edges={white, -{Stealth[round]}, very thick}] {
      (1) -> (2) -> (3) -> (4),
      (3) -> (6) -> (7),
    };

    \begin{scope}[on background layer]
      \def\dx{2ex};
      \def\dy{1.9ex};
      \path {
        (1) ++(15:-\dx) ++(105:\dy) coordinate (c1)
        (1) ++(15:-\dx) ++(105:-\dy) coordinate (c4)
        (4) ++(15:\dx) ++(105:\dy) coordinate (c2)
        (4) ++(15:\dx) ++(105:-\dy) coordinate (c3)
      };
      \fill[tolhighcontrastBlue, rounded corners=5pt] (c1) -- (c2) -- (c3) -- (c4) -- cycle;

      \def\dx{1.7ex};
      \def\dy{1.6ex};
      \path {
        (2) ++(15:-\dx) ++(105:0.9*\dy) coordinate (c1)
        (2) ++(15:-\dx) ++(105:-0.9*\dy) coordinate (c2)
        (3) ++(15:-\dx) ++(105:-0.9*\dy) coordinate (c3)
        (6) ++(-\dx,-\dy) coordinate (c4)
        (7) ++(-\dx,-\dy) coordinate (c5)
        (7) ++(\dx,-\dy) coordinate (c6)
        (7) ++(\dx,\dy) coordinate (c7)
        (6) ++(\dx,0) coordinate (c8)
        (3) ++(15:\dx) ++(105:-0.9*\dy) coordinate (c9)
        (3) ++(15:\dx) ++(105:0.9*\dy) coordinate (c10)
      };
      \fill[tolhighcontrastYellow, rounded corners=5pt] (c1) -- (c2) -- (c3) -- (c4) -- (c5) -- (c6) -- (c7) -- (c8) -- (c9) -- (c10) -- cycle;
    \end{scope}
    \node[font=\Large] (bhead) at (-0.7em,5em) { $\str{B} $ };
    \node[below=0em of bhead.south, anchor=north, inner sep=0pt] { forward extension };
  \end{scope}

  \begin{scope}[xshift=30em]
    \def\denom{3}
    \path[every node/.style=el] (0, 0)
       +(15:-3em) -- node[pos=0] (1) {a} node[pos=1/\denom] (2) {b} node[pos=2/\denom] (3) {c} node[pos=1] (4) {d} +(15:2.5em);

    \node [el,above=0.7em of 2] (5) { e };
    \node [el,below=0.7em of 3] (6) { f };
    \node [el,below right=0.7em of 6] (7) { g };

    \graph[use existing nodes, edges={white, -{Stealth[round]}, very thick}] {
      (1) -> (2) -> (3) -> (4),
      (5) -> (2),
      (3) -> (6) -> (7),
    };

    \begin{scope}[on background layer]
      \def\dx{2ex};
      \def\dy{1.9ex};
      \path {
        (1) ++(15:-\dx) ++(105:\dy) coordinate (c1)
        (1) ++(15:-\dx) ++(105:-\dy) coordinate (c4)
        (4) ++(15:\dx) ++(105:\dy) coordinate (c2)
        (4) ++(15:\dx) ++(105:-\dy) coordinate (c3)
      };
      \fill[tolhighcontrastBlue, rounded corners=5pt] (c1) -- (c2) -- (c3) -- (c4) -- cycle;

      \def\dx{1.7ex};
      \def\dy{1.6ex};
      \path {
        (5) ++(90:\dy) ++(0:-\dx) coordinate (c1)
        (2) ++(15:-\dx) ++(105:-0.9*\dy) coordinate (c2)
        (3) ++(15:-\dx) ++(105:-0.9*\dy) coordinate (c3)
        (6) ++(-\dx,-\dy) coordinate (c4)
        (7) ++(-\dx,-\dy) coordinate (c5)
        (7) ++(\dx,-\dy) coordinate (c6)
        (7) ++(\dx,\dy) coordinate (c7)
        (6) ++(\dx,0) coordinate (c8)
        (3) ++(15:\dx) ++(105:-0.8*\dy) coordinate (c9)
        (3) ++(15:\dx) ++(105:0.9*\dy) coordinate (c10)
        (3) ++(15:-\dx) ++(105:0.9*\dy) coordinate (c11)
        (5) ++(90:\dy) ++(0:\dx) coordinate (c12)
      };
      \fill[tolhighcontrastYellow, rounded corners=5pt] (c1) -- (c2) -- (c3) -- (c4) -- (c5) -- (c6) -- (c7) -- (c8) -- (c9) -- (c10) -- (c11) -- (c12) -- cycle;
    \end{scope}
    \node[font=\Large] (bhead) at (0,5em) { $\str{B} $ };
    \node[below=0em of bhead.south, anchor=north, inner sep=0pt] { backward extension };
  \end{scope}


\end{tikzpicture}

  \vspace{-1ex}
  \pause
  \begin{minipage}[t]{0.6\textwidth}
    \begin{definition}[$z$-similar elements]
      \vspace{2.5ex}
      \picelemOneA{} $= \langle \cdots\; \tups_{0} \ij{1} \tups_{1} \ij{2} \cdots \ij{z} \tups_{z} / x \rangle$

      \pause
      \vspace{2ex}
      \picelemOneB{} $= \langle \cdots\; \tupt_{0} \op{1} \tupt_{1} \op{2} \cdots \op{z} \tupt_{z} / y \rangle$
    \end{definition}
  \end{minipage}
  \begin{minipage}[t]{0.39\textwidth}
    \vspace{1.5ex}

    {\Large \picelemOneA{} $\approx_{z}$ \picelemOneB{} if:}
    \vspace{0.5ex}
    {\large\begin{enumerate}
      \item $x = y$,
      \item $\ij{k} = \op{k}$, and
      \item $(\str{A}, \tups_{k}) \bisimto_{\FGF}^{z} (\str{B}, \tupt_{k})$
    \end{enumerate}}
    \vspace{0.5ex}
    \hspace{3pt}for all $k \in [1,z]$.
  \end{minipage}
\end{frame}

\begin{frame}{Forward extension}
  \begin{center}
  given $\;\picelemOneA{} \approx_{z} \picelemOneB{}\;$ with
  \begin{math}
    \begin{array}{l @{{}={}\langle{}} c @{{}/{}} c @{{}\rangle}}
      \picelemOneA{} & \cdots \tups_{z-1} \ij{z} \tups_{z} & x \\[1ex]
      \picelemOneB{} & \cdots \tupt_{z-1} \op{z} \tupt_{z} & y
    \end{array}
  \end{math}

  \vspace{3em}
  \begin{minipage}{0.8\textwidth}
  \begin{minipage}{4em}
    \textbf{Case 1}\\{}
  \end{minipage}
  \begin{math}
    \begin{array}{rl @{{}={}\langle{}} c @{{}/{}} c @{{}\rangle}}
      \text{if} & \picelemTwoA{} & \cdots \tups_{z-1}\ij{z} \tups_{z} & x + 1 \\[1ex]
      \text{then} & \picelemTwoB{} & \cdots \tupt_{z-1}\op{z} \tupt_{z} & y + 1
    \end{array}
  \end{math}

  \vspace{2em}
  \begin{minipage}{4em}
    \textbf{Case 2}\\{}
  \end{minipage}
  \begin{math}
    \begin{array}{rl @{{}={}\langle{}} c @{{}/{}} c @{{}\rangle}}
      \text{if} & \picelemTwoA{} & \cdots \tups_{z} (\ni,\nj) \tups_{z+1} & (\nj{-}\ni{+}1) + 1 \\[1ex]
      \text{then} & \picelemTwoB{} & \cdots \tupt_{z} (\ni,\nj) \tupt_{z+1} & (\nj{-}\ni{+}1) + 1
    \end{array}
  \end{math}
  \end{minipage}
  \end{center}
\end{frame}

\section{Conclusion}

\end{document}
