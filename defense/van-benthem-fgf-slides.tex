\documentclass[169]{beamer}

\usetheme{auriga}
\usepackage{ifxetex}

\ifxetex
  \usepackage{polyglossia}
  \usepackage{fontspec}

  \setmainlanguage{english}
\else
  \usepackage[english]{babel}
\fi

\usepackage{hyperref}
\usepackage{blindtext}
\usepackage{graphicx}
\usepackage{tabularx}
\usepackage{listings}
\usepackage{tikz}
\usepackage{pifont}
\usepackage{fontawesome}
\usepackage{enumitem}
\usetikzlibrary{arrows, arrows.meta, decorations.markings, shapes, calc, positioning, fit, graphs, trees, matrix, backgrounds, bending, decorations.pathmorphing, decorations.pathreplacing, decorations.shapes, fadings, shadings, patterns, graphs.standard}
\usepackage{tikzpeople}
\usepackage{upquote}
\usepackage{tikzsymbols}
\usepackage{bookmark}
\usepackage{printlen}
\usepackage{xcolor}
\usepackage{mathtools}
\usepackage{amssymb}
\usepackage{amsmath}

\usepackage{appendixnumberbeamer}

\includeonlyframes{current}

\lstset{upquote=true}
%\setsansfont[BoldFont={Fira Sans SemiBold}]{Fira Sans Book}
\makeatletter
\newlength\beamerleftmargin
\setlength\beamerleftmargin{\Gm@lmargin}
\makeatother

\hypersetup{bookmarks=true,colorlinks=true,allcolors=blue}

\title{A Constructive Proof of a Van~Benthem Theorem for the Forward Guarded Fragment of First~Order~Logic}
\author{Benno Fünfstück}
\institute{Master's defense}
\date{24.06.2024}

../Lipics version/settings/tolcolors.tex
% Inline comments
\definecolor{ao(english)}{rgb}{0.0, 0.5, 0.0}
\definecolor{brickred}{rgb}{0.8, 0.25, 0.33}
\newcommand{\myundef}[1]{\textcolor{brickred}{\textbf{#1}}}
\newcommand{\possiblelie}[1]{\textcolor{brickred}{\textbf{#1}}}
\newcommand{\becareful}[1]{\textcolor{brickred}{\textbf{#1}}}
\newcommand{\bennof}[1]{\textbf{\textcolor{blue}{\textbf{#1}}}}
\newcommand{\bbe}[1]{\textbf{\textcolor{ao(english)}{#1}}}
\newcommand{\bbebox}[1]{\todo[inline,color=green!30]{\textbf{BB\@: }#1}\xspace}
\newcommand{\bbeside}[1]{\todo[color=green!30,size=\scriptsize,fancyline]{\textbf{BB\@: }#1}\xspace}
\newcommand{\bfbox}[1]{\todo[inline,color=blue!30]{\textbf{BF\@: }#1}\xspace}
\newcommand{\bfside}[1]{\todo[color=blue!30,size=\scriptsize,fancyline]{\textbf{BF\@: }#1}\xspace}

% logics:
\newcommand{\Logic}[1]{\ensuremath{\mathsf{#1}}} % a Logic
\newcommand{\logicL}{\Logic{L}} % some logic L
\newcommand{\Laffix}{\Logic{L}_{\mathsf{affix}}}   % L_affix
\newcommand{\Linfix}{\Logic{L}_{\mathsf{inf}}}   % L_infix
\newcommand{\Lsuffix}{\Logic{L}_{\mathsf{suf}}} % L_suffix
\newcommand{\Lprefix}{\Logic{L}_{\mathsf{pre}}} % L_prefix

\newcommand{\Gaffix}{\Logic{G}_{\mathsf{affix}}}   % G_affix
\newcommand{\Ginfix}{\Logic{G}_{\mathsf{inf}}}     % G_infix
\newcommand{\Gsuffix}{\Logic{G}_{\mathsf{suf}}}    % G_suffix
\newcommand{\Gprefix}{\Logic{G}_{\mathsf{pre}}}    % G_prefix

\newcommand{\GF}{\Logic{GF}}   % Guarded Fragment
\newcommand{\FGF}{\Logic{FGF}}   % Forward Guarded Fragment
\newcommand{\FO}{\Logic{FO}}   % First-Order Logic

% Complexity classses:
\newcommand{\complexityclass}[1]{\textsc{#1}} % any complexity class
\newcommand{\ExpTime}{\complexityclass{ExpTime}} % exponential time
\hyphenation{Exp-Time} % prevent "Ex-PTime" (see, e.g. Tobies'01, Glimm'07 ;-)
\newcommand{\NExpTime}{\complexityclass{NExpTime}} % nondeterministic exponential time
\hyphenation{NExp-Time} % prevent "Ex-PTime" (see, e.g. Tobies'01, Glimm'07 ;-)
\newcommand{\TwoExpTime}{\complexityclass{2ExpTime}} % doubly-exponential time
\newcommand{\TwoNExpTime}{\complexityclass{2NExpTime}} % doubly-nondeterministic-exponential time
\newcommand{\coNExpTime}{\complexityclass{coNExpTime}} % co nondeterministic exponential time
\hyphenation{coNExp-Time} 
\newcommand{\Tower}{\complexityclass{Tower}}
\newcommand{\LogSpace}{\complexityclass{LogSpace}}
\newcommand{\PSpace}{\complexityclass{PSpace}}
\newcommand{\PTime}{\complexityclass{PTime}}

% Others
\newcommand{\str}[1]{{\mathfrak{#1}}}
\DeclareRobustCommand{\unravel}[1]{\vv{\mathfrak{#1}}}
\newcommand{\deff}{\coloneqq}
\newcommand{\arity}{\mathsf{ar}}
\renewcommand{\iff}{\leftrightarrow}

\newcommand{\N}{{\mathbb{N}}}
\newcommand{\Z}{{\mathbb{Z}}} 
\newcommand{\Q}{{\mathbb{Q}}}
\newcommand{\V}{\mathbf{V}} 
\newcommand{\R}{\mathbf{R}} 
\newcommand{\Var}{\mathrm{Var}}
\newcommand{\sigSigma}{\Sigma}

\newcommand{\sqin}{%
  \mathrel{\vphantom{\sqsubset}\text{%
    \mathsurround=0pt
    \ooalign{$\sqsubset$\cr$-$\cr}%
  }}%
}

% Rel symbols
\newcommand{\rel}[1]{\mathrm{#1}}
\newcommand{\relP}{\rel{P}}
\newcommand{\relR}{\rel{R}}
\newcommand{\relQ}{\rel{Q}}
\newcommand{\relS}{\rel{S}}
\newcommand{\relT}{\rel{T}}
\newcommand{\relU}{\rel{U}}
\newcommand{\relA}{\rel{A}}
\newcommand{\relB}{\rel{B}}
\newcommand{\relC}{\rel{C}}
\newcommand{\relD}{\rel{D}}
\newcommand{\relE}{\rel{E}}
\newcommand{\relH}{\rel{H}}
\newcommand{\sig}{\mathsf{sig}}

% Tuples 
\newcommand{\emptytupl}{\epsilon}
\newcommand{\set}{\mathsf{set}}

% Domain elements
\newcommand{\elem}[1]{\mathrm{#1}}                           % domain element
\newcommand{\elema}{\elem{a}}                             % domain element a
\newcommand{\elemb}{\elem{b}}                             % domain element b
\newcommand{\elemc}{\elem{c}}                             % domain element c
\newcommand{\elemd}{\elem{d}}                             % domain element d
\newcommand{\eleme}{\elem{e}}                             % domain element e
\newcommand{\elemf}{\elem{f}}                               % domain element f
\newcommand{\elemg}{\elem{g}}                             % domain element g
\newcommand{\elemh}{\elem{h}}                             % domain element h
\newcommand{\elemi}{\elem{i}}                             % domain element i
\newcommand{\elemj}{\elem{j}}                             % domain element j
\newcommand{\elemo}{\elem{o}}                             % domain element o
\newcommand{\elemp}{\elem{p}}                             % domain element p
\newcommand{\elemq}{\elem{q}}                             % domain element q
\newcommand{\elemr}{\elem{r}}                             % domain element r
\newcommand{\elems}{\elem{s}}                             % domain element s
\newcommand{\elemt}{\elem{t}}                             % domain element t
\newcommand{\elemw}{\elem{w}}                             % domain element w
\newcommand{\elemv}{\elem{v}}                             % domain element v
\newcommand{\elemu}{\elem{u}}                             % domain element u
\newcommand{\elemx}{\elem{x}}                             % domain element u
\newcommand{\elemtuplea}{\overline{\elema}}                         % tuple of domain element a
\newcommand{\elemtupleb}{\overline{\elemb}}                         % tuple of domain element b
\newcommand{\elemtuplec}{\overline{\elemc}}                         % tuple of domain element c
\newcommand{\elemtupled}{\overline{\elemd}}                         % tuple of domain element d
\newcommand{\elemtuplee}{\overline{\eleme}}                         % tuple of domain element e
\newcommand{\elemtuplef}{\overline{\elemf}}                           % tuple of domain element f
\newcommand{\elemtupleg}{\overline{\elemg}}                         % tuple of domain element g
\newcommand{\elemtupleh}{\overline{\elemh}}                         % tuple of domain element h
\newcommand{\elemtuplei}{\overline{\elemi}}                         % tuple of domain element i
\newcommand{\elemtupleo}{\overline{\elemo}}                         % tuple of domain element o
\newcommand{\elemtuplep}{\overline{\elemp}}                         % tuple of domain element p
\newcommand{\elemtupleq}{\overline{\elemq}}                         % tuple of domain element q
\newcommand{\elemtupler}{\overline{\elemr}}                         % tuple of domain element r
\newcommand{\elemtuples}{\overline{\elems}}                         % tuple of domain element s
\newcommand{\elemtuplet}{\overline{\elemt}}                         % tuple of domain element t
\newcommand{\elemtuplew}{\overline{\elemw}}                         % tuple of domain element w
\newcommand{\elemtupleu}{\overline{\elemu}}                         % tuple of domain element u
\newcommand{\elemtuplev}{\overline{\elemv}}                         % tuple of domain element v
\newcommand{\elemtuplex}{\overline{\elemx}}                         % tuple of domain element v

\newcommand{\elemtupledfromto}[2]{\overline{\elemd}_{#1\ldots#2}}  % tuple of domelements d from #1 to #2
\newcommand{\elemtupleefromto}[2]{\overline{\eleme}_{#1\ldots#2}}  % tuple of domelements e from #1 to #2
\newcommand{\elemtuplecfromto}[2]{\overline{\elemc}_{#1\ldots#2}}  % tuple of domelements c from #1 to #2
\newcommand{\elemtuplebfromto}[2]{\overline{\elemb}_{#1\ldots#2}}  % tuple of domelements b from #1 to #2
\newcommand{\elemtupleafromto}[2]{\overline{\elema}_{#1\ldots#2}}  % tuple of domelements a from #1 to #2
\newcommand{\elemtupleffromto}[2]{{\elemtuplef}_{#1\dots#2}} % tuple of domelements f from #1 to #2
\newcommand{\elemtuplegfromto}[2]{{\elemtupleg}_{#1\dots#2}} % tuple of domelements g from #1 to #2

\DeclareRobustCommand{\elemtuptuplea}{\vv{\elema}}                         % tuple of domain element a
\DeclareRobustCommand{\elemtuptupleb}{\vv{\elemb}}                         % tuple of domain element b
\DeclareRobustCommand{\elemtuptuplec}{\vv{\elemc}}                         % tuple of domain element c
\DeclareRobustCommand{\elemtuptupled}{\vv{\elemd}}                         % tuple of domain element d
\DeclareRobustCommand{\elemtuptuplee}{\vv{\eleme}}                         % tuple of domain element e
\DeclareRobustCommand{\elemtuptuplef}{\vv{\elemf}}                           % tuple of domain element f
\DeclareRobustCommand{\elemtuptupleg}{\vv{\elemg}}                         % tuple of domain element g
\DeclareRobustCommand{\elemtuptupleh}{\vv{\elemh}}                         % tuple of domain element h
\DeclareRobustCommand{\elemtuptuplei}{\vv{\elemi}}                         % tuple of domain element i
\DeclareRobustCommand{\elemtuptuplej}{\vv{\elemj}}                         % tuple of domain element j
\DeclareRobustCommand{\elemtuptuplep}{\vv{\elemp}}                         % tuple of domain element p
\DeclareRobustCommand{\elemtuptupleq}{\vv{\elemq}}                         % tuple of domain element q
\DeclareRobustCommand{\elemtuptupler}{\vv{\elemr}}                         % tuple of domain element r
\DeclareRobustCommand{\elemtuptuples}{\vv{\elems}}                         % tuple of domain element s
\DeclareRobustCommand{\elemtuptuplet}{\vv{\elemt}}                         % tuple of domain element t
\DeclareRobustCommand{\elemtuptuplew}{\vv{\elemw}}                         % tuple of domain element w
\DeclareRobustCommand{\elemtuptupleu}{\vv{\elemu}}                         % tuple of domain element u
\DeclareRobustCommand{\elemtuptuplev}{\vv{\elemv}}                         % tuple of domain element v
\DeclareRobustCommand{\elemtuptuplex}{\vv{\elemx}}                         % tuple of domain element v

% Variables:
\newcommand{\var}[1]{\mathit{#1}}       % variable
\newcommand{\varx}{\var{x}}             % variable x
\newcommand{\vary}{\var{y}}             % variable y
\newcommand{\varz}{\var{z}}             % variable z
\newcommand{\varv}{\var{v}}             % variable v
\newcommand{\varu}{\var{u}}             % variable u
\newcommand{\varw}{\var{w}}             % variable w
\newcommand{\varh}{\var{h}}             % variable h
\newcommand{\vartuplex}{\overline{\varx}}    % tuple of variables x
\newcommand{\vartuplexomega}{\overline{\varx_{\omega}}}      % tuple of variables x_omega
\newcommand{\vartupley}{\overline{\vary}}                    % tuple of variables y
\newcommand{\vartupleyone}{\overline{\vary_1}}                    % tuple of variables y_1
\newcommand{\vartupleytwo}{\overline{\vary_2}}                    % tuple of variables y_2
\newcommand{\vartuplez}{\overline{\varz}}                    % tuple of variables z
\newcommand{\vartuplev}{\overline{\varv}}                    % tuple of variables v
\newcommand{\vartupleu}{\overline{\varu}}                    % tuple of variables u
\newcommand{\vartuplew}{\overline{\varw}}                    % tuple of variables w
\newcommand{\vartupleh}{\overline{\varh}}                    % tuple of variables h
\newcommand{\vartuplexfromto}[2]{\overline{\varx}_{#1\ldots#2}}  % tuple of variables x from #1 to #2
\newcommand{\vartupleyfromto}[2]{\overline{\vary}_{#1\ldots#2}}  % tuple of variables y from #1 to #2
\newcommand{\vartupleufromto}[2]{\overline{\varu}_{#1\ldots#2}}  % tuple of variables u from #1 to #2
\newcommand{\vartuplevfromto}[2]{\overline{\varv}_{#1\ldots#2}}  % tuple of variables v from #1 to #2
\newcommand{\vartuplewfromto}[2]{\overline{\varw}_{#1\ldots#2}}  % tuple of variables v from #1 to #2

% Theory
\newcommand{\theory}[1]{\mathcal{#1}}   % theory
\newcommand{\theoryT}{\theory{T}}       % theory T

% Types
\newcommand{\atp}[3]{\mathsf{atp}^{#1}_{#2}(#3)}
\newcommand{\tp}[3]{\mathsf{tp}^{#1}_{#2}(#3)}

% Tree unravelings
\newcommand{\seq}[1]{\mathsf{seq}(#1)}
\newcommand{\ctr}[1]{\mathsf{ctr}(#1)}
\newcommand{\relNext}{\rel{Next}}
\newcommand{\unraveldom}[1]{\vv{#1}}
\newcommand{\Seq}[1]{\mathsf{Seq}(\str{#1})}
\newcommand{\hist}[2]{\mathsf{hist}_{#1}(#2)}

% Restrictions
\renewcommand{\restriction}{\mathord{\upharpoonright}}
\newcommand{\restr}[2]{#1\restriction_{#2}} % the restriction of #1 to #2

% morphisms
\newcommand{\homo}[1]{\mathfrak{#1}}    % homomorphism
\newcommand{\homof}{\homo{f}}           % homomorphism f
\newcommand{\homog}{\homo{g}}           % homomorphism g
\newcommand{\homoh}{\homo{h}}           % homomorphism h
\newcommand{\homop}{\homo{p}}           % homomorphism p
\newcommand{\homoe}{\homo{e}}           % homomorphism e
\newcommand{\ishomoto}{\vartriangleleft} % is homomorhic to
\newcommand{\homeq}{\rightleftarrows} % homomorphically equivalent
\newcommand{\isoeq}{\cong} % isomorphic
\newcommand{\elemext}{\preceq} % elementary extension
\newcommand{\omegasat}[1]{\widehat{#1}}
\newcommand{\partisof}{\homo{f}}           % partial isomorphism f
\newcommand{\partisog}{\homo{g}}           % partial isomorphism g
\newcommand{\partisoh}{\homo{h}}           % partial isomorphism h

% bisimulations
\newcommand{\bisimulation}[1]{\mathcal{#1}} % a bisimulation
\newcommand{\bisimZ}{\bisimulation{Z}}
\newcommand{\bisimto}{\sim} % bisimilarity relation
\newcommand{\strbisimto}{\approx} % strong bisimilarity relation
\newcommand{\PartIso}[2]{\mathsf{Part}(#1,#2)}

% tikz helpers
\newcommand{\tikzdbg}{%
  \draw[step=5em,color=lightgray]%
    (current bounding box.south west) grid (0,0)%
    (current bounding box.north west) grid (0,0)%
    (current bounding box.south east) grid (0,0)%
    (current bounding box.north east) grid (0,0);%
  \fill[red] (0,0) circle (0.1);%
}

% Proof sketchs
\let\realproof\proof
\let\realendproof\endproof
% \newenvironment{proofsketch}{%
%   \renewcommand{\proofname}{\normalfont\emph{Proof Sketch}}\realproof}{\realendproof}

% hide proofs
\usepackage{apxproof}
\let\proof\appendixproof
\let\endproof\endappendixproof


\begin{document}

\begin{frame}
  \titlepage
\end{frame}

\begin{frame}{What is the ``Forward Guarded Fragment ($\FGF$)''?}
  \textbf{guarded}: quantification follows pattern ``$\exists{\elemtuplex} (\relR(\vartuplex, \vartupley) \land \varphi(\vartuplex, \vartupley))$''
  \vspace{0.5em}
  \begin{overprint}
    \onslide<-2>
      \begin{exampleblock}{Example 1}
        $\varphi_{1} = \exists{x_{1}x_{2}}\; (\relE(x_{1}, x_{2}) \land (\exists{x_{3}}(\relE(x_{2}, x_{3}) \land \relP(x_{3}))$
        \begin{center}
        \begin{tikzpicture}
  \tikzset{el/.style={circle,draw,fill=white,minimum width=2em,minimum height=2em}};
  \tikzset{relP/.style={tolhighcontrastBlue, line width=0.4em, -{Stealth[round]}, shorten >= 0.4em}};
  \tikzset{relE/.style={tolhighcontrastYellow, line width=0.25em, -{Stealth[round]}, shorten >= 0.2em}};
  \tikzset{relS/.style={tolhighcontrastRed, line width=0.15em, -{Stealth[round]}, shorten >= 0.2em}};
  \def\radiusP{0.7em};
  \def\radiusE{0.4em};
  \def\radiusS{0.2em};

  \matrix[matrix of nodes, every node/.style={el},name=n,column sep=4em, row sep=4em]  {
    |(n1)| $1$ & |(n2)| 2 & |(n3) [fill=tollightLightcyan]| 3 \\
  };
  \begin{scope}[on background layer]
    \fill[relE] (n1.east) circle [radius=\radiusE] coordinate (p1);
    \fill[relE] (n2.west) circle [radius=\radiusE] coordinate (p2);
    \draw[relE] (p1) to node[above]{E} (p2);
    \fill[relE] (n2.east) circle [radius=\radiusE] coordinate (p1);
    \fill[relE] (n3.west) circle [radius=\radiusE] coordinate (p2);
    \draw[relE] (p1) to node[above]{E} (p2);
  \end{scope}

  \node[above right=0em of n3, color=tollightLightcyan] { P };
\end{tikzpicture}

        \end{center}
      \end{exampleblock}
      \uncover<2>{
      \begin{alertblock}{Counterexample 1}
        $\varphi_{2} = \exists{x_{1}x_{2}}\; (\relE(x_{1}, x_{2}) \land (\exists{x_{3}}(\relE(x_{2}, x_{3}) \land \relE(x_{3}, x_{1}))$
        \begin{center}
        \input{res/counterex1.tex}
        \end{center}
      \end{alertblock}
      }
    \onslide<3->
      \begin{exampleblock}{More examples}
        \vspace{1em}
        \hspace{1em}
        \begin{minipage}{20em}
        $\varphi_{3} = \lnot \exists{x_{1}x_{2}}\; (\relG(x_{1}, x_{2}) \land \lnot \relE(x_{2}, x_{1}))$ \\[1em]
        $\varphi_{3} = \forall{x_{1}x_{2}}\; (\relG(x_{1}, x_{2}) \to \relE(x_{2}, x_{1}))$ \\[1em]
        $\varphi_{4} = \forall{x_{1}}\; (\relG(x_{1}, x_{1}) \to \exists{x_{2}} (\relP(x_{1},x_{2}, x_{1})))$
        \end{minipage}
      \end{exampleblock}
  \end{overprint}
\end{frame}

\begin{frame}{What is the ``Forward Guarded Fragment ($\FGF$)''?}
  \textbf{forward}: variables must appear in order $x_{1}$, $x_2$, \ldots
  \vspace{0.5em}
  \begin{alertblock}{Counterexamples}
    $\varphi_{3} = \forall{x_{1}x_{2}}\; (\relG(x_{1}, x_{2}) \to \relE(x_{2}, x_{1}))$ \\[1em]
    $\varphi_{5} = \exists{x_{1}}\; (\relE(x_{1}, x_{1}))$ \\[1em]
    $\varphi_{6} = \exists{x_{1}x_{2}x_{3}}\; (\relP(x_{1}, x_{2}, x_{3}) \land \relE(x_{1}, x_{3}))$ \\[1em]
    $\varphi_{7} = \exists{x_{2}x_{1}} \relE(x_{2}, x_{1})$
  \end{alertblock}

  \begin{exampleblock}{Examples}
    $\varphi_{1} = \exists{x_{1}x_{2}}\; (\relE(x_{1}, x_{2}) \land (\exists{x_{3}}(\relE(x_{2}, x_{3}) \land \relP(x_{3}))$ \\[1em]

    $\varphi_{8} = \exists{x_{1}x_{2}x_{3}}\; (\relP(x_{1}x_{2}x_{3}) \land \exists{x_{3}}(\relE(x_{2}, x_{3})))$ \\[1em]
    $\varphi_{9} = \exists{x_{1}x_{2}x_{3}}\; (\relP(x_{1}x_{2}x_{3}) \land \relE(x_{2}, x_{3}))$ \\[1em]
  \end{exampleblock}

  \vskip 0pt plus 1filll
\end{frame}

\begin{frame}{How expressive is $\FGF$?}
  \begin{center}
    \begin{tikzpicture}
  \tikzset{faded edge/.style={dash pattern=on 1pt off 1pt on 1pt off 1pt on 2pt off 1pt on 2pt off 1pt on 100pt}};
  \tikzset{faded edge reverse/.style={tips=false,dash pattern=on 5pt off 1pt on 5pt off 1pt on 2pt off 1pt on 2pt off 1pt on 1pt off 1pt on 1pt off 1pt on 1pt off 1pt on 1pt off 1pt on 1pt off 100pt}};

  \begin{scope}
    \graph[empty nodes, nodes={draw,circle}] {
        subgraph C_n [n=7,clockwise,radius=1.5cm];
    };
    \node[font=\LARGE] (strA) at (0,0) { $\str{A}$ };
    \draw[faded edge reverse] (1) -- (strA);
    \draw[faded edge reverse] (7) -- (2);
    \draw[faded edge reverse] (7) -- (4);
    \draw[faded edge reverse] (2) -- (6);
    \draw[faded edge reverse] (4) -- (1);
    \draw[faded edge reverse] (5) -- (3);
    \draw[faded edge reverse] (6) -- (3);
  \end{scope}

  \begin{scope}[xshift=15em]
    \graph[empty nodes, nodes={draw,circle}] {
        subgraph C_n [n=7,clockwise,radius=1.5cm];
    };
    \node[font=\LARGE] (strB) at (0,0) { $\str{B}$ };
    \draw[faded edge reverse] (1) -- (4);
    \draw[faded edge reverse] (1) -- (5);
    \draw[faded edge reverse] (7) -- (3);
    \draw[faded edge reverse] (2) -- (6);
    \draw[faded edge reverse] (4) -- (1);
    \draw[faded edge reverse] (5) -- (3);
    \draw[faded edge reverse] (6) -- (3);
    \draw[faded edge reverse] (3) -- (strB);
  \end{scope}

  \node[font=\LARGE] at ($(strA)!.5!(strB)$) { $\bisimto_{\FGF}$ };

\end{tikzpicture}

  \end{center}
  \begin{itemize}
    \item \emph{FGF-bisimulation}: equivalence relation on structures
    \item $\FGF$-formulae are $\FGF$-bisimulation-invariant
    \item \textbf{Can we express all $\FGF$-bisimulation-invariant $\FO$-definable properties in $\FGF$?}
  \end{itemize}
\end{frame}

\begin{frame}\frametitle<1>{Van Benthem's theorem}\frametitle<2>{Van Benthem's theorem (finitary version)}
  \begin{theorem}[Van Benthem's theorem for $\FGF$]
    A first-order formula $\varphi$ is equivalent to some $\FGF$-formula\only<2>{ \textbf{over finite structures}} if and only if it is invariant under $\FGF$-bisimulation\only<2>{ \textbf{over finite structures}}.
  \end{theorem}
\end{frame}

\begin{frame}{Part 1 of 2}
  First, look at the simpler example of modal logic to introduce concepts
\end{frame}

\begin{frame}[label=current]{Bisimulation for Modal Logic ($\ML$)}
  $\ML$ = quantification restricted to $\varphi(x) = \exists{y} (\relR(x,y) \land \psi(y))$
  \vspace{0.5em}
  \begin{definition}[$\ML$-bisimulation between $(\str{A}, s_{a})$ and $(\str{B}, s_{b})$]
    A $\ML$-bisimulation is a set $\mathcal{Z} \subseteq A \times B$ with $(s_{a}, s_{b}) \in \mathcal{Z}$ and:
    \begin{description}[itemsep=0.5em]
      \item[(Atomic Harmony)] if $(a,b) \in \mathcal{Z}$, then $a \in p_{i}^{\mathcal{A}}$ if and only if $b \in p_{i}^{\mathcal{B}}$,
      \item[(Forth)] if $(a,b) \in \mathcal{Z}$ and $(a,c) \in \relR^{\str{A}}$, there exists $d \in B$ with $(c,d) \in \mathcal{Z}$ and $(b,d) \in \relR^{\str{B}}$, and
      \item[(Back)] vice-versa of \textbf{(Forth)}
    \end{description}
  \end{definition}
\end{frame}

\begin{frame}[label=current]{$\ML$-bisimulation example}
  \begin{center}
    \begin{tikzpicture}
  \tikzset{el/.style={minimum height=3ex,minimum width=3ex,inner sep=0pt,outer sep=0pt,anchor=center}};
  \tikzset{nop/.style={el}};
  \tikzset{p1/.style={draw=black,rectangle,el}};
  \tikzset{eq1/.style={color=tolvibrantBlue,label distance=0em, label={[draw,circle,inner sep=0.1em,outer sep=0em,label distance=0.2em]45:\scriptsize{1}}}};
  \tikzset{eq2/.style={color=tolvibrantOrange,label={[draw,circle,inner sep=0.1em,outer sep=0em,label distance=0.4em]40:\scriptsize{2}}}};
  \tikzset{eq3/.style={color=tolvibrantTeal,label={[overlay,draw,circle,inner sep=0.1em,outer sep=0em,label distance=0em]60:\scriptsize{3}}}};
  \tikzset{eq1p/.style={color=tolvibrantBlue,label distance=0em, label={[draw,circle,inner sep=0.1em,outer sep=0em,label distance=0em]45:\scriptsize{1'}}}};
  \tikzset{eq2p/.style={color=tolvibrantOrange,label={[draw,circle,inner sep=0.1em,outer sep=0em,label distance=0.4em]-10:\scriptsize{2'}}}};

  \matrix[matrix of nodes, row sep=1.5em, matrix anchor=a1.base east] (t1) {
    |[nop,eq1] (a1)| \underline{a} \\
    |[p1,eq2]  (b1)| b \\
  };

  \matrix[matrix of nodes, row sep=1.5em, column sep=2em, right=6em of a1.base east, matrix anchor=a2.base west] (t2) {
    |[nop,eq1] (a2)| \underline{a} & |[p1,eq2]  (d2)| d \\
    |[p1,eq2]  (b2)| b & |[nop,eq1] (c2)| c \\
  };


  \path[->]
    (a1) edge[bend right] (b1) (b1) edge[bend right] (a1)
    (a2) edge (b2) (b2) edge (c2) (c2) edge (d2) (d2) edge (a2);

  \node[font=\Large] at ($(t1.east)!.5!(t2.west)$) { $\bisimto_{\Logic{ML}}$ };

  \node[below=0.5em of t1] (label1) { $\str{A}, \underline{a}$ };
  \node[at=(label1.base -| t2.center), anchor=base] (label2) { $\str{B}, \underline{a}$ };
\end{tikzpicture}

  \end{center}
\end{frame}

\begin{frame}[label=current]{$\ML$-bisimulation counterexample}
  \begin{center}
    \begin{tikzpicture}
  \tikzset{el/.style={minimum height=3ex,minimum width=3ex,inner sep=0pt,outer sep=0pt,anchor=center}};
  \tikzset{nop/.style={el}};
  \tikzset{p1/.style={draw=black,rectangle,el}};
  \tikzset{eq1/.style={color=tolvibrantBlue,label distance=0em, label={[draw,circle,inner sep=0.1em,outer sep=0em,label distance=0.2em]45:\scriptsize{1}}}};
  \tikzset{eq2/.style={color=tolvibrantOrange,label={[draw,circle,inner sep=0.1em,outer sep=0em,label distance=0.4em]40:\scriptsize{2}}}};
  \tikzset{eq3/.style={color=tolvibrantTeal,label={[overlay,draw,circle,inner sep=0.1em,outer sep=0em,label distance=0em]60:\scriptsize{3}}}};
  \tikzset{eq1p/.style={color=tolvibrantBlue,label distance=0em, label={[draw,circle,inner sep=0.1em,outer sep=0em,label distance=0em]45:\scriptsize{1'}}}};
  \tikzset{eq2p/.style={color=tolvibrantOrange,label={[draw,circle,inner sep=0.1em,outer sep=0em,label distance=0.4em]-10:\scriptsize{2'}}}};
  \tikzset{eq1/.style={color=tolvibrantBlue,label distance=0em, label={[draw,circle,inner sep=0.1em,outer sep=0em,label distance=0.2em]45:\scriptsize{1}}}};
  \tikzset{eq2/.style={color=tolvibrantOrange,label={[draw,circle,inner sep=0.1em,outer sep=0em,label distance=0em]-40:\scriptsize{2}}}};
  \matrix[matrix of nodes, row sep=1.5em, column sep=1.5em, yshift=0.5em, matrix anchor=a3.base west] (t3) {
    |[nop,eq1] (a3)| \underline{a}  &
    |[p1]  (b3)| b  &
    |[nop,eq2] (c3)| c  \\
  };

  \matrix[matrix of nodes, row sep=1.5em, column sep=1.5em, at={($(a3.base east -| c3.west)+(8em,0)$)}, matrix anchor=a4.base west] (t4) {
    |[nop,eq1p] (a4)| \underline{a}  &
    |[p1]  (b4)| b  &
    |[nop,eq3] (c4)| c  \\
    &
    |[nop,eq2p] (d4)| d &
    &
    \\
  };

  \path[->]
    (a3) edge (b3) (b3) edge (c3) (a3) edge[bend right] (c3)
    (a4) edge (b4) (b4) edge (c4) (a4) edge (d4) (d4) edge (c4)
  ;

  \node[font=\Large] (bisim2) at ($(t3.east)!.5!(t4.west)$) { $\nsim_{\Logic{ML}}$ };

  \node[at=(t3.center), yshift=-3em, anchor=base] (label3) { $\str{C}, \underline{a}$ };
  \node[at=(d4.center), yshift=-2em, anchor=base] (label4) { $\str{D}, \underline{a}$ };

  \begin{scope}[on background layer]
  %\tikzdbg
  \end{scope}
\end{tikzpicture}

  \end{center}
\end{frame}

\begin{frame}{Van Benthem for ML}

\end{frame}

\begin{frame}{Classical Proof}

\end{frame}

\begin{frame}{Tree Unraveling}

\end{frame}

\begin{frame}{Constructive Proof}

\end{frame}

\begin{frame}{Part 2 of 2}
  Now, adapt techniques to $\FGF$
\end{frame}

\begin{frame}{Bisimulation and van Benthem for FGF}

\end{frame}

\begin{frame}{Adapting the proof}

\end{frame}

\begin{frame}{Challenge 1: higher arity relations}

\end{frame}

\begin{frame}{HAH-unraveling}

\end{frame}

\begin{frame}{Problem with the HAH-unraveling}

\end{frame}

\begin{frame}{Game-based unraveling}

\end{frame}

\begin{frame}{Challenge 2: global modality and finiteness}

\end{frame}

\section{Conclusion}

\end{document}
