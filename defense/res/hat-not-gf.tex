\begin{tikzpicture}
  \tikzset{el/.style={circle,draw,fill=white,minimum width=2em,minimum height=2em}};
  \tikzset{relP/.style={tolhighcontrastBlue, line width=0.4em, -{Stealth[round]}, shorten >= 0.4em}};
  \tikzset{relE/.style={tolhighcontrastYellow, line width=0.25em, -{Stealth[round]}, shorten >= 0.2em}};
  \tikzset{relS/.style={tolhighcontrastRed, line width=0.15em, -{Stealth[round]}, shorten >= 0.2em}};
  \def\radiusP{0.7em};
  \def\radiusE{0.4em};
  \def\radiusS{0.2em};

  \matrix[matrix anchor=south west, matrix of nodes, every node/.style={el},name=n,column sep=2em, row sep=1.5em] at (0,0) {
    1 & 2 & 3 &[12em] a & b & c \\
     & &  &  & d &  \\
  };

  \begin{scope}[on background layer]
    \fill[relP] (n-1-1.north) circle [radius=\radiusP] coordinate (p1);
    \fill[relP] (n-1-2.north) circle [radius=\radiusP] coordinate (p2);
    \fill[relP] (n-1-3.north) circle [radius=\radiusP] coordinate (p3);
    \draw[relP] (p1) to[out=30,in=150] (p2) to[out=30,in=150] (p3);

    \fill[relP] (n-1-4.north) circle [radius=\radiusP] coordinate (p1);
    \fill[relP] (n-1-5.north) circle [radius=\radiusP] coordinate (p2);
    \fill[relP] (n-1-6.north) circle [radius=\radiusP] coordinate (p3);
    \draw[relP] (p1) to[out=30,in=150] (p2) to[out=30,in=150] (p3);

    \fill[relE] (n-1-2.east) circle [radius=\radiusE] coordinate (p1);
    \fill[relE] (n-1-3.west) circle [radius=\radiusE] coordinate (p2);
    \draw[relE] (p1) to (p2);

    \fill[relE] (n-1-5.east) circle [radius=\radiusE] coordinate (p1);
    \fill[relE] (n-1-6.west) circle [radius=\radiusE] coordinate (p2);
    \draw[relE] (p1) to (p2);

    \fill[relE] (n-1-5.south) circle [radius=\radiusE] coordinate (p1);
    \fill[relE] (n-2-5.north) circle [radius=\radiusE] coordinate (p2);
    \draw[relE] (p1) to (p2);
  \end{scope}

  \node[above right=1.5ex of n-1-3] {\huge{$\str{A}$}};
  \node[above left=1.5ex of n-1-4] {\huge{$\str{B}$}};
  \node at ($(n-1-3.east)!.5!(n-1-4.west)$) {\huge{$\bisimto_{\FGF}$}};

  \matrix[matrix anchor=north west, matrix of nodes, every node/.style={el},name=n,column sep=2em, row sep=2em]  at (0, -2em) {
    1 & 12 & 123 &[12em] a & ab & abc & b & bc \\
    3 & 2 & 23 & c & abd &  & bd & d \\
  };

  \begin{scope}[on background layer]
    \fill[relP] (n-1-1.north) circle [radius=\radiusP] coordinate (p1);
    \fill[relP] (n-1-2.north) circle [radius=\radiusP] coordinate (p2);
    \fill[relP] (n-1-3.north) circle [radius=\radiusP] coordinate (p3);
    \draw[relP] (p1) to[out=30,in=150] (p2) to[out=30,in=150] (p3);

    \fill[relP] (n-1-4.north) circle [radius=\radiusP] coordinate (p1);
    \fill[relP] (n-1-5.north) circle [radius=\radiusP] coordinate (p2);
    \fill[relP] (n-1-6.north) circle [radius=\radiusP] coordinate (p3);
    \draw[relP] (p1) to[out=30,in=150] (p2) to[out=30,in=150] (p3);

    \fill[relE] (n-1-2.east) circle [radius=\radiusE] coordinate (p1);
    \fill[relE] (n-1-3.west) circle [radius=\radiusE] coordinate (p2);
    \draw[relE] (p1) to (p2);

    \fill[relE] (n-2-2.east) circle [radius=\radiusE] coordinate (p1);
    \fill[relE] (n-2-3.west) circle [radius=\radiusE] coordinate (p2);
    \draw[relE] (p1) to (p2);

    \fill[relE] (n-1-5.east) circle [radius=\radiusE] coordinate (p1);
    \fill[relE] (n-1-6.west) circle [radius=\radiusE] coordinate (p2);
    \draw[relE] (p1) to (p2);

    \fill[relE] (n-1-5.south) circle [radius=\radiusE] coordinate (p1);
    \fill[relE] (n-2-5.north) circle [radius=\radiusE] coordinate (p2);
    \draw[relE] (p1) to (p2);

    \fill[relE] (n-1-7.east) circle [radius=\radiusE] coordinate (p1);
    \fill[relE] (n-1-8.west) circle [radius=\radiusE] coordinate (p2);
    \draw[relE] (p1) to (p2);

    \fill[relE] (n-1-7.south) circle [radius=\radiusE] coordinate (p1);
    \fill[relE] (n-2-7.north) circle [radius=\radiusE] coordinate (p2);
    \draw[relE] (p1) to (p2);
  \end{scope}

  \node[above right=3ex of n-1-3] {\huge{$\hatunravel{A}$}};
  \node[above=3ex of n-1-6] {\huge{$\hatunravel{B}$}};
  \node at ($(n-1-3.east)!.5!(n-1-4.west)$) {\huge{$\bisimto_{\FGF}$}};
\end{tikzpicture}
