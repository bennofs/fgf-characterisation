\begin{tikzpicture}
  % \node (headproperties) at (0em,-10em) { key structural properties of the unraveling };
  % \node [below=1em of headproperties] (headpropharmony) { 1. forward partial maps yield partial isomorphisms };
  % \node [below=1em of headpropharmony] (headpropintersect) { 2. intersections of guarded sets are infixes };


  \tikzset{faded edge/.style={dash pattern=on 1pt off 1pt on 1pt off 1pt on 2pt off 1pt on 2pt off 1pt on 100pt}};
  \tikzset{faded edge reverse/.style={tips=false,dash pattern=on 5pt off 1pt on 5pt off 1pt on 2pt off 1pt on 2pt off 1pt on 1pt off 1pt on 1pt off 1pt on 1pt off 1pt on 1pt off 1pt on 1pt off 100pt}};

  \tikzset{relP/.style={tolhighcontrastBlue, line width=0.25em, -{Stealth[round]}, shorten >= 0.4em, radius=0.4em}};
  \tikzset{relE/.style={tolhighcontrastYellow, line width=0.15em, -{Stealth[round]}, shorten >= 0.2em, radius=0.2em}};
  \tikzset{relS/.style={tolhighcontrastRed, line width=0.10em, -{Stealth[round]}, shorten >= 0.2em, radius=0.2em}};


  \node (prop1) { 1. If $(\elemtuples, \elemtuplet)$ is forward partial map, then $\homop: s_{i} \mapsto t_{i}$ is a partial isomorphism };

  \matrix[
  name=s,
  ampersand replacement=§,
  column sep=1em,
  matrix of nodes,
  below=1em of prop1.south west,
  matrix anchor=north west,
  nodes={
    draw, circle, minimum height=2.3ex, font=\small, inner sep=0.2ex},
  ] {
    {1} § {2} § {3} § {4} \\
  };
  \graph[use existing nodes, edges={-{Stealth[round]}}] { (s-1-1) -> (s-1-2) -> (s-1-3) -> (s-1-4) };

\end{tikzpicture}
