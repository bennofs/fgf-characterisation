\begin{tikzpicture}
  \tikzset{el/.style={rectangle,rounded corners=10pt,draw,fill=white,minimum width=2em,minimum height=2em,font=\small}};
  \tikzset{relP/.style={tolhighcontrastBlue, line width=0.4em, -{Stealth[round]}, shorten >= 0.4em}};
  \tikzset{relE/.style={tolhighcontrastYellow, line width=0.25em, -{Stealth[round]}, shorten >= 0.2em}};
  \tikzset{relS/.style={tolhighcontrastRed, line width=0.15em, -{Stealth[round]}, shorten >= 0.2em}};
  \tikzset{x2/.style={onslide=<4>{fill=tolhighcontrastYellow, text=white}}};
  \tikzset{x4/.style={onslide=<4>{fill=tolhighcontrastYellow, text=white}}};
  \def\radiusP{0.7em};
  \def\radiusE{0.4em};
  \def\radiusS{0.2em};

  \matrix[matrix of nodes, every node/.style={el},name=n,column sep=2em, row sep=2em, ampersand replacement=§,
  row 3/.style={onslide=<1>hidden},
  row 2/.style={onslide=<1>hidden},
  ]  {
    1 § |[x2]| 12 § |[x4]| 123 §[8em] a § |[x4]| ab                          § abc § |[uncover=<2->]| d  \\
    3 § 2         § 23         § c      § |[onslide=<1>{opacity=1}, x4]| abd § b   § bc \\
      §           §            §        §                                    § bd  §    \\
  };

  \begin{scope}[on background layer]
    \fill[relP] (n-1-1.north) circle [radius=\radiusP] coordinate (p1);
    \fill[relP] (n-1-2.north) circle [radius=\radiusP] coordinate (p2);
    \fill[relP] (n-1-3.north) circle [radius=\radiusP] coordinate (p3);
    \draw[relP] (p1) to[out=30,in=150] (p2) to[out=30,in=150] (p3);

    \fill[relP] (n-1-4.north) circle [radius=\radiusP] coordinate (p1);
    \fill[relP] (n-1-5.north) circle [radius=\radiusP] coordinate (p2);
    \fill[relP] (n-1-6.north) circle [radius=\radiusP] coordinate (p3);
    \draw[relP] (p1) to[out=30,in=150] (p2) to[out=30,in=150] (p3);

    \fill[relE] (n-1-2.east) circle [radius=\radiusE] coordinate (p1);
    \fill[relE] (n-1-3.west) circle [radius=\radiusE] coordinate (p2);
    \draw[relE] (p1) to (p2);

    \fill[relE] (n-1-5.east) circle [radius=\radiusE] coordinate (p1);
    \fill[relE] (n-1-6.west) circle [radius=\radiusE] coordinate (p2);
    \draw[relE] (p1) to (p2);

    \fill[relE] (n-1-5.south) circle [radius=\radiusE] coordinate (p1);
    \fill[relE] (n-2-5.north) circle [radius=\radiusE] coordinate (p2);
    \draw[relE] (p1) to (p2);

    \begin{scope}[every path/.style={onslide=<1>{hidden}}]
      \fill[relE] (n-2-2.east) circle [radius=\radiusE] coordinate (p1);
      \fill[relE] (n-2-3.west) circle [radius=\radiusE] coordinate (p2);
      \draw[relE] (p1) to (p2);

      \fill[relE] (n-2-6.east) circle [radius=\radiusE] coordinate (p1);
      \fill[relE] (n-2-7.west) circle [radius=\radiusE] coordinate (p2);
      \draw[relE] (p1) to (p2);

      \fill[relE] (n-2-6.south) circle [radius=\radiusE] coordinate (p1);
      \fill[relE] (n-3-6.north) circle [radius=\radiusE] coordinate (p2);
      \draw[relE] (p1) to (p2);
    \end{scope}
  \end{scope}

  \begin{scope}[onslide=<2->{hidden}]
    \node[above=3ex of n-1-2,font=\huge] {$\str{A}$};
    \node[above=3ex of n-1-6,font=\huge] {$\str{B}$};
  \end{scope}
  \begin{scope}[onslide=<1>{hidden}]
    \node[above=3ex of n-1-2,font=\huge] {$\hatunravel{A}$};
    \node[above=3ex of n-1-6,font=\huge] {$\hatunravel{B}$};
  \end{scope}
  \node (fgfbisim) at ($(n-1-3.east)!.5!(n-1-4.west)$) {\huge{$\bisimto_{\FGF}$}};
  \node[below=3em of fgfbisim.west, anchor=west, alert, uncover=<3->] { \huge{$\nsim_{\GF}$} };

  \def\rP{\textcolor{tolhighcontrastBlue}{\mathbf{P}}};
  \def\rE{\textcolor{tolhighcontrastYellow}{\mathbf{E}}};
  \node[below=3em of n-2-3, font=\huge, text width=20em, uncover=<3->] {
    $\begin{aligned}
      \exists{x_{2} x_{3}}\; [&\rE(x_{2}, x_{3})\.\land \\
                              &\forall{x_{4}} (\rE(x_{2}, x_{4}) \to \exists{x_{1}}(\rP(x_{1}, x_{2}, x_{4})))]
    \end{aligned}$
  };

  \begin{scope}[uncover=<4>, font=\Large]
    \node[anchor=135] at (n-1-1.-80) { $x_{1}$ };
    \node[anchor=135] at (n-1-2.-80) { $x_{2}$ };
    \node[anchor=135] at (n-1-3.-80) { $x_{4}$ };

    \node[anchor=135] at (n-1-4.-80) { $x_{1}$ };
    \node[anchor=135] at (n-1-5.-60) { $x_{2}$ };
    \node[anchor=135] at (n-2-5.-60) { $x_{4}$ };
  \end{scope}
\end{tikzpicture}
