\def\exunravelstruct#1{
\begin{tikzpicture}[baseline]
  \tikzset{reledges/.style={}};
  \tikzset{gaifman/.style={hidden}};
  \tikzset{#1}
  \tikzset{el/.style={circle,draw,fill=white,minimum width=2em,minimum height=2em}};
  \tikzset{relP/.style={tolhighcontrastBlue, line width=0.4em, -{Stealth[round]}, shorten >= 0.4em}};
  \tikzset{relE/.style={tolhighcontrastYellow, line width=0.25em, -{Stealth[round]}, shorten >= 0.2em}};
  \tikzset{relS/.style={tolhighcontrastRed, line width=0.15em, -{Stealth[round]}, shorten >= 0.2em}};
  \tikzset{adj/.style={densely dashed,thick,draw}};
  \def\radiusP{0.7em};
  \def\radiusE{0.4em};
  \def\radiusS{0.2em};

  \matrix[matrix of nodes, every node/.style={el}, column sep=4em, name=g, row sep=2em] {
    |(n1)| 1 \\ |(n2)| 2 \\ |(n3)| 3 \\ |(n4)| 4  \\
  };

  \def\relYellow{\textcolor{tolhighcontrastYellow}{E}};
  \def\relBlue{\textcolor{tolhighcontrastBlue}{P}};
  \def\relRed{\textcolor{tolhighcontrastRed}{S}};
  % \matrix[matrix of nodes, right=1.5em of g.east, yshift=2ex, matrix anchor=west, every node/.style={anchor=west}] {
  %   {$\relYellow^{\str{A}} = \{ (1,3), (3,4) \}$} \\
  %   {$\relRed^{\str{A}} = \{ (1,2), (2,3) \}$} \\
  %   {$\relBlue^{\str{A}} = \{ (1,2,3), (4,3,2) \}$} \\
  % };

  \begin{scope}[on background layer, every path/.style={reledges}]
    % 1 2 3
    \fill[relP] (n1.west) circle[radius=\radiusP] coordinate (p1);
    \fill[relP] (n2.west) circle[radius=\radiusP] coordinate (p2);
    \fill[relP] (n3.west) circle[radius=\radiusP] coordinate (p3);
    \draw[relP] (p1) to[out=-150,in=150] (p2) to[out=-150,in=150] (p3);

    % 4 3 2
    \fill[relP] (n4.east) circle[radius=\radiusP] coordinate (p1);
    \fill[relP] (n3.east) circle[radius=\radiusP] coordinate (p2);
    \fill[relP] (n2.east) circle[radius=\radiusP] coordinate (p3);
    \draw[relP] (p1) to[in=-30,out=30] (p2) to[in=-30,out=30] (p3);

    % 3 4
    \fill[relE] ($(n3.south) + (0.2em,0.0em)$) circle[radius=\radiusE] coordinate (p1);
    \fill[relE] (n4.north) circle[radius=\radiusE] coordinate (p2);
    \draw[relE] (p1) to (p2);

    % 1 3
    \fill[relE] (n1.120) circle[radius=\radiusE] coordinate (p1);
    \fill[relE] (n3.-120) circle[radius=\radiusE] coordinate (p2);
    \draw[relE,overlay] (p1) to[out=140,in=-140, looseness=1.2] (p2);
    \path (n2.north) +(0,3em) coordinate (x);

    % 1 2
    \fill[relS] (n1.south) circle[radius=\radiusS] coordinate (p1);
    \fill[relS] (n2.north) circle[radius=\radiusS] coordinate (p2);
    \draw[relS] (p1) to (p2);

    \fill[relS] (n2.south) circle[radius=\radiusS] coordinate (p1);
    \fill[relS] (n3.north) circle[radius=\radiusS] coordinate (p2);
    \draw[relS] (p1) to (p2);
  \end{scope}

  \graph[use existing nodes, edges={adj, gaifman}] {
    (n1) -> (n2),
    (n1) ->[bend left] (n3),
    (n2) -> (n3),
    (n3) -> (n4),
    (n4) ->[bend left] (n3),
    (n3) ->[bend left] (n2)
  };
\end{tikzpicture}
}
