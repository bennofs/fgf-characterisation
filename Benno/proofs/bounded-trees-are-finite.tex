\ifmainpart
\begin{restatable}{lemma}{boundedTreesAreFinite}\label{lem:bounded-trees-are-finite}
  If $\unravel{A}$ is the unraveling of a finite structure $\str{A}$, then $\mathcal{T}_{r,\ell}(\unravel{A})$ is finite for $\ell \in \N$ and $r \in \unraveldom{A}$.
\end{restatable}
\else
  \section{Lemma~\ref{lem:bounded-trees-are-finite}}
  \boundedTreesAreFinite*
\fi
\ifmainpart
\begin{proofsketch}
  For a finite structure, there are only a finite number of live tuples and thus for any fixed $k$, there are only a finite number of $k$-biseqs in this structure.
  Hence, the number of elements where the level of $\seq{e}$ is at most $2 * \ell$ is also finite, so $\mathcal{T}_{r,\ell}(\str{A})$ is finite.
\end{proofsketch}
\else
\begin{proof}
If $A$ is finite, then there are only a finite number of live tuples in $A$.
Recall that an $\ell$-biseq is just a word in $A^*{(\N\N{}A^{*})}^{\ell}$, consisting of live tuples from $A^{*}$ and indices into those tuples from $\N\N$.
Let $W$ be the maximum size of a live tuple.
Then an index into a live tuple must be in the range $[1,W]$, which is finite, so the set of possible indices is finite.
Thus there are only a finite number of $\ell$-biseqs, since both the set of live tuples and the set of indices is finite.
If $e \in \mathcal{T}_{r,\ell}(\str{A})$ with $\seq{e} = \sigma$, then the level of $\sigma$ is at most $2 * \ell$.
Thus, there are only a finite number of choices for $\sigma$.
For every $\sigma$, there are only a finite number of elements $e \in \unraveldom{A}$ with $\seq{e} = \sigma$.
It follows that there are only finitely many $e$ with $e \in \mathcal{T}_{r,\ell}(\str{A})$, so $\mathcal{T}_{r,\ell}(\str{A})$ is finite.
\end{proof}
\fi
