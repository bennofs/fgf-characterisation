\ifmainpart
\begin{restatable}{lemma}{companionTreeLike}\label{lem:companion-tree-like}
  Let $D_{a, \ell}(\unravel{A}_{\ell})$ be the set of elements $b$ for which there exists a $\relNext_{\ell}$-chain of length at most $\ell$ starting from $a$ and ending in $b$.
  For any element $a \in \unraveldom{A}_{\ell}$, the set $D_{a, \ell}$ is a tree rooted at $a$.
\end{restatable}
% \else
%   \section{Lemma~\ref{lem:companion-tree-like}}
%   \companionTreeLike*
\fi
\ifmainpart
% \begin{proofsketch}
%   Towards a contradiction, suppose that $D_{a,\ell}$ is not a tree.
%   Since all elements in $D_{a,\ell}$ are descendants of $a$, the only case in which $D_{a,\ell}$ is not a tree is if some element $e \in D_{a,\ell}$ has at least two parents $p_{1}, p_{2} \in D_{a,\ell}$.
%   In this case, $p_{1}$ and $p_{2}$ must be leaves of different subtrees $\mathcal{T}_{r_{1},\ell}$ and $\mathcal{T}_{r_{2},\ell}$ of the structure $\unravel{A}_{\ell}$ and both have sequences with level $2 * \ell$.
%   Now, at least one of $p_{1}$ or $p_{2}$ must be in a different subtree than $a$.
%   Reaching the root of that subtree from $a$ requires traversing at least one $\relNext$-edge.
%   Then, to reach the leaf $p_{1}$ or $p_{2}$ from the root requires an additional next-chain of length at least $\ell$ by \cref{lem:bounded-trees-shortest-next-path}.
%   In total, this is a next chain of length at least $\ell + 1$ and hence the element does not belong to $D_{a,\ell}$, a~contradiction.
% \end{proofsketch}
% \else
\begin{proof}
  Towards a contradiction, suppose that $D_{a,\ell}$ is not a tree.
  Since all elements in $D_{a,\ell}$ are descendants of $a$, the only case in which $D_{a,\ell}$ is not a tree is if some element $e \in D_{a,\ell}$ has at least two parents $p_{1}, p_{2} \in D_{a,\ell}$.
  Recall that the structure $\unravel{A}_{\ell}$ consists of subtrees $\mathcal{T}_{r,\ell}$.
  By construction, the only elements which have more than one parent in a finite $\ell$-unraveling are roots of such subtrees.
  Additionally, the parents of roots are leaves of other subtrees and have sequences with level $2 * \ell$.
  Hence, if $e$ has two different parents, we know that $e$ is a root of a subtree and the parents $p_{1}$ and $p_{2}$ are leaves of subtrees $\mathcal{T}_{r_{1},\ell}$ and $\mathcal{T}_{r_{2},\ell}$ of the structure $\unravel{A}_{\ell}$.
  Furthermore, both of those leaves $p_{1}$ and $p_{2}$ have sequences with level $2 * \ell$.

  The parents are leaves of \emph{different} subtrees since no two leaves of any subtree link to the same root.
  At least one of the elements $p_{1}$ and $p_{2}$ must therefore be in a different subtree than $a$.
  By symmetry, let us assume that $p_{1}$ is not in the same subtree as $a$.
  Reaching the root of that subtree from $a$ requires traversing at least one $\relNext_{\ell}$-edge.
  Then, to reach the leaf $p_{1}$ from the root requires an additional $\relNext_{\ell}$-chain of length at least $\ell$ by \cref{lem:bounded-trees-shortest-next-path}.
  In total, this is a $\relNext_{\ell}$-chain of length at least $\ell + 1$ and hence the element does not belong to $D_{a,\ell}$, a~contradiction.
\end{proof}
\fi
