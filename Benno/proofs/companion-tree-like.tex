\ifmainpart
\begin{restatable}{lemma}{companionTreeLike}\label{lem:companion-tree-like}
  Let $D_{a, \ell}(\unravel{A}_{\ell})$ be the set of elements $b$ for which there exists a next-chain starting from $a$ and ending in $b$ of length at most $\ell$.
  For any element $a \in \unraveldom{A}_{\ell}$, the set $D_{a, \ell}$ is a tree with root $a$.
\end{restatable}
\else
  \section{\cref{lem:companion-tree-like}}
  \companionTreeLike*
\fi
\ifmainpart
\begin{proofsketch}
  Towards a contradiction, suppose that $D_{a,\ell}$ is not a tree. Since all elements in $D_{a,\ell}$ are descendants of $a$, the only case in which $D_{a,\ell}$ is not a tree is if some element $e \in D_{a,\ell}$ has at least two parents $p_{1}, p_{2} \in D_{a,\ell}$.
  In this case, $p_{1}$ and $p_{2}$ must be leaves of different subtrees $\mathcal{T}_{r_{1},\ell}$ and $\mathcal{T}_{r_{2},\ell}$ of the structure $\unravel{A}_{\ell}$ and both have sequences with level $2 * \ell$.
  Now, at least one of $p_{1}$ or $p_{2}$ must be in a different subtree than $a$.
  Reaching the root of that subtree from $a$ requires traversing at least one $\relNext$-edge.
  Then, to reach the leaf $p_{1}$ or $p_{2}$ from the root requires an additional next-chain of length at least $\ell$ by \cref{lem:bounded-trees-shortest-next-path}.
  In total, this is a next chain of length at least $\ell + 1$ and hence the element does not belong to $D_{a,\ell}$, a~contradiction.
\end{proofsketch}
\else
\begin{proof}
  Assume by contradiction that $D_{a,\ell}$ is not a tree.
  Clearly, all elements in $D_{a,\ell}$ are descendants of $a$.
  Thus, if $D_{a,\ell}$ is not a tree, there must be an element $e \in D_{a,\ell}$ which has at least two parents $p_{1}, p_{2} \in D_{a,\ell}$.
  Recall that the structure $\str{A}$ is consists of subtrees $\mathcal{T}_{r,\ell}$.
  For $e$ to have two different parents, those parents must be leaves of different subtrees $\mathcal{T}_{r_{1},\ell}$ and $\mathcal{T}_{r_{2}, \ell}$ for different roots $r_{1}, r_{2} \in \unraveldom{A}$.
  At least one of the elements $p_{1}$ and $p_{2}$ must therefore be in a different subtree than $a$.
  By symmetry, let us assume that $p_{1}$ is not in the same subtree as $a$.
  Then the shortest next-chain from $a$ to $p_{1}$ must traverse through the whole subtree from $r_{1}$ to the leaf $p_{1}$.
  With every transition along the a $\relNext_{\ell}$-edge, the level of the sequence of an element increases by at most 1.
  Hence, the next-chain from $r_{1}$ to $p_{1}$ has a length of at least $\ell$.
  Since the shortest next-chain from $a$ to $p_{1}$ must go through $r_{1}$ (as $a$ is not in the same subtree as $p_{1}$) and reaching $r_{1}$ from $a$ also requires at least one next transition, the shortest next-chain from $a$ to $p_{1}$ is longer than $\ell$.
  Hence, $p_{1} \notin D_{a,\ell}$, a contradiction.
\end{proof}
\fi
