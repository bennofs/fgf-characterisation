\ifmainpart
\begin{restatable}{lemma}{lemNextIsConsecutive}\label{lem:next-is-consecutive}
For elements $s, t \in \unraveldom{A}$ with $\ctr{t} \ge 2$ and $(s,t) \in \relNext$, if $\elemtuplea \sqin \unraveldom{A}$ is the tuple such that $\seq{t} = \cdots \elemtuplea$, then $\pi(s) = a_{k-1}$ and $\pi(t) = a_{k}$ for $k = \ctr{t}$.
\end{restatable}
% \else
%   \section{Lemma~\ref{lem:next-is-consecutive}}
%   \lemNextIsConsecutive*
\fi
\ifmainpart
% \begin{proofsketch}
%   Let $\seq{s} = \cdots \elemtupleb$.
%   Since $(s,t) \in \relNext$, the biseq $\seq{t}$ extends $\seq{s}$.
%   Thus, $\elemtupleb$ and $\elemtuplea$ must an share an infix (by definition of biseqs).
%   The definition of $\relNext$ restricts the counters $\ctr{s}$ and $\ctr{t}$ such that the equality $b_{\ctr{s}} = a_{\ctr{t}-1}$ holds, implying the property of the lemma.
% \end{proofsketch}
% \else
\begin{proof}
  Let $\seq{t} = \cdots{}\elemtuplea$ for some element $t$ and a tuple $\elemtuplea \sqin \unraveldom{A}$.
  Note that $\pi(t) = a_{\ctr{t}} = a_{k}$ by definition of the projection $\pi$.
  We prove that $\pi(s) = a_{k-1}$ for an element $s$ with $(s,t) \in \relNext$ by case analysis on the two cases of $\relNext$.
  The~\ref{next:addctr} case is simple: in this case $\seq{s} = \seq{t}$ and $\ctr{s} = k - 1$, so the property follows directly from the definition of $\pi$.
  For the~\ref{next:addseq} case, let $\seq{t} = \seq{s} (i,j) \elemtuplea$ and $\seq{s} = \cdots \elemtupleb$.
  Further, in this case, $k = (j - i + 1) + 1$ and $\ctr{s} = j$.
  By the fact that $\seq{t}$ is a biseq, we know that $\elemtupleafromto{1}{j-i+1} = \elemtuplebfromto{i}{j}$.
  In particular, it follows that $\pi(s) = b_{j} = a_{j-i+1} = a_{k-1}$, concluding the proof.
\end{proof}
\fi
