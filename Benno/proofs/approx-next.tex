\ifmainpart
\begin{restatable}{lemma}{lemApproxNext}\label{lem:approx-next}
  Let $\str{A}_{\ell}$ and $\str{B}_{\ell}$ be the finite unravelings of structures $\str{A}$ and $\str{B}$ for some parameter $\ell \in \N$.
  For elements $a \in \unraveldom{A}_{\ell}$ and $b \in \unraveldom{B}_{\ell}$, if $a \approx_{z} b$ for some $z \in [1,\ell]$, then:
  \begin{description}
    \item[\desclabel{(succ)}{elemeq:succ}] If $(a, c) \in \relNext_{\ell}$ for $c \in \unraveldom{A}_{\ell}$, then there is $d \in \unraveldom{B}_{\ell}$ with $(b, d) \in \relNext_{\ell}$ and $c \approx_{z-1} d$.
    \item[\desclabel{(pred)}{elemeq:pred}] If $(c, a) \in \relNext_{\ell}$ for $c \in \unraveldom{A}_{\ell}$, then there is $d \in \unraveldom{B}_{\ell}$ with $(d, b) \in \relNext_{\ell}$ and $c \approx_{z-1} d$.
  \end{description}
\end{restatable}
\else
  \section{Lemma~\ref{lem:approx-next}}
  \lemApproxNext*
\fi
% \ifmainpart
% \begin{proofsketch}
%   Since $(a,c) \in \relNext_{\ell}$ or $(c,a) \in \relNext_{\ell}$, we have three cases:
%   \begin{romanenumerate}
%     \item only counter changed: $\seq{a} = \seq{c}$ and $\ctr{c} = \ctr{a} \pm 1$,
%     \item sequence extended: $\seq{c} = \mathsf{trunc}_{\ell}[\seq{a} (i,j) \elemtuples]$ and $\ctr{c} = j-i+1$, or
%     \item sequence reduced: $\seq{c}$ is such that $\seq{a} = \mathsf{trunc}_{\ell}[\seq{c} (i,j) \elemtuples]$.
%   \end{romanenumerate}
%   where
%   \begin{displaymath}
%     \mathsf{trunc}_{\ell} =
%     \begin{cases}
%       \sigma & \text{if $\sigma$ has level at most $2 * \ell$} \\
%       \text{level-$\ell$ suffix of $\sigma$} & \text{if $\sigma$ has level greater than $2 * \ell$.}
%     \end{cases}
%   \end{displaymath}
%   Note that case (\romannumeral2) is possible only for~\ref{elemeq:succ} and (\romannumeral3) only for~\ref{elemeq:pred}.

%   In case (\romannumeral1), $a$ and $c$ have the same history.
%   We can find an element $d$ with $(b,d) \in \relNext_{\ell}$ (or $(d,b) \in \relNext_{\ell}$) by incrementing (or decrementing) the counter of $b$ accordingly, keeping the history the same, thus $c \approx_{z-1} d$ (in fact, even $c \approx_{z} d$).
%   It can be seen that $\hist{1}{a} = \hist{1}{b}$ and $c \in \unraveldom{A}_{\ell}$ implies $d \in \unraveldom{B}_{\ell}$.

%   In case (\romannumeral2), we employ the fact that the last tuples of $\seq{a}$ and $\seq{b}$ have the same $\FGF_{z}$-type.
%   Thus, we can find a tuple $\elemtuplet$ such that $\seq{b} (i,j) \elemtuplet$ is a biseq and $\elemtuplet$ has the same $\FGF_{z{-}1}$-type as $\elemtuples$.
%   Next, we construct the element $d$ with $\seq{d} = \mathsf{trunc}_{\ell}[\seq{b}(i,j)\elemtuplet]$ and $\ctr{d} = (j-i+1) + 1$.
%   Cleary, $d \in \unraveldom{B}_{\ell}$ and $(b,d) \in \relNext_{\ell}$.
%   The function $\mathsf{trunc}_{\ell}$ preserves suffixes with level $z$ since $z \le \ell$, so it does not affect $z$-histories.
%   By our choice of $\elemtuplet$, we have $\hist{z-1}{c} = \hist{z-1}{d}$.
%   It follows that $c \approx_{z-1} d$, as wanted.

%   In case (\romannumeral3), by equivalence of $z$-histories, we know that $\seq{b} = \mathsf{trunc}_{\ell}[\sigma \cdots(i,j)\elemtuplet]$ for some tuple $\elemtuplet$ and a bismimulation sequence $\sigma$ with level at most $2 * \ell$.
%   Further, by equivalence of counters, $\ctr{b} = \ctr{a} = (j-i+1) + 1$.
%   By definition of the unraveling, there is an element $d \in \unraveldom{B}_{\ell}$ with $(d,b) \in \relNext_{\ell}$, which has $\seq{d} = \sigma$.
%   This element satisfies $c \approx_{z-1} d$.
% \end{proofsketch}
% \else
\begin{proof}
  We start with the simple case, where $\seq{a} = \seq{c}$ (Case \romannumeral1).
  We then consider the case of $\seq{a} \ne \seq{c}$ separately for~\ref{elemeq:succ} (Case \romannumeral2) and \ref{elemeq:pred} (Case \romannumeral3).

  \paragraph{Case \romannumeral1: $\seq{a} = \seq{c}$.}
  Let $a$ and $c$ be elements with $\seq{a} = \seq{c}$ and $(a,c) \in \relNext_{\ell}$ (or $(c,a) \in \relNext$ for the proof of~\ref{elemeq:pred}).
  By definition of $\relNext_{\ell}$, then $\ctr{c} = \ctr{a} + 1$ (or $\ctr{c} = \ctr{a} - 1$ for the proof of~\ref{elemeq:pred}).
  Construct the element $d = (\seq{b}, \ctr{b} + 1)$ (or $d = (\seq{b}, \ctr{b} - 1)$ for the proof of~\ref{elemeq:pred}).
  Note that by construction, $\ctr{c} = \ctr{d}$ and $\hist{z}{c} = \hist{z}{a} = \hist{z}{b} = \hist{z}{d}$.
  It follows that $c \approx_{z-1} d$ (in fact, even $c \approx_{z} d$).
  It remains to be shown that $d \in \unraveldom{B}_{\ell}$, \ie{} that $d$ is a bipoint and the level of $\seq{d}$ is less or equal to $2 * \ell$.
  The latter condition follows immediately from the fact that $b \in \unraveldom{B}_{\ell}$ and $\seq{d} = \seq{b}$.
  To show that $d$ is a bipoint, observe that $\seq{c}$ and $\seq{d}$ ``look similar'', since the $z$-histories of $c$ and $d$ are equal.
  This means that either $\seq{c} = \elemtuples$ and $\seq{d} = \elemtuplet$ or $\seq{c} = \cdots(i,j)\elemtuples$ and $\seq{d} = \cdots(i,j)\elemtuplet$, for tuples $\elemtuples \sqin A$ and $\elemtuplet \sqin B$ with $|\elemtuples| = |\elemtuplet|$ and a pair of indices $(i,j)$.
  Further, we have $\ctr{c} = \ctr{d}$.
  Inspecting the conditions from the definition of a bipoint, we see that in this case, $d$ is a bipoint exactly if $c$ is a bipoint.
  Therefore, as $c \in \unraveldom{A}_{\ell}$, it follows that $d$ is a bipoint.
  Hence $d \in \unraveldom{B}_{\ell}$.

  For the remaining two cases, we make use of the following auxilliary function which \emph{truncates} a biseq $\sigma$ to level at most $2 * \ell$, by only keeping a suffix of $\sigma$ if its level is too large.
  It is defined as follows:
  \begin{displaymath}
    \mathsf{trunc}_{\ell}[\sigma] =
    \begin{cases}
      \sigma & \text{if $\sigma$ has level at most $2 * \ell$,} \\
      \text{suffix of $\sigma$ with level $\ell$} & \text{if $\sigma$ has level greater than $2 * \ell$.}
    \end{cases}
  \end{displaymath}

  \paragraph{Case \romannumeral2: $\seq{a} \ne \seq{c}$, for~\ref{elemeq:succ}.}
  Let $a$ and $c$ be elements with $\seq{a} \ne \seq{c}$ and $(a,c) \in \relNext_{\ell}$.
  By definition of $\relNext_{\ell}$, this means that $\seq{c} = \mathsf{trunc}_{\ell}[\sigma]$ with $\sigma = \seq{a}(i,j)\elemtuples$ and $\ctr{c} = (j-i+1)+1$, for some tuple $\elemtuples \sqin A$ and indices $(i,j)$.
  We now employ the fact that the last tuples of $\seq{a}$ and $\seq{b}$ have the same $\FGF_{z}$-type.
  This is implied by the precondition $\hist{z}{a} = \hist{z}{b}$ of the lemma.
  Thus, we can find a tuple $\elemtuplet$ such that $\theta = \seq{b} (i,j) \elemtuplet$ is a biseq and $\elemtuplet$ has the same $\FGF_{z{-}1}$-type as $\elemtuples$.
  We claim that the bipoint $d = (\mathsf{trunc}_{\ell}[\theta], k)$ with $k = (j-i+1)+1$ satisfies the lemma.
  First, by construction $d \in \unraveldom{B}_{\ell}$ and $(b,d) \in \relNext_{\ell}$.
  Recall that $\ctr{c} = (j-i+1)+1$ and thus $\ctr{c} = \ctr{d}$.
  It remains to be shown that $d$ has the same $(z-1)$-history as $c$.
  Recall that $\seq{c} = \mathsf{trunc}_{\ell}[\sigma]$ and $\seq{d} = \mathsf{trunc}_{\ell}[\theta]$.
  Let us first consider the bipoints with untruncated biseqs, namely $c' = (\sigma, \ctr{c})$ and $d' = (\theta, \ctr{d})$.
  Note that these are not necessarily elements of $\str{A}_{\ell}$ or $\str{B}_{\ell}$, since their biseqs may have a level greater than $2 * \ell$.
  However, observe that since $\hist{z}{a} = \hist{z}{b}$ and we chose $\elemtuplet$ such that it has the same $\FGF_{z-1}$-type as $\elemtuplet$, we have that $\hist{z-1}{c'} = \hist{z-1}{d'}$.
  Furthermore, if $\sigma$ has some biseq $\rho$ of level $z-1$ as suffix, then $\mathsf{trunc}_{\ell}(\sigma)$ also has $\rho$ as suffix, since $z-1 \le \ell$ by the precondition $z \le \ell$ of the lemma.
  Since we have $\hist{z-1}{c'} = \hist{z-1}{d'}$, it follows that also $\hist{z-1}{c} = \hist{z-1}{d}$.
  Hence $c \approx_{z-1} d$ as wanted.

  \paragraph{Case \romannumeral3: $\seq{a} \ne \seq{c}$, for~\ref{elemeq:pred}.}
  Let $a$ and $c$ be elements with $\seq{a} \ne \seq{c}$ and $(c,a) \in \relNext_{\ell}$.
  By definition of $\relNext_{\ell}$, this means that $\seq{a} = \mathsf{trunc}_{\ell}[\sigma]$ with $\sigma = \seq{c}(i,j)\elemtuples$ and $\ctr{a} = (j-i+1)+1$, for some tuple $\elemtuples \sqin A$ and indices $(i,j)$.
  Let $b \in \unraveldom{B}_{\ell}$ be an element with $a \approx_{z} b$.
  Due to the equivalence of $z$-histories, it follows that $\seq{b} = \cdots (i,j) \elemtuplet$ for some indices $(i,j)$ and a tuple $\elemtuplet$.
  Importantly, this shows that the level of $\seq{b}$ is greater than 0.
  Recall that by definition of $\ell$-unravelings (\cref{def:finite-tree-unraveling}), all elements in an $\ell$-unraveling are reachable via a $\relNext_{\ell}$-chain from elements with a sequence level equal to 0.
  Since the level of $\seq{b}$ is greater than 0, the element $b$ must have at least one $\relNext_{\ell}$-parent in $\unraveldom{B}_{\ell}$.
  Let $d \in \unraveldom{B}_{\ell}$ be any $\relNext_{\ell}$-parent of $b$, \ie{} we have $(d,b) \in \relNext_{\ell}$.

  We claim that $c \approx_{z-1} d$.
  First, recall that $\ctr{a} = \ctr{b} = (j-i+1)+1$ and $\seq{b} = \cdots(i,j) \elemtuplet$.
  Remember that $(\seq{b}, j-i+1)$ is not a bipoint, since all bipoints with a sequence equal to $\seq{b}$ have a counter of at least $(j-i+1)+1$.
  Hence, if $(d,b) \in \relNext_{\ell}$, then $\seq{d} \ne \seq{b}$.

  At this point, we know that:
  \begin{enumerate}
    \item $(c,a) \in \relNext$ with $\seq{c} \ne \seq{a}$,
    \item $(d,b) \in \relNext$ with $\seq{d} \ne \seq{b}$,
    \item $\seq{a} = \mathsf{trunc}_{\ell}[\sigma]$ with $\sigma = \seq{c}(i,j)\elemtuples$, and
    \item $\seq{b} = \mathsf{trunc}_{\ell}[\theta]$ with $\theta = \seq{d}(i,j)\elemtuplet$.
  \end{enumerate}
  Items (1) and (2) imply $\ctr{c} = j = \ctr{d}$.
  Because of the precondition $a \approx_{z} b$ of the lemma, the elements $a$ and $b$ have equal $z$-histories.
  Since $\mathsf{trunc}_{\ell}$ preserves suffixes of level $z-1$ (see Case \romannumeral2), from items (3) and (4) above it follows that $c$ and $d$ have the same $(z-1)$-history.
  Therefore, $a \approx_{z-1} b$, as wanted.
\end{proof}
%\fi
