%!TEX root = ../main.tex
\chapter{Expressive Completeness}\label{chap:expressivity}
Expressive completeness of a fragment $\logicL$ of $\FO$ means that this fragment is complete with respect to some class of properties, \ie{} that for any property of this class, there is a formula in $\logicL$ that expresses this property.
Here, we focus on the special class of properties expressible as \emph{bisimulation-invariant} first-order formulae, which are formulae that do not distinguish between bisimilar structures.
More precisely, a first-order formula $\varphi$ is $\bisimto_{\logicL}$-invariant for a given notion of bisimulation $\bisimto_{\logicL}$ if whenever $\str{A}, \elemtuplea \bisimto_{\logicL} \str{B}, \elemtupleb$, then $\str{A}, \elemtuplea \models \varphi \Leftrightarrow \str{B}, \elemtupleb \models \varphi$.
Similarly, a formula $\varphi$ of $\FO$ is \emph{$\bisimto_{\logicL}$-invariant in the finite} if it does not distinguish between bisimilar structures which are finite (it may however distinguish between \emph{infinite} bisimilar structures, which makes this different from the general, infinite case).
We say that $\logicL$ is \emph{expressively complete} (in the finite) with respect to $\bisimto_{\logicL}$-invariant first-order properties if every $\bisimto_{\logicL}$-invariant formula is equivalent (in the finite) to some formula of $\logicL$.
In this chapter, we show that $\FGF$ is expressively complete with respect to $\bisimto_{\FGF}$-invariant first-order properties, both in the classical sense and in the finite.
We first present a short review of related preservation theorems in \cref{sec:related-work} and then prove a van Benthem theorem for $\FGF$.
We give the high-level structure of the proof, leaving concrete details to be filled in by later chapters.

We already saw in the previous chapter that every $\FGF$ formula is invariant under $\FGF$-bisimulation.
Combined with the result of this chapter, this means that the properties expressible as $\FGF$-formula are precisely those first-order properties which are $\bisimto_{\FGF}$-invariant.
In general, we say that a logic $\logicL$ is the \emph{$\bisimto_{\logicL}$-invariant fragment} of $\FO$ (in the finite) if all formulae of $\logicL$ are $\bisimto_{\logicL}$-invariant and every $\bisimto_{\logicL}$-invariant first-order formula is equivalent (in the finite) to some formula of $\logicL$.
This lets us formulate the central theorem of this thesis, which we prove at the end of this chapter.
\begin{restatable*}{maintheorem}{vanbenthemFGF}\label{thm:main}
The forward guarded fragment is the $\bisimto_{\FGF}$-invariant fragment of $\FO$, both in a sense of classical model theory and in the finite.
\end{restatable*}

\section{Related Work}\label{sec:related-work}
We focus on \emph{preservation theorems}, which characterize a logic $\logicL$ as a fragment of a richer logic $\logicL'$ as exactly the fragment that is able to express all properties preserved under some form of structural equivalence.
Other ways to measure the expressive power of logics, such as Lindström theorems which characterize $\logicL$ as being the maximally expressive logic that still has certain model theoretic properties~\cite{benthem2009} are out of scope for this work.

There is a large body of work extending van Benthem's classical result for basic modal logic~\cite{van1983modal} to other variants of modal logic.
Examples of this include:
\begin{itemize}
  \item Rosen's result shows that van Benthem's theorem also works in the finite~\cite{Rosen97}.
  \item Modal logics with multiple, global and inverse modalities are bisimulation invariant fragments of $\FO$ for notions of bisimulation, over both finite and infinite structures~\cite{Otto04}.
  \item Graded modal logic characterized as the counting bisimulation invariant fragment of $\FO$, over both finite and infinite relational structures \cite{derijke2000,otto2023}.
  \item $\GF$ is the $\GF$-bisimulation invariant fragment of $\FO$, both in the classical~\cite{AndrekaNB98} and finite setting~\cite{Otto2012}.
  \item Modal logic extended with fixed point operators ($\mu\ML$) is the bisimulation-invariant fragment of monadic second order logic ($\Logic{MSO}$) (Janin-Walukiwicz theorem)~\cite{janin1996}, and similar for the guarded fragment extended with fixed point operators~\cite{gradel2002}.
        However, whether this is true in the finite as well is an open question.
  \item Sabotage modal logic, which extends modal logic with an ``edge-deletion'' operator, is the sabotage-bisimulation invariant fragment of $\FO$~\cite{aucher2015}.
  \item Coalgebaric modal logic is, under some constraints, the behavioral-equivalence invariant fragment of Coalgebaric Predicate Logic\cite{litak2012}, both in the finite and the classical sense.
  \item Concept expressions of different description logics including \Logic{FL^{-}}, \Logic{FLU^-}, \Logic{ALC}, \Logic{ALCQIO}, \Logic{DL{-}Lite} and others are fragments of $\FO$ invariant under corresponding equivalence relations\cite{kurtonina1999, lutz2011, piro2013}.
  \item $\Logic{XPath}_{=}$, a logic able to express (parts of) the navigational aspect of the XPath XML query language, is a $\FO$ fragment invariant under a corresponding notion of bisimulation for different navigational axes (child, descendant)\cite{figueira2015}.
  \item \Logic{HornALC} is the Horn-simulation invariant fragment of $\FO$\cite{jung2019}.
  \item Weakly Aggregative Modal Logic, a polyadic modal logic, is the $wa$-bisimulation invariant fragment of $\FO$.
\end{itemize}

With exception of the guarded fragments and the last example, all these results are for logics where predicates have arity of at most two.
A possible reason for this may be that higher-arity relations are difficult to work with, as demonstrated by the proof of the van Benthem characterization of $\GF$\cite{Otto2012}.
Compared to an earlier result for $\GF$ with relations of arity at most 2\cite{Otto04}, the details of this proof are much more complex.
Luckily, since $\FGF$ is a subset of $\GF$ and any formula preserved under $\FGF$-bisimulation is also preserved under $\GF$-bisimulation, we can focus on characterizing $\FGF$ as a subset of $\GF$ here.
The full van Benthem theorem for $\FGF$ then follows from Otto's van Benthem theorem for $\GF$.

The proof technique in this thesis closely follows the approach taken by Otto to prove van Benthem theorems for variations of modal logic\cite{Otto04, otto2004a}.
The central part of this technique is the construction of structures called ``finite companions''.
Like Otto, we construct these companions by stopping the infinite unraveling at some depth and then relinking edges to roots, to preserve similarity with the original structure.
The method of unraveling that we employ is inspired by the construction that Bednarczyk used to show the $\ExpTime$-completeness of $\FGF$\cite{Bednarczyk21}.
Bednarczyk's construction however is not sufficient to construct the finite companions.
Our new method turns out to be similar to a method of unraveling already described by Andréka et al. when introducing the guarded fragment\cite{AndrekaNB98}.
We present a comparison to these two existing methods of unraveling in \cref{sec:existing-unravelings}.
In the next section, we introduce our approach to prove van Benthem for $\FGF$ following Otto in detail.

\section{Van Benthem for $\FGF$}\label{sec:van-benthem-theorem}
The high-level idea for the proof of the van Benthem theorem for $\FGF$ is to show that for any given $\bisimto_{\FGF}$ first-order formula $\varphi$, there is some number $\ell \in \N$ such that $\varphi$ is $\bisimto_{\FGF}^{\ell}$ invariant.
An observation by Otto~\cite[Obs.~13]{Otto04} then implies the above theorem.
In the following, we present this observation in detail for $\FGF$.
First, we can see that every $\bisimto_{\FGF}^{\ell}$ equivalence class is characterized by an $\FGF_{\ell}$ formula, as stated in the following lemma:
\begin{lemma}
  For any structure $(\str{A}, \elemtuplea)$ and $\ell \in \N$, there is a formula $\chi_{(\str{A}, \elemtuplea)} \in \FGF_{\ell}$ such that $(\str{A}, \elemtuplea) \models \chi_{(\str{A}, \elemtuplea)}$, but $(\str{B}, \elemtupleb) \not\models \chi_{\str{A}, \elemtuplea}$ for every structure $(\str{B}, \elemtupleb) \nsim^{\ell}_{\FGF} (\str{A}, \elemtuplea)$.
\end{lemma}
\begin{proof}
  Let $\Gamma$  be the set of all $\FGF_{\ell}$ formulae which are satisfied in $(\str{A}, \elemtuplea)$, so
  \begin{equation*}
    \Gamma = \{ \psi:\, \psi \in \FGF_{\ell} \text{ and } (\str{A}, \elemtuplea) \models \psi \}.
  \end{equation*}
  Since the number of logically distinct $\FGF_{\ell}$ formula is finite, there is a set $\Delta$ which is finite and logically equivalent to $\Gamma$.
  We show that $\chi_{(\str{A}, \elemtuplea)} = \bigwedge \Delta$ satisfies the above lemma.
  For any structure $(\str{B}, \elemtupleb)$ with $(\str{B}, \elemtupleb) \nsim_{\FGF}^{\ell} (\str{A}, \elemtuplea)$, by \cref{lem:FGF-bisimulations-work-well} there is a formula $\varphi$ in $\FGF_{\ell}$ such that $(\str{A}, \elemtuplea) \models \varphi$ but $(\str{B}, \elemtupleb) \not\models \varphi$.
  But $(\str{A}, \elemtuplea) \models \varphi$ means that $\varphi \in \Gamma$ and hence $\chi_{(\str{A},\elemtuplea)} \to \varphi$.
  Since $(\str{B}, \elemtupleb) \not\models \varphi$, we have $(\str{B}, \elemtupleb) \not\models \chi_{(\str{A}, \elemtuplea)}$ as required.
\end{proof}
Because the characteristic formula $\chi_{(\str{A}, \elemtuplea)}$ is an $\FGF_{\ell}$ formula, there are only a finite number of logically distinct characteristic formulae.
It follows that $\bisimto_{\FGF}^{\ell}$ has a finite number of equivalence classes.
For a $\bisimto_{\FGF}^{\ell}$-invariant formula $\varphi$, we can hence enumerate all equivalence classes of $\bisimto_{\FGF}^{\ell}$ in which $\varphi$ is satisfied, and take the disjunction of their characteristic formulae to obtain an equivalent $\FGF_{\ell}$ formula.
This leads to the following corollary:
\begin{corollary}\label{cor:ell-invariant-has-ell-formula}
  For any $\ell \in \N$, a $\bisimto_{\FGF}^{\ell}$-invariant $\FO$ formula $\varphi$ is equivalent to a formula in $\FGF_{\ell}$.
\end{corollary}

\noindent
Our proof for the van Benthem characterization of $\FGF$ relies on the van Benthem characterization of $\GF$ provided by Otto~\cite{Otto2012}.
For convenience, we restate this result here, in the form of the folowing theorem.
\begin{theorem}\label{thm:vanBenthem-for-GF}
  The guarded fragment is the $\bisimto_{\GF}$-invariant fragment of $\FO$, both in a sense of classical model theory and in the finite.
  More specifically, there is a computable function $\homof \colon \N \to \N$ such that every $\FO$-formula $\varphi$ of quantifier rank $q$ that is $\bisimto_{\GF}$-invariant (in the finite) is also $\bisimto^{\homof(q)}_{\GF}$-invariant (in the finite) and logically-equivalent (in the finite) to a $\GF$-formula of quantifier rank at most $\homof(q)$.
\end{theorem}
Additionally, we assume the following theorem, which is the main technical theorem whose proof we develop through \cref{chap:unraveling} and \cref{chap:finite}.
\begin{restatable*}{theorem}{maintechnicalthm}\label{thm:main-technical-thm}
  There exists a positive polynomial function~$\homop$ such that for every $\ell$ and every pair of $\homop(\ell)$-$\FGF$-bisimilar pointed structures, there exists $\FGF$-bisimilar companions $(\str{A}', \elemtuplea')$ and $(\str{B}', \elemtupleb')$ (that are finite if $\str{A}$ and $\str{B}$ are) of $(\str{A}, \elemtuplea)$ and $(\str{B}, \elemtupleb)$ such that $(\str{A}', \elemtuplea') \bisimto^{\ell}_{\GF} (\str{B}', \elemtupleb')$.
\end{restatable*}

\noindent
Finally, we show how these two theorems combined imply our main theorem, the van Benthem characterization of $\FGF$.
\ifmainpart
\vanbenthemFGF
\else
  \section{Main Theorem}
  \vanbenthemFGF*
\fi
\ifmainpart
\newsavebox{\diagupgrading}
\begin{lrbox}{\diagupgrading}{
  \begin{tikzpicture}
    \matrix[row sep=2em, column sep=3em]
    {
        \node (a) {$(\str{A}, \elemtuplea)$}; &
         \node[font=\large] (sim_ab) {$\bisimto_{\FGF}^{\homop(\mathit{g})}$}; &
        \node (b) {$(\str{B}, \elemtupleb)$}; &
        \node[anchor=west] {}; \\

        \node[font=\large] (sim_a_unravel) {$\bisimto_{\FGF}$}; &
        \node {}; &
        \node[font=\large] (sim_b_unravel) {$\bisimto_{\FGF}$}; &
        \node[anchor=west] {}; \\

        \node (a_unravel) {$(\str{A}', \elemtuplea')$}; &
        \node[font=\large] (sim_unravel) {$\bisimto_{\GF}^{\mathit{g}}$}; &
        \node (b_unravel) {$(\str{B}', \elemtupleb')$};
        \node[anchor=west] {}; \\
    };

    \draw[->] (a) -- (sim_ab) -> (b);
    \draw[dashed,->] (a) -- (sim_a_unravel) -> (a_unravel);
    \draw[->] (a_unravel) -- (sim_unravel) -> (b_unravel);
    \draw[dashed,->] (b_unravel) -- (sim_b_unravel) -> (b);
  \end{tikzpicture}
}
\end{lrbox}
\begin{proofsketch}
  The invariance of $\FGF$ under $\bisimto_{\FGF}$ was already shown in \cref{lem:FGF-bisimulations-work-well}, hence what is left to show is the expressive completeness.
  By \cref{cor:ell-invariant-has-ell-formula}, it suffices to show that for every first-order formula that is $\bisimto_{\FGF}$-invariant (in the finite) is also $\bisimto_{\FGF}^{\ell}$-invariant (in the finite) for some $\ell \in \N$.
  Like Otto, we employ an``upgrading'' to show this.
  For structures $\str{A}, \elemtuplea$ and $\str{B}, \elemtupleb$, the upgrading is shown in the following diagram:
  \begin{center}%
    \usebox{\diagupgrading}
  \end{center}
  where $\str{A'}, \elemtuplea'$, $\str{B'}, \elemtupleb'$ are the companions and $\homop$ is the polynomial function as given by \cref{thm:main-technical-thm}.
  For every $\bisimto_{\FGF}$-invariant (and hence also $\bisimto_{\GF}$-invariant) $\FO$-formula $\phi$, there is a $g$ such that $\phi$ is $\bisimto_{\GF}^{g}$ invariant (\cref{thm:vanBenthem-for-GF}).
  By the above diagram, this implies that $\phi$ is $\bisimto_{\FGF}^{\homop(g)}$ invariant.
  Since all the structures involved in the proof are finite if $\str{A}, \elemtuplea$ and $\str{B},  \elemtupleb$ are finite, this works for both classical model theory and in the finite, concluding the proof.
\end{proofsketch}
\else
\begin{proof}
  The invariance of $\FGF$ under $\bisimto_{\FGF}$ was already shown in \cref{lem:FGF-bisimulations-work-well}, hence what is left to show is the expressive completeness.
By \cref{cor:ell-invariant-has-ell-formula}, it suffices to show that for every first-order formula that is $\bisimto_{\FGF}$-invariant (in the finite) is also $\bisimto_{\FGF}^{\ell}$-invariant (in the finite) for some $\ell \in \N$.

Take any such first-order formula $\varphi$, and let $\homop$ be the function provided by~\cref{thm:main-technical-thm}.
As $\FGF$ is a fragment of $\GF$ (and hence $\bisimto_{\FGF}$-invariance implies $\bisimto_{\GF}$-invariance), we know that $\varphi$ is $\bisimto_{\GF}$-invariant (in the finite).
Thus $\varphi$ is also $\bisimto_{\GF}^{\mathit{g}}$-invariant (in the finite) for certain threshold $\mathit{g} \in \N$ provided by~\cref{thm:vanBenthem-for-GF}.
We aim to show that $\varphi$ is $\bisimto_{\FGF}^{\homop(\mathit{g})}$-invariant (in the finite).
To do so, let $(\str{A}, \elemtuplea)$ and $(\str{B}, \elemtupleb)$ be $\homop(\mathit{g})$-$\FGF$-bisimilar pointed structures, and suppose that $(\str{A}, \elemtuplea) \models \varphi$. It remains to show that $(\str{B}, \elemtupleb) \models \varphi$.
From~\cref{thm:main-technical-thm} we infer the existence of $\FGF$-bisimilar companions $(\str{A}', \elemtuplea')$ and $(\str{B}', \elemtupleb')$, that are finite if $\str{A}$ and $\str{B}$ are, for which $(\str{A}', \elemtuplea') \bisimto^{\mathit{g}}_{\GF} (\str{B}', \elemtupleb')$ holds.
The following diagram depicts what will happen next.
\begin{figure}[H]
  \centering
  \usebox{\diagupgrading}
  % \caption{Upgrading $\bisimto_{\FGF}^{\ell}$-bisimulation to $\bisimto_{\GF}^{\mathit{g}}$-bisimulation.}
\end{figure}

It is now clear that:
(i) $(\str{A}, \elemtuplea) \models \varphi$ (by assumption),
(ii) $(\str{A}', \elemtuplea') \models \varphi$ (by $\FGF$-bisimilarity and $\bisimto_{\FGF}$-invariance of $\varphi$),
(iii) $(\str{B}', \elemtupleb') \models \varphi$ (by $\bisimto_{\GF}^{\mathit{g}}$-invariance of $\varphi$),
(iv) $(\str{B}, \elemtupleb) \models \varphi$ (by $\FGF$-bisimilarity and $\bisimto_{\FGF}$-invariance of $\varphi$).
This concludes the proof.
\end{proof}
\fi

