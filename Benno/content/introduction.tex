%!TEX root = ../main.tex
\chapter{Introduction}\label{chap:introduction}

First-order logic ($\FO$)---also known as predicate logic---is undecidable in its full generality\cite[Sec. 1.1]{borger1997}.
For instance, there is no algorithm that for any first-order formula $\varphi$, decides whether it is valid (true in all models) or not.
For practical applications, this leaves two choices: either accept the undecidability and hope that it does not occur in ``real-world problems'' or use another, less expressive logic, limited in some way to become decidable.

One decidable fragment of first-order logic is the \emph{Forward Guarded Fragment}~($\FGF$), recently shown to have nice algorithmic properties\cite{Bednarczyk21}.
The forward guarded fragment derives from the popular \emph{Guarded Fragment} ($\GF$) by restricting the order of variables.
For example, consider the following list of $FO$ formulae:
\begin{enumerate}
  \item $\forall{x_{1}x_{2}x_{3}}\; ((\mathrm{dayAfter}(x_{1},x_{2}) \land \mathrm{dayAfter}(x_{2},x_{3})) \to \mathrm{dayAfter}(x_{1}, x_{3}))$ \\
        ``if $x_{3}$ is a day after $x_{2}$ and $x_{2}$ itself is a day after $x_{1}$, then $x_{3}$ is after $x_{1}$ as well (transitivity)'' \\
        \emph{not guarded}: no atom covers all the variables $x_{1}, x_{2}, x_{3}$
  \item $\forall{x_{1}}\; (\mathrm{sunny}(x_{1}) \to \exists{x_{2}} (\mathrm{dayAfter}(x_{2},x_{1}) \land \neg \mathrm{sunny}(x_{2})))$ \\
        ``for every sunny day, there is a day before which was not sunny'' \\
        \emph{not forward}: $x_{2}, x_{1}$ is in wrong order
  \item $\forall{x_{1}x_{2}x_{3}}\; (\mathrm{dayBetween}(x_{1},x_{2},x_{3}) \to ((\mathrm{mon}(x_{1}) \land \mathrm{fri}(x_{3})) \to \mathrm{workday}(x_{2})))$ \\
        ``days between monday and friday are workdays'' \\
        \emph{forward and guarded}
\end{enumerate}
The forward guarded fragment consists of formulae which are both forward and guarded, like the third formula in the above list.
Now, we ask the question: if we restrict first-order logic in this way, which properties can we express, and which not?
In this thesis, we give a precise answer to this question, in the form of a van Benthem characterization for $\FGF$.
Our characterization works both over the class of all models and in restriction to just finite models.
The later is a new result.

Van Benthem's theorem characterizes a logic using a notion of \emph{bisimulation}.
For a simple example of one notion of bisimulation, consider transition systems, which are structures consisting of a set of states with associated properties and a transition relation between the states.
Two transition systems are \emph{bisimilar} if there is a mapping (bisimulation) between states from one system to the other such that if state $a$ maps to state $b$, then:
\begin{enumerate}[(a)]
  \item $a$ and $b$ have the same properties, and
  \item if there is a transition from $a$ to $c$, then there is a corresponding transition to a state $d$ from $b$ and $c$ maps to $d$, and vice versa for a transition from $b$ to some $d$.
\end{enumerate}
This notion of bisimulation defines an equivalence on structures.
It is easy to see that certain kinds of formulae, in particular first-order translations of modal logic formulae, are preserved under this equivalence.
Here, ``preserved'' means that if there are two bisimilar models, the formula must either satisfy both of them or neither.
Now, the classical van Benthem theorem\cite{van1983modal} proves the much harder converse: that any $\FO$-formula which is preserved under this kind of bisimulation is equivalent to a first-order translation of a modal logic formula.

We show in this thesis that a similar characterization is possible for $\FGF$.
Employing a suitable notion of bisimulation, the $\FGF$ bisimulation $\bisimto_{\FGF}$, we prove that $\FGF$ is exactly the bisimulation invariant fragment of $\FO$.
In the classical setting, this is a known result\cite{BednarczykJ22}, but with a proof relying heavily on compactness, which is known to fail over finite models.
In the finite case, we may only assume invariance under $\bisimto_{\FGF}$-equivalent models that are finite, but we also only need to establish logical equivalence under finite models.
Hence, a van Benthem theorem for the class of all models does not imply a van Benthem theorem for the class of finite models.
However, the approach that we follow in this thesis works in both settings.
Succinctly, we obtain the following result: $\FGF \simeq \FO/{\bisimto_{\FGF}}$ (``$\FGF$ is exactly the $\bisimto_{\FGF}$-invariant fragment of $\FO$''), both in the finite and the infinite.
This shows that $\bisimto_{\FGF}$ is in fact the right notion of bisimulation for $\FGF$.

\section{Contributions and Outline}

Our main contribution is the proof of a van Benthem theorem for $\FGF$ both in the finite and in classical sense allowing for infinite models.
This requires a new kind of unraveling for $\FGF$ which we introduce in \ref{chap:unraveling}.
We show that this unraveling is canonical in the sense that it provides unique up to $\GF$-bisimulation models for $\FGF$ equivalence classes.
We believe that this property makes the unraveling useful on its own.
As a corollary, this shows that checking whether a given $\GF$ formula can be expressed as formula of $\FGF$ is decidable.

The structure of the thesis is as follows.
In \cref{chap:logics}, we introduce basic mathematical notation from model theory and review the definitions of the logics $\GF$ and $\FGF$.
In the following chapter \cref{chap:expressivity}, we first review related work on expressivity theorems and then show the main theorem \cref{thm:main}, a van Benthem characterization for $\FGF$, employing prior work from Otto.
In the proof, we assume the existence of particular models called finite companions for $\FGF$.
To this end, we introduce a new form of unraveling for $\FGF$  in \cref{chap:unraveling} and compare it to existing notions of unraveling for $\GF$ and $\FGF$.
We then employ this new method of unraveling in \cref{chap:finite} to construct the finite companions for $\FGF$, completing the proof of \cref{thm:main}.
Finally, \cref{chap:conclusion} summarizes the results and presents directions for future research.
