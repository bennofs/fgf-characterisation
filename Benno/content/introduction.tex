%!TEX root = ../main.tex
\chapter{Introduction}\label{chap:introduction}

First-order logic ($\FO$)---also known as predicate logic---is undecidable in its full generality\cite[Sec. 1.1]{borger1997}.
For instance, there is no algorithm that for any first-order formula $\varphi$, decides whether it is valid (true in all models) or not.
For practical applications, this leaves two choices: either accept the undecidability and hope that it does not occur in ``real-world problems'' or use another, less expressive logic, limited in some way to become decidable.

One decidable fragment of first-order logic is the \emph{Forward Guarded Fragment}~($\FGF$), recently shown to have nice algorithmic properties\cite{Bednarczyk21}.
The forward guarded fragment derives from the popular \emph{Guarded Fragment} ($\GF$) by restricting the order of variables.
For example, consider the following list of $FO$ formulae:
\begin{enumerate}
  \item $\forall{x_{1}x_{2}x_{3}}\; ((\mathrm{dayAfter}(x_{1},x_{2}) \land \mathrm{dayAfter}(x_{2},x_{3})) \to \mathrm{dayAfter}(x_{1}, x_{3}))$ \\
        ``if $x_{3}$ is a day after $x_{2}$ and $x_{2}$ itself is a day after $x_{1}$, then $x_{3}$ is after $x_{1}$ as well (transitivity)'' \\
        \emph{not guarded}: no atom covers all the variables $x_{1}, x_{2}, x_{3}$
  \item $\forall{x_{1}}\; (\mathrm{sunny}(x_{1}) \to \exists{x_{2}} (\mathrm{dayAfter}(x_{2},x_{1}) \land \neg \mathrm{sunny}(x_{2})))$ \\
        ``for every sunny day, there is a day before which was not sunny'' \\
        \emph{not forward}: $x_{2}, x_{1}$ is in wrong order
  \item $\forall{x_{1}x_{2}x_{3}}\; (\mathrm{dayBetween}(x_{1},x_{2},x_{3}) \to ((\mathrm{mon}(x_{1}) \land \mathrm{fri}(x_{3})) \to \mathrm{workday}(x_{2})))$ \\
        ``days between monday and friday are workdays'' \\
        \emph{forward and guarded}
\end{enumerate}
The forward guarded fragment consists of formulae which are both forward and guarded, like the third formula in the above list.
Here, $\FGF$ we characterize $\FGF$ as a subset of $\FO$ subject to \emph{syntactic} criteria.
With such a definition, we can say for every $\FO$ formula whether it belongs to $\FGF$ or not.
However, what if instead of asking whether a given $\FO$-formula $\varphi$ is in $\FGF$, we ask: is there an equivalent formula in $\FGF$, \ie is the property expressed by $\varphi$ also possible to express in $\FGF$?
This thesis provides an answer to this question in the form of a van Benthem characterization of $\FGF$, as explained below.
The proof presented here applies to both finitary and infinitary versions of this characterization.
While the infinite version was proven before, the finite version is a new result.

Van Benthem's theorem characterizes a logic using the notion of \emph{bisimulation}.
For a simple example of one notion of bisimulation, consider transition systems, which are structures consisting of a set of states with associated properties and a transition relation between the states.
Two transition systems are \emph{bisimilar} if there is a mapping (bisimulation) between states from one system to the other such that if state $a$ maps to state $b$, then:
\begin{enumerate}[(a)]
  \item $a$ and $b$ have the same properties, and
  \item if there is a transition from $a$ to $c$, then there is a corresponding transition to a state $d$ from $b$ and $c$ maps to $d$, and vice versa for a transition from $b$ to some $d$.
\end{enumerate}
This notion of bisimulation defines an equivalence on structures.
It is easy to see that certain kinds of formulae, in particular first-order translations of modal logic formulae, are preserved under this equivalence.
Here, ``preserved'' means that if there are two bisimilar models, the formula must either satisfy both of them or neither.
Now, the classical van Benthem theorem\cite{van1983modal} proves the much harder converse: that any $\FO$-formula which is preserved under this kind of bisimulation is equivalent to a first-order translation of a modal logic formula.

We show that a similar characterization is possible for $\FGF$.
Employing a suitable notion of bisimulation, the $\FGF$ bisimulation $\bisimto_{\FGF}$, we prove that $\FGF$ is exactly the bisimulation invariant fragment of $\FO$.
Our main focus is the finite model setting.
In the finite case, we may only assume invariance under $\bisimto_{\FGF}$-equivalent models that are finite, but we also only need to establish logical equivalence under finite models.
Hence, the van Benthem theorem differs between the finite and the infinite case.
However, the approach that we follow in this thesis works in both settings.
Succinctly, we obtain the following result: $\FGF \simeq \FO/{\bisimto_{\FGF}}$ (``$\FGF$ is exactly the $\bisimto_{\FGF}$-invariant fragment of $\FO$''), both in the finite and the infinite.
This shows that $\bisimto_{\FGF}$ is in fact the right notion of bisimulation for $\FGF$.

\section{Related Work}
% incorporate related work for expressive completeness paper section

The proof technique in this thesis closely follows the approach taken by Otto to prove van Benthem theorems for variations of modal logic\cite{Otto04, otto2004a}.
The central part of this technique is the construction of structures called ``finite companions''.
Like Otto, we construct these companions by stopping the infinite unraveling at some depth and then relinking edges to roots, to preserve similarity with the original structure.
The method of unraveling that we employ is inspired by the construction that Bednarczyk used to show the $\ExpTime$-completeness of $\FGF$\cite{Bednarczyk21}.
Bednarczyk's construction however is not sufficient to construct the finite companions.
Our new method turns out to be similar to a method of unraveling already described by Andréka et al. when introducing the guarded fragment\cite{AndrekaNB98}.
We present a comparison to these two existing methods of unraveling in \cref{sec:other-unravelings}.

Previous works on van Benthem style characterizations often focus on logics with two variables.
Examples of this, in addition to the aforementioned variants of basic modal logic, are:
\begin{itemize}
  \item graded modal logic characterized as the counting bisimulation invariant fragment of $\FO$, over both finite and infinite relational structures \cite{derijke2000,otto2023}.
  \item \Logic{HornALC} is the Horn-simulation invariant fragment of $\FO$\cite{jung2019},
  \item TBoxes of different description logics including \Logic{ALC}, \Logic{ALCQIQ}, \Logic{DL{-}Lite} and others are fragments of $\FO$ invariant under corresponding equivalence relations\cite{lutz2011, piro2013},
  \item $\Logic{XPath}_{=}$, a logic able to express (parts of) the navigational aspect of the XPath XML query language, is a $\FO$ fragment invariant under a corresponding notion of bisimulation for different navigational axes (child, descendant)\cite{figueira2015}.
\end{itemize}
One notable exemption is the uniform one-dimensional fragment ($\Logic{UF_{1}}$) studied by Hella et al.\cite{hella2014}, which like $\FGF$ allows unbounded number of variables.
Kierónski et al. describe a Ehrenfeucht-Fraïssé (EF) game for $\Logic{UF_{1}}$, which is similar to the notion of bisimulation.
However, they only prove the easier direction of van Benthem's theorem: that $\Logic{UF_{1}}$  is preserved under the equivalence defined by this game.
A proof or a failure for the converse is not found in the literature at this time.

A possible reason for the popularity of two-variable fragments may be that higher-arity relations or more than two variables are difficult to work with, as demonstrated by the proof of the van Benthem characterization of $\GF$\cite{Otto2012}.
Compared to an earlier result for $\GF$ with relations of arity at most 2\cite{Otto04}, the details of this proof are much more complex.
Luckily, since $\FGF$ is a subset of $\GF$ and any formula preserved under $\FGF$-bisimulation is also preserved under $\GF$-bisimulation, we can focus on characterizing $\FGF$ as a subset of $\GF$ here.
The full van Benthem theorem for $\FGF$ then follows from Otto's van Benthem theorem for $\GF$.

\section{Contributions and Outline}

Our main contribution is the proof of a van Benthem theorem for $\FGF$ over both in the sense of finite model theory and in the classical sense allowing for infinite models.
This requires a new kind of unraveling for $\FGF$ which we introduce in \ref{chap:unraveling}.
As a corollary, this shows that checking whether a given $\GF$ formula can be expressed as formula of $\FGF$ is decidable.

The structure of the thesis is as follows.
In \cref{chap:logics}, we introduce basic mathematical notation from model theory and review the definitions of the logics $\GF$ and $\FGF$.
In the following chapter \cref{chap:expressivity}, we formally state the main theorem \cref{TODO} of this thesis and describe the high-level structure of the proof technique, employing prior work from Otto.
To construct the finite companions required in the proof, we then introduce a new form of unraveling for $\FGF$ in \cref{chap:unraveling}, which we also compare to existing notions of unraveling for $\GF$ and $\FGF$.
In \cref{chap:finite}, we complete the proof by showing how to employ this new notion of unraveling to construct the finite companions.
Finally, \cref{chap:conclusion} summarizes the results and presents directions for future research.
