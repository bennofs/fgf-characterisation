% Proof sketchs
\let\realproof\proof
\let\realendproof\endproof
% \newenvironment{proofsketch}{%
%   \renewcommand{\proofname}{\normalfont\emph{Proof Sketch}}\realproof}{\realendproof}

% hide proofs
\let\proof\appendixproof
\let\endproof\endappendixproof

% \newtheorem{theorem}{Theorem}
% \newtheorem{lemma}[theorem]{Lemma}
% \newtheorem{corollary}[theorem]{Corollary}
% \newtheorem{proposition}[theorem]{Proposition}
% \newtheorem{exercise}[theorem]{Exercise}
% \newtheorem{definition}[theorem]{Definition}
% \newtheorem{conjecture}[theorem]{Conjecture}
% \newtheorem{observation}[theorem]{Observation}
% \theoremstyle{definition}
% \newtheorem{example}[theorem]{Example}
% \theoremstyle{remark}
% \newtheorem{note}[theorem]{Note}
% \newtheorem*{note*}{Note}
% \newtheorem{remark}[theorem]{Remark}
% \newtheorem*{remark*}{Remark}
% \theoremstyle{claimstyle}
% \newtheorem{claim}[theorem]{Claim}
% \newtheorem*{claim*}{Claim}


\newcommand{\problemdef}[4]{
  \smallskip
\begin{center}
\begin{tabular}{|r p{36em} |}
  \hline
\multicolumn{2}{|l|}{\rule{0pt}{2.6ex}\textbf{#1}} \\[.12ex]
\textit{Parameters:}     & #2\\[.12ex]
\textit{Input:}     & #3\\[.12ex]
\textit{Question:}  & #4 \\[.12ex]
\hline
\end{tabular}
\end{center}
\smallskip
}

\newcommand{\problemdeff}[3]{
  \smallskip
\begin{center}
\begin{tabular}{|r p{36em} |}
  \hline
\multicolumn{2}{|l|}{\rule{0pt}{2.6ex}\textbf{#1}} \\[.12ex]
\textit{Input:}     & #2\\[.12ex]
\textit{Question:}  & #3 \\[.12ex]
\hline
\end{tabular}
\end{center}
\smallskip
}


\definecolor{ForestGreen}{RGB}{34,139,34}
\definecolor{Salmon}{rgb}{1.0, 0.55, 0.41}
\definecolor{RawSienna}{rgb}{0.77, 0.12, 0.23}% \definecolor{carminered}{rgb}
\definecolor{MidnightBlue}{rgb}{0.0, 0.2, 0.4}
\definecolor{RedViolet}{rgb}{0.78, 0.08, 0.52}
\definecolor{TealBlue}{rgb}{0.21, 0.46, 0.53}
\definecolor{brandeisblue}{rgb}{0.0, 0.44, 1.0}

%%fakesection Theorems

\theoremstyle{definition}
\mdfdefinestyle{mdbluebox}{%
  roundcorner = 10pt,
  linewidth=1pt,
  skipabove=12pt,
  innerbottommargin=9pt,
  skipbelow=2pt,
  nobreak=true,
  linecolor=blue,
  backgroundcolor=TealBlue!8,
}
\declaretheoremstyle[
  headfont=\sffamily\bfseries\color{MidnightBlue},
  mdframed={style=mdbluebox},
  headpunct={\\[3pt]},
  postheadspace={0pt}
]{thmbluebox}

\mdfdefinestyle{mdredbox}{%
  linewidth=0.5pt,
  skipabove=12pt,
  frametitleaboveskip=5pt,
  frametitlebelowskip=0pt,
  skipbelow=2pt,
  frametitlefont=\bfseries,
  innertopmargin=4pt,
  innerbottommargin=8pt,
  nobreak=true,
  linecolor=RawSienna,
  backgroundcolor=Salmon!8,
}
\declaretheoremstyle[
  headfont=\bfseries\color{RawSienna},
  mdframed={style=mdredbox},
  headpunct={\\[3pt]},
  postheadspace={0pt},
]{thmredboxx}


\mdfdefinestyle{mdmyredbox}{%
  skipabove=8pt,
  linewidth=2pt,
  rightline=false,
  leftline=true,
  topline=false,
  bottomline=false,
  linecolor=RawSienna,
  backgroundcolor=Salmon!8,
}

\declaretheoremstyle[
  headfont=\bfseries\sffamily\color{ForestGreen!70!black},
  bodyfont=\normalfont,
  spaceabove=2pt,
  spacebelow=1pt,
  mdframed={style=mdmyredbox},
  headpunct={  },
]{thmmyredbox}

\mdfdefinestyle{mdgreenbox}{%
  skipabove=8pt,
  linewidth=2pt,
  rightline=false,
  leftline=true,
  topline=false,
  bottomline=false,
  linecolor=ForestGreen,
  backgroundcolor=ForestGreen!8,
}
\declaretheoremstyle[
  headfont=\bfseries\sffamily\color{ForestGreen!70!black},
  bodyfont=\normalfont,
  spaceabove=2pt,
  spacebelow=1pt,
  mdframed={style=mdgreenbox},
  headpunct={ },
  % qed={\qedDef},
]{thmgreenbox}

\declaretheoremstyle[
  headfont=\bfseries\sffamily\color{RawSienna!80!black},
  bodyfont=\normalfont,
  spaceabove=2pt,
  spacebelow=1pt,
  mdframed={style=mdmyredbox},
  headpunct={ },
]{thmredbox}


\declaretheoremstyle[
  headfont=\bfseries\sffamily\color{ForestGreen!70!black},
  bodyfont=\normalfont,
  spaceabove=2pt,
  spacebelow=1pt,
  mdframed={style=mdgreenbox},
  headpunct={},
]{thmgreenbox*}

\mdfdefinestyle{mdblackbox}{%
  skipabove=8pt,
  linewidth=3pt,
  rightline=false,
  leftline=true,
  topline=false,
  bottomline=false,
  linecolor=black,
  backgroundcolor=RedViolet!8!gray!8,
}
\declaretheoremstyle[
  headfont=\bfseries,
  bodyfont=\normalfont\small,
  spaceabove=0pt,
  spacebelow=0pt,
  mdframed={style=mdblackbox}
]{thmblackbox}

\mdfdefinestyle{mdbluebox}{%
  skipabove=8pt,
  linewidth=2pt,
  rightline=false,
  leftline=true,
  topline=false,
  bottomline=false,
  linecolor=brandeisblue,
  backgroundcolor=brandeisblue!8,
}
\declaretheoremstyle[
  headfont=\bfseries\sffamily\color{brandeisblue!70!black},
  bodyfont=\normalfont,
  spaceabove=2pt,
  spacebelow=1pt,
  mdframed={style=mdbluebox},
  headpunct={ },
]{thmbluebox}


\theoremstyle{definition}
\declaretheorem[numberwithin=chapter,style=thmgreenbox]{definition}
\declaretheorem[name=Example,sibling=definition,style=thmblackbox]{examplex}
\declaretheorem[name=Property,sibling=definition,style=thmblackbox]{property}
\declaretheorem[name=Remark,sibling=definition,style=thmblackbox]{remarkx}
\declaretheorem[name=Fact,sibling=definition,style=thmblackbox]{fact}
\declaretheorem[name=Corollary,sibling=definition,style=thmredboxx]{corollary}
\declaretheorem[name=Observation,sibling=definition,style=thmblackbox]{observation}
\declaretheorem[name=Theorem,sibling=definition,style=thmredboxx]{theorem}
\declaretheorem[name=Lemma,sibling=definition,style=thmbluebox]{lemma}
\declaretheorem[name=Main Theorem,,style=thmredboxx,numbered=no]{maintheorem}

\newtheorem{proposition}[definition]{Proposition}
\newtheorem{claim}[definition]{Claim}
\newtheorem*{induction*}{Induction Hypothesis}

\newenvironment{example}
{\pushQED{\qed}\renewcommand{\qedsymbol}{$\square$}\examplex}
{\popQED\endexamplex}

\newenvironment{remark}
{\pushQED{\qed}\renewcommand{\qedsymbol}{$\square$}\remarkx}
{\popQED\endremarkx}

\newenvironment{sproof}{%
  \renewcommand{\proofname}{Proof (sketch)}\proof}{\endproof}

\newcommand{\defstyle}[1]{\emph{\text{#1}}}
\newcommand{\indexgeneral}[2]{\index{#2}\defstyle{#1}}
\newcommand{\idindexgeneral}[1]{\indexgeneral{#1}{#1}}
\newcommand{\silentindexsymbol}[3]{\index[sym]{#3!#1 : #2}}
\newcommand{\indexsymbol}[3]{\silentindexsymbol{#1}{#2}{#3}{#1}}
\newcommand{\mathindexsymbol}[3]{\silentindexsymbol{$#1$}{#2}{#3}{#1}}
\newcommand{\silentformulaindexsymbol}[2]{\index[sym]{#2!#1}}
\newcommand{\formulaindexsymbol}[2]{{#1}\silentformulaindexsymbol{#1}{#2}}
\newcommand{\logicindexsymbol}[1]{\formulaindexsymbol{#1}{\domLogics}}

%Left / Right proof directions
\newcommand{\ProofRightarrow}{\noindent($\Rightarrow$):\xspace}
\newcommand{\ProofLeftarrow}{\noindent($\Leftarrow$):\xspace}

\newcommand{\ProofSupseteq}{\noindent($\supseteq$):\xspace}
\newcommand{\ProofSubseteq}{\noindent($\subseteq$):\xspace}

\newcommand{\vocab}[1]{\textbf{\color{ForestGreen!70!black}#1}}

% \makeatletter
% \renewenvironment{proof}[1][\proofname]{\par
%   \pushQED{\qed}%
%   \normalfont \topsep6\p@\@plus6\p@\relax
%   \list{}{\listparindent 0.5em
%           \itemindent    \z@
%           \labelwidth    \z@
%           \rightmargin   \leftmargin
%           \parsep        \z@ \@plus\p@}%
%   \item[\hskip\labelsep\itshape#1\@addpunct{.}]\ignorespaces
% }{%
%   \popQED\\\endlist\@endpefalse
% }
% \makeatother

\newcommand{\qedDef}{$\blacktriangleleft$}
