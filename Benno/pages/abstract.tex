\chapter*{Abstract}

\begingroup
\Large
\narrower
\narrower
The forward guarded fragment ($\FGF$) is a recently introduced decidable fragment of first-order logic ($\FO$) whose satisfiability problem is only $\ExpTime$-complete.
$\FGF$ derives from the guarded fragment ($\GF$) by adding a restriction on the order in which variables may appear in formulae, similar to the fluted fragment by Purdy.
$\FGF$ is a higher-arity generalization of multi-modal logic with global modalities, but in contrast to $\GF$, it does not capture logics with inverse modalities.

\vskip0.5\baselineskip
\noindent
In total analogy to the classic van Benthem characterization of modal logic, we show that $\FGF$ is exactly the fragment of $\FO$ which consists of formulae that are invariant under a suitable notion of bisimulation for $\FGF$.
Crucially, we employ techniques by Martin Otto originally designed for modal logic which also work if we restrict the class of models under consideration to finite models only.

\vskip0.5\baselineskip
\noindent
In the process, we develop a new notion of unraveling for $\FGF$, which produces tree-like models that satisfy the same $\FGF$ formulae as the original model.
The new notion of unraveling is closely linked to a notion of unraveling for $\GF$ and is interesting on its own.
As an application for our van Benthem style characterization, we show that whether a given $\GF$ has an equivalent formula in $\FGF$ is decidable.

\endgroup
