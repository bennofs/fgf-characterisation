% -*- TeX-engine: xetex -*-
\documentclass[draft]{scrartcl}

\usepackage{polyglossia}
\usepackage{fontspec}

\setmainlanguage{english}

\usepackage{blindtext}
\usepackage{amsthm}
\usepackage{amssymb}
\usepackage{amsmath}
\usepackage{hyperref}
\usepackage[capitalise]{cleveref}
\usepackage{booktabs}
\usepackage{tikz}
\usepackage{tabularx}
\usepackage{makecell}
\usepackage{placeins}
\usepackage{listings}
\usepackage{subcaption}
\usepackage{fixme}

\fxsetup{theme=color}

\title{FGF Notes}
\author{Benno Fünfstück}
\date{March 2022}

\newtheorem{theorem}{Theorem}[section]
\theoremstyle{definition}
\newtheorem{definition}[theorem]{Definition}
\newtheorem{lemma}[theorem]{Lemma}
\crefname{lemma}{Lemma}{Lemmas}
\crefname{definition}{Definition}{Definitions}
\crefname{theorem}{Theorem}{Theorems}

\newcommand{\last}[1]{\mathtt{last}(#1)}

\begin{document}

\maketitle

\section{Logics}

% exactly as in jair-main
\begin{definition}[Forward (Guarded) Fragment]
Let FF$(n)$ be the smallest fragment of FO satisfying:

\begin{itemize}
    \item an atom $\alpha(\bar{x})$ belongs to FF$(n)$ iff $\alpha$ is equality-free and $\bar{x}$ is an infix of $\bar{x}_{1\ldots{}n}$
    \item FF$(n)$ is closed under boolean connectives $\land, \lor, \neg, \rightarrow, \leftrightarrow$.
    \item If $\phi(\bar{x}_{1\ldots{}n+1}$ is in FF$(n+1)$, then $\exists{x_{n+1}}\phi(\bar{x}_{1\ldots{}n+1})$ and $\forall{x_{n+1}}\phi(\bar{x}_{1\ldots{}n+1})$ are in FF$(n)$.
\end{itemize}

The \emph{forward fragment} FF is the set FF$(0)$.
The \emph{forward guarded fragment} FGF is the \fxfatal*{Allow variable reuse}{intersection of FF with GF}.
\end{definition}

% as in jair-main.pdf
\begin{definition}[Forward type]
A $(\sigma,n)$-forward type is a maximally consistent conjunction of atoms of the form $\pm{}R(\bar{x}_{i\ldots{}i+ar(R)-1}$ for an index $1 \leq i \leq n - ar(R) + 1$.
\end{definition}

We denote by $\mathrm{ftp}_\mathfrak{A}(\bar{a})$ the unique $(\sigma,|\bar{a}|)$-forward type such that $\mathfrak{A} \models \mathrm{ftp}_\mathfrak{A}(\bar{a})$

\section{Back-and-forth conditions / Bisimulations / Games}

% adapted from Otto 2004
\begin{definition}[Connected GF back-and-forth conditions]
Let $\mathfrak{A}$ and $\mathfrak{B}$ be $\sigma$-structures, $\mathcal{Z}, \mathcal{Z'} \subseteq \mathtt{Part}(\mathfrak{A}, \mathfrak{B})$ sets of partial isomorphisms between $\mathfrak{A}$ and $\mathfrak{B}$.

$\mathcal{Z'}$ satisfies the connected GF back-and-forth conditions for $\mathcal{Z}$ if for every $p \in \mathcal{Z}$ the following holds:

\begin{description}
    \item[(gforth)] For any guarded subset $s'$ of $A$ with $s' \cap \mathtt{dom}(p)$ nonempty, there is some $p' \in \mathcal{Z}$ with $\mathtt{dom}(p') = s'$ such that $p$ and $p'$ agree on their common domain.
    \item[(gback)] For any guarded subset $t'$ of $B$ with $t' \cap \mathtt{im}(p)$ nonempty, there is some $p' \in \mathcal{Z}$ with $\mathtt{im}(p') = t'$ such that $p^{-1}$ and $p'^{-1}$ agree on their common domain.
\end{description}
\end{definition}

% adapted from jair-main.pdf
\begin{definition}[Connected FGF back-and-forth conditions]
Let $\mathfrak{A}$ and $\mathfrak{B}$ be $\sigma$-structures, $\mathcal{Z}, \mathcal{Z'} \subseteq \bigcup_{i=0}(A^i \times B^i)$ be mappings such that for any $(\bar{a}, \bar{b}) \in \mathcal{Z} \cup \mathcal{Z'}$ forward types are preserved: $\mathrm{ftp}_\mathfrak{A}(\bar{a}) = \mathrm{ftp}_\mathfrak{B}(\bar{b})$ (\textbf{atomic harmony}).

$\mathcal{Z'}$ satisfies the connected FGF back-and-forth conditions for $\mathcal{Z}$ if for every $(\bar{a}, \bar{b}) \in \mathcal{Z}$ the following conditions hold:

\begin{description}
    \item[(fgforth)] For a nonempty affix $\bar{a}_{i\ldots{}j}$ of $\bar{a}$ and a $\sigma$-live tuple $\bar{c}$ with $|c| \leq k$ in $\mathfrak{A}$ such that $\bar{a}_{i\ldots{}j} = \bar{c}_{1\ldots{}j-i+1}$ there is a tuple $\bar{d}$ with $\bar{b}_{i\ldots{}j} = \bar{d}_{1\ldots{}j-i+1}$ and $(\bar{c}, \bar{d}) \in \mathcal{Z'}$
    \item[(fgback)] For a nonempty affix $\bar{b}_{i\ldots{}j}$ of $\bar{b}$ and a $\sigma$-live tuple $\bar{d}$ with $|d| \leq k$ in $\mathfrak{B}$ such that $\bar{b}_{i\ldots{}j} = \bar{d}_{1\ldots{}j-i+1}$ there is a tuple $\bar{c}$ with $\bar{a}_{i\ldots{}j} = \bar{c}_{1\ldots{}j-i+1}$ and $(\bar{c}, \bar{d}) \in \mathcal{Z'}$
\end{description}
\end{definition}

\begin{definition}[Connected Bisimulation]
A set $\mathcal{Z}$ is a connected GF or FGF bisimulation between pointed $\sigma$-structures $\mathfrak{A}, \bar{a}$ and $\mathfrak{B}, \bar{b}$ if it satisfies the connected GF respective FGF back-and-forth conditions for itself and $(\bar{a}, \bar{b}) \in \mathcal{Z}$.
\end{definition}

We write $\mathfrak{A}, \bar{a} \sim_{\textrm{GF}} \mathfrak{B}, \bar{b}$ or $\mathfrak{A}, \bar{a} \sim_{\textrm{FGF}} \mathfrak{B}, \bar{b}$ if there exists a connected GF respective FGF bisimulation between pointed $\sigma$-structures $\mathfrak{A}, \bar{a}$ and $\mathfrak{B}, \bar{b}$.

\begin{definition}[k-Level Bisimulation]
A k-level bisimulation is a sequence of sets $\mathcal{Z}_0, \ldots, \mathcal{Z}_k$ such that each $\mathcal{Z}_{i - 1}$ for $0 < i \le k$ satisfies the associated back-and-forth conditions for $\mathcal{Z}_i$.
\end{definition}

We write $\mathfrak{A}, \bar{a} \sim_{\textrm{GF}, k} \mathfrak{B}, \bar{b}$ or $\mathfrak{A}, \bar{a} \sim_{\textrm{FGF}, k} \mathfrak{B}, \bar{b}$ if there exists a k-level connected GF respective FGF bisimulation between pointed $\sigma$-structures $\mathfrak{A}, \bar{a}$ and $\mathfrak{B}, \bar{b}$.

\begin{definition}[Global Bisimulation]
A global bisimulation between $\sigma$-structures $\mathfrak{A}$ and $\mathfrak{B}$ is a set $\mathcal{Z}$ such that for any guarded tuple $\bar{a} \in \mathfrak{A}$ there is a guarded tuple $\bar{b} \in \mathfrak{B}$ such that $(\bar{a}, \bar{b}) \in \mathcal{Z}$ and vice versa.
\end{definition}

\begin{definition}[Connected Forward Bisimulation Path]\label{def:bisim-path}
  Let $\bar{a}^{0}, \cdots, \bar{a}^{l}$ be a sequence of guarded tuples in some structure $\mathfrak{A}$.
  This sequence is a \emph{connected forward bisimulation path} from $\bar{a}^{0}$ to $\bar{a}^{i}$ if for all consecutive tuples $\bar{a}^{i}$ and $\bar{a}^{i+1}$,
  a nonempty prefix of $\bar{a}^{i+1}$ is an infix of $\bar{a}^{i}$ and $a^{i} \ne a^{i+1}$.
  A path is \emph{strict} if $\bar{a}^{i+1}$ contains at least one element not part of the common infix.
\end{definition}

\section{Tree unraveling}

% definition adapted from jair-main.pdf
\begin{definition}[HAT]
Let $\mathcal{P}$ be a prefix-closed set of sequences over $A$.\fxfatal{need to restrict to single root}
A tuple $\bar{a}$ is \emph{sequential} if there are elements $a_1, \ldots, a_i$ in $A$ and a sequence $p \in \mathcal{P}$ such that $\bar{a} = (p, p a_1, \ldots, p a_1 \cdots a_i)$.

A \emph{higher-arity tree} (HAT) is a structure $\mathfrak{T}$ with domain $P$ where for each relation $R$, every tuple $\bar{a} \in R^\mathfrak{T}$ is sequential.
\end{definition}

\begin{lemma}\label{lemma-seq-intersection}
The intersection of two sequential tuples $\bar{a}, \bar{b}$ in a HAT is an infix of both tuples and a prefix of at least one of the tuples.
\end{lemma}

\begin{proof}
\fxfatal{write proof}
\end{proof}

% definition exactly as in jair-main.pdf
\begin{definition}[Forward Gaifman Graph]
The \emph{forward Gaifman graph} $\mathit{fG}_\mathfrak{A}$ of a structure $\mathfrak{A}$ is the graph with the set of nodes equal to $A$ and edges between $d$ and $e$ whenever there are (possibly empty) tuples of elements $\bar{c}$, $\bar{f}$ and a relational symbol $R$ witnessing $(\bar{c}, d, e, \bar{f}) \in R^{\mathfrak{A}}$.
\end{definition}

\begin{definition}[Tree unraveling]
For a structure $\mathfrak{A}$ and an element $a \in A$, the \emph{tree unraveling} $\mathfrak{A}^*_a$ of $\mathfrak{A}$ from $a$ is a structure with domain $\mathtt{Paths}_a(\mathit{fG}_\mathfrak{A})$. The relations in $\mathfrak{A}^*$ are defined as:

\[
R^{\mathfrak{A}^*_a} = \{ \bar{x} \mid \text{$\last{\bar{x}} \in R^{\mathfrak{A}}$ and $\bar{x}$ is sequential} \}
\]
for every relational R, where $\last{\bar{x}}$ is the projection that maps the tuple $\bar{x}$ of paths to a tuple of each path's last element.
\end{definition}

\begin{definition}[Global tree unraveling]
The global tree unraveling $\mathfrak{A}^*$ of a structure $\mathfrak{A}$ without a distinguished element is defined as the disjoint union of all the tree unravelings $A^*_{a_i}$ from every $a_i \in A$.
\end{definition}

\begin{lemma}
$\mathfrak{A} \sim_{\textrm{FGF}} \mathfrak{A}^*$ for any structure $\mathfrak{A}$.
\end{lemma}

\begin{proof}
We show that:
\[ \mathcal{Z} = \left\{(\bar{a}, \bar{b}) \in \bigcup_{i=0}(A^i \times {A^*}^i) \mid \text{$\bar{a} = \last{\bar{b}}$ and $\bar{b}$ is sequential}\right\} \]
is a FGF-bisimulation between $\mathfrak{A}$ and $\mathfrak{A}^*$. Clearly, $(\epsilon, \epsilon) \in \mathcal{Z}$.

To show (atomic harmony), first note that for any $(\bar{a}, \bar{b}) \in \mathcal{Z}$ and any relation R of arity $|a| = |b|$, we have $\mathfrak{A} \models R(\bar{a})$ iff $\mathfrak{A}^* \models R(\bar{b})$ since by construction of $\mathcal{Z}$ both $\bar{a} = \last{\bar{b}}$ and $\bar{b}$ is sequential. For any indices $i \leq j$, $(\bar{a}_{i\ldots{}j}, \bar{b}_{i\ldots{}j})$ is also in $\mathcal{Z}$. Thus, all infixes also satisfy equal relations implying $\mathtt{ftp}_{\mathfrak{A}}(\bar{a}) = \mathtt{ftp}_{\mathfrak{A}^*}(\bar{b})$.

To show (fgforth), let $(\bar{a}, \bar{b}) \in \mathcal{Z}$, $\bar{a}_{i\ldots{}j}$ be an infix of $\bar{a}$ and $\bar{c}$ be a $\sigma$-live tuple with prefix $\bar{a}_{i\ldots{}j}$ as in the definition of FGF-bisimulation.
Since $\bar{b} = \last{\bar{a}}$ and $\bar{b}$ is sequential, we know that there is a path $p \in \mathtt{Paths}_e(\mathit{fG}_\mathfrak{A})$ from some element $e \in A$ such that $\bar{b}_{i\ldots{}j} = (p, p a_{i+1}, \ldots, p a_{i+1} \cdots a_j)$.
Because $\bar{c}$ is a live tuple in $\mathfrak{A}$ there is a path $c_1 \cdots c_n$ in $\mathit{fG}_\mathfrak{A}$. If $\bar{b}_{i\ldots{}j}$ is empty we pick $p = c_1$, a path from element $e = c_1$.
The tuple $\bar{d} = (p, p c_2, \ldots, p c_2 \cdots c_n)$ is sequential and satisfies $\last{\bar{d}} = \bar{c}$, so we have $(\bar{c}, \bar{d}) \in \mathcal{Z}$.
The paths $p \bar{c}_2 \cdots \bar{c}_{k}$ and $p \bar{a}_{i+1} \cdots \bar{a}_{i+k-1}$ are identical for all $k \leq j-i+1$ as $\bar{a}_{i\ldots{}j} = \bar{c}_{1\ldots{}j-i+1}$.
Thus $\bar{d}_{1\ldots{}j-i+1} = \bar{b}_{i\ldots{}j}$ as required by (fgforth).

To show (fgback), let $(\bar{a}, \bar{b}) \in \mathcal{Z}$, $\bar{b}_{i\ldots{}j}$ be an infix of $\bar{b}$ and $\bar{d}$ be a $\sigma$-live tuple with prefix $\bar{b}_{i\ldots{}j}$ as in the definition of FGF-bisimulation. The tuple $\bar{d}$ is sequential since it is $\sigma$-live in $\mathfrak{A}^*$. Then, taking $\bar{c} = \last{\bar{d}}$ satisfies $\bar{c}_{1\ldots{}j-i+1} = \bar{a}_{i\ldots{}j}$ and $(\bar{c}, \bar{d}) \in \mathcal{Z}$ as required by (fgback).
\end{proof}

\begin{definition}[Path-complete tree unraveling]
  Let $\mathfrak{A}, \bar{a}$ be a pointed structure.
  Let $S$ be the set of all strict paths in $\mathfrak{A}$ starting from the tuple $\bar{a}$.
  The \emph{path-complete tree unraveling} $A^{*}_{c,\bar{a}}$ is a structure with the domain equal to the smallest set $D$ such that:

  \begin{itemize}
    \item $(\bar{a}, i, a_{i}) \in D$ for every $a_{i} \in \bar{a}$
    \item
          for every path $\rho \in S$ with $|\rho| \geq 2$, let $\bar{p}^{(n-1)}$, $\bar{p}^{(n)}$ be the last two tuples of the path.
          Let $\bar{p}^{(n)}_{i}$ be the leftmost element of $\bar{p}^{(n)}$ which is not part of the infix of  $\bar{p}^{(n-1)}$.
          Then for every $i \leq j \leq |\bar{p}^{n}|$ also $ (\rho, j-i, \bar{p}^{(n)}_{j}) \in D$.
  \end{itemize}

  For every relation R we define the relation in $\mathfrak{A}^{*}_{c,\bar{a}}$ as follows:
  \[
    R^{\mathfrak{A}^{*}_{c,\bar{a}}} = \left\{
      (
      (\rho, i, \bar{s}_{0}),
      (\rho, i+1, \bar{s}_{1}),
      \ldots,
      (\rho \bar{p'}, 0, \bar{s}_{j}),
      (\rho \bar{p'}, 1, \bar{s}_{j+1}),
      \ldots,
      (\rho \bar{p'}, |\bar{s}|-j, \bar{s}_{|\bar{s}|})
      )
      \mid
      \bar{s} \in R^{\mathfrak{A}}
    \right\}
  \]
\end{definition}

\begin{lemma}
  $\mathfrak{A}^{*}_{c, \bar{a}}, ((\bar{a}, 0, \bar{a}_{0}), \ldots, (\bar{a}, |\bar{a}|, \bar{a}_{|\bar{a}|})) \sim_{\textrm{FGF}} \mathfrak{A}, \bar{a}$ for any pointed structure $\mathfrak{A}, \bar{a}$.
\end{lemma}

\begin{proof}
  We show that the following set is a FGF-bisimulation:

  \[
    \mathcal{Z} = \left\{
      (\bar{s}, \pi(\bar{s}))
      \mid \bar{s}\
      \textrm{is any infix of a guarded tuple in}\ \mathfrak{A}^{*}_{c, \bar{a}}
    \right\}
  \]

  where the projection $\pi$ maps every element $(\rho, i, \bar{x}_{j})$ of a guarded tuple to the element $\bar{x}_{j}$ in $A$.

  The set $\mathcal{Z}$ satisfies \textbf{(atomic harmony)} because $\bar{s}$ satisfies the same relations in $\mathfrak{A}^{*}_{c,\bar{a}}$ as $\pi(\bar{s})$ does in $\mathfrak{A}$, by definition.
  The same holds for any infixes as they are also contained in $\mathcal{Z}$.

  To show \textbf{(fgforth)}, let $(\bar{s}, \pi(\bar{s})) \in \mathcal{Z}$, $\bar{s}_{i\ldots{}j}$ be an infix of $\bar{s}$ and $\bar{s'}$ be a $\sigma$-live tuple with prefix $\bar{s}_{i\ldots{}j}$ as in the definition of FGF-bisimulation.
  The projection $\pi{\bar{s'}}$ then shares the same infix, and $(\bar{s'}, \pi(\bar{s'})) \in \mathcal{Z}$, as required.

  To show \textbf{(fgback)}, let $(\bar{s}, \pi(\bar{s})) \in \mathcal{Z}$, ${\pi(\bar{s})}_{i\ldots{}j}$ be an infix of $\pi(\bar{s})$ and $t'$ be a $\sigma$-live tuple with prefix ${\pi(\bar{s})}_{i\ldots{}j}$ as in the definition of FGF-bisimulation.
  If $j = i + |\bar{t'}| - 1$, then $\bar{t'}$ is an infix of $\pi(\bar{s})$.
  Thus, $(\bar{s}_{i\ldots{}j}, t') \in \mathcal{Z}$ satisfies the requirements.
  Otherwise let $(\rho, \_, \_) = \bar{s}_{i}$ be the first element of the shared infix.


\end{proof}


% \begin{definition}[bisimulation-level]
%   Let $\mathfrak{A}, \bar{a}$ and $\mathfrak{B}, \bar{b}$ be pointed structures and $\mathcal{Z}_0, \ldots, \mathcal{Z}_k$, be a k-level bisimulation between them.
%   The \emph{bisimulation level} $\mathtt{level}_{\mathcal{Z}}(\bar{c})$ of a tuple $\bar{c} \in A$ is the smallest $l$ such that there exists a tuple $\bar{x}$ with $(\bar{c}, \bar{x}) \in \mathcal{Z}_{k-l}$.
%   The bisimulation level of a tuple $\bar{d} \in B$ is defined symmetrically.
% \end{definition}

% \begin{lemma}
%   Let $\mathcal{Z}_{k}: \mathfrak{\hat{A}}, \bar{a} ~_{\textrm{FGF},k} \mathfrak{\hat{B}}, \bar{b}$ be a minimal k-level connected FGF bisimulation between HATs.
%   Then for all $0 \leq i \leq k$ and bisimilar tuples $(\bar{s}, \bar{t}) \in \mathcal{Z}_{i}$, the level is preserved: $\mathtt{level}(\bar{s}) = \mathtt{level}(\bar{t})$.
% \end{lemma}

% \begin{proof}
%   Proof by induction.
%   \paragraph{Base case for level 0}
%   Since the bisimulation $\mathcal{Z}$ is minimal, $Z_{k} = {(\bar{a}, \bar{b})}$.
%   Both $\bar{a}$ and $\bar{b}$ have equal level $0$, as required.
%   \paragraph{Inductive case for level $i' = i + 1$}
%   Let $s'$ be a tuple in $A$ (the case for $B$ is symmetric) with $level(\bar{s}') = i$.
%   Assume by contradiction that there is a tuple $\bar{t}'$ with $(\bar{s}', \bar{t}') \in \mathcal{Z}_{x}$ with $\level{t'} < i'$ for some $0 \leq x \leq k$.
%   Let $a$
%   Let $(\bar{s}', \bar{t}') \in Z_{i'}$.
%   Assume by contradiction and w.l.o.g by symmetry that $\mathtt{level}(\bar{s}') > \mathtt{level}(\bar{t}')$.
%   We know that $\mathtt{level}(\bar{s}') \leq i'$ since it appears in $Z_{i'}$.
%   Let $x = \mathtt{level}(\bar{t}')$, which must therefore be less than $i'$.
%   We also know that $\mathtt{level}(\bar{s}') = i'$ since otherwise both tuples in the pair $(\bar{s}', \bar{t}')$ would have a level less than $i'$ and thus have equal levels by inductive hypothesis.

%   Since $\mathcal{Z}$ is minimal, there must be $(\bar{s}, \bar{t}) \in Z_{i'+1}$ where $\bar{s}$ has an infix which is a prefix of $\bar{s'}$ and the same holds for $t$ and $t'$.
% \end{proof}

\begin{lemma}
If $\mathfrak{\hat{A}}, \bar{a}$ and $\mathfrak{\hat{B}}, \bar{b}$ are pointed HATs, then $\mathfrak{\hat{A}}, \bar{a} \sim_{\mathrm{FGF},k} \mathfrak{\hat{B}}, \bar{b}$ implies $\mathfrak{\hat{A}}, \bar{a} \sim_{\mathrm{GF},k} \mathfrak{\hat{B}}, \bar{b}$.
\end{lemma}


\begin{proof}
Let $\mathcal{Z}_k: \mathfrak{\hat{A}} \sim_{\textrm{FGF},k} \mathfrak{\hat{B}}$ be a history-preserving k-level FGF bisimulation between $\mathfrak{\hat{A}}$ and $\mathfrak{\hat{B}}$. We show that $Z$ is also a k-level GF-bisimulation.

First, we show that for tuples $\bar{a}, \bar{b}$ if $\mathtt{ftp}_\mathfrak{\hat{A}}(\bar{a}) = \mathtt{ftp}_\mathfrak{\hat{B}}$, then the mapping $p: \bar{a}_i \mapsto \bar{b}_i$ is a partial isomorphism in HATs.
Clearly, $p$ is bijective, as $|\bar{a}| = |\bar{b}|$.
We claim that $\bar{d} = (p(c_1), ..., p(c_n))$ satisfies exactly the same relations as $\bar{c}$, for any tuple $\bar{c}$ with elements from $\bar{a}$.
If $\bar{c}$ is an infix of $\bar{a}$, then this is true because either $-R(\bar{c})$ or $+R(\bar{c})$ must be part of the forward type of $\bar{a}$ and due to equality of forward types also in the forward type of $\bar{b}$.
If $\bar{c}$ is not an infix of $\bar{a}$, then $\bar{c}$ is not sequential, since any tuple of elements from $\bar{a}$ which is sequential must be an infix of $\bar{a}$.
Since $\bar{c}$ is not sequential, $\bar{d}$ is also not sequential, and for any relation $R$ we have $-R(\bar{c})$ and $-R(\bar{d})$ as $\mathfrak{\hat{A}}$ and $\mathfrak{\hat{B}}$ are HATs and therefore any tuple satisfying a relation must be sequential.

For (gforth), let $s'$ be any guarded subset of $\hat{A}$, and $p$ be the partial isomorphism for any pair of tuples $(\bar{a},\bar{b}) \in \mathcal{Z}$.
Since $s'$ is guarded and $\mathfrak{\hat{A}}$ is a HAT, there must be a tuple $\bar{c}$ such that each element of $s'$ appears in $\bar{c}$ exactly once and $\bar{c}$ is sequential.
Because $\bar{a}$ is guarded it must also be sequential.
Let $\bar{a}_{i\ldots{}j}$ be the intersection of $\bar{a}$ and $\bar{c}$, which is a an infix of $\bar{a}$ by~\cref{lemma-seq-intersection}.
If $\bar{a}_{i\ldots{}j}$ is a prefix of $\bar{c}$, then we directly obtain $\bar{d}$ using (fgforth) which satisfies the necessary conditions for (gforth).

Otherwise, $\bar{a}_{i\ldots{}j}$ is a prefix of $\bar{a}$.
Let $X_{}$

\fxfatal{case for $\bar{a}_{i\ldots{}j}$ not prefix of $\bar{c}$}

The case for (gback) is symmetrical.

\end{proof}



\end{document}
