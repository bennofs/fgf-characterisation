% -*- TeX-engine: xetex -*-
\documentclass[draft]{scrartcl}

\usepackage{ifxetex}
\usepackage{amsmath}

\ifxetex
  \usepackage{polyglossia}
  \usepackage{fontspec}

  \setmainlanguage{english}
\else
  \usepackage[english]{babel}
\fi

\usepackage{blindtext}
\usepackage{amsthm}
\usepackage{amssymb}
\usepackage{amsmath}
\usepackage{hyperref}
\usepackage[capitalise]{cleveref}
\usepackage{booktabs}
\usepackage{tikz}
\usepackage{tabularx}
\usepackage{makecell}
\usepackage{placeins}
\usepackage{listings}
\usepackage{caption}
\usepackage{subcaption}
\usepackage{fixme}
\usepackage{graphicx}
\usepackage[shortlabels]{enumitem}
\usepackage{pbox}
\usepackage{mathtools}

\setkeys{Gin}{draft=false} %load images in draft mode
\fxsetup{theme=color,status=draft}

\title{FGF Notes}
\author{Benno Fünfstück}
\date{March 2022}

\newtheorem{theorem}{Theorem}
\theoremstyle{definition}
\newtheorem{definition}[theorem]{Definition}
\newtheorem{lemma}[theorem]{Lemma}
\newtheorem{example}[theorem]{Example}
\newtheorem{observation}[theorem]{Observation}
\crefname{lemma}{Lemma}{Lemmas}
\crefname{definition}{Definition}{Definitions}
\crefname{theorem}{Theorem}{Theorems}
\crefname{observation}{Observation}{Observations}

\newcommand{\last}[1]{\mathtt{last}(#1)}
\newcommand{\trace}[1]{\mathtt{trace}(#1)}
\newcommand{\context}[1]{\mathtt{context}(#1)}
\newcommand{\struct}[1]{$\mathfrak{#1}$}

../Lipics version/settings/tolcolors.tex
\usetikzlibrary{calc}
\usetikzlibrary{backgrounds}
\usetikzlibrary{arrows}

\typeout{CONTENT START NOW}

\begin{document}

\maketitle

\section{Logics}

% exactly as in jair-main
\begin{definition}[Forward (Guarded) Fragment]
Let FF$(n)$ be the smallest fragment of FO satisfying:

\begin{itemize}
    \item an atom $\alpha(\bar{x})$ belongs to FF$(n)$ iff $\alpha$ is equality-free and $\bar{x}$ is an infix of $\bar{x}_{1\ldots{}n}$
    \item FF$(n)$ is closed under boolean connectives $\land, \lor, \neg, \rightarrow, \leftrightarrow$.
    \item If $\phi(\bar{x}_{1\ldots{}n+1}$ is in FF$(n+1)$, then $\exists{x_{n+1}}\phi(\bar{x}_{1\ldots{}n+1})$ and $\forall{x_{n+1}}\phi(\bar{x}_{1\ldots{}n+1})$ are in FF$(n)$.
\end{itemize}

The \emph{forward fragment} FF is the set FF$(0)$.
The \emph{forward guarded fragment} FGF is the \fxfatal*{Allow variable reuse}{intersection of FF with GF}.
\end{definition}

% as in jair-main.pdf
\begin{definition}[Forward type]
A $(\sigma,n)$-forward type is a maximally consistent conjunction of atoms of the form $\pm{}R(\bar{x}_{i\ldots{}i+ar(R)-1}$ for an index $1 \leq i \leq n - ar(R) + 1$.
\end{definition}

We denote by $\mathrm{ftp}_\mathfrak{A}(\bar{a})$ the unique $(\sigma,|\bar{a}|)$-forward type such that $\mathfrak{A} \models \mathrm{ftp}_\mathfrak{A}(\bar{a})$

\section{Back-and-forth conditions / Bisimulations / Games}

% adapted from Otto 2004
\begin{definition}[Connected GF back-and-forth conditions]
Let $\mathfrak{A}$ and $\mathfrak{B}$ be $\sigma$-structures, $\mathcal{Z}, \mathcal{Z'} \subseteq \mathtt{Part}(\mathfrak{A}, \mathfrak{B})$ sets of partial isomorphisms between $\mathfrak{A}$ and $\mathfrak{B}$.

$\mathcal{Z'}$ satisfies the connected GF back-and-forth conditions for $\mathcal{Z}$ if for every $p \in \mathcal{Z}$ the following holds:

\begin{description}
    \item[(gforth)] For any guarded subset $s'$ of $A$ with $s' \cap \mathtt{dom}(p)$ nonempty, there is some $p' \in \mathcal{Z}$ with $\mathtt{dom}(p') = s'$ such that $p$ and $p'$ agree on their common domain.
    \item[(gback)] For any guarded subset $t'$ of $B$ with $t' \cap \mathtt{im}(p)$ nonempty, there is some $p' \in \mathcal{Z}$ with $\mathtt{im}(p') = t'$ such that $p^{-1}$ and $p'^{-1}$ agree on their common domain.
\end{description}
\end{definition}

% adapted from jair-main.pdf
\begin{definition}[Connected FGF back-and-forth conditions]
Let $\mathfrak{A}$ and $\mathfrak{B}$ be $\sigma$-structures, $\mathcal{Z}, \mathcal{Z'} \subseteq \bigcup_{i=0}(A^i \times B^i)$ be mappings such that for any $(\bar{a}, \bar{b}) \in \mathcal{Z} \cup \mathcal{Z'}$ forward types are preserved: $\mathrm{ftp}_\mathfrak{A}(\bar{a}) = \mathrm{ftp}_\mathfrak{B}(\bar{b})$ (\textbf{atomic harmony}).

$\mathcal{Z'}$ satisfies the connected FGF back-and-forth conditions for $\mathcal{Z}$ if for every $(\bar{a}, \bar{b}) \in \mathcal{Z}$ the following conditions hold:

\begin{description}
    \item[(fgforth)] For a nonempty affix $\bar{a}_{i\ldots{}j}$ of $\bar{a}$ and a $\sigma$-live tuple $\bar{c}$ with $|c| \leq k$ in $\mathfrak{A}$ such that $\bar{a}_{i\ldots{}j} = \bar{c}_{1\ldots{}j-i+1}$ there is a tuple $\bar{d}$ with $\bar{b}_{i\ldots{}j} = \bar{d}_{1\ldots{}j-i+1}$ and $(\bar{c}, \bar{d}) \in \mathcal{Z'}$
    \item[(fgback)] For a nonempty affix $\bar{b}_{i\ldots{}j}$ of $\bar{b}$ and a $\sigma$-live tuple $\bar{d}$ with $|d| \leq k$ in $\mathfrak{B}$ such that $\bar{b}_{i\ldots{}j} = \bar{d}_{1\ldots{}j-i+1}$ there is a tuple $\bar{c}$ with $\bar{a}_{i\ldots{}j} = \bar{c}_{1\ldots{}j-i+1}$ and $(\bar{c}, \bar{d}) \in \mathcal{Z'}$
\end{description}
\end{definition}

\begin{definition}[Connected Bisimulation]
A set $\mathcal{Z}$ is a connected GF or FGF bisimulation between pointed $\sigma$-structures $\mathfrak{A}, \bar{a}$ and $\mathfrak{B}, \bar{b}$ if it satisfies the connected GF respective FGF back-and-forth conditions for itself and $(\bar{a}, \bar{b}) \in \mathcal{Z}$.
\end{definition}

We write $\mathfrak{A}, \bar{a} \sim_{\textrm{GF}} \mathfrak{B}, \bar{b}$ or $\mathfrak{A}, \bar{a} \sim_{\textrm{FGF}} \mathfrak{B}, \bar{b}$ if there exists a connected GF respective FGF bisimulation between pointed $\sigma$-structures $\mathfrak{A}, \bar{a}$ and $\mathfrak{B}, \bar{b}$.

\begin{definition}[k-Level Bisimulation]
A k-level bisimulation is a sequence of sets $\mathcal{Z}_0, \ldots, \mathcal{Z}_k$ such that each $\mathcal{Z}_{i - 1}$ for $0 < i \le k$ satisfies the associated back-and-forth conditions for $\mathcal{Z}_i$.
\end{definition}

We write $\mathfrak{A}, \bar{a} \sim_{\textrm{GF}, k} \mathfrak{B}, \bar{b}$ or $\mathfrak{A}, \bar{a} \sim_{\textrm{FGF}, k} \mathfrak{B}, \bar{b}$ if there exists a k-level connected GF respective FGF bisimulation between pointed $\sigma$-structures $\mathfrak{A}, \bar{a}$ and $\mathfrak{B}, \bar{b}$.

\begin{definition}[Global Bisimulation]
A global bisimulation between $\sigma$-structures $\mathfrak{A}$ and $\mathfrak{B}$ is a set $\mathcal{Z}$ such that for any guarded tuple $\bar{a} \in \mathfrak{A}$ there is a guarded tuple $\bar{b} \in \mathfrak{B}$ such that $(\bar{a}, \bar{b}) \in \mathcal{Z}$ and vice versa.
\end{definition}

\pagebreak

\section{Tree unraveling}

\begin{definition}[Bisimulation Path]\label{def:bisim-path}
  A (connected, forward) \emph{bisimulation path} of \emph{length} $n$ with \emph{start} $\bar{s}_{0}$ for a structure \struct{A} is a sequence $\rho = \bar{s}_{0}(i_{1}, j_{1})\bar{s}_{1}\cdots{}(i_{n}, j_{n})\bar{s}_{n}$ where:
  \begin{itemize}
    \item each $\bar{s}_{k}$ for $0 \le k \le n$ is a guarded tuple from \struct{A}
    \item each $(i_{k}, j_{k})$ for $1 \le k \le n$ are a pair of indices into $\bar{s}_{k-1}$, such that $\bar{s}_{k-1,i\ldots{}j} = \bar{s}_{k,1\ldots{j-i+1}}$
  \end{itemize}
\end{definition}

\begin{definition}[Strict Bisimulation Path]
  A bisimulation path $\rho = \bar{s}_{0}(i_{1}, j_{1})\bar{s}_{1}\cdots{}(i_{n}, j_{n})\bar{s}_{n}$ is called strict iff:

  \begin{enumerate}
    \item \textbf{tail is not empty}: $j_{k}-i_{k}+1 < |s_{k}|$ for $1 \le k \le n$, and
    \item \textbf{previous tail is used}: $j_{k} > j_{k-1}-i_{k-1} + 1$ for $2 \le k \le n$
  \end{enumerate}

  The set of all strict bisimulation paths starting from a tuple $\bar{s}_{0}$ is denoted by $\mathtt{Paths}_{\bar{s}_{0}}(\mathfrak{A})$.
\end{definition}

\begin{definition}[Bisimulation Path Predecessor]
  For any bisimulation path $\rho = \bar{s}_{0}(i_{1}, j_{1})\bar{s}_{1}\cdots{}(i_{n}, j_{n})\bar{s}_{n}$, define $\mathtt{pred}(\rho) = \bar{s}_{0}(i_{1}, j_{1})\bar{s}_{1}\cdots{}(i_{n-1}, j_{n-1})\bar{s}_{n-1}$ as the path $\rho$ without the last step.
\end{definition}

\begin{samepage}
\begin{example}[Bisimulation paths]\leavevmode\\*[1em]\label{ex:fgf-paths}
  {%
    \newcommand{\tups}{{\color{tolbrightYellowDarker}\bar{s}}}%
    \newcommand{\tupp}{{\color{tolbrightCyanDarker}\bar{p}}}%
    \newcommand{\tupt}{{\color{tolbrightGreen}\bar{t}}}%
    \newcommand{\tupq}{{\color{tolbrightPurple}\bar{q}}}%
  \begin{minipage}[t]{0.6\textwidth}
    Bisimulation paths in \cref{fig:fgf-struct-a}:
    \begin{enumerate}
      \item $\rho_{1} = \tups$
      \item $\rho_{2} = \tups(2,2)\tupp$
      \item $\rho_{3} = \tups(2,2)\tupp(2,2)\tupt$
      \item $\rho_{4} = \tups(2,2)\tupp(2,2)\tupt(2,2)\tupq$
      \item $\rho_{5} = \tups(2,2)\tupp(2,2)\tupt(2,3)\tupq$, not strict (1)
      \item $\rho_{7} = \tups(2,2)\tupp(2,3)\tupt$
      \item $\rho_{7} = \tups(2,2)\tupp(2,3)\tupt(2,2)\tupq$
      \item $\rho_{8} = \tups(2,2)\tupp(2,3)\tupt(2,3)\tupq$, not strict (1)
      \item $\rho_{9} = \tups(2,2)\tupp(3,3)\tupq$
      \item $\rho_{10} = \tups(2,3)\tupp$
      \item $\rho_{11} = \tups(2,3)\tupp(2,2)\tupt$, not strict (2)
      \item $\rho_{12} = \tups(2,3)\tupp(2,2)\tupt(2,2)\tupq$, not strict (2)
      \item $\rho_{13} = \tups(2,3)\tupp(2,2)\tupt(2,3)\tupq$, not strict (1)
      \item $\rho_{14} = \tups(2,3)\tupp(2,3)\tupt$
      \item $\rho_{15} = \tups(2,3)\tupp(2,3)\tupt(2,2)\tupq$
      \item $\rho_{16} = \tups(2,3)\tupp(2,3)\tupt(2,3)\tupq$, not strict (1)
      \item $\rho_{17} = \tups(2,3)\tupp(3,3)\tupq$
      \item $\rho_{18} = \tups(3,3)\tupt$
      \item $\rho_{19} = \tups(3,3)\tupt(2,2)\tupq$
      \item $\rho_{20} = \tups(3,3)\tupt(2,3)\tupq$, not strict (1)
    \end{enumerate}
  \end{minipage}\hfill
  \begin{minipage}[t]{0.35\textwidth}
    \vspace{0pt}
    \includegraphics[width=0.9\textwidth]{svg/fgf-unravel.svg/example-struct-1}
    \captionof{figure}{Example structure \struct{A}}\label{fig:fgf-struct-a} \leavevmode\\[-0.5em]
    \raggedright
    $\tups = (1,2,3)$, $\tupp = (2,3,4)$, $\tupt = (3,4,5)$, $\tupq = (4,5)$. \\

    \begin{enumerate}[label=(\arabic*)]
      \item not strict because tail in step from $\tupt$ to $\tupq$ is empty
      \item not strict because step from $\tupp$ to $\tupt$ does not use previous tail
    \end{enumerate}
  \end{minipage}
  }
\end{example}
\end{samepage}

\begin{example}[Traces of paths]
  Traces of some paths from \cref{ex:fgf-paths}:
  {%
    \newcommand{\pA}{{\color{tolbrightYellowDarker}\rho_1}}%
    \newcommand{\pB}{{\color{tolbrightCyanDarker}\rho_2}}%
    \newcommand{\pC}{{\color{tolbrightGreen}\rho_{18}}}%
    \newcommand{\pD}{{\color{tolbrightPurple}\rho_{19}}}%
    \newcommand{\tups}{{\color{tolbrightYellowDarker}\bar{s}}}%
    \newcommand{\tupp}{{\color{tolbrightCyanDarker}\bar{p}}}%
    \newcommand{\tupt}{{\color{tolbrightGreen}\bar{t}}}%
    \newcommand{\tupq}{{\color{tolbrightPurple}\bar{q}}}%
    \begin{align*}
      \trace{\tups} &= (\pA,1)(\pA,2)(\pA,3) \\
      \trace{\tups(3,3)\tupt(2,2)} &= (\pA,1)(\pA,2)(\pA,3)(\pC,2)(\pC,3)\\
      \trace{\tups(3,3)\tupt(2,2)\tupq} &= (\pA,1)(\pA,2)(\pA,3)(\pC,2)(\pD,2)\\
      \trace{\tups(2,2)\tupp} &= (\pA,1)(\pA,2)(\pB,2)(\pB,3)
    \end{align*}
  }
\end{example}

\begin{samepage}
\begin{definition}[Trace of a path]
  Let $\rho = \bar{s}_{0}(i_{1}, j_{1})\bar{s}_{1}\cdots{}(i_{n}, j_{n})\bar{s}_{n}$ be a bisimulation path, and $\rho_{\ldots{}k} = \bar{s}_{0}\cdots(i_{k},j_{k})\bar{s}_{k}$ for $0 \le k \le n$ be the k-length prefix of that path.
  The trace of $\rho$ is a sequence of tuples with the following structure, describing the elements traversed by the path:
  \begin{equation*}
  \trace{\rho} = (\rho_{\ldots{}0}, 1)(\rho_{\ldots{}0}, 2)\cdots{}(\rho_{\ldots{}0}, b_{0})(\rho_{\ldots{}1}, a_{1})\cdots{}(\rho_{\ldots{}1}, b_{1})\cdots(\rho_{\ldots{}n}, a_{n})\cdots(\rho_{\ldots{}n}, b_{n})
  \end{equation*}
  where the indices $a_{k}$, $b_{k}$ marking the start and end of each segment are defined as follows:
  \begin{align*}
    a_{0} &= 1                   & b_{n} &= |s_{n}| & \text{edge cases} \\
    a_{k} &= j_{k} - i_{k} + 2    &  b_{k-1} &= j_{k} & \text{for}\ 1 \le k \le n
  \end{align*}

  Every element in a trace has a projection $\pi$ to an element of the original structure.
  This projection is given by:
  \begin{equation*}
    \pi(\rho_{\ldots{}k},x) = s_{k,x}\ \text{for every $0 \le k \le n$}
  \end{equation*}
  The projection $\pi(\trace{\rho})$ of the whole trace is the sequence of the projections of individual trace elements.

  The set of all traces starting from a tuple $\bar{s}_{0}$ is denoted by $\mathtt{Traces}_{\bar{s}_{0}}(\mathfrak{A})$.
\end{definition}
\end{samepage}

As $\mathtt{Traces}_{\bar{a}}$ is a prefix-closed set, there is the natural preorder $\tau_{1} \le \tau_{2}$ iff $\tau_{1}$ is a prefix of $\tau_{2}$.

\begin{observation}
  If $(\rho,i) \in \trace{\sigma}$ then $\rho \le \sigma$ and $(\rho, i) \in \trace{\rho}$.
\end{observation}

The context of a path represents the elements of the trace which are used in the last step, and is defined as follows:

\begin{definition}[Context of a path]
  For a bisimulation path $\rho = \bar{s}_{0}(i_{1}, j_{1})\bar{s}_{1}\cdots{}(i_{n}, j_{n})\bar{s}_{n}$, the context is defined as: $\context{\rho} = \text{tuple of the last}\ |s_{n}|\ \text{elements of}\ \trace{\rho}$
\end{definition}

Note that $\pi(\context{\rho (i,j) \bar{s}_{n}}) = \bar{s}_{n}$ for any path $\rho$.
We write $\context{\rho_{1}} \le \context{\rho_{2}}$ iff $\rho_{1} \le \rho_{2}$ which is well-defined since $\mathtt{context}$ is an injective mapping.

\begin{definition}[Bisimilarity of paths]
  Let $\mathfrak{A},\bar{a} \sim_{\textrm{FGF},k} \mathfrak{B}, \bar{b}$ be two bisimilar structures.
  We define $\sim_{\textrm{FGF},k}$ for bisimulation paths inductively as:
  \begin{align*}
    \bar{a} &\sim_{\textrm{FGF},k} \bar{b} & \text{base case} \\
    \rho (i,j) \bar{s} &\sim_{\textrm{FGF,k}} \sigma (i,j) \bar{t} & \text{if}\ \rho \sim_{\textrm{FGF,k}} \sigma \ \text{and}\ \mathfrak{A},\bar{s} \sim_{k-|\rho|} \mathfrak{B},\bar{t}
  \end{align*}
\end{definition}

\begin{observation}\label{obs:path-context-bisim}
  Let $\rho$ and $\sigma$ be bisimulation paths in structures $\mathfrak{A}$ and $\mathfrak{B}$.
  Then $\rho \sim_{\textrm{FGF},k} \sigma$ implies $\mathfrak{A}, \pi(\context{\rho}) \sim_{\textrm{FGF},k+1-|\rho|} \mathfrak{B}, \pi(\context{\sigma})$ and the same holds by induction for any n-length prefixes $\rho_{\ldots{}n}, \sigma_{\ldots{}n}$ of $\rho$ and $\sigma$.
\end{observation}

\begin{definition}[Path-complete tree unraveling]
  For a structure \struct{A}, let
  \begin{equation*}
    \text{Traces}_{\bar{a}}(\mathfrak{A}) = \left\{ \trace{\rho} \mid \text{$\rho$ is a path in}\ \mathfrak{A} \right\}
  \end{equation*}
  be the set of all traces starting from $\bar{a}$ in $\mathfrak{A}$.
  The path-complete unraveling $\mathfrak{A}^{*,c}_{\bar{a}}, \bar{a}^{*}$ is the structure with domain $\bigcup_{\tau \in \text{Traces}_{\bar{a}}(\mathfrak{A})} \{ x \in \tau \}$ (elements of traces) and each relation $R_{i}$ defined as:
  \begin{equation*}
    R_{i}^{\mathfrak{A}^{*,c}_{a_{0}}} = \left\{ \bar{r} \mid \pi(\bar{r}) \in R_{i}^{\mathfrak{A}}\ \text{and there exists a path $\rho$ with}\ \bar{r} \sqsubseteq \context{\rho}\right\}
  \end{equation*}
  The tuple $\bar{a}$ is lifed to the unraveling as $\bar{a}^{*} = \context{\bar{a}}$.
\end{definition}

\begin{definition}[Trace of element in tree unraveling]
  For $(\rho, k) \in A^{*,c}_{\bar{a}}$ define $\trace{(\rho, k)}$ as the prefix of $\trace{\rho}$ up to but including the element $(\rho, k)$.
\end{definition}

\begin{samepage}
\begin{example}\leavevmode\\
    \vspace{0pt}
    \includegraphics[width=0.9\textwidth]{svg/fgf-unravel.svg/struct-1-unravel}
    \captionof{figure}{Unraveling of structure \struct{A} from \cref{fig:fgf-struct-a}}\label{fig:fgf-struct-a-unravel}
    \vspace{0.5em}
\end{example}
\end{samepage}

\begin{lemma}[Elements of tree unraveling form a prefix order]
  Let $(\rho, i) \le (\sigma, j)$ iff $(\rho, i) \in \trace{(\sigma, j)}$ for elements $(\rho, i)$ and $(\sigma, j)$ of some tree unraveling.
  This order is a preorder.
\end{lemma}

\begin{observation}
  For all guarded tuples $\bar{s} \in A^{*,c}_{\bar{a}}$ we have $s_{k} < s_{k+1}$ for $1 \le k < |\bar{s}|$ because they are consecutive elements of some trace.
\end{observation}

The tuples of a tree unraveling are also continuous, meaning that for $s_{i}, s_{j} \in \bar{s}$ with $i \le j$ also $x \in \bar{s}$ for all $s_{i} \le x \le s_{j}$.

\begin{lemma}\label{lemma:context-element-bounding}
  Let $\rho_{1}, \rho_{2} \in \mathtt{Paths}_{\bar{a}}(\mathfrak{A})$.
  If $\rho_{1} \le \rho_{2}$ then for any $e_{2} \in \context{\rho_{2}}$ there is $e_{1} \in \context{\rho_{1}}$ with $e_{1} \le e_{2}$.
\end{lemma}
\begin{proof}
  By induction over bisimulation paths of increasing length.
  Note that a contetxt's first element $\context{\rho_{2}}_{1}$ (which is the minimal element of the context) is always in $\context{\mathtt{pred}(\rho_{2})}$ by definition of bisimulation paths.
\end{proof}

\begin{samepage}
\begin{lemma}[Intersection of contexts]\label{lemma:context-intersection}
  Let $\rho, \rho'$ be bisimulation paths with the same starting tuple in some structure $\mathfrak{A}$.
  Then the intersection $\bar{\imath} = \context{\rho} \cap \context{\rho'}$ is an infix of both $\context{\rho}, \context{\rho'}$ and at least one of the following is true:

  \begin{enumerate}[label=a)]
    \item $\rho \le \rho'$ and $\bar{\imath}$ is a prefix of $\context{\rho'}$
    \item there exists a bisimulation path $\gamma$ with $\gamma < \rho$ and $\bar{\imath} \sqsubseteq \context{\gamma}$.
  \end{enumerate}
\end{lemma}
\end{samepage}

\begin{figure}
  \begin{subfigure}[b]{0.45\textwidth}
    \begin{tikzpicture}
  \draw[tolbrightYellow, line cap=round, line width=3em] (0,4em) -- (0,-4em) -- ++(6em, -6em);
  \draw[tolbrightGreen, line cap=round, line width=2em] (-0em,8em) -- ++(0,-15em) -- ++(-2em, -2em);

  \draw [black, line width=0.1em, fill=white] (0em, 8em) circle [radius=0.8em] node[anchor=center] {$c$};
  \draw [black, line width=0.1em, fill=white] (0em, 4em) circle [radius=0.8em] node[anchor=center] {$a$};
  \draw [black, line width=0.1em, fill=white] (0em, 0em) circle [radius=0.8em] node[anchor=center] {};
  \draw [black, line width=0.1em, fill=white] (0em, -4em) circle [radius=0.8em] node[anchor=center] {$b$};
  \draw [black, line width=0.1em, fill=white] (3em, -7em) circle [radius=0.8em] node[anchor=center] {};
  \draw [black, line width=0.1em, fill=white] (6em, -10em) circle [radius=0.8em] node[anchor=center] {};
  \draw [black, line width=0.1em, fill=white] (-2em, -9em) circle [radius=0.8em] node[anchor=center] {};

  \path[tolbrightYellowDarker] (4em, -13em) node[anchor=south west] { $\mathtt{context}(\rho')$ };
  \path[tolbrightGreen] (-4em, -10.75em) node[anchor=west] { $\mathtt{context}(\rho = \gamma)$ };
\end{tikzpicture}

    \caption{$\rho = \gamma \le \rho'$}
  \end{subfigure}{\unskip\ \vrule\ }
  \begin{subfigure}[b]{0.5\textwidth}
    \begin{tikzpicture}
  \draw[tolbrightCyan, line cap=round, line width=5em] (0,0em) -- ++(0,-4em) -- ++(3em, -3em);
  \shade[shading=axis, bottom color=tolbrightCyan, top color=white] (-2.5em,0em) rectangle (2.5em, 13em);
  \draw[tolbrightYellow, -round cap, line width=3em] (0,7em) -- (0,-4em) -- ++(7em, -7em);
  \shade[shading=axis, bottom color=tolbrightYellow, top color=white, middle color=tolbrightYellow!20!white] (-1.5em, 7em) rectangle (1.5em, 10em) arc [radius=1.5em, start angle=0, delta angle=180];
  %\shade[shading=axis, bottom color=tolbrightYellow, top color=white, top color=tolbrightYellow!20!white] (1.5em, 10em)
  \draw[tolbrightGreen, line cap=round, line width=2em] (-0em,5em) -- ++(0,-12em) -- ++(-2em, -2em);

  \draw [black, line width=0.1em, fill=white] (0em, 11em) circle [radius=0.8em] node[anchor=center] {};
  \draw [black, line width=0.1em, fill=white] (0em, 8em) circle [radius=0.8em] node[anchor=center] {$c$};
  \draw [black, line width=0.1em, fill=white] (0em, 4em) circle [radius=0.8em] node[anchor=center] {$a$};
  \draw [black, line width=0.1em, fill=white] (0em, 0em) circle [radius=0.8em] node[anchor=center] {};
  \draw [black, line width=0.1em, fill=white] (0em, -4em) circle [radius=0.8em] node[anchor=center] {$b$};
  \draw [black, line width=0.1em, fill=white] (3em, -7em) circle [radius=0.8em] node[anchor=center] {};
  \draw [black, line width=0.1em, fill=white] (6em, -10em) circle [radius=0.8em] node[anchor=center] {};
  \draw [black, line width=0.1em, fill=white] (-2em, -9em) circle [radius=0.8em] node[anchor=center] {};

  \path[tolbrightCyan] (3em, -1em) node[anchor=west] { $\mathtt{context}(\gamma)$ };
  \path[tolbrightYellowDarker] (4em, -13em) node[anchor=south west] { $\mathtt{context}(\rho')$ };
  \path[tolbrightGreen] (-4em, -10.75em) node[anchor=west] { $\mathtt{context}(\rho)$ };
\end{tikzpicture}

    \caption{$\gamma < \rho, \rho'$}
  \end{subfigure}
  \caption{Intersection of contexts}\label{fig:fgf-context-intersect}
\end{figure}

The following proof is illustrated in \cref{fig:fgf-context-intersect}.
\begin{proof}
  Let $a$ be the minimal and $b$ be the maximal element of the intersection $\bar{\imath}$.
  Because contexts are continuous, both $\context{\rho}$ and $\context{\rho'}$ contain all elements between $a$ and $b$ as an infix.
  Let $\gamma$ be the smallest bisimulation path such that $b \in \context{\gamma}$.
  Note that $\gamma \le \rho, \rho'$ since $b \in \context{\rho}, \context{\rho'}$.
  There are two cases:

  \paragraph{a) $\gamma = \rho$}
  Then we have $\rho = \gamma \le \rho'$.
  By \cref{lemma:context-element-bounding} we know that there must be an element $c \in \context{\rho}$ such that $c \le \context{\rho'}_{1}$.
  Thus $c \le \context{\rho'}_{1} \le b$ and $c, b \in \context{\rho}$.
  Because contexts are continuous, this implies that the $\context{\rho'}_{1}$ is in $\context{\rho}$ showing that the intersection of both is a prefix of $\context{\rho'}_{1}$.

  \paragraph{b) $\gamma < \rho$}
  By \cref{lemma:context-element-bounding} since $\gamma < \rho$ and $a \in \context{\rho}$ there must a $c \in \context{\gamma}$ such that $c \le a$.
  Because $c \le a \le b$ with $c, b \in \context{\gamma}$ we get $a \in \context{\gamma}$ so the whole intersection is an infix of $\context{\gamma}$.
\end{proof}

\begin{lemma}\label{lemma:fgf-context-path-length}
  For strict bisimulation paths $\rho, \rho'$ if $\context{\rho}$ and $\context{\rho'}$ share at least one element, then $|\rho'| \le |\rho| + w$ where $w$ is the width of the signature.
\end{lemma}
\begin{proof}
  Let $\context{\rho'} = (s_{1}, \ldots, s_{n})$ where we know that $n \le w$.
  Let $s_{i} = (\gamma, k)$ be an element that $\rho$ and $\rho'$ share.
  Because $\rho'$ is a strict bisimulation path, we have that for any two elements $s_{j} = (\sigma_{j}, v_{i})$ and $s_{j+1} = (\sigma_{j+1}, v_{j+1})$ always $|\sigma_{j+1}| \le |\sigma_{j}| + 1$ since they are consecutive elements of $\trace{\rho'}$.
  Thus $|\rho'| \le |\gamma| + (n - i) \le |\gamma| + w$.
  From $\gamma \le \rho$ (since $s_{i} \in \context{\rho}$) it follows that $|\gamma| \le |\rho|$ so we obtain the final bound $|\rho'| \le |\rho| + w$.
\end{proof}

\begin{theorem}[$\textrm{FGF}_{k}$ to $\textrm{GF}_{k'}$]
  Let $\mathfrak{A}, \bar{a} \sim_{\textrm{FGF}_{k}}\mathfrak{B}, \bar{b}$ be two connected-forward bisimilar $\Sigma$-structures.
  Then $\mathfrak{A}^{*,c}_{\bar{a}}, \bar{a}^{*} \sim_{\textrm{GF},k'} \mathfrak{B}^{*,c}_{\bar{b}}, \bar{b}^{*}$ for some $k'$ only dependent on $k$ and $\Sigma$.
\end{theorem}

\begin{proof}
  Let $w$ be the width of the signature of $\mathfrak{A}$.
  We show prove the above theorem for $k = k' * w$ where the bisimulation is given by the sequence of sets $\mathcal{Z}_{0}, \mathcal{Z}_{1}, \ldots, \mathcal{Z}_{k'}$ defined as follows:
  \begin{equation*}
    \mathcal{Z}_{h} = \left\{
      (\bar{s}, \bar{t})
      \ \middle|\
      \begin{aligned}
        & \text{there are strict bisimulation paths $\rho$, $\sigma$ such that:} \\
        & \ \text{(1)}\ \rho \sim_{\textrm{FGF}, w*k'} \sigma \\
        & \ \text{(2)}\ \bar{s} = \context{\rho}_{x\ldots{}y}\ \text{and}\ \bar{t} = \context{\sigma}_{x\ldots{}y}\ \text{for some}\ x, y \\
        & \ \text{(3)}\ |\rho|, |\sigma| \le 1 + w * (k' - h)
      \end{aligned}
    \right\}
  \end{equation*}

  First we show for any $(\bar{s}, \bar{t}) \in \mathcal{Z}_{h}$ that $\bar{s}, \bar{t}$ induce isomorphic substructures, so the mapping $\bar{s}_{i} \mapsto \bar{t}_{i}$ for $1 \le i \le |\bar{s}|$ is a partial isomorphism.
  Let $\rho$ and $\sigma$ be the paths for $\bar{s}$ and $\bar{t}$ as in the definition of $\mathcal{Z}_{h}$.
  Condition (a) implies by \cref{obs:path-context-bisim} that $\mathfrak{A}, \pi(\context{\rho}) \sim_{FGF,w*k'-|\rho|+1} \mathfrak{B}, \pi(\context{\sigma})$.
  By the atomic harmony property of FGF bisimulations this means that $\pi(\context{\rho})$ and $\pi(\context{\sigma})$ must have equal forward types.
  This also holds for $\context{\rho}$ and $\context{\sigma}$ since by definition of the unraveling those have the same forward type as the projection.
  Because elements of the unraveling are ordered, substructures with equal forward types are isomorphic.
  As $\bar{s}$ and $\bar{t}$ are simply slices of $\context{\rho}$ and $\context{\sigma}$ they must be isomorphic substructes as well.

  Next we give the proof for \texttt{(gforth)} and \texttt{(gback)} follows by symmetry.
  Let $(\bar{s}, \bar{t}) \in \mathcal{Z}_{h}$ and $\bar{s'} \in \mathfrak{A}^{*,c}_{\bar{a}}$ be a guarded tuple sharing at least one element with $\bar{s}$.
  Let $\rho$ and $\sigma$ be the strict bisimulation paths for $\bar{s}$ and $\bar{t}$ as in the definition of $Z_{h}$.
  Since $\bar{s}'$ is guarded there is also a strict bisimulation path $\rho'$ such that $\bar{s}' \sqsubseteq \context{\rho'}$.
  We show that there is a path $\sigma'$ with $|\sigma'| \le |\sigma| + w$ such that $\rho' \sim_{\textrm{FGF}, w*k'} \sigma'$ and whenever $\context{\rho}_{v} = \context{\rho'}_{v}$ for some indices $v,v'$ then also $\context{\sigma}_{v} = \context{\sigma'}_{v'}$.

  Following \cref{lemma:context-intersection} we differentiate two cases for the common elements $\bar{c} = \context{\rho} \cap \context{\rho'}$:

  \begin{description}[left=0pt]
  \item[a) $\rho \le \rho'$ and $\bar{c}$ is a prefix of $\context{\rho'}$] Let $\rho' = \rho(i_{1}, j_{1})\bar{x}_{1} \cdots (i_{l}, j_{l})\bar{x}_{l}$.
  Then we can find $\bar{y}_{1},\ldots,\bar{y}_{l}$ such that $\sigma' = \sigma(i_{1}, j_{1})\bar{y}_{l}\cdots(i_{l},j_{l})\bar{y}_{l}$ and $\rho' \sim_{\mathrm{FGF,w*k'}} \sigma'$ by applying $l$ times the (fgforth) property of $\mathfrak{A}$.
  Because the sequence of indices $(i_{l},j_{l})$ is the same for both $\rho'$ and $\sigma'$ we get $\context{\sigma}_{v} = \context{\sigma'}_{v'}$ when $\context{\rho}_{v} = \context{\rho'}_{v'}$ as required.
  Since $\context{\sigma}$ and $\context{\sigma'}$ share at least one element \cref{lemma:fgf-context-path-length} guarantees that $|\sigma'| \le |\sigma| + w$.

  \item[b) exists path $\gamma < \rho$ with $\bar{c} \sqsubseteq \context{\gamma}$] Let $\rho = \gamma (i_{1}, j_{1})\bar{x}_{1}\cdots(i_{l}, j_{l})\bar{x}_{l}$.
  From condition (1) $\rho \sim_{\mathrm{FGF}, w*k'} \sigma$ it follows that $\sigma = \delta(i_{1}, j_{1})\cdots(i_{l}, j_{l})\bar{x}_{l}$ for some bisimulation path $\delta$ with $\gamma \sim_{\mathrm{FGF, w*k'}} \delta$ and thus $(\context{\gamma}, \context{\delta} \in \mathcal{Z}_{h}$.
  We assume by induction that we have a $\sigma'$ such that $\context{\delta}_{v} = \context{\sigma'}_{v'}$ if $\context{\gamma}_{v} = \context{\rho'}_{v'}$.
  Because $\gamma$ is strictly smaller than $\rho$ this induction terminates.
  Since all the shared elements between $\context{\rho}$ and $\context{\rho'}$ are also in $\context{\gamma}$ and the sequence of indices $(i_{l}, j_{l})$ is exactly the same for both $\rho$ and $\sigma$ this implies that $\context{\sigma}_{v} = \context{\sigma'}_{v}$ if $\context{\rho}_{v} = \context{\rho}_{v'}$.
  \end{description}

  We have now found a path $\sigma'$ such that every element shared by $\context{\rho}$ and $\context{\rho'}$ is also shared by $\context{\sigma}$ and $\context{\sigma'}$.
  Let $i',j'$ be indices such that $\bar{s}' = \context{\rho'}_{i'\ldots{}j'}$ and then define $t' = \context{\rho'}_{i'\ldots{}j'}$.
  Clearly $(\bar{s}', \bar{t}') \in \mathcal{Z}_{h+1}$ with bisimulation paths $\rho', \sigma'$.
  $\bar{t}$ also shares the same elements with $\bar{t}'$ as $\bar{s}$ does with $\bar{s}'$ because the contexts for $\rho, \sigma$ and $\rho', \sigma'$ do and the tuples are simply infixes of those contexts at the same positions.
\end{proof}

\end{document}
