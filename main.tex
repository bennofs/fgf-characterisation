% -*- TeX-engine: xetex -*-
\documentclass[draft]{scrartcl}

\usepackage{ifxetex}
\usepackage{amsmath}

\ifxetex
  \usepackage{polyglossia}
  \usepackage{fontspec}

  \setmainlanguage{english}
\else
  \usepackage[english]{babel}
\fi

\usepackage{blindtext}
\usepackage{amsthm}
\usepackage{amssymb}
\usepackage{amsmath}
\usepackage{hyperref}
\usepackage[capitalise]{cleveref}
\usepackage{booktabs}
\usepackage{tikz}
\usepackage{tabularx}
\usepackage{makecell}
\usepackage{placeins}
\usepackage{listings}
\usepackage{caption}
\usepackage{subcaption}
\usepackage{fixme}
\usepackage{graphicx}
\usepackage[shortlabels]{enumitem}
\usepackage{pbox}
\usepackage{mathtools}

\setkeys{Gin}{draft=false} %load images in draft mode
\fxsetup{theme=color,status=draft}

\title{FGF Notes}
\author{Benno Fünfstück}
\date{\today}

\newtheorem{theorem}{Theorem}
\theoremstyle{definition}
\newtheorem{definition}[theorem]{Definition}
\newtheorem{lemma}[theorem]{Lemma}
\newtheorem{example}[theorem]{Example}
\newtheorem{observation}[theorem]{Observation}
\crefname{lemma}{Lemma}{Lemmas}
\crefname{definition}{Definition}{Definitions}
\crefname{theorem}{Theorem}{Theorems}
\crefname{observation}{Observation}{Observations}

\newcommand{\first}[1]{\mathtt{first}(#1)}
\newcommand{\last}[1]{\mathtt{last}(#1)}
\newcommand{\trace}[1]{\mathtt{trace}(#1)}
\newcommand{\context}[1]{\mathtt{context}(#1)}
\newcommand{\str}[1]{\mathfrak{#1}}
\newcommand{\seqset}[1]{\mathtt{Seq}_{#1}}
\newcommand{\seq}[1]{\mathtt{seq}({#1})}
\newcommand{\num}[1]{\mathtt{num}({#1})}
\newcommand{\dom}[1]{\mathtt{dom}({#1})}
\newcommand{\dist}[2]{\mathtt{dist}({#1},{#2})}
\newcommand{\nextrel}[2]{\mathit{Next}({#1},{#2})}
\newcommand{\lift}[1]{\mathtt{lift}({#1})}
\newcommand{\sij}{_{i\ldots{}j}}

\definecolor{tolbrightBlue}{HTML}{4477AA}
\definecolor{tolbrightRed}{HTML}{EE6677}
\definecolor{tolbrightGreen}{HTML}{228833}
\definecolor{tolbrightYellow}{HTML}{CCBB44}
\colorlet{tolbrightYellowDarker}{tolbrightYellow!70!black}
\definecolor{tolbrightCyan}{HTML}{66CCEE}
\colorlet{tolbrightCyanDarker}{tolbrightCyan!70!black}
\definecolor{tolbrightPurple}{HTML}{AA3377}
\definecolor{tolbrightGrey}{HTML}{BBBBBB}
\definecolor{tolhighcontrastBlue}{HTML}{004488}
\definecolor{tolhighcontrastYellow}{HTML}{DDAA33}
\definecolor{tolhighcontrastRed}{HTML}{BB5566}
\definecolor{tolvibrantOrange}{HTML}{EE7733}
\definecolor{tolvibrantBlue}{HTML}{0077BB}
\definecolor{tolvibrantCyan}{HTML}{33BBEE}
\definecolor{tolvibrantMagenta}{HTML}{EE3377}
\definecolor{tolvibrantRed}{HTML}{CC3311}
\definecolor{tolvibrantTeal}{HTML}{009988}
\definecolor{tolvibrantGrey}{HTML}{BBBBBB}
\definecolor{tolmutedRose}{HTML}{CC6677}
\definecolor{tolmutedIndigo}{HTML}{332288}
\definecolor{tolmutedSand}{HTML}{DDCC77}
\definecolor{tolmutedGreen}{HTML}{117733}
\definecolor{tolmutedCyan}{HTML}{88CCEE}
\definecolor{tolmutedWine}{HTML}{882255}
\definecolor{tolmutedTeal}{HTML}{44AA99}
\definecolor{tolmutedOlive}{HTML}{999933}
\definecolor{tolmutedPurple}{HTML}{AA4499}
\definecolor{tolmutedPalegrey}{HTML}{DDDDDD}
\definecolor{tolmediumcontrastLightblue}{HTML}{6699CC}
\definecolor{tolmediumcontrastDarkblue}{HTML}{004488}
\definecolor{tolmediumcontrastLightyellow}{HTML}{EECC66}
\definecolor{tolmediumcontrastDarkred}{HTML}{994455}
\definecolor{tolmediumcontrastDarkyellow}{HTML}{997700}
\definecolor{tolmediumcontrastLightred}{HTML}{EE99AA}
\definecolor{tolpalePaleblue}{HTML}{BBCCEE}
\definecolor{tolpalePalered}{HTML}{FFCCCC}
\definecolor{tolpalePalegreen}{HTML}{CCDDAA}
\definecolor{tolpalePaleyellow}{HTML}{EEEEBB}
\definecolor{tolpalePalecyan}{HTML}{CCEEFF}
\definecolor{tolpalePalegrey}{HTML}{DDDDDD}
\definecolor{toldarkDarkblue}{HTML}{222255}
\definecolor{toldarkDarkred}{HTML}{663333}
\definecolor{toldarkDarkgreen}{HTML}{225522}
\definecolor{toldarkDarkyellow}{HTML}{666633}
\definecolor{toldarkDarkcyan}{HTML}{225555}
\definecolor{toldarkDarkgrey}{HTML}{555555}
\definecolor{tollightLightblue}{HTML}{77AADD}
\definecolor{tollightOrange}{HTML}{EE8866}
\definecolor{tollightLightyellow}{HTML}{EEDD88}
\definecolor{tollightPink}{HTML}{FFAABB}
\definecolor{tollightLightcyan}{HTML}{99DDFF}
\definecolor{tollightMint}{HTML}{44BB99}
\definecolor{tollightPear}{HTML}{BBCC33}
\definecolor{tollightOlive}{HTML}{AAAA00}
\definecolor{tollightPalegrey}{HTML}{DDDDDD}

\usetikzlibrary{calc}
\usetikzlibrary{backgrounds}
\usetikzlibrary{arrows}
\usetikzlibrary{arrows.meta}

\typeout{CONTENT START NOW}

\begin{document}

\maketitle

\section{Logics}

% exactly as in jair-main
\begin{definition}[Forward (Guarded) Fragment]
Let FF$(n)$ be the smallest fragment of FO satisfying:

\begin{itemize}
  \item an atom $\alpha(\bar{x})$ belongs to FF$(n)$ iff $\alpha$ is equality-free and $\bar{x}$ is an infix of $\bar{x}_{1\ldots{}n}$
  \item $\mathrm{FF}(n-1) \subseteq \mathrm{FF}(n)$
  \item FF$(n)$ is closed under boolean connectives $\land, \lor, \neg, \rightarrow, \leftrightarrow$.
  \item If $\phi(\bar{x}_{1\ldots{}n+1}$ is in FF$(n+1)$, then $\exists{x_{n+1}}\phi(\bar{x}_{1\ldots{}n+1})$ and $\forall{x_{n+1}}\phi(\bar{x}_{1\ldots{}n+1})$ are in FF$(n)$.
\end{itemize}

The \emph{forward fragment} FF is the set FF$(0)$.
The \emph{forward guarded fragment} FGF is the intersection of FF with GF.
\end{definition}

% as in jair-main.pdf
\begin{definition}[Forward type]
A $(\sigma,n)$-forward type is a maximally consistent conjunction of atoms of the form $\pm{}R(\bar{x}_{i\ldots{}i+ar(R)-1}$ for an index $1 \leq i \leq n - ar(R) + 1$.
\end{definition}

We denote by $\mathrm{ftp}_\mathfrak{A}(\bar{a})$ the unique $(\sigma,|\bar{a}|)$-forward type such that $\mathfrak{A} \models \mathrm{ftp}_\mathfrak{A}(\bar{a})$

\section{Back-and-forth conditions / Bisimulations / Games}

% adapted from Otto 2004
\begin{definition}[GF back-and-forth conditions]
Let $\mathfrak{A}$ and $\mathfrak{B}$ be $\sigma$-structures, $\mathcal{Z}, \mathcal{Z'} \subseteq \mathtt{Part}(\mathfrak{A}, \mathfrak{B})$ sets of partial isomorphisms between $\mathfrak{A}$ and $\mathfrak{B}$.

$\mathcal{Z'}$ satisfies the GF back-and-forth conditions for $\mathcal{Z}$ if for every $p \in \mathcal{Z}$ the following holds:

\begin{description}
    \item[(gforth)] For any guarded subset $s'$ of $A$, there is some $p' \in \mathcal{Z}$ with $\mathtt{dom}(p') = s'$ such that $p$ and $p'$ agree on their common domain.
    \item[(gback)] For any guarded subset $t'$ of $B$, there is some $p' \in \mathcal{Z}$ with $\mathtt{im}(p') = t'$ such that $p^{-1}$ and $p'^{-1}$ agree on their common domain.
\end{description}
\end{definition}

% adapted from jair-main.pdf
\begin{definition}[FGF back-and-forth conditions]
Let $\mathfrak{A}$ and $\mathfrak{B}$ be $\sigma$-structures, $\mathcal{Z}, \mathcal{Z'} \subseteq \bigcup_{i=0}(A^i \times B^i)$ be mappings such that for any $(\bar{a}, \bar{b}) \in \mathcal{Z} \cup \mathcal{Z'}$ forward types are preserved: $\mathrm{ftp}_\mathfrak{A}(\bar{a}) = \mathrm{ftp}_\mathfrak{B}(\bar{b})$ (\textbf{atomic harmony}).

$\mathcal{Z'}$ satisfies the FGF back-and-forth conditions for $\mathcal{Z}$ if for every $(\bar{a}, \bar{b}) \in \mathcal{Z}$ the following conditions hold:

\begin{description}
    \item[(fgforth)] For an infix $\bar{a}_{i\ldots{}j}$ of $\bar{a}$ and a $\sigma$-live tuple $\bar{c}$ with $|c| \leq k$ in $\mathfrak{A}$ such that $\bar{a}_{i\ldots{}j} = \bar{c}_{1\ldots{}j-i+1}$ there is a tuple $\bar{d}$ with $\bar{b}_{i\ldots{}j} = \bar{d}_{1\ldots{}j-i+1}$ and $(\bar{c}, \bar{d}) \in \mathcal{Z'}$
    \item[(fgback)] For an infix $\bar{b}_{i\ldots{}j}$ of $\bar{b}$ and a $\sigma$-live tuple $\bar{d}$ with $|d| \leq k$ in $\mathfrak{B}$ such that $\bar{b}_{i\ldots{}j} = \bar{d}_{1\ldots{}j-i+1}$ there is a tuple $\bar{c}$ with $\bar{a}_{i\ldots{}j} = \bar{c}_{1\ldots{}j-i+1}$ and $(\bar{c}, \bar{d}) \in \mathcal{Z'}$
\end{description}
\end{definition}

\begin{definition}[Bisimulation]
A set $\mathcal{Z}$ is a GF or FGF bisimulation between $\sigma$-structures $\mathfrak{A}, \bar{a}$ and $\mathfrak{B}, \bar{b}$ if it satisfies the GF respective FGF back-and-forth conditions for itself and $(\bar{a}, \bar{b}) \in \mathcal{Z}$.
\end{definition}

We write $\mathfrak{A}, \bar{a} \sim_{\textrm{GF}} \mathfrak{B}, \bar{b}$ or $\mathfrak{A}, \bar{a} \sim_{\textrm{FGF}} \mathfrak{B}, \bar{b}$ if there exists a GF respective FGF bisimulation between $\sigma$-structures $\mathfrak{A}, \bar{a}$ and $\mathfrak{B}, \bar{b}$.

\begin{definition}[k-Level Bisimulation]
A k-level bisimulation is a sequence of sets $\mathcal{Z}_0, \ldots, \mathcal{Z}_k$ such that each $\mathcal{Z}_{i - 1}$ for $0 < i \le k$ satisfies the associated back-and-forth conditions for $\mathcal{Z}_i$ and sets $Z_{j}$ satisfy atomic harmony.
\end{definition}

\begin{lemma}
  If $\str{A}, \bar{a} \sim_{\mathrm{FGF},k} \str{B},\bar{b}$ then $\str{A}, \bar{a}$ and $\str{B}, \bar{b}$ satisfy the same FGF formulae of quantifier rank $\le k$.
\end{lemma}
\begin{proof}
  First consider quantifier-free (quantifier rank zero) formulae.
  If $\str{A}, \bar{a} \sim_{\mathrm{FGF},k} \str{B}, \bar{b}$ then $\bar{a}$ and $\bar{b}$ must have equal forward types by ``atomic harmony''.
  This means that they satisfy the same atomic formulae.
  By structural induction, this also holds for boolean combinations of formulae.
  Therefore, $\str{A}, \bar{a}$ and $\str{B}, \bar{b}$ satisfy the same quantifier-free FGF-formulae.

  Now by strong induction, let $\phi(x_{1}, \ldots, x_{n})$ be a FGF-formulae and assume the lemma is true for all $k$ less than the quantifier rank of $\phi$.
  Consider the case where $\phi(x_{1}, \ldots, x_{n}) = \exists{x_{(n+1)}\cdots{}x_{m}} \alpha(x_{i}, \ldots, x_{j}) \land \psi(x_{i}, \ldots, x_{j})$ ($\alpha$ is the atomic guard).
  Let $k$ be the quantifier rank of $\phi$, then $\alpha(x_{i},\ldots,x_{j}) \land \psi(x_{i}, x_{j})$ has quantifier rank $k - 1$.
  By symmetry, assume w.l.o.g. $\str{A}, (a_{1}, \ldots, a_{n}) \models \phi$.
  Thus there exist $a_{n+1}, \ldots, a_{j}$ such that $\str{A} \models \alpha(a_{i}, \ldots, a_{j}) \land \psi(a_{i}, \ldots, a_{j})$.
  As $(a_{i}, \ldots, a_{j})$ is guarded by $\alpha$, we can apply ``gforth'' to find $b_{n+1}, b_{j}$ with $\str{A}, (a_{i}, \ldots, a_{j}) \sim_{\mathrm{FGF},(k-1)} \str{B}, (b_{i}, \ldots, b_{j})$.
  By the induction hypothesis, $\str{B}, (b_{i}, \ldots, b_{j}) \models \alpha(b_{i}, \ldots, b_{j}) \land \psi(b_{i}, \ldots, b_{j})$ and therefore also $\str{B}, (b_{1}, \ldots, b_{n}) \models \phi$.

  Again, we have that the above also holds for all boolean combinations of formulae as the lemma is preserved under boolean combinations.
  As $\forall{\bar{x}} \alpha(\bar{x}) \rightarrow \psi(\bar{x}) = \neg \exists{\bar{x}} \alpha(\bar{x}) \land \neg \psi(\bar{x})$, this also handles the case of universal quantification.
\end{proof}

\begin{lemma}
  If $\str{A}$ and $\str{B}$ are $\omega$-saturated and $\str{A}, \bar{a}$ and $\str{B}, \bar{b}$ satisfy the same FGF formulae of quantifier rank $\le k$, then $\str{A}, \bar{a} \sim_{\mathrm{FGF}, k} \str{B}, \bar{b}$.
\end{lemma}
\begin{proof}
  We show that the sets $Z_{0}, \ldots, Z_{k}$ defined as follows form a valid $k$-level FGF bisimulation:
  \begin{equation*}
  \mathcal{Z}_{k} = \left\{ (\bar{a}, \bar{b}) \in \bigcup_{n < \omega} (A^{n} \times B^{n}) \mid \str{A}, \bar{a} \equiv_{\mathrm{FGF},k} \str{B},\bar{b} \right\}
  \end{equation*}
  Each $Z_{k}$ satisfies ``atomic harmony'' as $\str{A}, \bar{a} \equiv_{\mathrm{FGF},k} \str{B},\bar{b}$ implies that $\bar{a}$ and $\bar{b}$ satisfy the same atomic formulae.

  To show ``fgforth'', let $(\bar{a}, \bar{b}) \in Z_{k}$ and $\bar{c} = \bar{a}_{i\ldots{}j}\bar{e}$ be a live tuple in $\str{A}$.
  Consider the $|e|$-type (over $\bar{b}_{i\ldots{}j}$) of all tuples $\bar{f}$ where $\str{A}, \bar{a}_{i\ldots{}j}\bar{e} \equiv_{\mathrm{FGF},k-1} \str{B}, \bar{b}_{i\ldots{j}}\bar{f}$:
  \begin{align*}
    \Sigma^{\bar{b}_{i\ldots{j}}}(\bar{f}) = \left\{ \phi(\bar{b}_{i\ldots{}j}\bar{f}) \mid \phi(\bar{x}) \in {\mathrm{FGF}(|c|)}_{k-1}\ \text{and}\ \str{A}, \bar{c} \models \phi(\bar{x}) \right\}
  \end{align*}
  We claim that $\Sigma^{\bar{b}_{i\ldots{}j}}(\bar{f})$ is realized in $\str{B}$.
  By $\omega$-saturation and compactness it suffices to show that it is finitely satisfiable in $\str{B}$.
  Let $\Delta \subseteq \Sigma^{\bar{b}_{i\ldots{}j}}(\bar{f})$ be a finite subset.
  Then $\delta(\bar{x}_{i\ldots{}i+|c|}) = \bigwedge \Delta[\bar{b}_{i\ldots{}j} \mapsto \bar{x}_{i\ldots{}j}, \bar{f} \mapsto \bar{x}_{(j+1)\ldots{}(j+|e|)}]$ is a $\mathrm{FGF}_{k-1}$ formula.
  Since $\bar{c}$ is live, there must a relation $\alpha$ such that $\bar{A}, \bar{c} \models \alpha(x_{1\ldots{}|c|})$.
  We can now construct the $\mathrm{FGF}_{k}$-formula $\chi = \exists{x_{j+1}\cdots{}x_{j+|e|}} (\alpha(x_{i\ldots{}i+|c|}) \land \delta(\bar{x}_{i\ldots{}i+|c|}))$.
  We have $\str{A}, \bar{a} \models \chi$ (witnessed by $\bar{e}$) and thus also $\str{B}, \bar{b} \models \chi$ as they satisfy the same $\mathrm{FGF}_{k}$ formulae by assumption.
  So there must be $\bar{f}$ satisfying $\str{B}, \bar{b}_{i\ldots{}j}\bar{f} \models \delta(\bar{x}_{i\ldots{}i+|c|})$, which by construction means that $\str{B}, \bar{f} \models \Delta$.
\end{proof}

\pagebreak

\section{Tree unraveling}

\subsection{Infinite full tree unraveling}

\begin{definition}[Bisimulation Sequences]
  The set of bisimulation sequences $\seqset{\str{A}}$ for a structure $\str{A}$ consists of all sequences of the form $\bar{a}_{0}(i_{1}, j_{1})\bar{a}_{1}\cdots{}(i_{l},j_{l})\bar{a}_{l}$, where:
  \begin{enumerate}[(1)]
    \item $a_{i}$ is a live tuple in $\str{A}$ for every $0 \le i \le l$,
    \item for every $1 \le k \le l$, $(i_{k}, j_{k})$ is a pair of indices into $\bar{a}_{k-1}$ such that $\bar{a}_{k-1,i_{k}\ldots{}j_{k}} = \bar{a}_{k,1\ldots{\dist{i_{k}}{j_{k}}}}$,
    \item $\dist{i_{k}}{j_{k}} < |a_{k}|$ for $1 \le k \le l$, and
    \item $j_{k} > \dist{i_{k-1}}{j_{k-1}}$ for $2 \le k \le n$
  \end{enumerate}
  This set is partially ordered with $\sigma_{1} \le \sigma_{2}$ iff $\sigma_{1}$ is a prefix of $\sigma_{2}$ for two sequences $s_{1}, s_{2}$.
\end{definition}

\begin{definition}[Unraveling Domain]
  The unraveling domain $\mathtt{Dom}^{\rightarrow}_{\str{A}}$ of a structure $\str{A}$ is defined as:
  \begin{equation*}
    \mathtt{Dom}^{\rightarrow}_{\str{A}}
      = (\seqset{\str{A}} \times \mathbb{N})
        \setminus
        \left\{ (\sigma(i,j)\bar{a}, k) \mid k \le \dist{i}{j}\ \text{or}\ k > |a| \right\}
  \end{equation*}
  Let ``$\mathit{Next}$'' be the binary relation such that $((\rho, k), (\rho', k')) \in \mathit{Next}$ iff:
  \begin{enumerate}[(a)]
    \item $\rho' = \rho$ and $k' = k + 1$, or
    \item $\rho' = \rho (i,j) \bar{a}$ for some $i, j, \bar{a}$ and $k = j, k' = \dist{i}{j} + 1$
  \end{enumerate}
  Let $e = (\rho, k) \in \mathtt{Dom}^{\rightarrow}$ be a domain element and $\bar{a}$ be the tuple so that $\sigma = \cdots{}\bar{a}$.
  We will use the following notation:
  \begin{enumerate}
    \item $\pi(e) = \bar{a}_{k} \in A$ (the projection of $e$)
    \item $\seq{e} = \rho \in \seqset{\str{A}}$ (the sequence of $e$)
    \item $\num{e} = k \in \mathbb{N}$ (the counter of $e$)
  \end{enumerate}
\end{definition}


\begin{definition}[Infinite Full Tree Unraveling]
  The infinite full tree unraveling for a base structure $\str{A}$ is a structure $\str{A}^{\rightarrow{}}$ with domain $\mathtt{Dom}^{\rightarrow}_{\str{A}}$.
  Each relation $R_{i}$ in the signature of $\str{A}$ is interpreted in $\str{A}^{\rightarrow}$ as:
  \begin{equation*}
    \overrightarrow{r} \in R_{i}^{\str{A}^{\rightarrow}} \iff \pi[\overrightarrow{r}] \in R_{i}^{\str{A}} \land \bigwedge_{k=1}^{|\overrightarrow{r}|-1}{(\nextrel{\overrightarrow{r}_{k}}{\overrightarrow{r}_{k+1}})} \land |\overrightarrow{r}| \le \mathtt{bound}(\overrightarrow{r})
  \end{equation*}
  where $\mathtt{bound}(\overrightarrow{r}) = \num{\overrightarrow{r}_{|\overrightarrow{r}|}}$ (counter of last element of $\overrightarrow{r}$)
\end{definition}

% \begin{definition}[Lifting tuples and bisimulation sequences]
%   For each tuple $\bar{a} \in \str{A}$, we define a lifted tuple $\lift{\bar{a}} \in \str{A}^{\rightarrow}$ as follows:
%   \begin{equation*}
%     \lift{\bar{a}} = ((\bar{a}, 1), \ldots{}, (\bar{a}, |a|))
%   \end{equation*}
%   The mapping can be extended to bisimulation sequences.
%   Let $\rho = \sigma(i,j)\bar{a}$ be a bisimulation sequence, then the $\lift{\rho}$ is defined such that:
%   \begin{alignat*}{3}
%     &{\lift{\rho}}_{1\ldots{}\dist{i}{j}} &&= {\lift{\sigma}}_{i\ldots{}j} && \\
%     &{\lift{\rho}}_{x} &&= (\rho, x) && \qquad\text{for $\dist{i}{j} < x \le |\bar{a}|$}
%   \end{alignat*}
% \end{definition}

\pagebreak

\begin{lemma}
  $\str{A} \sim_{\mathrm{FGF}} \str{A}^{\rightarrow{}}$
\end{lemma}
\begin{proof}
  We show that the set $\mathcal{Z}$, defined as follows, is FGF bisimulation:
  \begin{equation*}
    (\bar{a}, \bar{a}^{*}) \in \mathcal{Z} \iff
    \bar{a} = \pi[\bar{a}^{*}] \text{\ and\ }
    \bar{a}^{*} \text{\ is live}
  \end{equation*}

  \begin{description}
    \item[(atomic harmony)]
          Let $(\bar{a}, \bar{a}^{*}) \in \mathcal{Z}$.
          We show that $\bar{a}_{i\ldots{}j} \in R^{\str{A}} \iff \bar{a}^{*}_{i\ldots{}j} \in R^{\str{A}^{\rightarrow}}$.
          From $\bar{a} = \pi[\bar{a}^{*}]$ follows that $\bar{a}_{i\ldots{}j} = \pi[\bar{a}^{*}_{i\ldots{}j}]$.
          By definition of the unraveling $\bar{a}^{*}_{i\ldots{}j} \in R^{\str{A}^{\rightarrow}}$ implies $\pi[\bar{a}^{*}_{i\ldots{}j}] \in R^{\str{A}}$ leaving only the ``$\implies$'' direction to be shown.
          We prove the three conditions required for $\bar{a}^{*}_{i\ldots{}j} \in R^{\str{A}^{\rightarrow}}$:
          \begin{itemize}
            \item
                  $\pi[\bar{a}^{*}_{i\ldots{}j}] \in R^{\mathfrak{A}}$ follows directly from $\bar{a}_{i\ldots{}j} \in R^{\mathfrak{A}}$.
            \item
                  $\bigwedge_{k=1}^{|\bar{a}^{*}_{i\ldots{}j}|-1}{(\nextrel{\bar{a}^{*}_{i\ldots{}j,k}}{{}\bar{a}^{*}_{i\ldots{}j,k+1}})}$ is true since $\bar{a}^{*}$ is live so $\nextrel{a^{*}_{k}}{a^{*}_{k+1}}$ for all $1 \le k < |\bar{a}^{*}|$.
            \item
                  To show that $|\bar{a}^{*}_{i\ldots{j}}| \le \mathtt{bound}(\bar{a}^{*}_{i\ldots{}j})$, let $\bar{a}^{*} = ((\sigma_{1}, x_{1}), \ldots{}, (\sigma_{n}, x_{n}))$.
                  We know that for all $1 < k \le |\bar{a}^{*}|$ we have $\nextrel{(\sigma_{k-1}, x_{k-1})}{(\sigma_{k}, x_{k})}$ since $\bar{a}^{*}$ is live.
                  This implies $x_{k-1} \ge x_{k} - 1$ as can be seen by analysing the two cases of the definition of $\mathit{Next}$.
                  Since $\bar{a}^{*}$ is live we have $\mathtt{bound}(\bar{a}^{*}) = x_{n} \ge n = |\bar{a}^{*}|$.
                  It follows that $x_{k} \ge k$ for all $k$, so $\mathtt{bound}(\bar{a}^{*}_{i\ldots{}j}) = x_{j} \ge j \ge |\bar{a}^{*}_{i\ldots{j}}|$ as required.
          \end{itemize}
    \item[(fgforth)]
          Let $(\bar{a}, \bar{a}^{*}) \in \mathcal{Z}$ and $\bar{b} \in \mathfrak{A}$ be a live tuple such that $\bar{a}\sij = \bar{b}_{1\ldots{}\dist{i}{j}}$ for some $i,j$.
          We show that there exists $\bar{b}^{*}$ such that $\bar{a}^{*}\sij = \bar{b}^{*}_{1\ldots{}\dist{i}{j}}$ and $(\bar{b}, \bar{b}^{*}) \in \mathcal{Z}$.

          We first consider the case where $\bar{b}_{1\ldots{}\dist{i}{j}}$ is a proper, non-empty prefix of $\bar{b}$.
          Let $\sigma$ and $k$ be the bisimulation sequence and index such that $b_{j} = (\sigma, k)$.
          We choose $\bar{b}^{*}$ to be the tuple such that:
          \begin{alignat*}{2}
            &\bar{b}^{*}_{1\ldots{}\dist{i}{j}} &&= \bar{a}\sij \\
            &\bar{b}^{*}_{x} &&= (\sigma (i,j) \bar{b}, x) \qquad \text{for\ }\dist{i}{j} < x \le |\bar{b}|
          \end{alignat*}
          Note that \fxnote*{more detail}{$\sigma (i,j) \bar{b}$ is a valid bisimulation sequence}, $\bigwedge_{k=1}^{|\bar{b}^{*}|-1}{\nextrel{b^{*}_{k}}{b^{*}_{k+1}}}$ and $|\bar{b}^{*}| \le \mathtt{bound}(\bar{b}^{*}) = |\bar{b}|$.
          Since $\pi[\bar{b}^{*}] = \bar{b}$ and $\bar{b}$ is live, it follows that $\bar{b}^{*}$ is also live, so $(\bar{b}, \bar{b}^{*}) \in \mathcal{Z}$ as required.

          Now consider the case where $\bar{b}_{1\ldots{}\dist{i}{j}}$ is not a proper prefix, so it is either empty or equal to $\bar{b}$.
          If it is equal to $\bar{b}$, then $\bar{b}$ is an infix of $\bar{a}$.
          So by atomic harmony, $\bar{b}^{*} = \bar{a}^{*}\sij$ is live since $\bar{b}$ is live and satisfies $(\bar{b}, \bar{b}^{*}) \in \mathcal{Z}$.
          If it is empty, let $\bar{b}^{*} = ((\bar{b}, 1), \ldots{}, (\bar{b}, |b|))$.
          Again $\bar{b}^{*}$ is live since $\bar{b}$ is live and satisfies $(\bar{b}, \bar{b}^{*}) \in \mathcal{Z}$.

          % We define $$
          %  $\bar{
          % First consider the case where $\bar{a}$ and $\bar{b}$ share no elements, so $|\bar{a}\sij| = 0$.
          % As $(\bar{b}, \lift{\bar{b}}) \in \mathcal{Z}$, we can pick $\bar{b}^{*} = \lift{\bar{b}}$ to satisfy the conditions.
    \item[(fgfback)]
          This case follows directly from the definition of the unraveling.
          Let $(\bar{a}, \bar{a}^{*}) \in \mathcal{Z}$ and $\bar{b}^{*} \in \mathfrak{A}^{\rightarrow}$ be a live tuple such that $\bar{a}^{*}\sij = \bar{b}^{*}_{1\ldots{}\dist{i}{j}}$ for some $i,j$.
          Then $\bar{b} = \pi[\bar{b}^{*}]$ also has $\bar{a}_{i\ldots{}j} = \bar{b}_{1\ldots{}\dist{i}{j}}$ (since the projection is pointwise) and satisfies $(\bar{b}, \bar{b}^{*}) \in \mathcal{Z}$ as required.
  \end{description}
\end{proof}

\pagebreak

\begin{lemma}\label{lem:fgf-intersection-continuous}
  If two tuples $\overrightarrow{a}, \overrightarrow{a}' \in \str{A}^{\rightarrow}$ intersect, then $\overrightarrow{a} \cap \overrightarrow{a}'$ is an infix of both $\overrightarrow{a}, \overrightarrow{a}'$ and
  $\overrightarrow{a} \cap \overrightarrow{a}'$ is a prefix of at least one of $\overrightarrow{a}, \overrightarrow{a}'$.
\end{lemma}

\begin{lemma}
  If $\str{A} \sim_{\mathrm{FGF},2*W*l} \str{B}$, then $\str{A}^{\rightarrow} \sim_{\mathrm{GF},l} \str{B}^{\rightarrow}$
\end{lemma}
\begin{proof}
  Let $\mathcal{Z}_{0}, \ldots, \mathcal{Z}_{l}$ be a sequence of sets, each defined as follows:
  \fxnote{
    $\seq{a_{x}} \approx_{\alpha} \seq{b_{x}}$ is almost implied by $\seq{a_{n}} \approx_{\alpha} \seq{b_{n}}$ due to the definition of ``$\mathit{Next}$''.
    The only difference is that since $\seq{a_{1}}$ is shorter, this increases the ``lookback'' (how many elements from the end of the sequence of $a_n$ we consider) by at most the width.
  }{\begin{equation*}
      (\overrightarrow{a}, \overrightarrow{b}) \in \mathcal{Z}_{\alpha}\ \iff \left\{
        \begin{array}{lll}
          |\overrightarrow{a}| = |\overrightarrow{b}| & \text{with\ } n = |a| = |b| & \wedge \\
          \num{\overrightarrow{a}_{x}} = \num{\overrightarrow{b}_{x}} & \text{for all\ } 1 \le x \le |\overrightarrow{a}| & \wedge \\
          \seq{\overrightarrow{a}_{x}} \approx_{W * \alpha} \seq{\overrightarrow{b}_{x}} & \text{for all\ } 1 \le x \le |\overrightarrow{a}| &
        \end{array}
      \right.
    \end{equation*}}
  where $\sigma \approx_{\beta} \rho$ for $\sigma = \bar{s}_{0}\cdots{}(i_{n}, j_{n})\bar{s}_{n}$ and $\rho = \bar{r}_{0}\cdots{}(v_{m}, w_{m})\bar{r}_{m}$ iff:
  \begin{enumerate}
    \item
          last $\beta$ indices match: $i_{n-x} = v_{m-x} \wedge j_{n-x} = u_{m-x}$ for all $x \le \beta$, and
    \item
         last $\beta$ tuples are bisimilar: $\bar{s}_{n-x} \sim_{\mathrm{FGF},\beta} \bar{r}_{m-x}$ for all $x \le \beta$
  \end{enumerate}
  We show that each $\mathcal{Z}_{\alpha-1}$ has the back-and-forth property for $\mathcal{Z}_{\alpha}$ and satisfies atomic harmony.

  \paragraph{Atomic harmony}
  Let $(\overrightarrow{a}, \overrightarrow{b}) \in \mathcal{Z}_{\alpha}$.
  Let $\sigma = \cdots{}\bar{s}$ be a bisimulation sequence ending in $\bar{s}$.
  Observe that for any tuple $\overrightarrow{a} = (\overrightarrow{a}_{1}, \ldots{}, \overrightarrow{a}_{n})$ with $\seq{\overrightarrow{a}_{n}} = \sigma$, we have that $\pi(\overrightarrow{a}_{x}) = s_{\num{\overrightarrow{a}_{x}}}$.
  Thus, by definition of $\mathcal{Z}_{a}$ and ``$\approx_{\beta}$'' it follows that $\pi[\overrightarrow{a}] \sim_{\mathrm{FGF},W*\alpha} \pi[\overrightarrow{b}]$ which implies that $\mathrm{ftp}(\overrightarrow{a}) = \mathrm{ftp}(\overrightarrow{b})$.
  Since forward-type equivalence is the same as full atomic type equivalence in tree unravelings, this shows that $\mathcal{Z}_{\alpha}$ respects atomic harmony.

  \paragraph{Back-and-forth property}
  Now let $(\overrightarrow{a}, \overrightarrow{b}) \in \mathcal{Z}_{\alpha}$ and $\overrightarrow{a}'$ be a guarded tuple in $\str{A}^{\rightarrow}$.
  Since all guarded tuples in an unraveling are also live, \cref{lem:fgf-intersection-continuous} applies so the common elements of $\overrightarrow{a} \cap \overrightarrow{a}'$ must be a prefix of one of $\overrightarrow{a}, \overrightarrow{a}'$.
  We call the case where $\overrightarrow{a} \cap \overrightarrow{a}'$ is a prefix of $\overrightarrow{a}'$ the \emph{forward case}, and the other case the \emph{backward case}.

  \paragraph{The forward case.}
  Let $\overrightarrow{a}\sij$ be the infix of $\overrightarrow{a}$ which is a prefix of $\overrightarrow{a}'$.
  We construct a tuple $\overrightarrow{b}'$ with $\overrightarrow{b}'_{1\ldots{}\dist{i}{j}} = \overrightarrow{b}\sij$ and $(\overrightarrow{a}', \overrightarrow{b}') \in \mathcal{Z}_{\alpha-1}$.
  Since $\overrightarrow{b}'_{1\ldots{}\dist{i}{j}} = \overrightarrow{b}\sij$ \fxwarning*{This only works if the overlap is nonempty. But if the overlap is empty, we can simply treat this as restarting the game, so we handle this case the same way that we handle the start of the game}{we already know the first elements} of $\overrightarrow{b}'$.
  For some $x$, let $\num{\overrightarrow{b}'_{x}} = \num{\overrightarrow{a}'_{x}}$ and $\seq{\overrightarrow{b}'_{x}} \approx_{\beta} \seq{\overrightarrow{a}'_{x}}$.
  We show how to find $\overrightarrow{b}'_{x+1}$ with $\num{\overrightarrow{b}'_{x+1}} = \num{\overrightarrow{a}'_{x+1}}$ and $\seq{\overrightarrow{b}'_{x+1}} \approx_{\beta} \seq{\overrightarrow{a}'_{x+1}}$.

  If $\seq{\overrightarrow{a}'_{x+1}} = \seq{\overrightarrow{a}_{x}}$, then we can set $\overrightarrow{b}'_{x+1} = (\seq{\overrightarrow{b}'}_{x}, \num{\overrightarrow{a}'_{x+1}})$ and are done.
  By definition of ``$\mathit{Next}$'', the only remaining case is that $\seq{\overrightarrow{a}'_{x+1}} = \seq{\overrightarrow{a}'_{x}} (v, w) \bar{t}$.
  Let $\seq{\overrightarrow{a}'_{x}} = \cdots{}\bar{s}$ and $\seq{\overrightarrow{b}'_{x}} = \cdots{}\bar{o}$.
  By definition of ``$\approx_{\beta}$'', we know that $\bar{s} \sim_{\mathrm{FGF},\beta} \bar{o}$.
  Using the \textbf{(fgforth)} property of FGF, we can find $\bar{p}$ such that $\bar{t} \sim_{\mathrm{FGF},\beta-1} \bar{p}$ and $\bar{p}_{1\ldots{}\dist{v}{w}} = \bar{o}_{v\ldots{}w}$.
  Then $\rho = \seq{\overrightarrow{b}'_{x}} (v,w) \bar{p}$ is a valid bisimulation sequence and $\rho \approx_{\beta-1} \seq{\overrightarrow{a}'_{x+1}}$ by construction.
  So $\overrightarrow{b}_{x+1} = (\rho, \num{\overrightarrow{a}_{x+1}})$ has the required properties.

  We need to apply the above procedure at most $|\overrightarrow{b}'| = |\overrightarrow{a}'| \le W$ times to construct all elements of $\overrightarrow{b}'$.
  Therefore we have $\seq{\overrightarrow{b}'_{x}} \approx_{W*\alpha-W} \seq{\overrightarrow{a}'_{x}}$ for all $1 \le x \le |\overrightarrow{b}'|$.
  As we also preserve the equivalence of counters with each step, it follows that $(\overrightarrow{a}', \overrightarrow{b}') \in \mathcal{Z}_{\alpha-1}$.

  \paragraph{The backward case.}
  Let $\overrightarrow{a}'\sij$ be a prefix of $\overrightarrow{a}$.
  We construct a tuple $\overrightarrow{b}'$ with $\overrightarrow{b}'\sij = \overrightarrow{b}_{1\ldots{}\dist{i}{j}}$ and $(\overrightarrow{a}', \overrightarrow{b}') \in \mathcal{Z}_{\alpha-1}$.
  We know that $\overrightarrow{b}'_{i}$ must be equal to $\overrightarrow{b}_{1}$.
  We show how to find $\overrightarrow{b}'_{x-1}$ with $\num{\overrightarrow{b}'_{x-1}} = \num{\overrightarrow{a}'_{x-1}}$ and $\seq{\overrightarrow{b}'_{x-1}} \approx_{\beta-1} \seq{\overrightarrow{a}_{x-1}}$ given that $\num{\overrightarrow{b}'_{x}} = \num{\overrightarrow{a}'_{x}}$ and $\seq{\overrightarrow{b'_{x}}} \approx_{\beta} \seq{\overrightarrow{a'}_{x}}$.

  If $\seq{\overrightarrow{a}'_{x-1}} = \seq{\overrightarrow{a}'_{x}}$, we can set $\overrightarrow{b}'_{x-1} = (\seq{\overrightarrow{b}'_{x}}, \num{\overrightarrow{b}'_{x}} - 1)$ because in this case $\num{\overrightarrow{a}'_{x-1}} = \num{\overrightarrow{a}'_{x}} - 1$, so we are done.
  By definition of ``$\mathit{Next}$'', the only remaining case is that $\seq{\overrightarrow{a}'_{x}} = \seq{\overrightarrow{a}'_{x-1}} (v,w) \bar{s}$ and $\num{\overrightarrow{a}'_{x-1}} = w$.
  By definition of ``$\approx_{\beta}$'', there must be a tuple $\bar{o}$ such that $\seq{\overrightarrow{b'}_{x}} = \rho (v,w) \bar{o}$ where $\rho \approx_{\beta-1} \seq{\overrightarrow{a}'_{x-1}}$.
  So $\overrightarrow{b}'_{x-1} = (\rho, w)$ has the required properties.

  We need to apply the above procedure at most $i \le |\overrightarrow{a}'| \le W$ times to find all elements in $|\overrightarrow{b}'_{1\ldots{}i}|$.
  Thus $\seq{\overrightarrow{b}'_{x}} \approx_{W*a - W} \seq{\overrightarrow{a}'_{x}}$ for all $1 \le x \le i$.
  Because of $\overrightarrow{b}'\sij = \overrightarrow{b}_{1\ldots{}\dist{i}{j}}$, we now know all elements of $\overrightarrow{b}'_{1\ldots{}j}$.
  We can complete the tuple $\overrightarrow{b}'_{1\ldots{}j}$ to the tuple $\overrightarrow{b}'$ by applying the construction for the forward case starting from $\overrightarrow{b}_{j}$.
  Then we have $(\overrightarrow{a}', \overrightarrow{b}') \in \mathcal{Z}_{\alpha-1}$.
\end{proof}

\subsection{Finite tree unraveling}

\begin{definition}[Finite tree unraveling]
  Let $\mathfrak{A}^{(\sigma, s) \rightarrow}$ denote the tree structure having $(\sigma, s)$ as the root and containing all descendants according to the ``$\mathit{Next}$'' relation.
  Let $\mathfrak{A}^{(\sigma, s) \rightarrow}_{\theta}$ be the finite truncation of that tree to elements $e$ for which $|\seq{e}| \le |\sigma| + \theta$.
  Given two elements $(a, a') \in \mathit{Next}$, then $a'$ is called a \emph{cut element} if $a$ is in the truncated tree but $a'$ is not.

  Let $\mathcal{T}$ be the set of trees for the roots $(\sigma, s)$ which have $|\sigma| \le \beta$.
  Let $M \in \mathbb{N}$ be such that for each tree, all cut elements can be uniquely assigned an index from the sequence $1, 2, \ldots, M$.
  The domain of the finite tree unraveling is the union of $M$ copies of each tree in $\mathcal{T}$.
  The relation ``$\mathit{Next}_{fin}$'' is defined on this domain such that $(a, b) \in \mathit{Next}_{fin}$ iff:
  \begin{enumerate}[(a)]
    \item $(a, b) \in \mathit{Next}$ and $a$ and $b$ are part of the same tree, or
    \item $(a, a') \in \mathit{Next}$ for some cut element $a'$ with associated index $m$, and all the following are true:
        \begin{enumerate}
          \item $b$ is the root of the $m$-th copy of a tree in T
          \item $\num{b} = \num{a'}$
          \item $\seq{b}$ is the length-$\beta$ suffix of $\seq{a'}$
        \end{enumerate}
  \end{enumerate}
\end{definition}

\end{document}
