%!TEX root = main.tex
\pdfoutput=1
\documentclass[a4paper,UKenglish,cleveref, autoref, thm-restate, hideLIPIcs]{lipics-v2021}
\bibliographystyle{plainurl}
%!TEX root = ../main.tex
\title{Expressive Completeness\\ of the Forward Guarded Fragment}
\titlerunning{Expressive Completeness of the Forward Guarded Fragment}

\author{Bartosz Bednarczyk}
{Computational Logic Group, Technische Universit{\"a}t  Dresden, Germany \and 
Institute of Computer Science, University of Wroc\l aw, Poland
\and \url{https://bartoszjanbednarczyk.github.io/}}
{bartosz.bednarczyk@cs.uni.wroc.pl}
{https://orcid.org/0000-0002-8267-7554}
{supported by the ERC Consolidator Grant No. 771779 (DeciGUT).}

\author{Benno Fünfstück }
{Technische Universit{\"a}t  Dresden, Germany}
{benno.fuenfstueck@tu-dresden.de}
{}
{}

\authorrunning{B. Bednarczyk and B. Fünfstück}
\Copyright{Bartosz Bednarczyk and Benno Fünfstück}

\ccsdesc[500]{Theory of computation~Finite Model Theory}
\keywords{
ordered fragments, 
guarded fragment, 
(finite) model theory, 
expressive power,
Van Benthem theorem
}

% \category{}
% \relatedversion{Full version of this paper is available on arXiV~\cite{fullversion}.}
% \supplement{}
% \nolinenumbers %uncomment to disable line numbering
% \acknowledgements{}



\pdfinfo{
/Title (ATowards a Model Theory of Ordered Logics: Expressivity and Interpolation)
/Author (Bartosz Bednarczyk, Reijo Jaakkola)
/TemplateVersion (2022)
}
\usepackage{ifxetex}

\ifxetex
  \usepackage{polyglossia}
  \usepackage{fontspec}

  \setmainlanguage{english}
\else
  \usepackage[english]{babel}
\fi

\usepackage{hyperref}
\usepackage{blindtext}
\usepackage{graphicx}
\usepackage{tabularx}
\usepackage{listings}
\usepackage{tikz}
\usepackage{pifont}
\usepackage{fontawesome}
\usepackage{enumitem}
\usetikzlibrary{arrows, arrows.meta, decorations.markings, shapes, calc, positioning, fit, graphs, trees, matrix, backgrounds, bending, decorations.pathmorphing, decorations.pathreplacing, decorations.shapes, fadings, shadings, patterns, graphs.standard}
\usepackage{tikzpeople}
\usepackage{upquote}
\usepackage{tikzsymbols}
\usepackage{bookmark}
\usepackage{printlen}
\usepackage{xcolor}
\usepackage{mathtools}
\usepackage{amssymb}
\usepackage{amsmath}

% Inline comments
\definecolor{ao(english)}{rgb}{0.0, 0.5, 0.0}
\definecolor{brickred}{rgb}{0.8, 0.25, 0.33}
\newcommand{\myundef}[1]{\textcolor{brickred}{\textbf{#1}}}
\newcommand{\possiblelie}[1]{\textcolor{brickred}{\textbf{#1}}}
\newcommand{\becareful}[1]{\textcolor{brickred}{\textbf{#1}}}
\newcommand{\bennof}[1]{\textbf{\textcolor{blue}{\textbf{#1}}}}
\newcommand{\bbe}[1]{\textbf{\textcolor{ao(english)}{#1}}}
\newcommand{\bbebox}[1]{\todo[inline,color=green!30]{\textbf{BB\@: }#1}\xspace}
\newcommand{\bbeside}[1]{\todo[color=green!30,size=\scriptsize,fancyline]{\textbf{BB\@: }#1}\xspace}
\newcommand{\bfbox}[1]{\todo[inline,color=blue!30]{\textbf{BF\@: }#1}\xspace}
\newcommand{\bfside}[1]{\todo[color=blue!30,size=\scriptsize,fancyline]{\textbf{BF\@: }#1}\xspace}

% logics:
\newcommand{\Logic}[1]{\ensuremath{\mathsf{#1}}} % a Logic
\newcommand{\logicL}{\Logic{L}} % some logic L
\newcommand{\Laffix}{\Logic{L}_{\mathsf{affix}}}   % L_affix
\newcommand{\Linfix}{\Logic{L}_{\mathsf{inf}}}   % L_infix
\newcommand{\Lsuffix}{\Logic{L}_{\mathsf{suf}}} % L_suffix
\newcommand{\Lprefix}{\Logic{L}_{\mathsf{pre}}} % L_prefix

\newcommand{\Gaffix}{\Logic{G}_{\mathsf{affix}}}   % G_affix
\newcommand{\Ginfix}{\Logic{G}_{\mathsf{inf}}}     % G_infix
\newcommand{\Gsuffix}{\Logic{G}_{\mathsf{suf}}}    % G_suffix
\newcommand{\Gprefix}{\Logic{G}_{\mathsf{pre}}}    % G_prefix

\newcommand{\GF}{\Logic{GF}}   % Guarded Fragment
\newcommand{\FGF}{\Logic{FGF}}   % Forward Guarded Fragment
\newcommand{\FO}{\Logic{FO}}   % First-Order Logic

% Complexity classses:
\newcommand{\complexityclass}[1]{\textsc{#1}} % any complexity class
\newcommand{\ExpTime}{\complexityclass{ExpTime}} % exponential time
\hyphenation{Exp-Time} % prevent "Ex-PTime" (see, e.g. Tobies'01, Glimm'07 ;-)
\newcommand{\NExpTime}{\complexityclass{NExpTime}} % nondeterministic exponential time
\hyphenation{NExp-Time} % prevent "Ex-PTime" (see, e.g. Tobies'01, Glimm'07 ;-)
\newcommand{\TwoExpTime}{\complexityclass{2ExpTime}} % doubly-exponential time
\newcommand{\TwoNExpTime}{\complexityclass{2NExpTime}} % doubly-nondeterministic-exponential time
\newcommand{\coNExpTime}{\complexityclass{coNExpTime}} % co nondeterministic exponential time
\hyphenation{coNExp-Time} 
\newcommand{\Tower}{\complexityclass{Tower}}
\newcommand{\LogSpace}{\complexityclass{LogSpace}}
\newcommand{\PSpace}{\complexityclass{PSpace}}
\newcommand{\PTime}{\complexityclass{PTime}}

% Others
\newcommand{\str}[1]{{\mathfrak{#1}}}
\DeclareRobustCommand{\unravel}[1]{\vv{\mathfrak{#1}}}
\newcommand{\deff}{\coloneqq}
\newcommand{\arity}{\mathsf{ar}}
\renewcommand{\iff}{\leftrightarrow}

\newcommand{\N}{{\mathbb{N}}}
\newcommand{\Z}{{\mathbb{Z}}} 
\newcommand{\Q}{{\mathbb{Q}}}
\newcommand{\V}{\mathbf{V}} 
\newcommand{\R}{\mathbf{R}} 
\newcommand{\Var}{\mathrm{Var}}
\newcommand{\sigSigma}{\Sigma}

\newcommand{\sqin}{%
  \mathrel{\vphantom{\sqsubset}\text{%
    \mathsurround=0pt
    \ooalign{$\sqsubset$\cr$-$\cr}%
  }}%
}

% Rel symbols
\newcommand{\rel}[1]{\mathrm{#1}}
\newcommand{\relP}{\rel{P}}
\newcommand{\relR}{\rel{R}}
\newcommand{\relQ}{\rel{Q}}
\newcommand{\relS}{\rel{S}}
\newcommand{\relT}{\rel{T}}
\newcommand{\relU}{\rel{U}}
\newcommand{\relA}{\rel{A}}
\newcommand{\relB}{\rel{B}}
\newcommand{\relC}{\rel{C}}
\newcommand{\relD}{\rel{D}}
\newcommand{\relE}{\rel{E}}
\newcommand{\relH}{\rel{H}}
\newcommand{\sig}{\mathsf{sig}}

% Tuples 
\newcommand{\emptytupl}{\epsilon}
\newcommand{\set}{\mathsf{set}}

% Domain elements
\newcommand{\elem}[1]{\mathrm{#1}}                           % domain element
\newcommand{\elema}{\elem{a}}                             % domain element a
\newcommand{\elemb}{\elem{b}}                             % domain element b
\newcommand{\elemc}{\elem{c}}                             % domain element c
\newcommand{\elemd}{\elem{d}}                             % domain element d
\newcommand{\eleme}{\elem{e}}                             % domain element e
\newcommand{\elemf}{\elem{f}}                               % domain element f
\newcommand{\elemg}{\elem{g}}                             % domain element g
\newcommand{\elemh}{\elem{h}}                             % domain element h
\newcommand{\elemi}{\elem{i}}                             % domain element i
\newcommand{\elemj}{\elem{j}}                             % domain element j
\newcommand{\elemo}{\elem{o}}                             % domain element o
\newcommand{\elemp}{\elem{p}}                             % domain element p
\newcommand{\elemq}{\elem{q}}                             % domain element q
\newcommand{\elemr}{\elem{r}}                             % domain element r
\newcommand{\elems}{\elem{s}}                             % domain element s
\newcommand{\elemt}{\elem{t}}                             % domain element t
\newcommand{\elemw}{\elem{w}}                             % domain element w
\newcommand{\elemv}{\elem{v}}                             % domain element v
\newcommand{\elemu}{\elem{u}}                             % domain element u
\newcommand{\elemx}{\elem{x}}                             % domain element u
\newcommand{\elemtuplea}{\overline{\elema}}                         % tuple of domain element a
\newcommand{\elemtupleb}{\overline{\elemb}}                         % tuple of domain element b
\newcommand{\elemtuplec}{\overline{\elemc}}                         % tuple of domain element c
\newcommand{\elemtupled}{\overline{\elemd}}                         % tuple of domain element d
\newcommand{\elemtuplee}{\overline{\eleme}}                         % tuple of domain element e
\newcommand{\elemtuplef}{\overline{\elemf}}                           % tuple of domain element f
\newcommand{\elemtupleg}{\overline{\elemg}}                         % tuple of domain element g
\newcommand{\elemtupleh}{\overline{\elemh}}                         % tuple of domain element h
\newcommand{\elemtuplei}{\overline{\elemi}}                         % tuple of domain element i
\newcommand{\elemtupleo}{\overline{\elemo}}                         % tuple of domain element o
\newcommand{\elemtuplep}{\overline{\elemp}}                         % tuple of domain element p
\newcommand{\elemtupleq}{\overline{\elemq}}                         % tuple of domain element q
\newcommand{\elemtupler}{\overline{\elemr}}                         % tuple of domain element r
\newcommand{\elemtuples}{\overline{\elems}}                         % tuple of domain element s
\newcommand{\elemtuplet}{\overline{\elemt}}                         % tuple of domain element t
\newcommand{\elemtuplew}{\overline{\elemw}}                         % tuple of domain element w
\newcommand{\elemtupleu}{\overline{\elemu}}                         % tuple of domain element u
\newcommand{\elemtuplev}{\overline{\elemv}}                         % tuple of domain element v
\newcommand{\elemtuplex}{\overline{\elemx}}                         % tuple of domain element v

\newcommand{\elemtupledfromto}[2]{\overline{\elemd}_{#1\ldots#2}}  % tuple of domelements d from #1 to #2
\newcommand{\elemtupleefromto}[2]{\overline{\eleme}_{#1\ldots#2}}  % tuple of domelements e from #1 to #2
\newcommand{\elemtuplecfromto}[2]{\overline{\elemc}_{#1\ldots#2}}  % tuple of domelements c from #1 to #2
\newcommand{\elemtuplebfromto}[2]{\overline{\elemb}_{#1\ldots#2}}  % tuple of domelements b from #1 to #2
\newcommand{\elemtupleafromto}[2]{\overline{\elema}_{#1\ldots#2}}  % tuple of domelements a from #1 to #2
\newcommand{\elemtupleffromto}[2]{{\elemtuplef}_{#1\dots#2}} % tuple of domelements f from #1 to #2
\newcommand{\elemtuplegfromto}[2]{{\elemtupleg}_{#1\dots#2}} % tuple of domelements g from #1 to #2

\DeclareRobustCommand{\elemtuptuplea}{\vv{\elema}}                         % tuple of domain element a
\DeclareRobustCommand{\elemtuptupleb}{\vv{\elemb}}                         % tuple of domain element b
\DeclareRobustCommand{\elemtuptuplec}{\vv{\elemc}}                         % tuple of domain element c
\DeclareRobustCommand{\elemtuptupled}{\vv{\elemd}}                         % tuple of domain element d
\DeclareRobustCommand{\elemtuptuplee}{\vv{\eleme}}                         % tuple of domain element e
\DeclareRobustCommand{\elemtuptuplef}{\vv{\elemf}}                           % tuple of domain element f
\DeclareRobustCommand{\elemtuptupleg}{\vv{\elemg}}                         % tuple of domain element g
\DeclareRobustCommand{\elemtuptupleh}{\vv{\elemh}}                         % tuple of domain element h
\DeclareRobustCommand{\elemtuptuplei}{\vv{\elemi}}                         % tuple of domain element i
\DeclareRobustCommand{\elemtuptuplej}{\vv{\elemj}}                         % tuple of domain element j
\DeclareRobustCommand{\elemtuptuplep}{\vv{\elemp}}                         % tuple of domain element p
\DeclareRobustCommand{\elemtuptupleq}{\vv{\elemq}}                         % tuple of domain element q
\DeclareRobustCommand{\elemtuptupler}{\vv{\elemr}}                         % tuple of domain element r
\DeclareRobustCommand{\elemtuptuples}{\vv{\elems}}                         % tuple of domain element s
\DeclareRobustCommand{\elemtuptuplet}{\vv{\elemt}}                         % tuple of domain element t
\DeclareRobustCommand{\elemtuptuplew}{\vv{\elemw}}                         % tuple of domain element w
\DeclareRobustCommand{\elemtuptupleu}{\vv{\elemu}}                         % tuple of domain element u
\DeclareRobustCommand{\elemtuptuplev}{\vv{\elemv}}                         % tuple of domain element v
\DeclareRobustCommand{\elemtuptuplex}{\vv{\elemx}}                         % tuple of domain element v

% Variables:
\newcommand{\var}[1]{\mathit{#1}}       % variable
\newcommand{\varx}{\var{x}}             % variable x
\newcommand{\vary}{\var{y}}             % variable y
\newcommand{\varz}{\var{z}}             % variable z
\newcommand{\varv}{\var{v}}             % variable v
\newcommand{\varu}{\var{u}}             % variable u
\newcommand{\varw}{\var{w}}             % variable w
\newcommand{\varh}{\var{h}}             % variable h
\newcommand{\vartuplex}{\overline{\varx}}    % tuple of variables x
\newcommand{\vartuplexomega}{\overline{\varx_{\omega}}}      % tuple of variables x_omega
\newcommand{\vartupley}{\overline{\vary}}                    % tuple of variables y
\newcommand{\vartupleyone}{\overline{\vary_1}}                    % tuple of variables y_1
\newcommand{\vartupleytwo}{\overline{\vary_2}}                    % tuple of variables y_2
\newcommand{\vartuplez}{\overline{\varz}}                    % tuple of variables z
\newcommand{\vartuplev}{\overline{\varv}}                    % tuple of variables v
\newcommand{\vartupleu}{\overline{\varu}}                    % tuple of variables u
\newcommand{\vartuplew}{\overline{\varw}}                    % tuple of variables w
\newcommand{\vartupleh}{\overline{\varh}}                    % tuple of variables h
\newcommand{\vartuplexfromto}[2]{\overline{\varx}_{#1\ldots#2}}  % tuple of variables x from #1 to #2
\newcommand{\vartupleyfromto}[2]{\overline{\vary}_{#1\ldots#2}}  % tuple of variables y from #1 to #2
\newcommand{\vartupleufromto}[2]{\overline{\varu}_{#1\ldots#2}}  % tuple of variables u from #1 to #2
\newcommand{\vartuplevfromto}[2]{\overline{\varv}_{#1\ldots#2}}  % tuple of variables v from #1 to #2
\newcommand{\vartuplewfromto}[2]{\overline{\varw}_{#1\ldots#2}}  % tuple of variables v from #1 to #2

% Theory
\newcommand{\theory}[1]{\mathcal{#1}}   % theory
\newcommand{\theoryT}{\theory{T}}       % theory T

% Types
\newcommand{\atp}[3]{\mathsf{atp}^{#1}_{#2}(#3)}
\newcommand{\tp}[3]{\mathsf{tp}^{#1}_{#2}(#3)}

% Tree unravelings
\newcommand{\seq}[1]{\mathsf{seq}(#1)}
\newcommand{\ctr}[1]{\mathsf{ctr}(#1)}
\newcommand{\relNext}{\rel{Next}}
\newcommand{\unraveldom}[1]{\vv{#1}}
\newcommand{\Seq}[1]{\mathsf{Seq}(\str{#1})}
\newcommand{\hist}[2]{\mathsf{hist}_{#1}(#2)}

% Restrictions
\renewcommand{\restriction}{\mathord{\upharpoonright}}
\newcommand{\restr}[2]{#1\restriction_{#2}} % the restriction of #1 to #2

% morphisms
\newcommand{\homo}[1]{\mathfrak{#1}}    % homomorphism
\newcommand{\homof}{\homo{f}}           % homomorphism f
\newcommand{\homog}{\homo{g}}           % homomorphism g
\newcommand{\homoh}{\homo{h}}           % homomorphism h
\newcommand{\homop}{\homo{p}}           % homomorphism p
\newcommand{\homoe}{\homo{e}}           % homomorphism e
\newcommand{\ishomoto}{\vartriangleleft} % is homomorhic to
\newcommand{\homeq}{\rightleftarrows} % homomorphically equivalent
\newcommand{\isoeq}{\cong} % isomorphic
\newcommand{\elemext}{\preceq} % elementary extension
\newcommand{\omegasat}[1]{\widehat{#1}}
\newcommand{\partisof}{\homo{f}}           % partial isomorphism f
\newcommand{\partisog}{\homo{g}}           % partial isomorphism g
\newcommand{\partisoh}{\homo{h}}           % partial isomorphism h

% bisimulations
\newcommand{\bisimulation}[1]{\mathcal{#1}} % a bisimulation
\newcommand{\bisimZ}{\bisimulation{Z}}
\newcommand{\bisimto}{\sim} % bisimilarity relation
\newcommand{\strbisimto}{\approx} % strong bisimilarity relation
\newcommand{\PartIso}[2]{\mathsf{Part}(#1,#2)}

% tikz helpers
\newcommand{\tikzdbg}{%
  \draw[step=5em,color=lightgray]%
    (current bounding box.south west) grid (0,0)%
    (current bounding box.north west) grid (0,0)%
    (current bounding box.south east) grid (0,0)%
    (current bounding box.north east) grid (0,0);%
  \fill[red] (0,0) circle (0.1);%
}

% Proof sketchs
\let\realproof\proof
\let\realendproof\endproof
% \newenvironment{proofsketch}{%
%   \renewcommand{\proofname}{\normalfont\emph{Proof Sketch}}\realproof}{\realendproof}

% hide proofs
\usepackage{apxproof}
\let\proof\appendixproof
\let\endproof\endappendixproof


\begin{document}

\maketitle

\begin{abstract}
\bbebox{TODO.}
\end{abstract}

\section{Introduction}\label{sec:intro}
\bbebox{TODO.}


\section{Preliminaries}\label{sec:preliminaries}
We employ standard terminology from (finite and classical) model theory~\cite{Hodges97,Libkin04}.
All the logics considered here will be fragments of the first-order logic ($\FO$) over purely-relational equality-free vocabularies, under the usual syntax and semantics. 

We fix a countably infinite set $\V \deff \{x_i \mid i \in \N\}$ of variables. 
Throughout this paper all the formulae will use only variables from this set.
With $\sig(\varphi)$ we denote the set of relational symbols appearing in $\varphi$. 
We use $\arity(\relR)$ to denote the arity of $\relR$.
For a logic $\logicL$ and a signature~$\sigma$ we use~$\logicL[\sigma]$ in place of~$\{ \varphi \in \logicL \mid \sig(\varphi) \subseteq \sigma \}$.
Moreover, if an $\ell \in \N$ is given we use $\logicL_\ell$ to denote the restriction of $\logicL$ to formulae of quantifier rank (i.e.\ the maximal number of nested quantifiers) at most~$\ell$. 
We write $\varphi(\vartuplex)$ to indicate that all free variables from $\varphi$ are members of $\vartuplex$. If $\vartuplex$ contains precisely the free
variables of $\varphi$, then we will emphasise this separately.
A formula without free variables will be called a \emph{sentence}. 
Given a structure $\str{A}$ and $B \subseteq A$, we will use $\restr{\str{A}}{B}$ to denote the \emph{substructure} of $\str{A}$ induced by $B$.\\

\noindent \textbf{Tuples and subsequences.}
An $n$-tuple is a sequence with $n$ elements. The $0$-tuple is denoted with $\emptytupl$.
We use $\vartuplexfromto{i}{j}$ to denote the $(j{-}i{+}1)$-tuple $\varx_i, \varx_{i+1}, \ldots, \varx_j$.
We say that $\vartuplexfromto{i}{j}$ is an infix of a tuple $\vartuplexfromto{k}{l}$ if $k \leq i \leq j \leq l$ holds. 
To improve readability, tuples of tuples will be denoted by arrows, for instance with $\elemtuptuplea$.
For a set $S$, we write $\vartuplex \sqin S$ iff $\varx_i \in S$ for all indices $1 \leq i \leq |\vartuplex|$, where $|\vartuplex|$ denotes the length of~$\vartuplex$. 
A tuple $\elemtuplea \sqin A$ is \emph{$\sigma$-live} (or simply \emph{live} if a signature $\sigma$ is known from the context) in $\str{A}$ if $|\elemtuplea| \leq 1$ or~$\elemtuplea \in \relR^{\str{A}}$ for some~$\relR \in \sigma$.\\

\noindent \textbf{Guarded fragments.}
We start by recalling the definition of the \emph{guarded fragment}~\cite{AndrekaNB98}, \ie the fragment of $\FO$ obtained by requiring that blocks of quantifiers are appropriately relativised by atoms.
Formally $\GF$ is the smallest fragment of $\FO$  such that:
\begin{itemize}\itemsep0em
    \item Every atomic formula is in $\GF$;
    \item $\GF$ is closed under boolean connectives $\land, \lor, \neg, \to$;
    \item If $\varphi(\vartuplex, \vartupley)$ is in $\GF$ and $\alpha(\vartuplex, \vartupley)$ is an atom containing all free variables of $\varphi$ and $\vartupley$ is a tuple of variables then both $\forall{\vartupley} \; (\alpha(\vartuplex, \vartupley) \to \varphi(\vartuplex, \vartupley))$ and $\exists{\vartupley} \; (\alpha(\vartuplex, \vartupley) \land \varphi(\vartuplex, \vartupley))$ are in $\GF$; 
    \item If $\varphi(\varx)$ has only a single free-variable $\varx$, then $\forall{\varx}\; \varphi$ and $\exists{\varx}\; \varphi$ are in $\GF$.
\end{itemize}
The atoms $\alpha$ appearing in the 3rd item of the above definition are called \emph{guard}.
We stress that in definition of a quantifier rank for $\GF$ we count treat quantifiers $\exists{\vartuplex}$ introducing tuples of variables as as a single quantifier (not as an abbreviation for a block of quantifiers).

The \emph{forward guarded fragment} $\FGF$~\cite{Bednarczyk21} restrict $\GF$ in a way that the allowed sequences of atoms are \emph{infixes} (a.k.a.\@ contiguous subsequences) of the sequence of already-introduced variables in the order of their quantification.
A formal definition comes next, which will be follow by a bunch of examples.
$\FGF(n)$ for $n \in \N$ is the smallest fragment of $\FO$ satisfying:
\begin{itemize}\itemsep0em
    \item An atom $\alpha(\vartuplex)$ belongs to $\FGF(n)$ if $\alpha$ is equality-free and $\vartuplex$ is an infix of $\vartuplexfromto{1}{n}$.
    \item $\FGF(n)$ is closed under boolean connectives $\land, \lor, \neg, \to, \iff$;
    \item If $\varphi$ is in $\FGF(n{+}k)$ for a positive $k$ and $\alpha(\vartuplex, \vartupley)$ is an atom containing all free variables of $\varphi$ and $\vartupley$ is a $k$-tuple of variables then  $\forall{\vartupley} \; (\alpha(\vartuplex, \vartupley) \to \varphi(\vartuplex, \vartupley))$ and $\exists{\vartupley} \; (\alpha(\vartuplex, \vartupley) \land \varphi(\vartuplex, \vartupley))$ are both in $\FGF(n)$; 
    \item If $\varphi(\varx_1) \in \FGF(1)$ has only a single free-variable $\varx_1$, then $\forall{\varx_1}\; \varphi$ and $\exists{\varx_1}\; \varphi$ are in $\FGF(0)$.
\end{itemize}
We use $\FGF$ to denote $\FGF(0)$. 
Note that $\FGF(0)$ is solely composed of sentences. 

\bbebox{TODO:\@ Introduce some examples of sentences that (i) are in $\FGF$ (ii) are in $\GF$ but not in $\FO$ and sentences that are (iii) in $\GF$ but not in $\FGF$.}

\noindent \textbf{Bisimulations for guarded fragments.}
%
We next present notion of bisimulation relations tailored towards $\GF$ and $\FGF$.
Fix a logic $\logicL$, a finite signature~$\sigma$, a $\sigma$-structure $\str{A}$ and an $n$-tuple of elements $\elemtuplea \sqin A$. 
The $\logicL[\sigma]$-type of $\elemtuplea$ in $\str{A}$, denoted with~$\tp{\logicL[\sigma]}{\str{A}}{\elemtuplea}$, consists of all $\logicL[\sigma]$-formulae with free variables~$\vartuplexfromto{1}{n}$ that are satisfied by $\elemtuplea$ in $\str{A}$. 

For a signature $\sigma$ and $\sigma$-structures $\str{A}$ and $\str{B}$ we denote with $\PartIso{\str{A}}{\str{B}}$ the set of all partial isomorphisms between $\str{A}$ and $\str{B}$.\bbeside{Parametrize by $\sigma$.} 
For $\bisimZ, \bisimZ' \in \PartIso{\str{A}}{\str{B}}$ we say that $\bisimZ'$ satisfies back-and-forth conditions for $\bisimZ$ if for every partial isomorphism $\partisof \in \bisimZ$ we have:
%
\begin{description}\itemsep0em
  \item[\desclabel{(gForth)}{desc:gforth}] For any $\sigma$-live $\elemtuplea \sqin A$, there is $\partisog \in \bisimZ'$ with the domain $\elemtuplea$ such that $\partisof$ and $\partisog$ agree on their common domain. 
  \item[\desclabel{(gBack)}{desc:gback}] For any $\sigma$-live $\elemtupleb \sqin B$, there is $\partisog \in \bisimZ'$ with the image $\elemtuplea$ such that $\partisof$ and $\partisog$ agree on their common domain. 
\end{description}
The set $\bisimZ$ is a $\GF[\sigma]$-\emph{bisimulation} between $\str{A}$ and $\str{B}$ if it itself satisfies \ref{desc:gforth} and~\ref{desc:gback} conditions given above.
An $\ell$-$\GF[\sigma]$-bisimulation between $\str{A}$ and $\str{B}$ is a sequence $\bisimZ_0, \bisimZ_1, \ldots, \bisimZ_\ell$ of partial isomorphisms from $\PartIso{\str{A}}{\str{B}}$ such that for all $i > 0$ we have that $\bisimZ$ satisfies~\ref{desc:gforth} and~\ref{desc:gback} conditions for $\bisimZ_{i{-}1}$.
We say that pointed structures $(\str{A}, \elemtuplea)$\bbeside{pointed structures undef} and $(\str{B}, \elemtupleb)$ are $\GF[\sigma]$-\emph{bisimilar}, denoted $(\str{A}, \elemtuplea) \bisimto_{\GF[\sigma]} (\str{B}, \elemtupleb)$ if there exists a $\GF[\sigma]$-bisimulation  between $\str{A}$ and $\str{B}$ containing the partial isomorphism $\elemtuplea \mapsto \elemtupleb$.
Analogously we speak about $\ell$-$\GF[\sigma]$-\emph{bisimilar}; Notation: $(\str{A}, \elemtuplea) \bisimto_{\GF[\sigma]}^{\ell} (\str{B}, \elemtupleb)$.
We links bisimulations and logical equivalence as follows:

\begin{lemma}
For any finite signature $\sigma$ and $\sigma$-structures $(\str{A}, \elemtuplea), (\str{B}, \elemtupleb)$ we have that:\\
(a) $(\str{A}, \elemtuplea) \bisimto_{\GF[\sigma]} (\str{B}, \elemtupleb)$ implies $(\str{A}, \elemtuplea) \equiv_{\GF[\sigma]} (\str{B}, \elemtupleb)$;\\
(b) $(\str{A}, \elemtuplea) \bisimto_{\GF[\sigma]}^{\ell} (\str{B}, \elemtupleb)$ implies $(\str{A}, \elemtuplea) \equiv_{\GF_\ell[\sigma]} (\str{B}, \elemtupleb)$;\\
Moreover, if $\str{A}$ and $\str{B}$ are $\omega$-saturated\bbeside{undef}, then the reverse implication hold.
\end{lemma}
\begin{proof}
\bbebox{TODO: Benno. Either prove it or find a suitable reference (the second option is preferred.)}
\end{proof}


\clearpage
\bibliography{references}

\end{document}