% Inline comments
\definecolor{ao(english)}{rgb}{0.0, 0.5, 0.0}
\definecolor{brickred}{rgb}{0.8, 0.25, 0.33}
\newcommand{\myundef}[1]{\textcolor{brickred}{\textbf{#1}}}
\newcommand{\possiblelie}[1]{\textcolor{brickred}{\textbf{#1}}}
\newcommand{\becareful}[1]{\textcolor{brickred}{\textbf{#1}}}
\newcommand{\bennof}[1]{\textbf{\textcolor{blue}{\textbf{#1}}}}
\newcommand{\bbe}[1]{\textbf{\textcolor{ao(english)}{#1}}}
\newcommand{\bbebox}[1]{\todo[inline,color=green!30]{\textbf{BB\@: }#1}\xspace}
\newcommand{\bbeside}[1]{\todo[color=green!30,size=\scriptsize,fancyline]{\textbf{BB\@: }#1}\xspace}
\newcommand{\bfbox}[1]{\todo[inline,color=blue!30]{\textbf{BF\@: }#1}\xspace}
\newcommand{\bfside}[1]{\todo[color=blue!30,size=\scriptsize,fancyline]{\textbf{BF\@: }#1}\xspace}

% logics:
\newcommand{\Logic}[1]{\ensuremath{\mathsf{#1}}} % a Logic
\newcommand{\logicL}{\Logic{L}} % some logic L
\newcommand{\Laffix}{\Logic{L}_{\mathsf{affix}}}   % L_affix
\newcommand{\Linfix}{\Logic{L}_{\mathsf{inf}}}   % L_infix
\newcommand{\Lsuffix}{\Logic{L}_{\mathsf{suf}}} % L_suffix
\newcommand{\Lprefix}{\Logic{L}_{\mathsf{pre}}} % L_prefix

\newcommand{\Gaffix}{\Logic{G}_{\mathsf{affix}}}   % G_affix
\newcommand{\Ginfix}{\Logic{G}_{\mathsf{inf}}}     % G_infix
\newcommand{\Gsuffix}{\Logic{G}_{\mathsf{suf}}}    % G_suffix
\newcommand{\Gprefix}{\Logic{G}_{\mathsf{pre}}}    % G_prefix

\newcommand{\GF}{\Logic{GF}}   % Guarded Fragment
\newcommand{\FGF}{\Logic{FGF}}   % Forward Guarded Fragment
\newcommand{\FO}{\Logic{FO}}   % First-Order Logic

% Complexity classses:
\newcommand{\complexityclass}[1]{\textsc{#1}} % any complexity class
\newcommand{\ExpTime}{\complexityclass{ExpTime}} % exponential time
\hyphenation{Exp-Time} % prevent "Ex-PTime" (see, e.g. Tobies'01, Glimm'07 ;-)
\newcommand{\NExpTime}{\complexityclass{NExpTime}} % nondeterministic exponential time
\hyphenation{NExp-Time} % prevent "Ex-PTime" (see, e.g. Tobies'01, Glimm'07 ;-)
\newcommand{\TwoExpTime}{\complexityclass{2ExpTime}} % doubly-exponential time
\newcommand{\TwoNExpTime}{\complexityclass{2NExpTime}} % doubly-nondeterministic-exponential time
\newcommand{\coNExpTime}{\complexityclass{coNExpTime}} % co nondeterministic exponential time
\hyphenation{coNExp-Time} 
\newcommand{\Tower}{\complexityclass{Tower}}
\newcommand{\LogSpace}{\complexityclass{LogSpace}}
\newcommand{\PSpace}{\complexityclass{PSpace}}
\newcommand{\PTime}{\complexityclass{PTime}}

% Others
\newcommand{\str}[1]{{\mathfrak{#1}}}
\newcommand{\deff}{\coloneqq}
\newcommand{\arity}{\mathsf{ar}}
\renewcommand{\iff}{\leftrightarrow}

\newcommand{\N}{{\mathbb{N}}}
\newcommand{\Z}{{\mathbb{Z}}} 
\newcommand{\Q}{{\mathbb{Q}}}
\newcommand{\V}{\mathbf{V}} 
\newcommand{\R}{\mathbf{R}} 
\newcommand{\Var}{\mathrm{Var}}
\newcommand{\sigSigma}{\Sigma}

\newcommand{\sqin}{%
  \mathrel{\vphantom{\sqsubset}\text{%
    \mathsurround=0pt
    \ooalign{$\sqsubset$\cr$-$\cr}%
  }}%
}

% Rel symbols
\newcommand{\rel}[1]{\mathrm{#1}}
\newcommand{\relP}{\rel{P}}
\newcommand{\relR}{\rel{R}}
\newcommand{\relQ}{\rel{Q}}
\newcommand{\relS}{\rel{S}}
\newcommand{\relT}{\rel{T}}
\newcommand{\relU}{\rel{U}}
\newcommand{\relA}{\rel{A}}
\newcommand{\relB}{\rel{B}}
\newcommand{\relC}{\rel{C}}
\newcommand{\relD}{\rel{D}}
\newcommand{\relE}{\rel{E}}
\newcommand{\relH}{\rel{H}}
\newcommand{\sig}{\mathsf{sig}}

% Tuples 
\newcommand{\emptytupl}{\epsilon}
\newcommand{\set}{\mathsf{set}}

% Domain elements
\newcommand{\elem}[1]{\mathrm{#1}}                           % domain element
\newcommand{\elema}{\elem{a}}                             % domain element a
\newcommand{\elemb}{\elem{b}}                             % domain element b
\newcommand{\elemc}{\elem{c}}                             % domain element c
\newcommand{\elemd}{\elem{d}}                             % domain element d
\newcommand{\eleme}{\elem{e}}                             % domain element e
\newcommand{\elemf}{\elem{f}}                               % domain element f
\newcommand{\elemg}{\elem{g}}                             % domain element g
\newcommand{\elemh}{\elem{h}}                             % domain element h
\newcommand{\elemi}{\elem{i}}                             % domain element i
\newcommand{\elemj}{\elem{j}}                             % domain element j
\newcommand{\elemw}{\elem{w}}                             % domain element w
\newcommand{\elemv}{\elem{v}}                             % domain element v
\newcommand{\elemu}{\elem{u}}                             % domain element u
\newcommand{\elemtuplea}{\overline{\elema}}                         % tuple of domain element a
\newcommand{\elemtupleb}{\overline{\elemb}}                         % tuple of domain element b
\newcommand{\elemtuplec}{\overline{\elemc}}                         % tuple of domain element c
\newcommand{\elemtupled}{\overline{\elemd}}                         % tuple of domain element d
\newcommand{\elemtuplee}{\overline{\eleme}}                         % tuple of domain element e
\newcommand{\elemtuplef}{\overline{\elemf}}                           % tuple of domain element f
\newcommand{\elemtupleg}{\overline{\elemg}}                         % tuple of domain element g
\newcommand{\elemtupleh}{\overline{\elemh}}                         % tuple of domain element h
\newcommand{\elemtuplei}{\overline{\elemi}}                         % tuple of domain element i
\newcommand{\elemtuplej}{\overline{\elemj}}                         % tuple of domain element j
\newcommand{\elemtuplew}{\overline{\elemw}}                         % tuple of domain element w
\newcommand{\elemtupleu}{\overline{\elemu}}                         % tuple of domain element u
\newcommand{\elemtuplev}{\overline{\elemv}}                         % tuple of domain element v

\newcommand{\elemtupledfromto}[2]{\overline{\elemd}_{#1\ldots#2}}  % tuple of domelements d from #1 to #2
\newcommand{\elemtupleefromto}[2]{\overline{\eleme}_{#1\ldots#2}}  % tuple of domelements e from #1 to #2
\newcommand{\elemtuplecfromto}[2]{\overline{\elemc}_{#1\ldots#2}}  % tuple of domelements c from #1 to #2
\newcommand{\elemtuplebfromto}[2]{\overline{\elemb}_{#1\ldots#2}}  % tuple of domelements b from #1 to #2
\newcommand{\elemtupleafromto}[2]{\overline{\elema}_{#1\ldots#2}}  % tuple of domelements a from #1 to #2
\newcommand{\elemtupleffromto}[2]{{\elemtuplef}_{#1\dots#2}} % tuple of domelements f from #1 to #2
\newcommand{\elemtuplegfromto}[2]{{\elemtupleg}_{#1\dots#2}} % tuple of domelements g from #1 to #2


\newcommand{\elemtuptuplea}{\vv{\elema}}                         % tuple of domain element a
\newcommand{\elemtuptupleb}{\vv{\elemb}}                         % tuple of domain element b
\newcommand{\elemtuptuplec}{\vv{\elemc}}                         % tuple of domain element c
\newcommand{\elemtuptupled}{\vv{\elemd}}                         % tuple of domain element d
\newcommand{\elemtuptuplee}{\vv{\eleme}}                         % tuple of domain element e
\newcommand{\elemtuptuplef}{\vv{\elemf}}                           % tuple of domain element f
\newcommand{\elemtuptupleg}{\vv{\elemg}}                         % tuple of domain element g
\newcommand{\elemtuptupleh}{\vv{\elemh}}                         % tuple of domain element h
\newcommand{\elemtuptuplei}{\vv{\elemi}}                         % tuple of domain element i
\newcommand{\elemtuptuplej}{\vv{\elemj}}                         % tuple of domain element j
\newcommand{\elemtuptuplew}{\vv{\elemw}}                         % tuple of domain element w
\newcommand{\elemtuptupleu}{\vv{\elemu}}                         % tuple of domain element u
\newcommand{\elemtuptuplev}{\vv{\elemv}}                         % tuple of domain element v





% Variables:
\newcommand{\var}[1]{\mathit{#1}}       % variable
\newcommand{\varx}{\var{x}}             % variable x
\newcommand{\vary}{\var{y}}             % variable y
\newcommand{\varz}{\var{z}}             % variable z
\newcommand{\varv}{\var{v}}             % variable v
\newcommand{\varu}{\var{u}}             % variable u
\newcommand{\varw}{\var{w}}             % variable w
\newcommand{\varh}{\var{h}}             % variable h
\newcommand{\vartuplex}{\overline{\varx}}    % tuple of variables x
\newcommand{\vartuplexomega}{\overline{\varx_{\omega}}}      % tuple of variables x_omega
\newcommand{\vartupley}{\overline{\vary}}                    % tuple of variables y
\newcommand{\vartupleyone}{\overline{\vary_1}}                    % tuple of variables y_1
\newcommand{\vartupleytwo}{\overline{\vary_2}}                    % tuple of variables y_2
\newcommand{\vartuplez}{\overline{\varz}}                    % tuple of variables z
\newcommand{\vartuplev}{\overline{\varv}}                    % tuple of variables v
\newcommand{\vartupleu}{\overline{\varu}}                    % tuple of variables u
\newcommand{\vartuplew}{\overline{\varw}}                    % tuple of variables w
\newcommand{\vartupleh}{\overline{\varh}}                    % tuple of variables h
\newcommand{\vartuplexfromto}[2]{\overline{\varx}_{#1\ldots#2}}  % tuple of variables x from #1 to #2
\newcommand{\vartupleyfromto}[2]{\overline{\vary}_{#1\ldots#2}}  % tuple of variables y from #1 to #2
\newcommand{\vartupleufromto}[2]{\overline{\varu}_{#1\ldots#2}}  % tuple of variables u from #1 to #2
\newcommand{\vartuplevfromto}[2]{\overline{\varv}_{#1\ldots#2}}  % tuple of variables v from #1 to #2
\newcommand{\vartuplewfromto}[2]{\overline{\varw}_{#1\ldots#2}}  % tuple of variables v from #1 to #2

% Theory
\newcommand{\theory}[1]{\mathcal{#1}}   % theory
\newcommand{\theoryT}{\theory{T}}       % theory T

% Types
\newcommand{\tp}[3]{\mathsf{tp}^{#1}_{#2}(#3)}


% Restrictions
\renewcommand{\restriction}{\mathord{\upharpoonright}}
\newcommand{\restr}[2]{#1\restriction_{#2}} % the restriction of #1 to #2

% morphisms
\newcommand{\homo}[1]{\mathfrak{#1}}    % homomorphism
\newcommand{\homof}{\homo{f}}           % homomorphism f
\newcommand{\homog}{\homo{g}}           % homomorphism g
\newcommand{\homoh}{\homo{h}}           % homomorphism h
\newcommand{\ishomoto}{\vartriangleleft} % is homomorhic to
\newcommand{\homeq}{\rightleftarrows} % homomorphically equivalent
\newcommand{\isoeq}{\cong} % isomorphic
\newcommand{\elemext}{\preceq} % elementary extension
\newcommand{\omegasat}[1]{\widehat{#1}}
\newcommand{\partisof}{\homo{f}}           % partial isomorphism f
\newcommand{\partisog}{\homo{g}}           % partial isomorphism g
\newcommand{\partisoh}{\homo{h}}           % partial isomorphism h

% bisimulations
\newcommand{\bisimulation}[1]{\mathcal{#1}} % a bisimulation
\newcommand{\bisimZ}{\bisimulation{Z}}
\newcommand{\bisimto}{\sim} % bisimilarity relation
\newcommand{\strbisimto}{\approx} % strong bisimilarity relation
\newcommand{\PartIso}[2]{\mathsf{Part}(#1,#2)}
