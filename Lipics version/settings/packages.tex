%!TEX root = ../main.tex

\frenchspacing % Spacing and typesetting
\usepackage{microtype} % Micro typing improvements
\usepackage{ellipsis} % improves whitespacing around "..."s
\usepackage[abbreviations,british]{foreign} % for better typing i.e., e.g. etc. 

% Knowledge package
\usepackage{xcolor}
\usepackage{hyperref}

\definecolor{darkmidnightblue}{rgb}{0.0, 0.2, 0.4}
\definecolor{persianplum}{rgb}{0.44, 0.11, 0.11}

\hypersetup{
  colorlinks  = true, % Colours links instead of ugly boxes
  citecolor   = persianplum, % Colour of citations
  urlcolor    = persianplum, % Colour for external hyperlinks,
  linkcolor   = persianplum
}

% TODO notes
\usepackage{tikz}
\usepackage{todonotes}
\usepackage{xspace}

% Math stuff
\let\Bbbk\relax
\usepackage{mathtools}
\usepackage{amssymb} 
\usepackage{amsmath}
\usepackage{amsthm}
\usepackage{apxproof}
%\usepackage{thmtools}

% Pictures
\usepackage{caption}
\usepackage{float}
\usepackage{subcaption}
\usepackage{marvosym}

% Arrows
\usepackage[h]{esvect}

% Tikz settings by Nicolas Markey
\usetikzlibrary{arrows, decorations.markings, shapes, calc, positioning, fit, graphs}

\colorlet{jaune}{yellow!80!green}
\def\enjaune#1{\textcolor{jaune!70!red!70!green}{#1}}
\colorlet{vert}{green!45!black}
\def\envert#1{\textcolor{vert}{#1}}
\colorlet{bleu}{blue!70!black}
\def\enbleu#1{\textcolor{bleu}{#1}}
\colorlet{rouge}{red!80!black}
\def\enrouge#1{\textcolor{rouge}{#1}}

\tikzstyle{minirond}=[draw, circle, minimum height=2mm, inner sep=0pt]
\tikzstyle{ptrond}=[draw, circle, minimum height=2.5mm]
\tikzstyle{medrond}=[draw, circle, minimum height=5mm]
\tikzstyle{rond}=[draw, circle, minimum height=7mm]

\tikzstyle{carre}=[draw,minimum width=6mm,minimum height=6mm]
\tikzstyle{medcarre}=[draw,minimum width=4mm,minimum height=4mm]
\tikzstyle{ptcarre}=[draw,minimum width=2.8mm,minimum height=2.8mm]
\tikzstyle{minicarre}=[draw,minimum width=1.5mm,minimum height=1.5mm,inner sep=0pt]
%
\tikzstyle{rouge}=[draw=red,fill=red!20!white]
\tikzstyle{vert}=[draw=green!80!black,fill=green!80!black!20!white]
\tikzstyle{jaune}=[draw=yellow!60!red,fill=yellow!60!red!30!white]
\tikzstyle{bleu}=[draw=blue,fill=blue!40!white]
\tikzstyle{gris}=[draw=black!80!white,fill=black!40!white]
%
\tikzstyle{rougef}=[draw=red,fill=red!60!white]
\tikzstyle{vertf}=[draw=green!80!black,fill=green!80!black!60!white]
\tikzstyle{jaunef}=[draw=yellow!80!black,fill=yellow!80!black!60!white]
\tikzstyle{bleuf}=[draw=blue,fill=blue!70!white]
%
\tikzstyle{rjaune}=[style=rond,style=jaune]
\tikzstyle{rbleu}=[style=rond,style=bleu]
\tikzstyle{rvert}=[style=rond,style=vert]
\tikzstyle{rrouge}=[style=rond,style=rouge]
\tikzstyle{rgris}=[style=rond,style=gris]
%
\tikzstyle{cjaune}=[style=carre,style=jaune]
\tikzstyle{cbleu}=[style=carre,style=bleu]
\tikzstyle{cvert}=[style=carre,style=vert]
\tikzstyle{crouge}=[style=carre,style=rouge]
\tikzstyle{cgris}=[style=carre,style=gris]
%
\tikzstyle{rjaunef}=[style=rond,style=jaunef]
\tikzstyle{rbleuf}=[style=rond,style=bleuf]
\tikzstyle{rvertf}=[style=rond,style=vertf]
\tikzstyle{rrougef}=[style=rond,style=rougef]

% Double colours in nodes: https://tex.stackexchange.com/questions/343354/tikz-rectangle-with-diagonal-fill-two-colors
\usetikzlibrary{shadows}
\tikzset{
diagonal fill/.style 2 args={fill=#2, path picture={
\fill[#1, sharp corners] (path picture bounding box.south west) -|
                         (path picture bounding box.north east) -- cycle;}},
reversed diagonal fill/.style 2 args={fill=#2, path picture={
\fill[#1, sharp corners] (path picture bounding box.north west) |- 
                         (path picture bounding box.south east) -- cycle;}}
}

\usetikzlibrary{hobby,backgrounds,calc,trees}

% Category-theory-like diagrams
% https://tikzcd.yichuanshen.de/
\usepackage{tikz-cd}
\usetikzlibrary{matrix}
\usepackage{stmaryrd}
\usepackage{stackengine}
\usepackage{xspace}
\usepackage{xcolor}
\usepackage[linesnumbered,ruled,vlined]{algorithm2e}
\newcommand\mycommfont[1]{\footnotesize\ttfamily\textcolor{blue}{#1}}
\SetCommentSty{mycommfont}
\SetKwInput{KwData}{Input}
\renewcommand*{\algorithmcfname}{Procedure}
\renewcommand*{\algorithmautorefname}{procedure}
\newtheorem{fact}{Fact}
