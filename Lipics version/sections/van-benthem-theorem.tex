%!TEX root = ../main.tex

\section{Expressive Completeness}\label{sec:van-benthem-theorem}


Let $\logicL$ be a fragment of $\FO$. 
A first-order formula~$\varphi$ is called \emph{$\bisimto_{\logicL}$-invariant}
if for all pointed structures $(\str{A},\elemtuplea)$ and $(\str{B}, \elemtupleb)$ we have that  
$(\str{A},\elemtuplea) \bisimto_{\logicL[\sig(\varphi)]} (\str{B}, \elemtupleb)$ implies
$\str{A} \models \varphi[\elemtuplea] \Leftrightarrow \str{B} \models \varphi[\elemtupleb]$, \ie if $\logicL[\sig(\varphi)]$-bisimilar structures agree on the satisfaction of $\varphi$.
Similarly, $\varphi$ is called \emph{$\bisimto_{\logicL}$-invariant in the finite} if the above implication holds for all \emph{finite} pointed structures $(\str{A},\elemtuplea)$ and $(\str{B}, \elemtupleb)$.
A logic $\logicL$ is the \emph{$\bisimto_{\logicL}$-invariant fragment} (in the finite) of $\FO$ if every formula of~$\logicL$ is $\logicL$-invariant (in the finite) and every $\logicL$-invariant first-order formula is equivalent (in the finite) to some formula of $\logicL$.
A celebrated result of van Benthem states that modal logic $\mathrm{K}$ is the bisimulation-invariant fragment of first-order logic~\cite[Thm.~2.2.1]{AndrekaNB98}. The constructive version of the theorem (which works also in the finite) was provided later by Rosen~\cite[Prop.~4]{Rosen97}.
A~constructive analogous of van Benthem's theorem for $\GF$ was given by Otto~\cite{Otto2012}.
\begin{theorem}\label{thm:vanBenthem-for-GF}
  The guarded fragment is the $\bisimto_{\GF}$-invariant fragment of $\FO$, both in a sense of classical model theory and in the finite.
  More specifically, for any finite signature $\sigma$, there is an \possiblelie{exponential}\bbeside{Asked Otto. Waiting for the answer. EDIT: He thinks that this is true. But I'm not sure whether we can write it this way...} function $\homoe \colon \N \to \N$ such that every $\FO[\sigma]$-formula $\varphi$ of quantifier rank $q$ that is $\bisimto_{\GF[\sigma]}$-invariant (in the finite) is also $\bisimto^{\homoe(q)}_{\GF[\sigma]}$-invariant (in the finite) and logically-equivalent (in the finite) to a $\GF[\sigma]$-formula of quantifier rank at most $\homoe(q)$.
\end{theorem}
In~\cite[Thm. 5]{BednarczykJ22} a non-constructive version of van Benthem's theorem for $\FGF$ is given. 
Unfortunately the proof heavily relies on model-theoretic toolkit like compactness, that is well-known to fail in the finite~\cite[Prop.~3.2]{Libkin04}. Thus the proof of~\cite[Thm.~5]{BednarczykJ22} does not provide any results in the finite model scenario.
The goal of this paper is to provide its constructive analogue, that can also be applied to the finite case.
Our main technical contribution is:
\begin{theorem}\label{thm:main-technical-thm}
  Let $\sigma$ be a finite signature. 
  There exists a positive polynomial function~$\homop$ such that for every $\ell$ and every pair of $\ell$-$\FGF[\sigma]$-bisimilar pointed $\sigma$-structures, there exists $\FGF[\sigma]$-bisimilar companions $(\str{A}', \elemtuplea')$ and $(\str{B}', \elemtupleb')$ (that are finite if $\str{A}$ and $\str{B}$ are) of $(\str{A}, \elemtuplea)$ and $(\str{B}, \elemtupleb)$ such that $(\str{A}', \elemtuplea') \bisimto^{\homop(\ell)}_{\GF[\sigma]} (\str{B}', \elemtupleb')$.
\end{theorem}
We next show how~\cref{thm:main-technical-thm} implies a van Benthem's style characterisation of $\FGF$. More precisely:
\begin{theorem}
The forward guarded fragment is the $\bisimto_{\FGF}$-invariant fragment of $\FO$, both in a sense of classical model theory and in the finite.
\end{theorem}
\begin{proof}
By employing previous developments of Otto~\cite[Obs.~13]{Otto04}, uniformly stated for fragments of first-order logic, it suffices to show that for every first-order formula that is $\bisimto_{\FGF}$-invariant (in the finite) is also $\bisimto_{\FGF}^{\ell}$-invariant (in the finite) for some $\ell \in \N$.

Take any such first-order formula $\varphi$, and let $\homog$ be the function provided by~\cref{thm:main-technical-thm}. 
As $\FGF$ is a fragment of $\GF$, we know that $\varphi$ is $\bisimto_{\GF}$-invariant (in the finite).
Thus $\varphi$ is also $\bisimto_{\GF}^{\mathit{g}}$-invariant (in the finite) for $\mathit{g} \in \N$ provided by~\cref{thm:vanBenthem-for-GF}.
Let $\ell$ be minimal number for which $\homop(\ell) \geq \mathit{g}$ holds (it exists by the positivity of $\homop$).
We will show that $\varphi$ is $\bisimto_{\FGF}^{\ell}$-invariant (in the finite).
To do so, let $(\str{A}, \elemtuplea)$ and $(\str{B}, \elemtupleb)$ be $\ell$-$\FGF[\sigma]$-bisimilar pointed $\sigma$-structures, and suppose that $(\str{A}, \elemtuplea) \models \varphi$. It remains to show that $(\str{B}, \elemtupleb) \models \varphi$.
From~\cref{thm:main-technical-thm} we infer the existence of $\FGF$-bisimilar companions $(\str{A}', \elemtuplea')$ and $(\str{B}', \elemtupleb')$, that are finite if $\str{A}$ and $\str{B}$ are, for which $(\str{A}', \elemtuplea') \bisimto^{\homop(\ell)}_{\GF[\sigma]} (\str{B}', \elemtupleb')$ holds. 
The following diagram depicts what will happen next.
\begin{figure}[H]
  \centering
  \begin{tikzpicture}
    \matrix[row sep=2em, column sep=3em]
    {
        \node (a) {$(\str{A}, \elemtuplea)$}; &
         \node[font=\large] (sim_ab) {$\bisimto_{\FGF}^{\ell}$}; &
        \node (b) {$(\str{B}, \elemtupleb)$}; &
        \node[anchor=west] {}; \\

        \node[font=\large] (sim_a_unravel) {$\bisimto_{\FGF}$}; &
        \node {}; &
        \node[font=\large] (sim_b_unravel) {$\bisimto_{\FGF}$}; &
        \node[anchor=west] {}; \\

        \node (a_unravel) {$(\str{A}', \elemtuplea')$}; &
        \node[font=\large] (sim_unravel) {$\bisimto_{\GF}^{\mathit{g}}$}; &
        \node (b_unravel) {$(\str{B}', \elemtupleb')$}; 
        \node[anchor=west] {}; \\
    };

    \draw[->] (a) -- (sim_ab) -> (b);
    \draw[dashed,->] (a) -- (sim_a_unravel) -> (a_unravel);
    \draw[->] (a_unravel) -- (sim_unravel) -> (b_unravel);
    \draw[dashed,->] (b_unravel) -- (sim_b_unravel) -> (b);
  \end{tikzpicture}
  % \caption{Upgrading $\bisimto_{\FGF}^{\ell}$-bisimulation to $\bisimto_{\GF}^{\mathit{g}}$-bisimulation.}
  \label{fig:fgf-to-gf-upgrade}
\end{figure}

It is now clear that: 
(i) $(\str{A}, \elemtuplea) \models \varphi$ (by assumption),
(ii) $(\str{A}', \elemtuplea') \models \varphi$ (by $\FGF[\sigma]$-bisimilarity and $\bisimto_{\FGF}$-invariance of $\varphi$),
(iii) $(\str{B}', \elemtupleb') \models \varphi$ (by $\bisimto_{\GF}^{\mathit{g}}$-invariance of $\varphi$ and the choice of $\ell$),
(iv) $(\str{B}, \elemtupleb) \models \varphi$ (by $\FGF[\sigma]$-bisimilarity and $\bisimto_{\FGF}$-invariance of $\varphi$). 
This concludes the proof.
\end{proof}

% Van Benthem's theorem shows that modal logic corresponds to the bisimulation-invariant subset of first order logic~\cite{van1983modal}, which holds even in finite structures~\cite{Rosen97}.
% We can prove an analogous theorem for the forward guarded fragment, using our modified notion of bisimulation:

% \begin{theorem}
%   The following two statements are equivalent:
%   \begin{enumerate}
%     \item $\phi(\bar{x}) \in \Logic{FO}$ is invariant under \FGF-bisimulation in finite models
%     \item $\phi(\bar{x})$ is logically equivalent to a \FGF~formula in finite models
%   \end{enumerate}
% \end{theorem}

% The prove this theorem, we use the following oberservation by Otto~\cite{Otto04}:
% \begin{observation}
%   Let $\bisimto$ have finite approximations $\bisimto^{\ell}, \ell \in \mathbb{N}$.
%   Assume that each $\bisimto^{\ell}$ has finite index for every finite relational vocabulary.
%   Let $\Logic{L} = \bigcup_{\ell}{\Logic{L}^{\ell}}$ be a logic, each stratum $\Logic{L}^{\ell}$ closed under disjunctions.
%   Assume that each $\Logic{L}^{\ell}$ is invariant under $\bisimto^{\ell}$ and that each $\bisimto^{\ell}$-class is definable by a formula of $\Logic{L}^{\ell}$.

%   Then the following are equivalent, both in the sense of classical model theory and of finite model theory:
%   \begin{enumerate}[(i)]
%     \item Every FO formula that is $\bisimto$-invariant is invariant under $\bisimto^{\ell}$ for some $\ell$.
%     \item Every FO formula that is $\bisimto$-invariant is equivalent to some formula in $\Logic{L}$.
%   \end{enumerate}
% \end{observation}

% We have shown previously that $\FGF$, $\bisimto_{\FGF}$ and $\bisimto_{\FGF}^{\ell}$ satisfy the preconditions of this observation.
% Hence it is sufficient to show condition (i).
% Following Otto, we prove this condition by upgrading \FGF~bisimulation to a stronger form of bisimulation: \GF~bisimulation:

% \begin{proof}
%   Let $\phi$ be an $\FO$ formula invariant under $\bisimto_{\FGF}$.
%   $\phi$ is then also invariant under $\bisimto_{\GF}$, since $\bisimto_{\GF}$ is stronger than $\bisimto_{\FGF}$.
%   This implies that there exists a finite $g$ such that $\phi$ is invariant under $\bisimto_{\GF}^{g}$, as a result of Otto's proof of the van Benthem theorem for \GF~\cite{Otto2012}.

%   By theorem \todo[inline,inlinewidth=5cm,noinlinepar]{TODO: upgrading theorem}, there is a finite $\ell$ and $\theta$ for this $g$ such that for arbitrary models, $~^{\ell}$-\FGF-bisimilarity implies that their $\theta$-unravelings are $g$-\GF-bisimilar.
%   Let $\str{A}, \bar{a} \bisimto_{\FGF}^{\ell} \str{B}, \bar{b}$ be two $\ell$-\FGF-bisimilar models.
%   Then, $\str{A}, \bar{a} \models \phi(\bar{x}) \implies \str{B}, \bar{b} \models \phi(\bar{x})$ (and vice-versa by symmetry), as can be seen following \cref{fig:fgf-to-gf-upgrade}:
%   \begin{itemize}
%     \item $\phi(\bar{x})$ is preserved in $\str{A}^{\rightarrow}_{\theta}, \overrightarrow{a}$, since $\phi(\bar{x})$ is preserved under \FGF-bisimulation by assumption and the unraveling is fully \FGF-bisimlar
%     \item $\phi(\bar{x})$ is preserved in $\str{B}^{\rightarrow}_{\theta}, \overrightarrow{b}$ since the unravelings are $\bisimto_{\GF}^{g}$ equivalent and $\phi$ is invariant under $\bisimto_{\GF}^{g}$
%     \item $\phi(\bar{x})$ is preserved in $\str{B}, \bar{b}$ again since the unravelings are \FGF-bisimilar
%   \end{itemize}
%   This shows that $\phi$ is invariant under $\bisimto_{\FGF}^{\ell}$, completing the proof.
% \end{proof}

% \begin{figure}
%   \centering
%   \begin{tikzpicture}
%     \matrix[row sep=2em, column sep=3em]
%     {
%         \node (a) {$\str{A}, \bar{a}$}; &
%         \node[font=\large] (sim_ab) {$\bisimto_{\FGF}^{\ell}$}; &
%         \node (b) {$\str{B}, \bar{b}$}; &
%         \node[anchor=west] {$\ell$ depends on $g$}; \\

%         \node[font=\large] (sim_a_unravel) {$\bisimto_{\FGF}$}; &
%         \node {}; &
%         \node[font=\large] (sim_b_unravel) {$\bisimto_{\FGF}$}; &
%         \node[anchor=west] {full FGF-bisimilarity, without bound}; \\

%         \node (a_unravel) {$\str{A}^{\rightarrow}_{\theta}, \overrightarrow{a}$}; &
%         \node[font=\large] (sim_unravel) {$\bisimto_{\GF}^{g}$}; &
%         \node (b_unravel) {$\str{B}^{\rightarrow}_{\theta}, \overrightarrow{b}$}; &
%         \node[anchor=west] {$g$ given by Otto, depends on the quantifier rank of $\phi$}; \\
%     };

%     \draw[->] (a) -- (sim_ab) -> (b);
%     \draw[dashed,->] (a) -- (sim_a_unravel) -> (a_unravel);
%     \draw[->] (a_unravel) -- (sim_unravel) -> (b_unravel);
%     \draw[dashed,->] (b_unravel) -- (sim_b_unravel) -> (b);
%   \end{tikzpicture}
%   \caption{Upgrading FGF to GF bisimulation}
%   \label{fig:fgf-to-gf-upgrade}
% \end{figure}
