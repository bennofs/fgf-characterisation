%!TEX root = ../main.tex

\section{Tree-Like Unraveling}\label{sec:unraveling}
We will now describe the construction of the companion structures required in \cref{thm:main-technical-thm}.
Let $\str{A}$ be a structure with an associated binary relation $\relNext \subseteq A \times A$.
A \emph{parent} of an element $e \in A$ is an element $p_{e} \in A$ such that $(p_{e}, e) \in A$.
A \emph{root} is an element which has no parents.
The structure $\str{A}$ is a \emph{forest} if every element has at most one parent and there is at least one root.
A \emph{tree} is a forest with exactly one root.
Using the well-known unraveling for transation systems, it can be shown that every satisfiable modal logic formula is satisfiable in a tree model.
The same holds true for \FGF-formula, using the HAF-unraveling introduced by Bednarczyk~\cite[Sec 3.3]{Bednarczyk21}.
Even though the tree unraveling of a structure $\str{A}, \elemtuplea$ can be infinite for finite $\str{A}$,
for many modal logic variants, companions for a van Benthem style characterisation can be constructed based on the infinite unraveling~\cite{Otto04}.

Unfortuntately, constructing a finite version of the HAF-unraveling is not enough to prove \cref{thm:main-technical-thm}.
Consider the two finite structures shown in \cref{fig:unravel-haf}, which are both already fully HAF-unraveled, but can be distinguished by the GF sentence $\forall b,c. E(b,c) \implies \exists a. P(a,b,c)$.
This shows that even if the HAF-unraveling is finite, it does not provide GF-bisimilar companions.
\begin{figure}
  \centering
    \begin{tikzpicture}[baseline=(current bounding box.north)]
        \draw[tolbrightGreen, line cap=round, line width=2em] (-0em,8em) -- ++(0,-8em);
        \draw[tolbrightYellow, line cap=round, line width=0.5em, -{Latex[length=2em]}] (0,4em) -- (0,0em);

        \draw [black, line width=0.1em, fill=white] (0em, 8em) circle [radius=0.8em] node[anchor=center] {1};
        \draw [black, line width=0.1em, fill=white] (0em, 4em) circle [radius=0.8em] node[anchor=center] {2};
        \draw [black, line width=0.1em, fill=white] (0em, 0em) circle [radius=0.8em] node[anchor=center] {3};

        \begin{scope}[xshift=10em]
            \draw[tolbrightGreen, line cap=round, line width=2em] (-0em,8em) -- ++(0,-8em);
            \draw[tolbrightYellow, line cap=round, line width=0.5em, -{Latex[length=2em]}] (0em,4em) -> (6em,2em);
            \draw[tolbrightYellow, line cap=round, line width=0.5em, -{Latex[length=2em]}] (0,4em) -- (0,0em);

            \draw [black, line width=0.1em, fill=white] (0em, 8em) circle [radius=0.8em] node[anchor=center] {1};
            \draw [black, line width=0.1em, fill=white] (0em, 4em) circle [radius=0.8em] node[anchor=center] {2};
            \draw [black, line width=0.1em, fill=white] (0em, 0em) circle [radius=0.8em] node[anchor=center] {3};
            \draw [black, line width=0.1em, fill=white] (6em, 2em) circle [radius=0.8em] node[anchor=center] {3'};

            \node[tolbrightGreen] at (-2em, 6em) {P};
            \node[tolbrightYellow] at (-1.5em, 2em) {E};
            \node[tolbrightYellow] at (3em, 4em) {E};
        \end{scope}

        \node[font=\Large] at (5em, 4em) {$\sim_{FGF}$};

        \node[tolbrightGreen] at (-2em, 6em) {P};
        \node[tolbrightYellow] at (-1.5em, 2em) {E};
    \end{tikzpicture}%
    \caption{relations are always top to bottom, so the green area marks the relation $\relP(1,2,3)$}%
    \label{fig:unravel-haf}
\end{figure}

% introduce unraveling domain
    % bisimulation sequences
    % counter
% describe finite unraveling
% proof: finite unraveling is FGF-bisimilar to orig structure
% lemma: intersection between any two live tuples is infix
