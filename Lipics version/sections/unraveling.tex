%!TEX root = ../main.tex

\section{Tree-Like Unraveling}\label{sec:unraveling}
We now describe a tree unraveling for \FGF, which we later use to construct the companion structures required in \cref{thm:main-technical-thm}.
Let $\str{A}$ be a structure with an associated binary relation $\relNext \subseteq A \times A$.
For $(e_{1}, e_{2}) \in \relNext$, we call $e_{2}$ a \emph{child} of $e_{1}$ and $e_{1}$ a $\emph{parent}$ of $e_{2}$.
A \emph{root} is an element which has no parents.
The set of $\emph{descendants}$ for an element $e$ is the smallest set which contains $e$ itself and every child of each element in the set.
The structure $\str{A}$ is a \emph{forest} if every element has at most one parent and there is at least one root.
A \emph{tree} is a forest with exactly one root.
Using the well-known unraveling for transation systems, it can be shown that every satisfiable modal logic formula is satisfiable in a tree model.
The same holds true for \FGF-formula, using the HAF-unraveling introduced by Bednarczyk~\cite[Sec 3.3]{Bednarczyk21}.
Even though the tree unraveling of a structure $\str{A}, \elemtuplea$ can be infinite for finite $\str{A}$,
for many modal logic variants, companions for a van Benthem style characterisation can be constructed based on the infinite unraveling~\cite{Otto04}.

Unfortuntately, constructing a finite version of the HAF-unraveling is not enough to prove \cref{thm:main-technical-thm}.
Consider the two finite structures shown in \cref{fig:unravel-haf}, which are both already fully HAF-unraveled, but can be distinguished by the GF sentence $\forall b,c. E(b,c) \implies \exists a. P(a,b,c)$.
This shows that even if the HAF-unraveling is finite, it does not provide GF-bisimilar companions.
\begin{figure}
  \centering
    \begin{tikzpicture}[baseline=(current bounding box.north)]
        \draw[tolbrightGreen, line cap=round, line width=2em] (-0em,8em) -- ++(0,-8em);
        \draw[tolbrightYellow, line cap=round, line width=0.5em, -{Latex[length=2em]}] (0,4em) -- (0,0em);

        \draw [black, line width=0.1em, fill=white] (0em, 8em) circle [radius=0.8em] node[anchor=center] {1};
        \draw [black, line width=0.1em, fill=white] (0em, 4em) circle [radius=0.8em] node[anchor=center] {2};
        \draw [black, line width=0.1em, fill=white] (0em, 0em) circle [radius=0.8em] node[anchor=center] {3};

        \begin{scope}[xshift=10em]
            \draw[tolbrightGreen, line cap=round, line width=2em] (-0em,8em) -- ++(0,-8em);
            \draw[tolbrightYellow, line cap=round, line width=0.5em, -{Latex[length=2em]}] (0em,4em) -> (6em,2em);
            \draw[tolbrightYellow, line cap=round, line width=0.5em, -{Latex[length=2em]}] (0,4em) -- (0,0em);

            \draw [black, line width=0.1em, fill=white] (0em, 8em) circle [radius=0.8em] node[anchor=center] {1};
            \draw [black, line width=0.1em, fill=white] (0em, 4em) circle [radius=0.8em] node[anchor=center] {2};
            \draw [black, line width=0.1em, fill=white] (0em, 0em) circle [radius=0.8em] node[anchor=center] {3};
            \draw [black, line width=0.1em, fill=white] (6em, 2em) circle [radius=0.8em] node[anchor=center] {3'};

            \node[tolbrightGreen] at (-2em, 6em) {P};
            \node[tolbrightYellow] at (-1.5em, 2em) {E};
            \node[tolbrightYellow] at (3em, 4em) {E};
        \end{scope}

        \node[font=\Large] at (5em, 4em) {$\sim_{FGF}$};

        \node[tolbrightGreen] at (-2em, 6em) {P};
        \node[tolbrightYellow] at (-1.5em, 2em) {E};
    \end{tikzpicture}%
    \caption{Two FGF-bisimilar HAFs which are not GF-bisimilar. Relations are drawn top to bottom, so the green area marks the relation $\relP(1,2,3)$}%
    \label{fig:unravel-haf}
\end{figure}
We construct an unraveling that fulfills the following theorem:
\begin{theorem}\label{thm:inf-unraveling-upgrading}
  Let $\str{A}, \elemtuplea \bisimto_{\FGF} \str{B}, \elemtupleb$ for two pointed $\sigma$-structures.
  Then there are unravelings $\unravel{A}, \elemtuptuplea$ and $\unravel{B}, \elemtuptupleb$ which are both:
  \begin{itemize}
    \item $\FGF$-similar to the original structures: $\unravel{A}, \elemtuptuplea \bisimto_{\FGF} \str{A}, \elemtuplea$ and $\unravel{B}, \elemtuptupleb \bisimto_{\FGF} \str{B}, \elemtupleb$
    \item $\GF$-bisimilar: $\unravel{A}, \elemtuptuplea \bisimto_{\GF} \unravel{B}, \elemtuptupleb$
  \end{itemize}
\end{theorem}

\noindent \textbf{Domain of the unraveling}
Let $\str{A}, \elemtuplea^{(0)}$ be a pointed $\sigma$-structure.
A \emph{bisimulation sequence} of length $\ell$ is a sequence of the form $\elemtuplea^{(0)}(i^{(1)}, j^{(1)})\elemtuplea^{(1)}\cdots(i^{(\ell)}, j^{(\ell)})\elemtuplea^{(\ell)}$, where each $a^{(k)}$ is a live tuple in $\str{A}$ and $i^{(k)}, j^{(k)}$ are indices such that $\elemtupleafromto{i^{(k)}}{j^{(k)}}^{(k-1)} = \elemtupleafromto{1}{j^{(k)}-i^{(k)}+1}^{k}$.
These bisimulation sequences arise naturally when considering bisimilar structures to $\str{A}$.
If $\str{B}, \elemtupleb^{(0)}$ is a $\sigma$-structure that is \FGF-bisimilar to $\str{A}, \elemtuplea^{(0)}$, then we can apply \ref{bisim:fforth} $\ell$ times to find tuples $\elemtupleb^{1}, \ldots, \elemtupleb^{\ell}$ for a corresponding bisimulation sequence in $\str{B}$.
Note that the infix selected by the indices $(i^{(k)}, j^{(k)})$ is not required to be maximal.
This is illustrated in the following examples of bisimulation sequences.
\begin{figure}[H]
  \centering
    \begin{minipage}[t]{0.2\textwidth}
        \raggedleft
        \vspace{0pt}
        \includegraphics[scale=0.5]{res/example-struct-1}
    \end{minipage}
    \hspace{4em}
    \begin{minipage}[t]{0.6\textwidth}
      {%
      \newcommand{\tups}{{\color{tolbrightYellowDarker}\elemtuples}}%
      \newcommand{\tupp}{{\color{tolbrightCyanDarker}\elemtuplep}}%
      \newcommand{\tupt}{{\color{tolbrightGreen}\elemtuplet}}%
      \newcommand{\tupq}{{\color{tolbrightPurple}\elemtupleq}}%
      The picture on the left shows a structure with the relations: $\relS(1,2,3)$, $\relP(2,3,4)$, $\relT(3,4,5)$, $\relQ(4,5)$

      \vspace{1ex}
      Let $\tups = (1, 2, 3), \tupp = (2, 3, 4), \tupt = (3, 4, 5), \tupq = (4,5)$.

      \vspace{1ex}
      Some examples of bisimulation sequences in this structure are:
      \begin{itemize}
          \item $\tups(3,3)\tupt$
          \item $\tups(2,3)\tupp(2,3)\tupt$
      \end{itemize}

      Bisimulation sequences are not required to use maximal infixes, so the following are also valid bisimulation sequences:
      \begin{itemize}
          \item $\tups(2,2)\tupp$
          \item $\tups(3,3)\tupt(2,2)\tupq$
      \end{itemize}
      }
    \end{minipage}
    \caption{Examples for bisimulation sequences}
\end{figure}

Let $\Seq{A}$ be the set of bisimulation sequences for a structure $\str{A}$.
Consider the final tuple of a bisimulation sequence.
This tuple has a prefix of elements which are shared with the previous tuple, and a suffix of unshared elements.
We now introduce a counter to distinguish these unshared elements.
The \emph{unraveling domain} for $\str{A}$ is a set, defined as follows:
\bfside{This looks really ugly. Any way to write this in a better way?}
\begin{equation*}
\unraveldom{A} = \Seq{A} \times \mathbb{N} \setminus \left\{ (\sigma(i,j)\bar{a}, k) \colon\, k \le j-i+1\ \text{or}\ k > |a| \right\}
\end{equation*}
For an element $e \in \unraveldom{A}$ where $e = (\rho, k)$, we use the notation $\seq{e} = \rho$ and $\ctr{e} = k$ to denote the sequence and the counter of this element, respectively.
Let $\elemtuplea$ be the final tuple of $\seq{e}$, so $\seq{e} = \cdots \elemtuplea$.
Since the counter is an index into final tuple, we can define the projection $\pi(e)$ as: $\pi(e) = \elema_{\ctr{e}}$.
The unraveling domain is a forest\bfside{does this require some justification or is it obvious enough?}, with the binary relation $\relNext \subseteq \unraveldom{A} \times \unraveldom{A}$ defined such that $(s, t) \in \relNext$ iff either:
\begin{description}
  \item[\desclabel{(addCtr)}{next:addctr}] $\seq{s} = \seq{t}$, $\ctr{t} = \ctr{s} + 1$, or
  \item[\desclabel{(addSeq)}{next:addseq}] $\seq{t} = \seq{s} (i,j) \elemtuplea$ for some $i, j, \elemtuplea$ and $\ctr{s} = j, \ctr{t} = (j - i + 1) + 1$
\end{description}

\noindent
The definition of $\relNext$ has the following nice property: if $(s,t) \in \relNext$ and $t$ projects to $a_{k}$ for $k > 1$, then $s$ projects to $a_{k-1}$, where $k = \ctr{t}$ and $\elemtuplea$ is the tuple such that $\seq{t} = \cdots \elemtuplea$.
We prove this by case analysis on the two cases of $\relNext$.
The~\ref{next:addctr} case is simple: in this case $\seq{s} = \seq{t}$ and $\ctr{s} = k - 1$, so the property follows directly from the definition of $\pi$.
For the~\ref{next:addseq} case, let $\seq{t} = \seq{s} (i,j) \elemtuplea$ and $\seq{s} = \cdots \elemtupleb$.
Further, in this case $k = (j - i + 1) + 1$ and $\ctr{s} = j$.
By the fact that $\seq{t}$ is a bisimulation sequence, we know that $\elemtupleafromto{1}{j-i+1} = \elemtuplebfromto{i}{j}$.
In particular, this implies $\pi(s) = b_{j} = a_{j-i+1} = a_{k-1}$ as wanted.

\noindent
\textbf{Relations in the unraveling}
A tuple of elements $(e_{1}, \ldots, e_{n})$ is a \emph{next chain} if adjacent elements are related by $\relNext$, so $(e_{i}, e_{i+1}) \in \relNext$ for all $i$.
We define the tree unraveling such that relations are only realized by tuples which are next chains.
Let $\str{A}, \elemtuplea$ be a $\sigma$-structure and $\unraveldom{A}$ be the unraveling domain.
Let $\elemtuptuplea = ((\elemtuplea, 1), \ldots, (\elemtuplea, |\elemtuplea|))$.
The \emph{tree unraveling} $\unravel{A}, \elemtuptuplea$ is the tree with root $(\elemtuplea, 1)$ and all descendants according to the relation $\relNext$.
If $\relR \in \sigma$, then $\elemtuptupler \in \relR^{\unravel{A}}$ if and only if:
\begin{enumerate}
  \item $\pi[\elemtuptupler] \in \relR^{}$,
  \item $\bigwedge_{k=1}^{|\elemtuptupler|-1}{(\elemr_{k},\elemr_{k+1}) \in \relNext}$, and
  \item $|\elemtuptupler| \le \mathtt{bound}(\elemtuptupler)$, where $\mathtt{bound}(\elemtuptupler) = \ctr{\elemtuptupler_{|\elemtuptupler|}}$ (the counter of the last element of $\elemtuptupler$)
\end{enumerate}
The second condition ensures that live tuples are next chains.
The third condition is illustrated in the following example.
Recall the left structure from \cref{fig:unravel-haf}.
Let $\elemtuplep = (1,2,3)$ and $\elemtuplee = (2,3)$.
The picture below shows the tree unraveling with root $(\elemtuplep, 1)$.
\begin{figure}[H]
  \centering
  \begin{tikzpicture}
    \draw[tolbrightGreen, line cap=round, line width=2.2em] (-0em,8em) -- ++(0,-8em);
    \draw[tolbrightYellow, line cap=round, line width=0.5em, -{Latex[length=2em]}] (0em,4em) -> (6em,1em);
    \draw[tolbrightYellow, line cap=round, line width=0.5em, -{Latex[length=2em]}] (0,4em) -- (0,0em);

    \draw [black, line width=0.1em, fill=white] (0em, 8em) ellipse (1em and 0.8em) node[anchor=center] {$(\elemtuplep, 1)$};
    \draw [black, line width=0.1em, fill=white] (0em, 4em) ellipse (1em and 0.8em) node[anchor=center] {$(\elemtuplep, 2)$};
    \draw [black, line width=0.1em, fill=white] (0em, 0em) ellipse (1em and 0.8em) node[anchor=center] {$(\elemtuplep, 3)$};
    \draw [black, line width=0.1em, fill=white] (6em, 0em) ellipse (3em and 1em) node[anchor=center] {$(\elemtuplep(2,2)\elemtuplee, 2)$};

    \node[tolbrightGreen] at (-2em, 6em) {P};
    \node[tolbrightYellow] at (-1.5em, 2em) {E};
    \node[tolbrightYellow] at (3em, 4em) {E};
  \end{tikzpicture}
\end{figure}
First, observe that the structure is isomorphic to the structure on the right in \cref{fig:unravel-haf}.
Thus, at least in this example, \cref{thm:inf-unraveling-upgrading} is not violated.
Consider the tuples $\elemtuptuplea = ((\elemtuplep, 1), (\elemtuplep, 2), (\elemtuplep, 3))$ and $\elemtuptupleb = ((\elemtuplep, 1), (\elemtuplep, 2), (\elemtuplep(2,2)\elemtuplee, 2))$.
These tuples have equal projections: $\pi(\elemtuptuplea) = \pi(\elemtuptupleb) = \elemtuplep$.
But $\elemtuptuplea \in \relP^{\unravel{A}}$ while $\elemtuptupleb \notin \relP^{\unravel{A}}$.
This is because $\mathtt{bound}(\elemtuptupleb) = 2$ and $|\elemtuptupleb| = 3$, so the bound for $\elemtuptupleb$ is not large enough.
Note how the bisimulation sequence of the last element of $\elemtuptupleb$ is $\elemtuplep(2,2)\elemtuplee$, which has the final tuple $\elemtuplee$.
So this element represents choosing the tuple $\elemtuplee$ during \FGF-bisimulation.
Once we are at the tuple $\elemtuplee$, we cannot go back to $\elemtuplep$ again, since $\elemtuplep$ does not have a prefix that is shared with $\elemtuplee$.
Thus, the element does not need to be included in the relation $\relP$.

We now prove \cref{thm:inf-unraveling-upgrading} for this unraveling.
\begin{proof}
  \bfbox{write proof}
\end{proof}
