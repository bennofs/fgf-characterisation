%!TEX root = ../main.tex

\section{Finite Companions}\label{sec:finite}
Let $\unravel{A}, \elemtuptuplea$ be a tree unraveling of a finite structure $\str{A}, \elemtuplea$ over a signature $\sigma$.
We describe the finite unraveling $\unravel{A}_{\ell}, \elemtuptuplea_{\ell}$ which will be $\ell$-$\FGF$-bisimilar to $\unravel{A}, \elemtuptuplea$ (and also to $\str{A}, \elemtuplea$ by transitivity of bisimilarity).
For an element $r \in \unraveldom{A}$, we construct the finite tree $\mathcal{T}_{r,\ell}(\unravel{A})$ as follows:
\begin{equation*}
  \mathcal{T}_{r,\ell}(\unravel{A}) = \left\{ e \in \unraveldom{A}:\, e\ \text{is a descendant of $r$ and the length of $\seq{e}$ is at most $2 * \ell$} \right\}\bbe{.}
\end{equation*}
Note that this is a tree since for elements $a$ and $b$, if $b$ is a descendant of $a$ then the length of $\seq{b}$ is never less than that of $\seq{a}$.
Recall that a bisimulation sequence of length $\ell$ is just a word in $A^*{(\N\N{}A^{*})}^{\ell}$.
If $A$ is finite, then there are only a finite number of bisimulation sequences of length $\ell$ because the indices chosen from $\N$ are bounded by the maximum size of live tuples in $\str{A}$, which is not larger than the (finite) maximum arity of any relation $\relR \in \sigma$.
Thus, the number of different bisimulation sequences of length $\ell$ for the finite structure $\str{A}$ is finite.
For a given sequence, there are only a finite number of elements in the domain of the unraveling.
Because the sequence length of elements in $\mathcal{T}_{r,\ell}$ is bounded by $2 * \ell$, it follows that $\mathcal{T}_{r, \ell}$ is finite.
Now consider the set $S$ composed of trees for roots $r$ where the length of $\seq{r}$ is at most $\ell$.
An element $b$ is called a \emph{cut element} of a tree in $S$ if it itself is not part of the tree but $(a,b) \in \relNext$ for some $a$ in the tree.
Let $M \in \mathbb{N}$ be such that each cut element of a tree in $S$ can be assigned an index from $1, \ldots, M$, which is possible since the trees in $S$ are finite.
The domain of the structure $\unravel{A}_{\ell}$ consists of $m$ copies of each tree in $S$.
The relations are defined as in $\unravel{A}$, but with the relation $\relNext$ replaced by by its finitary variant $\relNext_{\ell}$, where $(a,b) \in \relNext_{\ell}$ if:
\begin{itemize}
  \item $(a,b) \in \relNext$, or
  \item $(a,a') \in \relNext$ where $a'$ is a cut element with associated index $m$, and all of the following are true:
        \begin{enumerate}
          \item $b$ is the root of the $m$-th copy of tree in $S$,
          \item $\ctr{b} = \ctr{a'}$,
          \item $\seq{b}$ is the length-$\ell$ suffix of $\seq{a'}$.
        \end{enumerate}
\end{itemize}

\noindent
The structure $\unravel{A}_{\ell}$ is tree-like, in the sense of the following lemma:
\begin{lemma}\label{lem:companion-tree-like}
  Let $D_{a, \ell}(\unravel{A}_{\ell})$ be the set of elements $b$ for which there exists a next-chain starting from $a$ and ending in $b$ of length at most $\ell$.
  For any element $a \in \unraveldom{A}_{\ell}$, the set $D_{a, \ell}$ is a tree.
\end{lemma}
\begin{proof}
  \bfbox{write proof}
\end{proof}
Let $\unravel{A}_{\ell}$ and $\unravel{B}_{\ell}$ be finite unravelings of some $\sigma$-structures $\str{A}$ and $\str{B}$.
We employ the tree-like property of the unraveling to show that for sufficient large $\ell$, there is a partial isomorphism between any two $\sigma$-live tuples of the same size and with the same $\FGF$-type from $\unravel{A}_{\ell}$ and $\unravel{B}_{\ell}$, respectively.
Let $W$ be the maximal arity of symbols from $\sigma$.
Let $\ell > W$ and consider tuples $\elemtuptuples \sqin \unraveldom{A}_{\ell}$ and $\elemtuptuplet \sqin \unraveldom{B}_{\ell}$ with equal $\FGF$-types and equal size $k$.
We claim that the map $\mu_{(\elemtuptuples, \elemtuptuplet)}$, that maps $s_{i}$ to $t_{i}$ for all $i \in [1, k]$, is a partial isomorphism.
Let $\elemtuptupler$ be a tuple with $\elemtuptupler \sqin \set(\elemtuptuples)$ and $\elemtuptupler \in \relR^{\str{A}}$ for some relation $\relR \in \sigma$.
We first show that $\elemtuptupler$ is an infix of $\elemtuptuples$.
Suppose to the contrary that $\elemtuptupler$ is not an infix of $\elemtuptuples$, thus there is an $i$ such that $r_{i} = s_{v}$ and $r_{i+1} = s_{w}$ with $v \ne w - 1$.
But then both $(s_{v}, s_{w}) \in \relNext_{\ell}$ and $(s_{w-1}, s_{w}) \in \relNext_{\ell}$ since $\elemtuptupler$ and $\elemtuptuples$ are $\sigma$-live.
This is a contradiction, since $D_{s_{1}, W}$ defined in \cref{lem:companion-tree-like} includes $s_{v}, s_{w-1}$ and $s_{w}$ but cannot be a tree since $s_{w}$ would have two parents $s_{v}$ and $s_{w-1}$.
Therefore, $\elemtuptupler$ is an infix of $\elemtuptuples$.
It follows from equality of $\FGF$-types that $\mu_{(\elemtuptuples, \elemtuptuplet)}[\elemtuptupler] \in \relR^{\str{B}}$.
Since the argument is symmetric, we conclude that elements of $\elemtuptuples$ and $\elemtuptuplet$ satisfy the same relations and so $\mu_{(\elemtuptuples, \elemtuptuplet)}$ is a partial isomorphism.

Next, we show that the finite unravelings $\unravel{A}_{\ell}$ and $\unravel{B}_{\ell}$ are $n$-$\GF$-bisimilar given that the base structures $\str{A}$ and $\str{B}$ are $m$-$\FGF$-bisimilar for sufficient large $m$ and $\ell$.
The proof starts by describing a notion of $z$-\emph{similar} elements from $\unravel{A}_{\ell}$ and $\unravel{B}_{\ell}$, for a natural number $z$ giving the level of similarity.
We then show that relating tuples composed of $z$-similar elements forms a $n$-$GF$-bisimulation between $\unravel{A}_{\ell}$ and $\unravel{B}_{\ell}$, for sufficient large $z$.
Consider the elements $e \in \unraveldom{A}_{\ell}$ and $f \in \unraveldom{B}_{\ell}$ with $\seq{e} = \cdots\elemtuples$ and $\seq{f} = \cdots\elemtuplet$.
Assume that $\ctr{e} = \ctr{f}$ and that $\str{A}, \elemtuples \bisimto_{\FGF}^{z} \str{B}, \elemtuplet$ for some $z$.
This means that Spoiler can make up to $z$ moves in starting from $\str{A}, \elemtuples$ in the bisimulation game, and all those moves are matched by Duplicator in $\str{B}, \elemtuplet$.
Let $g$ be descendant of $e$ with distance $d$ from $e$ and $\seq{g} = \seq{e}\cdots(i,j)\elemtuplev$.
Then there is descendant $h$ of $f$ with $\seq{h} = \seq{f}\cdots(i,j)\elemtuplew$ and $\str{A}, \elemtuplev \bisimto_{\FGF}^{z-d} \str{B}, \elemtuplew$.
Thus, the subtrees rooted at $e$ and $f$ in $\unravel{A}_{\ell}$ and $\unravel{B}_{\ell}$ look similar.
However, this tells us nothing about the parents of $e$ and $f$.
The parents of $e$ and $f$ depend on their bisimulation sequences $\seq{e}$ and $\seq{f}$.
For $z$-similar elements $e$ and $f$, we want the last $z$ moves of the bisimulation sequences to match.
For example, if $\seq{e} = \cdots\elemtuplev(i,j)\elemtuples$, then we want $\seq{f} = \cdots\elemtuplew(i,j)\elemtuplet$ with $\str{A}, \elemtuplev \bisimto_{\FGF}^{z} \str{B}, \elemtuplew$, in addition to $\str{A}, \elemtuples \bisimto_{\FGF}^{z} \str{B}, \elemtuplet$, and so on for the moves before that.
Formally, we define $z$-similar elements as follows:
\begin{definition}[$z$-similar elements]
 Let $e \in \unraveldom{A}_{\ell}$ and $f \in \unraveldom{B}_{\ell}$ and $z$ be a natural number.
 Let $\seq{e}$ and $\seq{f}$ either have the same length or a length greater than $z$.
 Let $h$ be this length or $z$, whichever is smaller.
 Then $e$ and $f$ are $z$-similar, written $e \approx_{z} f$, if:
 \begin{itemize}
   \item $\ctr{e} = \ctr{f}$,
   \item $\seq{e}$ has suffix $\elemtuples^{(0)}\cdots(i^{(h)},j^{(h)})\elemtuples^{(h)}$, and
   \item $\seq{f}$ has suffix $\elemtuplet^{(0)}\cdots(i^{(h)},j^{(h)})\elemtuplet^{(h)}$
 \end{itemize}
 where $\str{A}, \elemtuples^{(k)} \bisimto_{\FGF}^{z} \str{B}, \elemtuplet^{(k)}$ for all $k \in [0,h]$.
\end{definition}

We now show that if elements $a$ and $b$ are $z$-similar, then following a $\relNext$-edge leads to $(z-1)$-similar elements.
In other words, if $c$ is a parent or child of $a$, then there exists a parent respective child of $b$ which is $(z-1)$-similar to $c$:
\begin{lemma}\label{lem:approx-next}
  Let $a \in \unraveldom{A}_{\ell}$ and $b \in \unraveldom{B}_{\ell}$. If $a \approx_{z} b$ for $z \in [1,\ell]$, then:
  \begin{description}
  \item[\desclabel{(succ)}{elemeq:succ}] If $(a, c) \in \relNext_{\ell}$ for $c \in \unraveldom{A}_{\ell}$, then there is $d \in \unraveldom{B}_{\ell}$ with $(b, d) \in \relNext_{\ell}$ and $c \approx_{z-1} d$.
  \item[\desclabel{(pred)}{elemeq:pred}] If $(c, a) \in \relNext_{\ell}$ for $c \in \unraveldom{A}_{\ell}$, then there is $d \in \unraveldom{B}_{\ell}$ with $(d, b) \in \relNext_{\ell}$ and $c \approx_{z-1} d$.
\end{description}
\end{lemma}
\begin{proof}
  First observe that if $\seq{a} = \seq{c}$ and we choose $d$ with $\seq{d} = \seq{b}$ and $\ctr{d} = \ctr{c}$, then $a \approx_{z} b$ if and only if $c \approx_{z} d$.
  Further, if $(a,c) \in \relNext$ or $(c,a) \in \relNext_{\ell}$, then also $(b,d) \in \relNext_{\ell}$ or $(d,b) \in \relNext_{\ell}$, respectively.
  This proves~\ref{elemeq:succ} and~\ref{elemeq:pred} for the case that $\seq{a} = \seq{c}$.

  We now consider the case that $\seq{a} \neq \seq{c}$, which corresponds to~\ref{next:addseq} case from the definition of $\relNext$.
  Let $a \in \unraveldom{A}_{\ell}$ and $b \in \unraveldom{B}_{\ell}$ with $a \approx_{z} b$.
  Let
  $\seq{a} = \cdots \elemtuples^{(0)} \cdots (i^{(h)}, j^{(h)})\elemtuples^{(h)}$ and $\seq{b} = \cdots \elemtuplet^{(0)} \cdots (i^{(h)}, j^{(h)})\elemtuplet^{(h)}$, where $h$ is $z$ or the length of the sequences, whichever is smaller, as in the definition of $\approx_{z}$.
  For all $k \in [0,h]$, we have $\str{A}, \elemtuples^{(k)} \bisimto_{\FGF}^{z} \str{B}, \elemtuplet^{(k)}$ since $a \approx_{z} b$.

  We treat~\ref{elemeq:succ} and~\ref{elemeq:pred} separately.
  For~\ref{elemeq:succ}, let $c \in \unraveldom{A}_{\ell}$ with $\seq{c} \neq \seq{a}$ and $(a,c) \in \relNext_{\ell}$.
  Then there are indices $v,w$ and a tuple $\elemtupleo \sqin A$ such that $\seq{c}$ is equal to $\seq{a} (v,w) \elemtupleo$ or the $\ell$-length suffix of that sequence.
  We can find a tuple $\elemtuplep$ for which $\seq{b}(v,w)\elemtuplep$ is a bisimulation sequence and $\str{A}, \elemtupleo \bisimto_{\FGF}^{z-1} \str{B}, \elemtuplep$ by employing~\ref{bisim:fforth} since $\str{A}, \elemtuples^{(h)} \bisimto_{\FGF}^{z} \str{B}, \elemtuplet^{(h)}$.
  Let $d \in \unraveldom{B}_{\ell}$ be the element with:
  \begin{itemize}
    \item $\seq{d}$ equal to $\seq{b} (v,w) \elemtuplep$ or if that sequence is longer than $2 * \ell$, to the length-$\ell$ suffix of that sequence,
    \item $\ctr{d} = (w-v+1) + 1$, which by definition of $\relNext$ is equal to $\ctr{c}$.
  \end{itemize}
  By construction $(b, d) \in \relNext_{\ell}$.
  Note that the length of $\seq{e}$ and $\seq{d}$ increases by exactly one compared to $\seq{a}$ and $\seq{b}$, except if one of them reaches the length $2 * \ell$ and is truncated to a suffix.
  If no truncation occurs, then $\seq{c}$ and $\seq{d}$ have the same length if $\seq{a}$ and $\seq{b}$ have, so clearly $c \approx_{z-1} d$, as we extended them with bisimilar tuples and the same pair of indices.
  Otherwise, to reach the truncation length of $2 * \ell$, either $\seq{a}$ or $\seq{b}$ must have a length of at least $\ell$.
  As $z \le \ell$, by $a \approx_{z} b$ in this case both sequences are at least of length $z$, thus $h$ equals $z$.
  Then $\seq{c}$ and $\seq{d}$ also are at least of length $z$, since even if they are truncated to a suffix of length $\ell$, still $\ell \ge z$.
  It follows that the sequences $\elemtuples^{(2)}\cdots\elemtuples^{(z)}(v,w)\elemtupleo$ and $\elemtuplet^{(2)}\cdots\elemtuples^{(z)}\elemtuplep$, which are of length $z-1$, are suffixes of $\seq{c}$ and $\seq{d}$ respectively.
  Therefore, $c \approx_{z-1} d$ as wanted.

  For~\ref{elemeq:pred}, let $c \in \unraveldom{A}_{\ell}$ with $\seq{c} \neq \seq{a}$ and $(c,a) \in \relNext_{\ell}$.
  In this case, there are indices $v,w$ and a tuple $\elemtupleo \sqin A$ for which $\seq{a} = \sigma (v,w) \elemtupleo$, where $\sigma$ is either $\seq{c}$ or the $\ell-1$-length suffix $\seq{c}$.
  Since $a \approx_{z} b$, we have $\ctr{a} = \ctr{b} = (w-v+1) + 1$ and $\seq{b} = \theta (v,w) \elemtuplep$ for some bisimulation sequence $\theta$ and a tuple $\elemtuplep \sqin B$.
  Note that $\ctr{c} = w$ by definition of $\relNext$.
  Further, $(\sigma, w) \approx_{z-1} (\theta, w)$, since $a \approx_{z} b$ and $\sigma$ and $\theta$ are constructed from $\seq{a}$ and $\seq{b}$ by removing the last step.
  Let $d = (\theta, w)$.
  Clearly $(d,b) \in \relNext_{\ell}$ and if $\sigma = \seq{c}$, then $c \approx_{z-1} d$ follows directly.
  If $\seq{c} \neq \sigma$, then $\sigma$ is a suffix of $\seq{c}$ with length $\ell-1$.
  As $\ell-1 \ge z-1$, the last $z-1$ steps of $\sigma$ and $\seq{c}$ are equal in this case.
  Therefore, $(\sigma, w) \approx_{z-1} (\theta, w)$ implies $(\seq{c}, w) \approx_{z-1} (\theta, w)$, so $c \approx_{z-1} d$ as wanted.
  \end{proof}
\begin{description}
  \item[\desclabel{(global)}{elemeq:global}] If $\str{A} \bisimto_{\FGF}^{2*z + 1} \str{B}$, then for any $e \in \unraveldom{A}_{\ell}$, there is $f \in \unraveldom{B}_{\ell}$ such that $e \approx_{z} f$.
\end{description}

Let us now consider tuples of $z$-similar elements.
Let $\elemtuptuplea = (a_{1}, \ldots, a_{k})$ and $\elemtuptupleb = (b_{1}, \ldots, b_{k})$ where $a_{i} \in \unraveldom{A}_{\ell}$ and $b_{i} \in \unraveldom{B}_{\ell}$ for finite unravelings $\unravel{A}_{\ell}$ and $\unravel{B}_{\ell}$.
We show that if $\elemtuptuplea$ and $\elemtuptupleb$ are live and $a_{i} \approx_{z} b_{i}$ for every $i \in [1,k]$, then they have the same $\FGF$-type.
As seen before, in finite unravelings this implies that there exists a partial isomorphism between $\elemtuptuplea$ and $\elemtuptupleb$.
Let $\elemtuples$ and $\elemtuplet$ be the tuples from $\str{A}$ and $\str{B}$ such that $\seq{a_{k}} = \cdots \elemtuples$ and $\seq{b_{k}} = \cdots \elemtuplet$.
Since $a_{k} \approx_z b_{k}$, we have $\elemtuples \bisimto_{\FGF}^{z} \elemtuplet$ which in particular implies that $\elemtuples$ and $\elemtuplet$ have the same $\FGF$-type.
By the properties of $\relNext$, we know that $\pi[\elemtuptuplea] = \elemtuples_{(\ctr{a_{k}}-k)\ldots{}\ctr{a_{k}}}$ and likewise $\pi[\elemtuptupleb] = \elemtuplet_{(\ctr{b_{k}}-k)\ldots{}\ctr{b_{k}}}$.
As $\ctr{a_{k}} = \ctr{b_{k}}$, both $\elemtuptuplea$ and $\elemtuptupleb$ project to infixes $\elemtuples_{i\ldots{}j}$ and $\elemtuplet_{i\ldots{}j}$ over the common range $i\ldots{}j$ for $i = \ctr{a_{k}} - k$ and $j = \ctr{a_{k}}$.
Hence, $\FGF$-types of $\pi[\elemtuptuplea]$ and $\pi[\elemtuptupleb]$ are also equal.
Now consider an infix $\elemtuptuplea_{x\ldots{}y}$ and the corresponding infix $\elemtuptupleb_{x\ldots{}y}$, for some indices $x$ and $y$.
We show that the three conditions of the definition of relations in the unraveling are satisfied for $\elemtuptuplee_{i\ldots{}j}$ exactly if they are satified for $\elemtuptuplef_{i\ldots{}j}$:
\begin{enumerate}
  \item $\pi[\elemtuptuplea_{x\ldots{}y}] \in \relR^{\str{A}}$ if and only if $\pi[\elemtuptuplea_{x\ldots{}y}] \in \relR^{\str{B}}$, because $\pi[\elemtuptuplea]$ and $\pi[\elemtuptupleb]$ have equal $\FGF$-types,
  \item $\elemtuptuplea_{x\ldots{}y}$ is a next-chain if and only if $\elemtuptupleb_{x\ldots{}y}$ is a next-chain, because both infixes are always next-chains since $\elemtuptuplea$ and $\elemtuptupleb$ are next-chains as they live,
  \item $|\elemtuptuplea_{x\ldots{}y}| \le \mathtt{bound}(\elemtuptuplea_{x\ldots{}y})$ if and only if $|\elemtuptupleb_{x\ldots{}y}| \le \mathtt{bound}(\elemtuptuplef_{x\ldots{}y})$ because $|\elemtuptuplea_{x\ldots{}y}| = |\elemtuptupleb_{x\ldots{}y}| = y-x+1$ and $\mathtt{bound}$ is equal since $\ctr{e_{j}} = \ctr{f_{j}}$, which follows from the precondition $e_{j} \approx_{z} f_{j}$.
\end{enumerate}
Therefore, infixes of $\elemtuptuplea$ and $\elemtuptupleb$ are in the same relations.
We conclude that $\elemtuptuplea$ and $\elemtuptupleb$ have equal $\FGF$-types.

Finally, we show that the partial isomorphisms between tuples with $z$-similar elements form a $\GF$-bisimulation.
Let $\unravel{A}_{\ell}$ and $\unravel{B}_{\ell}$ be finite unravelings of $\sigma$-structures $\str{A}$ and $\str{B}$, for a finite signature $\sigma$.
We define a sequence of sets $\bisimZ_{0}, \ldots, \bisimZ_{n} \subseteq \PartIso{\unravel{A}_{\ell}}{\unravel{B}_{\ell}}$ as follows:
\begin{equation*}
  \bisimZ_{k} = \left\{
    \mu_{(\elemtuptuples, \elemtuptuplet)}:\,
    \elemtuptuples\ \text{and}\ \elemtuptuplet\ \text{are live and}\ \
    s_{i} \approx_{W * k} t_{i}\ \text{for all indices $i$}
  \right\}
\end{equation*}
where $W$ is the maximal arity of symbols from $\sigma$ and $\mu_{(\elemtuptuples, \elemtuptuplet)}$ is the partial isomorphism between $\elemtuptuples$ and $\elemtuptuplet$, which exists since they have $z$-similar elements.
We claim that $\bisimZ_{k-1}$ satisfies the~\ref{bisim:back} and~\ref{bisim:forth} conditions for $\bisimZ_{k}$, for $k \in [1, n]$.
We only show~\ref{bisim:forth} as the proof for~\ref{bisim:back} is symmetric.
Let $\mu_{(\elemtuptuples, \elemtuptuplet)} \in \bisimZ_{k}$, for some live tuples $\elemtuptuples$ and $\elemtuptuplet$.

For structures $\str{A}$ and $\str{B}$ and every element $a \in \unraveldom{A}_{\ell}$, we can find a $z$-similar element $b \in \unraveldom{B}_{\ell}$, given that $\str{A} \bisimto_{\FGF}^{z} \str{B}$, as follows.
Let $\seq{a} = \elemtuples^{(0)}\cdots(i^{{n}}, j^{(n)})\elemtuples^{(n)}$ and $m$ be the minimum of $z$ and $n$.
Using~\ref{bisim:fforth} $m$ times, we obtain live tuples $\elemtuplet^{(0)}, \ldots, \elemtuplet^{(m)}$ with $(\elemtuples^{(n-m)}, \elemtuplet^{(0)}) \in \bisimZ_{2*z}, \ldots, (\elemtuples^{(n)}, \elemtuplet^{(m)}) \in \bisimZ_{2*z - m}$ such that $\elemtuplet^{(0)}(i^{(n-m+1)}, j^{(n-m+1)})\elemtuplet^{(1)}\cdots{}(i^{n}, j^{n})\elemtuplet^{(m)}$ is a bisimulation sequence.
For any bisimulation it holds that $\bisimZ_{0} \supseteq \bisimZ_{1} \supseteq \cdots{} \supseteq \bisimZ_{2*z+1}$, thus for every x, $(\elemtuples^{(n-m+x)}, \elemtuplet^{(x)}) \in \bisimZ_{2*z - m}$.
Hence, since $z \le 2*z - m$, for every $x$, the tuples $\elemtuples^{(n-m+x)}$ and $\elemtuplet^{(x)}$ are $z$-$\FGF$-bisimilar.
We can now construct the element $b$ with $\ctr{b} = \ctr{a}$ and $\seq{b} = \elemtuplet^{(0)}(i^{(n-m+1)}, j^{(n-m+1)})\elemtuplet^{(1)}\cdots{}(i^{n}, j^{n})\elemtuplet^{(m)}$.
This satisfies $a \approx_{z} b$, as wanted.
