%!TEX root = ../main.tex

\section{Finite Companions}\label{sec:finite}
Let $\unravel{A}, \elemtuptuplea$ be a tree unraveling of a finite structure $\str{A}, \elemtuplea$ over a (finite) signature $\sigma$.
We describe the finite unraveling $\unravel{A}_{\ell}, \elemtuptuplea_{\ell}$ which will be $\ell$-$\FGF$-bisimilar to $\unravel{A}, \elemtuptuplea$ (and also $\ell$-$\FGF$-bisimilar to $\str{A}, \elemtuplea$ by transitivity of bisimilarity).
For an element $r \in \unraveldom{A}$, we construct the finite tree $\mathcal{T}_{r,\ell}(\unravel{A})$ as follows:
\begin{equation*}
  \mathcal{T}_{r,\ell}(\unravel{A}) = \left\{ e \in \unraveldom{A}:\, e\ \text{is a descendant of $r$ and the length of $\seq{e}$ is at most $2 * \ell$} \right\}.
\end{equation*}

$\mathcal{T}_{r,\ell}$ is substructure of $\unravel{A}$.
It is a tree with root $r$ because for every element $e \in \mathcal{T}_{r,\ell} \setminus {r}$, the parent of $e$ is also in $\mathcal{T}_{r,\ell}$.
This is easy to see: if $p$ is the parent of $e$, then $\seq{p}$ has a length lower or equal to the length of $\seq{e}$ and clearly $p$ is a descendant of $r$ as well, so $p$ is in $\mathcal{T}_{r,\ell}$.
Furthermore, this tree is finite:
\begin{lemma}\label{lem:bounded-trees-are-finite}
  If $\unravel{A}$ is the unraveling of a finite structure $\str{A}$ with the finite signature $\sigma$, then $\mathcal{T}_{r,\ell}$ is finite for $\ell \in \N$ and $r \in \unraveldom{A}$.
\end{lemma}
\begin{proof}
If $A$ is finite, then there only a finite number of live tuples in $A$.
Recall that a bisimulation sequence of length $\ell$ is just a word in $A^*{(\N\N{}A^{*})}^{\ell}$, consisting of live tuples from $A^{*}$ and indices into those tuples from $\N\N$.
Let $W$ be the maximum size of a live tuple.
Then an index into a live tuple must be in the range $[1,W]$, which is finite, so the set of possible indices is finite.
Thus there are only a finite number of bisimulation sequences with length $\ell$, since both the set of live tuples and the set of indices is finite.
If $e \in \mathcal{T}_{r,\ell}$ with $\seq{e} = \sigma$, then the length of $\sigma$ is at most $2 * \ell$.
Thus, there are only a finite number of choices for $\sigma$.
For every $\sigma$, there are only a finite number of elements $e \in \unraveldom{A}$ with $\seq{e} = \sigma$.
It follows that there are only finitely many $e$ with $e \in \mathcal{T}_{r,\ell}$, so $\mathcal{T}_{r,\ell}$ is finite.
\end{proof}
Now consider the set $\mathcal{S}$ of trees with $\mathcal{S} = \left\{ \mathcal{T}_{r,\ell}:\, \seq{r}\ \text{has length at most $\ell$}\right\}$.
An element $b$ is called a \emph{cut element} of a tree $\mathcal{T} \in \mathcal{S}$ if it itself is not in $\mathcal{T}$ but $(a,b) \in \relNext$ for some $a$ in $\mathcal{T}$.
Intuitively, the cut elements are elements $e$ where the length of $\seq{e}$ is $2 * \ell + 1$ (so $e$ is not in $\mathcal{T}$) and $\ctr{e}$ is minimal (so there is a $\relNext$-edge from some element $\mathcal{T}$).
The set of all cut elements for trees in $\mathcal{S}$ is finite, because $\mathcal{S}$ is finite and the cut elements of a tree $\mathcal{T}_{r,\ell} \in \mathcal{S}$ are contained in $\mathcal{T}_{r,\ell+1}$ which is finite by \cref{lem:bounded-trees-are-finite}.
Enumerate all cut elements of trees in $\mathcal{S}$ with indices from $1, \ldots, M$, where $M$ is the number of cut elements.
The domain of the structure $\unravel{A}_{\ell}$ consists of $M$ copies of each tree in $\mathcal{S}$.
The relations are defined as in $\unravel{A}$, but with the relation $\relNext$ replaced by by its finitary variant $\relNext_{\ell}$, where $(a,b) \in \relNext_{\ell}$ if:
\begin{itemize}
  \item $(a,b) \in \relNext$, or
  \item $(a,a') \in \relNext$ where $a'$ is a cut element with the associated index $m$, and all of the following are true:
        \begin{enumerate}
          \item $b$ is the root of the $m$-th copy of tree in $S$,
          \item $\ctr{b} = \ctr{a'}$,
          \item $\seq{b}$ is the length-$\ell$ suffix of $\seq{a'}$.
        \end{enumerate}
\end{itemize}

\noindent
The structure $\unravel{A}_{\ell}$ is tree-like, in the sense of the following lemma:
\begin{lemma}\label{lem:companion-tree-like}
  Let $D_{a, \ell}(\unravel{A}_{\ell})$ be the set of elements $b$ for which there exists a next-chain starting from $a$ and ending in $b$ of length at most $\ell$.
  For any element $a \in \unraveldom{A}_{\ell}$, the set $D_{a, \ell}$ is a tree.
\end{lemma}
\begin{proof}
  \bfbox{write proof}
\end{proof}

In the following, let $\elemtuplea \simeq \elemtupleb$ denote that $\restr{\str{A}}{\set(\elemtuplea)}$ is isomorphic to $\restr{\str{B}}{\set(\elemtupleb)}$, for tuples $\elemtuplea$ and $\elemtupleb$ from some structures $\str{A}$ and $\str{B}$.
Let $\elemtuptuplea$ and $\elemtuptupleb$ be live tuples from $\unravel{A}_{\ell}$ and $\unravel{B}_{\ell}$.
We employ \cref{lem:companion-tree-like} to show that $\tp{\FGF}{\unravel{A}_{\ell}}{\elemtuptuplea} = \tp{\FGF}{\unravel{B}_{\ell}}{\elemtuptupleb}$ implies $\elemtuptuplea \simeq \elemtuptupleb$, if $\ell$ is large enough:
\begin{lemma}\label{lem:fgf-type-iso}
  Let $\elemtuptuplea$ and $\elemtuptupleb$ be live tuples in finite unravelings $\unravel{A}_{\ell}$ and $\unravel{B}_{\ell}$ of $\sigma$-structures $\str{A}$ and $\str{B}$ for a signature $\sigma$.
  If $\ell > W$, where $W$ is the maximum arity of a relation in $\sigma$, then $\tp{\FGF}{\unravel{A}_{\ell}}{\elemtuptuplea} = \tp{\FGF}{\unravel{B}_{\ell}}{\elemtuptupleb}$ implies $\elemtuptuplea \simeq \elemtuptupleb$, witnessed by the isomorphism $\mu_{(\elemtuptuplea, \elemtuptupleb)}:\, a_{i} \mapsto b_{i}$ for every $i \in [1, k]$.
\end{lemma}
\begin{proof}
Let $\ell > W$ and consider live tuples $\elemtuptuplea \sqin \unraveldom{A}_{\ell}$ and $\elemtuptupleb \sqin \unraveldom{B}_{\ell}$ with equal $\FGF$-types and equal size $k$.
We claim that the map $\mu_{(\elemtuptuplea, \elemtuptupleb)}:\, a_{i} \mapsto b_{i}$ for all $i \in [1, k]$, is an isomorphism between $\restr{\str{\unravel{A}_{\ell}}}{\set(\elemtuptuplea)}$ and $\restr{\str{\unravel{B}_{\ell}}}{\set(\elemtuptupleb)}$.
Let $\elemtuptupler$ be a tuple with $\elemtuptupler \sqin \set(\elemtuptuplea)$ and $\elemtuptupler \in \relR^{\str{A}}$ for some relation $\relR \in \sigma$.
We first show that $\elemtuptupler$ is an infix of $\elemtuptuplea$.
Suppose to the contrary that $\elemtuptupler$ is not an infix of $\elemtuptuplea$, thus there is an $i$ such that $r_{i} = a_{v}$ and $r_{i+1} = a_{w}$ with $v \ne w - 1$.
But then both $(a_{v}, a_{w}) \in \relNext_{\ell}$ and $(a_{w-1}, a_{w}) \in \relNext_{\ell}$ since $\elemtuptupler$ and $\elemtuptuplea$ are $\sigma$-live.
This is a contradiction, since $D_{a_{1}, W}$ defined in \cref{lem:companion-tree-like} includes $a_{v}, a_{w-1}$ and $a_{w}$ but cannot be a tree since $a_{w}$ would have two parents $a_{v}$ and $a_{w-1}$.
Therefore, $\elemtuptupler$ is an infix of $\elemtuptuplea$.
It follows from equality of $\FGF$-types that $\mu_{(\elemtuptuplea, \elemtuptupleb)}[\elemtuptupler] \in \relR^{\str{B}}$.
Since the argument is symmetric, we conclude that elements of $\elemtuptuplea$ and $\elemtuptupleb$ satisfy the same relations so $\mu_{(\elemtuptuplea, \elemtuptupleb)}$ is an isomorphism and $\elemtuptuplea \simeq \elemtuptupleb$.
\end{proof}

Our main goal for the remainder of this section is to show that $\unravel{A}_{\ell} \bisimto_{\GF}^{n} \unravel{B}_{\ell}$, given that $\str{A} \bisimto_{\FGF}^{m} \str{B}$ for some $m$ and $\ell$ that depends on only $n$.
For this proof, we first introduce the notion of $z$-\emph{similar} elements, which is a relation between elements from $\unravel{A}_{\ell}$ and $\unravel{B}_{\ell}$.
We later lift this relation to tuples to construct the $n$-$\GF$-bisimulation.
\begin{definition}[$z$-similar elements]
 Let $e \in \unraveldom{A}_{\ell}$ and $f \in \unraveldom{B}_{\ell}$ and $z$ be a natural number.
 Let $\seq{e}$ and $\seq{f}$ either both have a length greater than $z$ or have the same length.
 If they have lengths greater than $z$, let $h$ be $z$, otherwise let $h$ be their common length.
 Then $e$ and $f$ are $z$-similar, written $e \approx_{z} f$, if:
 \begin{itemize}
   \item $\ctr{e} = \ctr{f}$,
   \item $\seq{e}$ has suffix $\elemtuples^{(0)}\cdots(i^{(h)},j^{(h)})\elemtuples^{(h)}$, and
   \item $\seq{f}$ has suffix $\elemtuplet^{(0)}\cdots(i^{(h)},j^{(h)})\elemtuplet^{(h)}$
 \end{itemize}
 where $\str{A}, \elemtuples^{(k)} \bisimto_{\FGF}^{z} \str{B}, \elemtuplet^{(k)}$ for all $k \in [0,h]$.
\end{definition}
We now give an intuitive description of this definition, followed by a lemma making this intuition precise.
\bfside{I am not sure if this intuitive description is necessary. It is mostly a more vague version of the proof below. If you think that this is too vague, it might make sense to just remove it?}
The role of $z$ in the definition of $z$-similar elements is twofold.
First, $z$ determines the length of the suffixes of $\seq{e}$ and $\seq{f}$ that we consider.
These suffixes correspond to the last $z$ moves in the play of the $\FGF$-game represented by $\seq{e}$ and $\seq{f}$, respectively.
For $z$-similar elements, we require that the last $z$ moves of both $\seq{e}$ and $\seq{f}$ are equal (or all the moves, if these sequences are shorter than $z$ moves), i.e.\ at move $k$, they both pick the infix at same range of indices $i^{(k)}\ldots{}j^{(k)}$.
This allows us to ``go backwards'' along $\relNext$-edges in the tree, in the following sense:
Let us say that a parent $c \in A$ of an element $a \in A$ ``undoes'' the last move of $a$ if its sequence $\seq{c}$ is equal $\seq{a}$ but without the final move.
Now, if $a \in A$ and $b \in B$ are $z$-similar elements, and there is an element $c \in A$ which ``undoes'' the last move of $a \in A$, then there is also an element $d \in B$ which ``undoes'' the same last move of $b$, since the last $z$ moves of $\seq{a}$ and $\seq{b}$ are the same.
The resulting elements $b$ and $d$ are then $(z-1)$-similar, since the sequences $\seq{b}$ and $\seq{d}$ are one move shorter.
In the second role, $z$ is the bound on the bisimulation equivalence $\elemtuples^{(k)} \bisimto_{\FGF}^{z} \elemtuplet^{(k)}$.
In this role, $z$ determines how many turns we can play in the $\FGF$-game with the starting position $\str{A}, \elemtuples^{(k)}$ if we want each move to be matched in a play with the starting position $\str{B}, \elemtuplet^{(k)}$.
This allows us to ``go forwards'' along $\relNext$-edges in the tree, in the following sense:
Let us say that $c \in A$ ``adds'' a move to an element $a \in A$ if $c$ is a child of $a$ and $\seq{c}$ is equal to $\seq{a}$ but with one additional move at the end.
Now, if $a \in A$ and $b \in B$ are $z$-similar elements, and there is an element $c \in A$ which ``adds'' a move to $a \in A$, then there is also an element $d \in B$ which ``adds'' a move to $b$.
This is because the final tuples of $\seq{a}$ and $\seq{b}$ are $z$-$\FGF$-bisimilar, so if we extend $\seq{a}$ in some way, we can find a move in $\str{B}$ to extend $\seq{b}$ in the same way.
The resulting elements $b$ and $d$ are then $(z-1)$-similar, since we used up one of the $z$ turns, leaving $z-1$ remaining turns.
The fact that, starting from $z$-similar elements $a$ and $b$, we can go both ``backwards'' and ``forwards'', is stated in the following lemma:
\begin{lemma}\label{lem:approx-next}
  Let $a \in \unraveldom{A}_{\ell}$ and $b \in \unraveldom{B}_{\ell}$. If $a \approx_{z} b$ for some $z \in [1,\ell]$, then:
  \begin{description}
  \item[\desclabel{(succ)}{elemeq:succ}] If $(a, c) \in \relNext_{\ell}$ for $c \in \unraveldom{A}_{\ell}$, then there is $d \in \unraveldom{B}_{\ell}$ with $(b, d) \in \relNext_{\ell}$ and $c \approx_{z-1} d$.
  \item[\desclabel{(pred)}{elemeq:pred}] If $(c, a) \in \relNext_{\ell}$ for $c \in \unraveldom{A}_{\ell}$, then there is $d \in \unraveldom{B}_{\ell}$ with $(d, b) \in \relNext_{\ell}$ and $c \approx_{z-1} d$.
\end{description}
\end{lemma}
\begin{proof}
  First observe that if $\seq{a} = \seq{c}$ and we choose $d$ with $\seq{d} = \seq{b}$ and $\ctr{d} = \ctr{c}$, then $a \approx_{z} b$ if and only if $c \approx_{z} d$.
  Further, if $(a,c) \in \relNext$ or $(c,a) \in \relNext_{\ell}$, then also $(b,d) \in \relNext_{\ell}$ or $(d,b) \in \relNext_{\ell}$, respectively.
  This proves~\ref{elemeq:succ} and~\ref{elemeq:pred} for the case that $\seq{a} = \seq{c}$.

  We now consider the case that $\seq{a} \neq \seq{c}$, which corresponds to~\ref{next:addseq} case from the definition of $\relNext$.
  Let $a \in \unraveldom{A}_{\ell}$ and $b \in \unraveldom{B}_{\ell}$ with $a \approx_{z} b$.
  Let
  $\seq{a} = \cdots \elemtuples^{(0)} \cdots (i^{(h)}, j^{(h)})\elemtuples^{(h)}$ and $\seq{b} = \cdots \elemtuplet^{(0)} \cdots (i^{(h)}, j^{(h)})\elemtuplet^{(h)}$, where $h$ is $z$ or the length of the sequences, whichever is smaller, as in the definition of $\approx_{z}$.
  For all $k \in [0,h]$, we have $\str{A}, \elemtuples^{(k)} \bisimto_{\FGF}^{z} \str{B}, \elemtuplet^{(k)}$ since $a \approx_{z} b$.

  We treat~\ref{elemeq:succ} and~\ref{elemeq:pred} separately.
  For~\ref{elemeq:succ}, let $c \in \unraveldom{A}_{\ell}$ with $\seq{c} \neq \seq{a}$ and $(a,c) \in \relNext_{\ell}$.
  Then there are indices $v,w$ and a tuple $\elemtupleo \sqin A$ such that $\seq{c}$ is equal to $\seq{a} (v,w) \elemtupleo$ or the $\ell$-length suffix of that sequence.
  We can find a tuple $\elemtuplep$ for which $\seq{b}(v,w)\elemtuplep$ is a bisimulation sequence and $\str{A}, \elemtupleo \bisimto_{\FGF}^{z-1} \str{B}, \elemtuplep$ by employing~\ref{bisim:fforth} since $\str{A}, \elemtuples^{(h)} \bisimto_{\FGF}^{z} \str{B}, \elemtuplet^{(h)}$.
  Let $d \in \unraveldom{B}_{\ell}$ be the element with:
  \begin{itemize}
    \item $\seq{d}$ equal to $\seq{b} (v,w) \elemtuplep$ or if that sequence is longer than $2 * \ell$, to the length-$\ell$ suffix of that sequence,
    \item $\ctr{d} = (w-v+1) + 1$, which by definition of $\relNext$ is equal to $\ctr{c}$.
  \end{itemize}
  By construction $(b, d) \in \relNext_{\ell}$.
  Note that the length of $\seq{e}$ and $\seq{d}$ increases by exactly one compared to $\seq{a}$ and $\seq{b}$, except if one of them reaches the length $2 * \ell$ and is truncated to a suffix.
  If no truncation occurs, then $\seq{c}$ and $\seq{d}$ have the same length if $\seq{a}$ and $\seq{b}$ have, so clearly $c \approx_{z-1} d$, as we extended them with bisimilar tuples and the same pair of indices.
  Otherwise, to reach the truncation length of $2 * \ell$, either $\seq{a}$ or $\seq{b}$ must have a length of at least $\ell$.
  As $z \le \ell$, by $a \approx_{z} b$ in this case both sequences are at least of length $z$, thus $h$ equals $z$.
  Then $\seq{c}$ and $\seq{d}$ also are at least of length $z$, since even if they are truncated to a suffix of length $\ell$, still $\ell \ge z$.
  It follows that the sequences $\elemtuples^{(2)}\cdots\elemtuples^{(z)}(v,w)\elemtupleo$ and $\elemtuplet^{(2)}\cdots\elemtuples^{(z)}\elemtuplep$, which are of length $z-1$, are suffixes of $\seq{c}$ and $\seq{d}$ respectively.
  Therefore, $c \approx_{z-1} d$ as wanted.

  For~\ref{elemeq:pred}, let $c \in \unraveldom{A}_{\ell}$ with $\seq{c} \neq \seq{a}$ and $(c,a) \in \relNext_{\ell}$.
  In this case, there are indices $v,w$ and a tuple $\elemtupleo \sqin A$ for which $\seq{a} = \sigma (v,w) \elemtupleo$, where $\sigma$ is either $\seq{c}$ or the $\ell-1$-length suffix $\seq{c}$.
  Since $a \approx_{z} b$, we have $\ctr{a} = \ctr{b} = (w-v+1) + 1$ and $\seq{b} = \theta (v,w) \elemtuplep$ for some bisimulation sequence $\theta$ and a tuple $\elemtuplep \sqin B$.
  Note that $\ctr{c} = w$ by definition of $\relNext$.
  Further, $(\sigma, w) \approx_{z-1} (\theta, w)$, since $a \approx_{z} b$ and $\sigma$ and $\theta$ are constructed from $\seq{a}$ and $\seq{b}$ by removing the last step.
  Let $d = (\theta, w)$.
  Clearly $(d,b) \in \relNext_{\ell}$ and if $\sigma = \seq{c}$, then $c \approx_{z-1} d$ follows directly.
  If $\seq{c} \neq \sigma$, then $\sigma$ is a suffix of $\seq{c}$ with length $\ell-1$.
  As $\ell-1 \ge z-1$, the last $z-1$ steps of $\sigma$ and $\seq{c}$ are equal in this case.
  Therefore, $(\sigma, w) \approx_{z-1} (\theta, w)$ implies $(\seq{c}, w) \approx_{z-1} (\theta, w)$, so $c \approx_{z-1} d$ as wanted.
\end{proof}

Let us now consider live tuples $\elemtuptuplea \sqin \unraveldom{A}_{\ell}$ and $\elemtuptupleb \sqin \unraveldom{B}_{\ell}$ with $\elemtuptuplea = (a_{1}, \ldots, a_{k})$ and $\elemtuptupleb = (b_{1}, \ldots, b_{k})$ where $a_{i} \approx_{z} b_{i}$ for every $i \in [1,k]$.
We say that $\elemtuptuplea$ and $\elemtuptupleb$ are \emph{tuples of $z$-similar elements}.
We show that in this case, $\elemtuptuplea$ and $\elemtuptupleb$ have the same $\FGF$-type, which further implies by \cref{lem:fgf-type-iso} that they are isomorphic.
Let $\elemtuples$ and $\elemtuplet$ be the tuples from $\str{A}$ and $\str{B}$ such that $\seq{a_{k}} = \cdots \elemtuples$ and $\seq{b_{k}} = \cdots \elemtuplet$.
Since $a_{k} \approx_z b_{k}$, we have $\elemtuples \bisimto_{\FGF}^{z} \elemtuplet$ which in particular implies that $\elemtuples$ and $\elemtuplet$ have the same $\FGF$-type.
Observe that $\pi[\elemtuptuplea] = \elemtuples_{(\ctr{a_{k}}-k)\ldots{}\ctr{a_{k}}}$ and likewise $\pi[\elemtuptupleb] = \elemtuplet_{(\ctr{b_{k}}-k)\ldots{}\ctr{b_{k}}}$.
This follows from \cref{lem:projection-next} since $\elemtuptuplea$ and $\elemtuptupleb$ are live and thus must be next-chains.
As $\ctr{a_{k}} = \ctr{b_{k}}$, both $\elemtuptuplea$ and $\elemtuptupleb$ project to infixes $\elemtuples_{i\ldots{}j}$ and $\elemtuplet_{i\ldots{}j}$ over the common range $i\ldots{}j$ for $i = \ctr{a_{k}} - k$ and $j = \ctr{a_{k}}$.
Hence, $\FGF$-types of $\pi[\elemtuptuplea]$ and $\pi[\elemtuptupleb]$ are also equal.
Now consider an infix $\elemtuptuplea_{x\ldots{}y}$ and the corresponding infix $\elemtuptupleb_{x\ldots{}y}$, for some indices $x$ and $y$.
Our claim is that $\elemtuptuplea_{x\ldots{}y}$ and $\elemtuptupleb_{x\ldots{}y}$ are in the same relations.
We prove this claim by examining the three conditions of the definition of relations in the unraveling in turn.
We show that each condition is true for $\elemtuptuplea_{x\ldots{}y}$ if and only if it is true for $\elemtuptupleb_{x\ldots{}y}$.
Let $\relR$ be any relational symbol. Then:
\begin{enumerate}
  \item $\pi[\elemtuptuplea_{x\ldots{}y}] \in \relR^{\str{A}}$ if and only if $\pi[\elemtuptuplea_{x\ldots{}y}] \in \relR^{\str{B}}$, because $\pi[\elemtuptuplea]$ and $\pi[\elemtuptupleb]$ have equal $\FGF$-types,
  \item $\elemtuptuplea_{x\ldots{}y}$ is a next-chain if and only if $\elemtuptupleb_{x\ldots{}y}$ is a next-chain, because both infixes are always next-chains since $\elemtuptuplea$ and $\elemtuptupleb$ are next-chains as they are live,
  \item $|\elemtuptuplea_{x\ldots{}y}| \le \mathtt{bound}(\elemtuptuplea_{x\ldots{}y})$ if and only if $|\elemtuptupleb_{x\ldots{}y}| \le \mathtt{bound}(\elemtuptupleb_{x\ldots{}y})$ because $|\elemtuptuplea_{x\ldots{}y}| = |\elemtuptupleb_{x\ldots{}y}| = y-x+1$ and $\mathtt{bound}(\elemtuptuplea_{x\ldots{}y}) = \mathtt{bound}(\elemtuptupleb_{x\ldots{}y})$ since $\ctr{a_{j}} = \ctr{b_{j}}$ follows from $a_{j} \approx_{z} b_{j}$.
\end{enumerate}
Therefore, infixes of $\elemtuptuplea$ and $\elemtuptupleb$ are in the same relations.
We conclude that $\elemtuptuplea$ and $\elemtuptupleb$ have equal $\FGF$-types.

Finally, we show that the partial isomorphisms between tuples with $z$-similar elements form a $\GF$-bisimulation.
Let $\unravel{A}_{\ell}$ and $\unravel{B}_{\ell}$ be finite unravelings of $\sigma$-structures $\str{A}$ and $\str{B}$, for a finite signature $\sigma$.
We define a sequence of sets $\bisimZ_{0}, \ldots, \bisimZ_{n} \subseteq \PartIso{\unravel{A}_{\ell}}{\unravel{B}_{\ell}}$ as follows:
\begin{equation*}
  \bisimZ_{k} = \left\{
    \mu_{(\elemtuptuples, \elemtuptuplet)}:\,
    \elemtuptuples\ \text{and}\ \elemtuptuplet\ \text{are live and}\ \
    s_{i} \approx_{W * k} t_{i}\ \text{for all indices $i$}
  \right\}
\end{equation*}
where $W$ is the maximal arity of symbols from $\sigma$ and $\mu_{(\elemtuptuples, \elemtuptuplet)}$ is the partial isomorphism between $\elemtuptuples$ and $\elemtuptuplet$, which exists since they have $z$-similar elements.
We claim that $\bisimZ_{k-1}$ satisfies the~\ref{bisim:back} and~\ref{bisim:forth} conditions for $\bisimZ_{k}$, for $k \in [1, n]$.
We only show~\ref{bisim:forth} as the proof for~\ref{bisim:back} is symmetric.
Let $\mu_{(\elemtuptuples, \elemtuptuplet)} \in \bisimZ_{k}$, for some live tuples $\elemtuptuples$ and $\elemtuptuplet$.

For structures $\str{A}$ and $\str{B}$ and every element $a \in \unraveldom{A}_{\ell}$, we can find a $z$-similar element $b \in \unraveldom{B}_{\ell}$, given that $\str{A} \bisimto_{\FGF}^{z} \str{B}$, as follows.
Let $\seq{a} = \elemtuples^{(0)}\cdots(i^{{n}}, j^{(n)})\elemtuples^{(n)}$ and $m$ be the minimum of $z$ and $n$.
Using~\ref{bisim:fforth} $m$ times, we obtain live tuples $\elemtuplet^{(0)}, \ldots, \elemtuplet^{(m)}$ with $(\elemtuples^{(n-m)}, \elemtuplet^{(0)}) \in \bisimZ_{2*z}, \ldots, (\elemtuples^{(n)}, \elemtuplet^{(m)}) \in \bisimZ_{2*z - m}$ such that $\elemtuplet^{(0)}(i^{(n-m+1)}, j^{(n-m+1)})\elemtuplet^{(1)}\cdots{}(i^{n}, j^{n})\elemtuplet^{(m)}$ is a bisimulation sequence.
For any bisimulation it holds that $\bisimZ_{0} \supseteq \bisimZ_{1} \supseteq \cdots{} \supseteq \bisimZ_{2*z+1}$, thus for every x, $(\elemtuples^{(n-m+x)}, \elemtuplet^{(x)}) \in \bisimZ_{2*z - m}$.
Hence, since $z \le 2*z - m$, for every $x$, the tuples $\elemtuples^{(n-m+x)}$ and $\elemtuplet^{(x)}$ are $z$-$\FGF$-bisimilar.
We can now construct the element $b$ with $\ctr{b} = \ctr{a}$ and $\seq{b} = \elemtuplet^{(0)}(i^{(n-m+1)}, j^{(n-m+1)})\elemtuplet^{(1)}\cdots{}(i^{n}, j^{n})\elemtuplet^{(m)}$.
This satisfies $a \approx_{z} b$, as wanted.
