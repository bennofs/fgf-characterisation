%!TEX root = ../main.tex

\section{Finite Companions}\label{sec:finite}
Let $\unravel{A}, \elemtuptuplea$ be a tree unraveling of a finite structure $\str{A}, \elemtuplea$.
We describe the finite unraveling $\unravel{A}_{\ell}, \elemtuptuplea_{\ell}$ which will be $\ell$-$\FGF$-bisimilar to $\unravel{A}, \elemtuptuplea$ (and also $\ell$-$\FGF$-bisimilar to $\str{A}, \elemtuplea$ by transitivity of bisimilarity).
For an element $r \in \unraveldom{A}$, we construct the finite tree $\mathcal{T}_{r,\ell}(\unravel{A})$ as follows:
\begin{equation*}
  \mathcal{T}_{r,\ell}(\unravel{A}) = \left\{ e \in \unraveldom{A}:\, e\ \text{is a descendant of $r$ and the level of $\seq{e}$ is at most $2 * \ell$} \right\}.
\end{equation*}

\noindent
$\mathcal{T}_{r,\ell}$ is substructure of $\unravel{A}$.
It is a tree with root $r$ because for every element $e \in \mathcal{T}_{r,\ell} \setminus \{r\}$, the parent of $e$ is also in $\mathcal{T}_{r,\ell}$.
This is easy to see: if $p$ is the parent of $e$, then $p$ is a descendant of $r$ and $\seq{p}$ has a level lower or equal to the level of $\seq{e}$, so $p$ is in $\mathcal{T}_{r,\ell}$.
Furthermore, this tree is finite:
\begin{lemma}\label{lem:bounded-trees-are-finite}
  If $\unravel{A}$ is the unraveling of a finite structure $\str{A}$, then $\mathcal{T}_{r,\ell}$ is finite for $\ell \in \N$ and $r \in \unraveldom{A}$.
\end{lemma}
\begin{proofsketch}
  For a finite structure, there are only a finite number of live tuples and thus for any fixed $k$, there are only a finite number of $k$-bisimulation sequences in this structure.
  Hence, the number of elements where the level of $\seq{e}$ is at most $2 * \ell$ is also finite, so $\mathcal{T}_{r,\ell}$ is finite.
\end{proofsketch}
\begin{proof}
If $A$ is finite, then there only a finite number of live tuples in $A$.
Recall that a $\ell$-bisimulation sequence is just a word in $A^*{(\N\N{}A^{*})}^{\ell}$, consisting of live tuples from $A^{*}$ and indices into those tuples from $\N\N$.
Let $W$ be the maximum size of a live tuple.
Then an index into a live tuple must be in the range $[1,W]$, which is finite, so the set of possible indices is finite.
Thus there are only a finite number of $\ell$-bisimulation sequences, since both the set of live tuples and the set of indices is finite.
If $e \in \mathcal{T}_{r,\ell}$ with $\seq{e} = \sigma$, then the level of $\sigma$ is at most $2 * \ell$.
Thus, there are only a finite number of choices for $\sigma$.
For every $\sigma$, there are only a finite number of elements $e \in \unraveldom{A}$ with $\seq{e} = \sigma$.
It follows that there are only finitely many $e$ with $e \in \mathcal{T}_{r,\ell}$, so $\mathcal{T}_{r,\ell}$ is finite.
\end{proof}
Now consider the set $\mathcal{S}$ of trees with $\mathcal{S} = \left\{ \mathcal{T}_{r,\ell}:\, \seq{r}\ \text{has level at most $\ell$}\right\}$.
An element $b$ is called a \emph{cut element} of a tree $\mathcal{T} \in \mathcal{S}$ if it itself is not in $\mathcal{T}$ but $(a,b) \in \relNext$ for some $a$ in $\mathcal{T}$.
Intuitively, the cut elements are elements $e$ where the level of $\seq{e}$ is $2 * \ell + 1$ (so $e$ is not in $\mathcal{T}$) and $\ctr{e}$ is minimal (so there is a $\relNext$-edge from some element in $\mathcal{T}$).
The set of all cut elements for trees in $\mathcal{S}$ is finite, because $\mathcal{S}$ is finite and the cut elements of a tree $\mathcal{T}_{r,\ell} \in \mathcal{S}$ are contained in $\mathcal{T}_{r,\ell+1}$ which is finite by \cref{lem:bounded-trees-are-finite}.
Enumerate all cut elements of trees in $\mathcal{S}$ with indices from $1, \ldots, M$, where $M$ is the number of cut elements.
The domain of the structure $\unravel{A}_{\ell}$ consists of $M$ copies of each tree in $\mathcal{S}$.
The relations are defined as in $\unravel{A}$, but with the relation $\relNext$ replaced by by its finitary variant $\relNext_{\ell}$, where $(a,b) \in \relNext_{\ell}$ if:
\begin{itemize}
  \item $(a,b) \in \relNext$, or
  \item $(a,a') \in \relNext$ where $a'$ is a cut element with the associated index $m$, and all of the following are true:
        \begin{enumerate}
          \item $b$ is the root of the $m$-th copy of tree in $S$,
          \item $\ctr{b} = \ctr{a'}$,
          \item $\seq{b}$ is the level-$\ell$ suffix of $\seq{a'}$.
        \end{enumerate}
\end{itemize}

\noindent
The structure $\unravel{A}_{\ell}$ is tree-like, in the sense of the following lemma:
\begin{lemma}\label{lem:companion-tree-like}
  Let $D_{a, \ell}(\unravel{A}_{\ell})$ be the set of elements $b$ for which there exists a next-chain starting from $a$ and ending in $b$ of length at most $\ell$.
  For any element $a \in \unraveldom{A}_{\ell}$, the set $D_{a, \ell}$ is a tree with root $a$.
\end{lemma}
\begin{proofsketch}
  By contradiction: if $D_{a,\ell}$ is not a tree, then some element $e \in D_{a,\ell}$ has at least two parents $p_{1}, p_{2} \in D_{a,\ell}$.
  In this case, $p_{1}$ and $p_{2}$ must be leaves of different subtrees $\mathcal{T}_{r_{1},\ell}$ and $\mathcal{T}_{r_{2},\ell}$ of the structure $\unravel{A}_{\ell}$ and have sequences with level $2 * \ell$.
  Now, at least one of $p_{1}$ or $p_{2}$ must be in a different subtree than $a$.
  The shortest next-chain ending at that element and starting from $a$ must traverse the full subtree $\mathcal{T}_{r_{1},\ell}$ or $\mathcal{T}_{r_{2},\ell}$, which requires at least a next-chain of length $\ell + 1$ (at least one next transition to enter the tree, $\ell$ next transitions to reach a leaf with a sequence with level $2 * \ell$).
  Hence the element cannot be in $D_{a,\ell}$, a contradiction.
\end{proofsketch}
\begin{proof}
  Assume by contradiction that $D_{a,\ell}$ is not a tree.
  Clearly, all elements in $D_{a,\ell}$ are descendants of $a$.
  Thus, if $D_{a,\ell}$ is not a tree, there must be an element $e \in D_{a,\ell}$ which has at least two parents $p_{1}, p_{2} \in D_{a,\ell}$.
  Recall that the structure $\str{A}$ is consists of subtrees $\mathcal{T}_{r,\ell}$.
  For $e$ to have two different parents, those parents must be leaves of different subtrees $\mathcal{T}_{r_{1},\ell}$ and $\mathcal{T}_{r_{2}, \ell}$ for different roots $r_{1}, r_{2} \in \unraveldom{A}$.
  At least one of the elements $p_{1}$ and $p_{2}$ must therefore be a different subtree than $a$.
  By symmetry, let us assume that $p_{1}$ is not in the same subtree as $a$.
  Then the shortest next-chain from $a$ to $p_{1}$ must traverse through the whole subtree from $r_{1}$ to the leaf $p_{1}$.
  With every transition along the a $\relNext_{\ell}$-edge, the level of the sequence of an element increases by at most 1.
  Hence, the next-chain from $r_{1}$ to $p_{1}$ has a length of at least $\ell$.
  Since the shortest next-chain from $a$ to $p_{1}$ must go through $r_{1}$ (as $a$ is not in the same subtree as $p_{1}$) and reaching $r_{1}$ from $a$ also requires at least one next transition, the shortest next-chain from $a$ to $p_{1}$ is longer than $\ell$.
  Hence, $p_{1} \notin D_{a,\ell}$, a contradiction.
\end{proof}

In the following, let $\elemtuplea \isoeq \elemtupleb$ denote that $\restr{\str{A}}{\set(\elemtuplea)}$ is isomorphic to $\restr{\str{B}}{\set(\elemtupleb)}$, for tuples $\elemtuplea$ and $\elemtupleb$ from some structures $\str{A}$ and $\str{B}$.
Let $\elemtuptuplea$ and $\elemtuptupleb$ be live tuples from $\unravel{A}_{\ell}$ and $\unravel{B}_{\ell}$.
We employ \cref{lem:companion-tree-like} to show that $\atp{\FGF}{\unravel{A}_{\ell}}{\elemtuptuplea} = \atp{\FGF}{\unravel{B}_{\ell}}{\elemtuptupleb}$ implies $\elemtuptuplea \simeq \elemtuptupleb$, if $\ell$ is large enough:
\begin{lemma}\label{lem:fgf-type-iso}
  Let $\elemtuptuplea$ and $\elemtuptupleb$ be live tuples in finite unravelings $\unravel{A}_{\ell}$ and $\unravel{B}_{\ell}$ of structures $\str{A}$ and $\str{B}$ for a parameter $\ell$.
  If $\ell > \arity(\Sigma)$, then $\atp{\FGF}{\unravel{A}_{\ell}}{\elemtuptuplea} = \atp{\FGF}{\unravel{B}_{\ell}}{\elemtuptupleb}$ implies $\elemtuptuplea \isoeq \elemtuptupleb$, witnessed by the isomorphism $\mu_{(\elemtuptuplea, \elemtuptupleb)}:\, a_{i} \mapsto b_{i}$ for every $i \in [1, k]$.
\end{lemma}
\begin{proofsketch}
  If $\elemtuptuplea = (a_{1}, \ldots, a_{n})$, then permutations of $\elemtuptuplea$ changing the order of the components are never live, since they cannot be next-chains due to the tree structure of $\unravel{A}_{\ell}$.
  Similarly, all live tuples consisting of a subset of the elements of $\elemtuptuplea$ must be infixes of $\elemtuptuplea$.
  Hence, the atomic-$\FGF$-type fully characterizes the atomic relations among the set of elements in $\elemtuptuplea$.
  Therefore, equivalence of atomic-$\FGF$-types is sufficient to ensure isomorphism in tree unravelings.
\end{proofsketch}
\begin{proof}
Let $\ell > W$ and consider live tuples $\elemtuptuplea \sqin \unraveldom{A}_{\ell}$ and $\elemtuptupleb \sqin \unraveldom{B}_{\ell}$ with equal atomic-$\FGF$-types and equal size $k$.
We claim that the map $\mu_{(\elemtuptuplea, \elemtuptupleb)}:\, a_{i} \mapsto b_{i}$ for all $i \in [1, k]$, is an isomorphism between $\restr{\str{\unravel{A}_{\ell}}}{\set(\elemtuptuplea)}$ and $\restr{\str{\unravel{B}_{\ell}}}{\set(\elemtuptupleb)}$.
Let $\elemtuptupler$ be a tuple with $\elemtuptupler \sqin \set(\elemtuptuplea)$ and $\elemtuptupler \in \relR^{\str{A}}$ for some relation $\relR \in \Sigma$.
We first show that $\elemtuptupler$ is an infix of $\elemtuptuplea$.
Suppose to the contrary that $\elemtuptupler$ is not an infix of $\elemtuptuplea$, thus there is an $i$ such that $r_{i} = a_{v}$ and $r_{i+1} = a_{w}$ with $v \ne w - 1$.
But then both $(a_{v}, a_{w}) \in \relNext_{\ell}$ and $(a_{w-1}, a_{w}) \in \relNext_{\ell}$ since $\elemtuptupler$ and $\elemtuptuplea$ are live.
This is a contradiction, since $D_{a_{1}, W}$ defined in \cref{lem:companion-tree-like} includes $a_{v}, a_{w-1}$ and $a_{w}$ but cannot be a tree since $a_{w}$ would have two parents $a_{v}$ and $a_{w-1}$.
Therefore, $\elemtuptupler$ is an infix of $\elemtuptuplea$.
It follows from equality of atomic-$\FGF$-types that $\mu_{(\elemtuptuplea, \elemtuptupleb)}[\elemtuptupler] \in \relR^{\str{B}}$.
Since the argument is symmetric, we conclude that elements of $\elemtuptuplea$ and $\elemtuptupleb$ satisfy the same relations so $\mu_{(\elemtuptuplea, \elemtuptupleb)}$ is an isomorphism and $\elemtuptuplea \simeq \elemtuptupleb$.
\end{proof}

Our main goal for the remainder of this section is to show that $\unravel{A}_{\ell} \bisimto_{\GF}^{n} \unravel{B}_{\ell}$, given that $\str{A} \bisimto_{\FGF}^{m} \str{B}$ for some $m$ and $\ell$ which depend only on $n$.
For this proof, we first introduce the notion of $z$-\emph{similar} elements, which is a relation between elements from $\unravel{A}_{\ell}$ and $\unravel{B}_{\ell}$.
We later lift this relation to tuples to construct the $n$-$\GF$-bisimulation.
Intuitively, the idea behind the notion of $z$-similarity is to put elements in relation which behave similar for up to $z$ transitions along $\relNext$-edges, in both forward and backward direction.
In other words, we want to define $z$-similarity such that if $a \in \unraveldom{A}_{\ell}$ and $b \in \unraveldom{B}_{\ell}$ are $z$-similar elements, then for every parent (or child) $c \in \unraveldom{A}_{\ell}$ of $a$, there exists a parent (or child) $d \in \unraveldom{B}_{\ell}$ of $b$ which is $(z-1)$-similar to $c$.
Let us first consider the simpler case of only achieving this property for children (i.e.\ forward transitions along $\relNext$-edges).
We say that elements $a \in \unraveldom{A}_{\ell}$ and $b \in \unraveldom{B}_{\ell}$ are $z$-\emph{forward-similar} if $\ctr{a} = \ctr{b}$, $\seq{a} = \cdots \elemtuples$ and $\seq{b} = \cdots \elemtuplet$ for tuples $\elemtuples$ and $\elemtuplet$ that satisfy $\tp{\FGF_{z}}{\str{A}}{\elemtuples} = \tp{\FGF_{z}}{\str{B}}{\elemtuplet}$.
Note that $\tp{\FGF_{z}}{\str{A}}{\elemtuples} = \tp{\FGF_{z}}{\str{B}}{\elemtuplet}$ is equivalent to $\str{A}, \elemtuples \bisimto_{\FGF}^{z} \str{B}, \elemtuplet$ by \cref{lem:FGF-bisimulations-work-well}.
If we have such $z$-forward-similar elements $a$ and $b$ and $c \in \unraveldom{A}_{\ell}$ is a child of $a$, then by definition of $\relNext_{\ell}$ there are two possible situations: either (\romannumeral1) $c$ has the same sequence as $a$ but an incremented counter, or (\romannumeral2) $c$ adds a move to the sequence of $a$ (possibly removing elements from the beginning of the sequence if the result is too long).
In both cases, we can find a child $b$ that is $(z-1)$-forward-similar to $c$: in case (\romannumeral1) we simply increment the counter of $b$, whereas in case (\romannumeral2) we employ~\ref{bisim:fforth} to find a move in $\str{B}$ that we can add to the end of the sequence of $b$ to obtain a child $d$ of $b$ so that the (new) last tuples of $\seq{c}$ and $\seq{d}$ have equal $\FGF_{z-1}$-types.

We now consider the general case, including backward transitions.
As we saw above, forward transitions along $\relNext$-edges extend the sequence of an element by adding moves to the end.
If we allow backward transitions, then we need to consider the opposite case: that moves are removed from the end of the sequence.
It is thus no longer sufficient to only look at the type of the last tuple in a sequence.
Instead, we need to consider the types of the last $z$ tuples.
For this, we introduce the auxilliary notion of $z$-histories:
\begin{definition}[$z$-history]
  Let $\str{A}$ be a structure and $\ell$ be a parameter.
  The $z$-\emph{history} of an element $a \in \unraveldom{A}_{\ell}$ with $\seq{a} = \elemtuples^{(0)} \cdots (i^{n}, j^{n})\elemtuples^{n}$ for some $n \in \N$, tuples $\elemtuples^{(0)}, \ldots, \elemtuples^{(n)}$ and indices $i^{(1)}, \ldots, i^{(n)}$ and $j^{(1)}, \ldots, j^{(n)}$ is denoted by $\hist{z}{a}$ and defined as:
  \begin{equation*}
    \hist{z}{a} = \tp{\FGF_{z}}{\str{A_{\ell}}}{\elemtuples^{(n-h)}}(i^{(n-h+1)}, j^{(n-h+1)})\tp{\FGF_{z}}{\str{A_{\ell}}}{\elemtuples^{(n-h+1)}}\cdots(i^{n}, j^{n})\tp{\FGF_{z}}{\str{A}_{\ell}}{\elemtuples^{(n)}}
  \end{equation*}
  where $h = \min(z, n)$. If $h < z$, then $\hist{z}{a}$ is called a \emph{short $z$-history}.
\end{definition}
Intuitively, we obtain the history of an element by replacing the individual tuples $\elemtuples^{(0)}, \elemtuples^{(1)}, \ldots$ in the bisimulation sequence with their $\FGF_{z}$-types $\tp{\FGF_{z}}{\str{A}_{\ell}}{\elemtuples^{(0)}}, \tp{\FGF_{z}}{\str{A}_{\ell}}{\elemtuples^{(1)}}, \ldots$ and truncating this sequence to only the last $z$ steps.
In the case that the original sequence already has less than $z$ steps, then this truncation does nothing and we get a short $z$-history, also with less than $z$ steps.
This definition is analogous to a construction used by Otto in the upgrading from global bisimulation to two-way global bisismulation for modal logic~\cite[Def. 39]{Otto04}.
Histories can be compared across unravelings of different structures, since they no longer refer to concrete elements from the base structure.
For two unravelings $\str{A}_{\ell}$ and $\str{B}_{\ell}$ of different base structures $\str{A}$ and $\str{B}$, we define $z$-similar elements to be those elements which have equal $z$-histories and equal counters:
\begin{definition}[$z$-similar elements]
 Let $e \in \unraveldom{A}_{\ell}$ and $f \in \unraveldom{B}_{\ell}$ be domain elements, and $z$ be a natural number. Then $e$ and $f$ are $z$-similar, written $e \approx_{z} f$, if $\ctr{e} = \ctr{f}$ and $\hist{z}{e} = \hist{z}{f}$.
\end{definition}
Note that if $e$ and $f$ are $z$-similar and $\hist{z}{e}$ is a short $z$-history, then $\hist{z}{f}$ must also be a short $z$-history.
So if either $\seq{e}$ or $\seq{f}$ has a level less than $z$, then $\seq{e}$ and $\seq{f}$ have the same level.
Compared to the notion of $z$-forward-similarity, $z$-similarity requires not just the $\FGF_{z}$-type of the last tuple in the sequences to match.
Rather, it additionally requires that the types of the $z$ last tuples of the sequences are equal.
Because of this, $z$-similar elements have both $(z-1)$-similar children and $(z-1)$-similar parents, as we prove with the next lemma:
\begin{lemma}\label{lem:approx-next}
  Let $\str{A}_{\ell}$ and $\str{B}_{\ell}$ be the finite unravelings of structures $\str{A}$ and $\str{B}$ for some parameter $\ell \in \N$.
  For elements $a \in \unraveldom{A}_{\ell}$ and $b \in \unraveldom{B}_{\ell}$, if $a \approx_{z} b$ for some $z \in [1,\ell]$, then:
  \begin{description}
    \item[\desclabel{(succ)}{elemeq:succ}] If $(a, c) \in \relNext_{\ell}$ for $c \in \unraveldom{A}_{\ell}$, then there is $d \in \unraveldom{B}_{\ell}$ with $(b, d) \in \relNext_{\ell}$ and $c \approx_{z-1} d$.
    \item[\desclabel{(pred)}{elemeq:pred}] If $(c, a) \in \relNext_{\ell}$ for $c \in \unraveldom{A}_{\ell}$, then there is $d \in \unraveldom{B}_{\ell}$ with $(d, b) \in \relNext_{\ell}$ and $c \approx_{z-1} d$.
  \end{description}
\end{lemma}
\begin{proofsketch}
  Since $(a,c) \in \relNext_{\ell}$ or $(c,a) \in \relNext_{\ell}$, we have three cases:
  \begin{romanenumerate}
    \item only counter changed: $\seq{a} = \seq{c}$ and $\ctr{c} = \ctr{a} \pm 1$,
    \item sequence extended: $\seq{c} = \mathsf{trunc}_{\ell}[\seq{a} (i,j) \elemtuples]$ and $\ctr{c} = j-i+1$, or
    \item sequence reduced: $\seq{c}$ is such that $\seq{a} = \mathsf{trunc}_{\ell}[\seq{c} (i,j) \elemtuples]$.
  \end{romanenumerate}
  where
  \begin{displaymath}
    \mathsf{trunc}_{\ell} =
    \begin{cases}
      \sigma & \text{if $\sigma$ has level at most $2 * \ell$} \\
      \text{level-$\ell$ suffix of $\sigma$} & \text{if $\sigma$ has level greater than $2 * \ell$.}
    \end{cases}
  \end{displaymath}
  Note that case (\romannumeral2) is possible only for~\ref{elemeq:succ} and (\romannumeral3) only for~\ref{elemeq:pred}.

  In case (\romannumeral1), $a$ and $c$ have the same history.
  We can find an element $d$ with $(b,d) \in \relNext_{\ell}$ (or $(d,b) \in \relNext_{\ell}$) by incrementing (or decrementing) the counter of $b$ accordingly, keeping the history the same, thus $c \approx_{z-1} d$ (in fact, even $c \approx_{z} d$).
  It can be seen that $\hist{1}{a} = \hist{1}{b}$ and $c \in \unraveldom{A}_{\ell}$ implies $d \in \unraveldom{B}_{\ell}$.

  In case (\romannumeral2), we employ the fact that the last tuples of $\seq{a}$ and $\seq{b}$ have the same $\FGF_{z}$-type.
  Thus, we can find a tuple $\elemtuplet$ such that $\seq{b} (i,j) \elemtuplet$ is a bisimulation sequence and $\elemtuplet$ has the same $\FGF_{z-1}$-type as $\elemtuples$.
  Next, we construct the element $d$ with $\seq{d} = \mathsf{trunc}_{\ell}[\seq{b}(i,j)\elemtuplet]$ and $\ctr{d} = (j-i+1) + 1$.
  Cleary, $d \in \unraveldom{B}_{\ell}$ and $(b,d) \in \relNext_{\ell}$.
  The function $\mathsf{trunc}_{\ell}$ preserves suffixes with level $z$ since $z \le \ell$, so it does not affect $z$-histories.
  By our choice of $\elemtuplet$, we have $\hist{z-1}{c} = \hist{z-1}{d}$.
  It follows that $c \approx_{z-1} d$, as wanted.

  In case (\romannumeral3), by equivalence of $z$-histories, we know that $\seq{b} = \mathsf{trunc}_{\ell}[\sigma \cdots(i,j)\elemtuplet]$ for some tuple $\elemtuplet$ and a bismimulation sequence $\sigma$ with level at most $2 * \ell$.
  Further, by equivalence of counters, $\ctr{b} = \ctr{a} = (j-i+1) + 1$.
  By definition of the unraveling, there is an element $d \in \unraveldom{B}_{\ell}$ with $(d,b) \in \relNext_{\ell}$, which has $\seq{d} = \sigma$.
  This element satisfies $c \approx_{z-1} d$.
\end{proofsketch}
\begin{proof}
  Since $(a,c) \in \relNext_{\ell}$ or $(c,a) \in \relNext_{\ell}$, we have three cases:
  \begin{romanenumerate}
    \item only counter changed: $\seq{a} = \seq{c}$ and $\ctr{c} = \ctr{a} \pm 1$,
    \item sequence extended: $\seq{c} = \mathsf{trunc}_{\ell}[\seq{a} (i,j) \elemtuples]$ and $\ctr{c} = j-i+1$, or
    \item sequence reduced: $\seq{c}$ is such that $\seq{a} = \mathsf{trunc}_{\ell}[\seq{c} (i,j) \elemtuples]$.
  \end{romanenumerate}
  where
  \begin{displaymath}
    \mathsf{trunc}_{\ell} =
    \begin{cases}
      \sigma & \text{if $\sigma$ has level at most $2 * \ell$} \\
      \text{level-$\ell$ suffix of $\sigma$} & \text{if $\sigma$ has level greater than $2 * \ell$}
    \end{cases}
  \end{displaymath}.
  Note that case (\romannumeral2) is possible only for~\ref{elemeq:succ} and (\romannumeral3) only for~\ref{elemeq:pred}.

  In case (\romannumeral1), $a$ and $c$ have the same history.
  We can find an element $d$ with $(b,d) \in \relNext_{\ell}$ (or $(d,b) \in \relNext_{\ell}$) by incrementing (or decrementing) the counter of $b$ accordingly, keeping the history the same, thus $c \approx_{z-1} d$ (in fact, even $c \approx_{z} d$).
  We claim that this element $d$ is in $\unraveldom{B}_{\ell}$.
  We consider two cases: either $\seq{a} = \elemtuples$ or $\seq{a} = \cdots (i,j) \elemtuples$, for some tuple $\elemtuples$.
  Since $b$ has the same $z$-history as $a$, there is a tuple $\elemtuplet$ with the same $\FGF_{z}$-type as $s$ such that $\seq{b} = \elemtuplet$ or $\seq{b} = \cdots (i,j) \elemtuplet$, respectively.
  Since $\elemtuples$ and $\elemtuplet$ have equal types, $|\elemtuplet| = |\elemtuples|$.
  Further, we have $\ctr{c} = \ctr{d}$.
  We can see that in this case, the conditions for $d$ to be in $\unraveldom{B}_{\ell}$ are equivalent to the conditions for $c$ to be in $\unraveldom{A}_{\ell}$.
  As we know that $c \in \unraveldom{A}_{\ell}$, it follows that $d$ is in $\unraveldom{B}_{\ell}$.

  In case (\romannumeral2), we employ the fact that the last tuples of $\seq{a}$ and $\seq{b}$ have the same $\FGF_{z}$-type, due to their equal $z$-histories.
  Thus, we can find a tuple $\elemtuplet$ such that $\seq{b} (i,j) \elemtuplet$ is a bisimulation sequence and $\elemtuplet$ has the same $\FGF_{z-1}$-type as $\elemtuples$.
  Next, we construct the element $d$ with $d = (\mathsf{trunc}_{\ell}, \ctr{c})$ for $\sigma = \seq{b}(i,j)\elemtuplet$.
  By construction, $d \in \unraveldom{B}_{\ell}$ and $(b,d) \in \relNext_{\ell}$ (recall that $\ctr{c} = (j-i+1)+1$).
  Observe that $\mathsf{trunc}_{\ell}$ preserves suffixes with level $z$, since $z \le \ell$.
  Hence, the $(z-1)$-history of the element $c$ is equal to the $(z-1)$-history of an element $e$ with $e = (\seq{a}(i,j)\elemtuples, \ctr{c})$.
  Similary, the $(z-1)$-history of $d$ is equal to the $(z-1)$-history of an element $f$ with $f = (\seq{b}(i,j)\elemtuplet, \ctr{d})$.
  As $\elemtuples$ and $\elemtuplet$ have the same $\FGF_{z-1}$-type, we can see that $\hist{z-1}{e} = \hist{z-1}{f}$ and hence $\hist{z-1}{c} = \hist{z-1}{d}$.
  It follows that $c \approx_{z-1} d$, as wanted.

  In case (\romannumeral3), by equivalence of $z$-histories, we know that $\seq{b} = \mathsf{trunc}_{\ell}[\sigma(i,j)\elemtuplet]$ for some tuple $\elemtuplet$ and a bisimulation sequence $\sigma$ with level at most $2 * \ell$.
  Further, by equivalence of counters, $\ctr{b} = \ctr{a} = (j-i+1) + 1$.
  By definition of the unraveling, there is an element $d \in \unraveldom{B}_{\ell}$ with $(d,b) \in \relNext_{\ell}$, which has $\seq{d} = \sigma$.
  Again, because $\mathsf{trunc}_{\ell}$ preserves suffixes with level $z$, the $z$-history of $b$ is equal to the $z$-history of an element $b'$ with $b' = (\sigma(i,j)\elemtuplet, \ctr{b})$.
  As $\seq{d} = \sigma$, the $(z-1)$-history of $d$ is equal to the $z$-history of $b'$ minus the last step.
  As $b$ and $a$ have the same $z$-history and the $(z-1)$-history of $c$ is also equal to the $z$-history of $a$ minus the last step, it follows that $c$ and $d$ have equal $(z-1)$-histories.
  Further, both $\ctr{c} = j$ and $\ctr{d} = j$.
  This follows from $(c,a) \in \relNext_{\ell}$ and $(d,b) \in \relNext_{\ell}$.
  Hence, $c \approx_{z-1} d$ as required.
\end{proof}

We now lift the notion of $z$-similarity from elements to tuples of elements.
Let $\elemtuptuplea \sqin \unraveldom{A}_{\ell}$ and $\elemtuptupleb \sqin \unraveldom{B}_{\ell}$ be live tuples of equal size $k$ with $\elemtuptuplea = (a_{1}, \ldots, a_{k})$ and $\elemtuptupleb = (b_{1}, \ldots, b_{k})$.
We say that $\elemtuptuplea$ and $\elemtuptupleb$ are \emph{tuples of $z$-similar elements} if $a_{i} \approx_{z} b_{i}$ for every $i \in [1,k]$.
We show that in this case, $\elemtuptuplea$ and $\elemtuptupleb$ have the same atomic-$\FGF$-type, which further implies by \cref{lem:fgf-type-iso} that they are isomorphic.
Let $\elemtuples$ and $\elemtuplet$ be the tuples from $\str{A}$ and $\str{B}$ such that $\seq{a_{k}} = \cdots \elemtuples$ and $\seq{b_{k}} = \cdots \elemtuplet$.
Since $a_{k} \approx_z b_{k}$, we know that $\elemtuples$ and $\elemtuplet$ have the same atomic-$\FGF$-type.
Observe that $\pi[\elemtuptuplea] = \elemtuples_{(\ctr{a_{k}}-k)\ldots{}\ctr{a_{k}}}$ and likewise $\pi[\elemtuptupleb] = \elemtuplet_{(\ctr{b_{k}}-k)\ldots{}\ctr{b_{k}}}$.
This follows from \cref{lem:projection-next} since $\elemtuptuplea$ and $\elemtuptupleb$ are live and thus must be next-chains.
As $\ctr{a_{k}} = \ctr{b_{k}}$, both $\elemtuptuplea$ and $\elemtuptupleb$ project to infixes $\elemtuples_{i\ldots{}j}$ and $\elemtuplet_{i\ldots{}j}$ over the common range $i\ldots{}j$ for $i = \ctr{a_{k}} - k$ and $j = \ctr{a_{k}}$.
Hence, atomic-$\FGF$-types of $\pi[\elemtuptuplea]$ and $\pi[\elemtuptupleb]$ are also equal.
Now consider an infix $\elemtuptuplea_{x\ldots{}y}$ and the corresponding infix $\elemtuptupleb_{x\ldots{}y}$, for some indices $x$ and $y$.
Our claim is that $\elemtuptuplea_{x\ldots{}y}$ and $\elemtuptupleb_{x\ldots{}y}$ are in the same relations.
We prove this claim by examining the three conditions of the definition of relations in the unraveling in turn.
We show that each condition is true for $\elemtuptuplea_{x\ldots{}y}$ if and only if it is true for $\elemtuptupleb_{x\ldots{}y}$.
Let $\relR$ be any relational symbol. Then:
\begin{enumerate}
  \item $\pi[\elemtuptuplea_{x\ldots{}y}] \in \relR^{\str{A}}$ if and only if $\pi[\elemtuptuplea_{x\ldots{}y}] \in \relR^{\str{B}}$, because $\pi[\elemtuptuplea]$ and $\pi[\elemtuptupleb]$ have equal atomic-$\FGF$-types,
  \item $\elemtuptuplea_{x\ldots{}y}$ is a next-chain if and only if $\elemtuptupleb_{x\ldots{}y}$ is a next-chain, because both infixes are always next-chains since $\elemtuptuplea$ and $\elemtuptupleb$ are next-chains as they are live,
  \item $|\elemtuptuplea_{x\ldots{}y}| \le \bound{\elemtuptuplea_{x\ldots{}y}}$ if and only if $|\elemtuptupleb_{x\ldots{}y}| \le \bound{\elemtuptupleb_{x\ldots{}y}}$ because $|\elemtuptuplea_{x\ldots{}y}| = |\elemtuptupleb_{x\ldots{}y}| = y-x+1$ and $\bound{\elemtuptuplea_{x\ldots{}y}} = \bound{\elemtuptupleb_{x\ldots{}y}}$ since $\ctr{a_{j}} = \ctr{b_{j}}$ follows from $a_{j} \approx_{z} b_{j}$.
\end{enumerate}
Therefore, infixes of $\elemtuptuplea$ and $\elemtuptupleb$ are in the same relations.
We conclude that $\elemtuptuplea$ and $\elemtuptupleb$ have equal $\FGF$-types.
Combining this result with \cref{lem:fgf-type-iso}, we get the following corollary:
\begin{corollary}\label{cor:tuple-similar-iso}
  Let $\elemtuptuplea$ and $\elemtuptupleb$ be live tuples of size $k$ in finite unravelings $\unravel{A}_{\ell}$ and $\unravel{B}_{\ell}$ for structures $\str{A}$, $\str{B}$ and a parameter $\ell$.
  If $\ell > \arity(\Sigma)$ and $a_{i} \approx_{z} a_{i}$ for all $i \in [1,k]$ and some $z \in \N$, then $\elemtuptuplea \isoeq \elemtuptupleb$, witnessed by the isomorphism $\mu_{(\elemtuptuplea, \elemtuptupleb)}:\, a_{i} \mapsto b_{i}$ for every $i \in [1, k]$.
\end{corollary}
Finally, we employ these isomorphisms to construct a $n$-$\GF$-bisimulation between finite unravelings:
\begin{lemma}
  Let $\str{A}$ and $\str{B}$ be structures and $W = \arity(\Sigma)$.
  If $\str{A} \bisimto_{\FGF}^{2 * W * n} \str{B}$, then there is a $n$-$GF$-bisimulation between $\unravel{A}_{W * n}$ and $\unravel{B}_{W * n}$, given by the sequence of sets $\bisimZ_{0}, \ldots, \bisimZ_{n} \subseteq \PartIso{\unravel{A}_{W * n}}{\unravel{B}_{W * n}}$ defined as:
  \begin{equation*}
    \bisimZ_{k} = \left\{
      \mu_{(\elemtuptuples, \elemtuptuplet)}:\,
      \elemtuptuples\ \text{and}\ \elemtuptuplet\ \text{are live and}\ \
      s_{i} \approx_{W * k} t_{i}\ \text{for all indices $i$}
    \right\}
  \end{equation*}
  where $\mu_{(\elemtuptuples, \elemtuptuplet)}$ is the isomorphism between $\elemtuptuples$ and $\elemtuptuplet$ as in \cref{cor:tuple-similar-iso}.
\end{lemma}
\begin{proofsketch}
  We show that $\bisimZ_{k-1}$ satisfies~\ref{bisim:forth} for $\bisimZ_{k}$, the proof for the condition~\ref{bisim:back} is symmetric.
  Let $\elemtuptuplea$ be a live tuple in $\unravel{A}_{W * n}$ and $\mu_{(\elemtuptuples, \elemtuptuplet)} \in \bisimZ_{k}$.
  We differentiate between two cases: (i) $\elemtuptuplea$ and $\elemtuptuples$ have at least one common element and (ii) $\elemtuptuplea$ and $\elemtuptuples$ have no common elements.
  \begin{romanenumerate}
    \item
    If there are common elements, then due to the tree-like nature of the unraveling, there are indices $i$ and $j$ such that the infix $\elemtuptuples_{i\ldots{}j}$ contains all the common elements of $\elemtuptuplea$ and $\elemtuptuples$, and the remaining elements $s_{1}, \ldots s_{i-1}, s_{j+1}, \ldots, s_{|\elemtuptuples|}$ are all distinct from elements of $\elemtuptuplea$.
    These common elements are also an infix of $\elemtuptuplea$, so there are indices $v$ and $w$ such that $\elemtuptuplea_{v\ldots{}w} = \elemtuptuples_{i\ldots{}j}$ and the remaining elements $a_1, \ldots, a_{v-1}, a_{w+1}, \ldots, a_{|\elemtuptuplea|}$ are all distinct from elements of $\elemtuptuples$.
    We iteratively apply~\ref{elemeq:pred} and~\ref{elemeq:succ} from \cref{lem:approx-next} to find elements $b_{1}, \ldots, b_{v-1}$ and $b_{w+1}, \ldots, b_{|\elemtuptuplea|}$ from $\unravel{B}_{W * n}$ such that $\elemtuptupleb$ with $\elemtuptupleb = b_{1}\cdots{}b_{v-1}\elemtuptuplet_{i\ldots{}j}b_{w+1}\cdots{}b_{|a|}$ is a tuple of elements that are each $(W*(k-1))$-similar to the corresponding elements of $\elemtuptuplea$.
    Additionally, $\elemtuptupleb$ is live since $\elemtuptuplea$ is live and they have the same atomic-$\FGF$-type.
    Thus the isomorphism $\mu_{(\elemtuptuplea,\elemtuptupleb)}$ is in $Z_{k-1}$.
    The common domain of $\mu_{(\elemtuptuples,\elemtuptuplet)}$ and $\mu_{(\elemtuptuplea,\elemtuptupleb)}$ is $\set(\elemtuptuplea_{v\ldots{}w})$, on which $\mu_{(\elemtuptuples,\elemtuptuplet)}$ and $\mu_{(\elemtuptuplea,\elemtuptupleb)}$ agree since $\elemtuptuplea_{v\ldots{}w} = \elemtuptuples_{i\ldots{}j}$ and $\elemtuptupleb_{v\ldots{}w} = \elemtuptuplet_{i\ldots{}j}$.
    This satsifies the requirements for~\ref{bisim:forth}.

    \item
    If $\elemtuptuplea$ and $\elemtuptuples$ have no common elements, then we first find an element $b_{1} \in \unraveldom{B}_{W * n}$ that is $(W * k)$-similar to $a_{1}$.
    We can find such an element by employing~\ref{bisim:forth} $W * k$ times, ``replaying'' the $(W*k)$-history of $a_{1}$ in the structure $\unravel{B}_{W * n}$.
    Similar to case (i), we now use~\ref{elemeq:succ} to find elements $b_{2}, \ldots, b_{|\elemtuptuplea|} \in \unraveldom{B}_{W * n}$ such that the tuple $\elemtuptupleb$ with $\elemtuptupleb = (b_{1}, \ldots, b_{|\elemtuptuplea|})$ is $W * (k - 1)$-similar to $\elemtuptuplea$.
    Now $\mu_{(\elemtuptuplea, \elemtuptupleb)} \in Z_{k-1}$ satisfies the requirements for~\ref{bisim:forth}.
  \end{romanenumerate}
\end{proofsketch}
\begin{proof}
  We show that $\bisimZ_{k-1}$ satisfies~\ref{bisim:forth} for $\bisimZ_{k}$, the proof for the condition~\ref{bisim:back} is symmetric.
  Let $\elemtuptuplea$ be a live tuple in $\unravel{A}_{W * n}$ and $\mu_{(\elemtuptuples, \elemtuptuplet)} \in \bisimZ_{k}$.
  We differentiate between two cases: (i) $\elemtuptuplea$ and $\elemtuptuples$ have at least one common element and (ii) $\elemtuptuplea$ and $\elemtuptuples$ have no common elements.
  \begin{romanenumerate}
    \item
    If there are common elements, then we claim that due to the tree-like nature of the unraveling, there are indices $i$ and $j$ such that the infix $\elemtuptuples_{i\ldots{}j}$ contains all the common elements of $\elemtuptuplea$ and $\elemtuptuples$, and the remaining elements $s_{1}, \ldots s_{i-1}, s_{j+1}, \ldots, s_{|\elemtuptuples|}$ are all distinct from elements of $\elemtuptuplea$.
    Assume by contradiction that this is not the case, thus there are elements $s_{x}, s_{y} \in \set(\elemtuptuplea)$ with $x < y$ and $s_{y-1} \notin \set(\elemtuptuplea)$.
    Since $s_{x}, s_{y} \in \set(\elemtuptuplea)$ and $\elemtuptuplea$ is live, either $s_{x} a_{u} a_{u+1} \cdots s_{y}$ or $s_{y} a_{u} a_{u+1} \cdots s_{x}$ for some elements $a_{u}, a_{u+1}$ of $\elemtuptuplea$ is a next chain.
    However, since $\elemtuptuples$ is live, there exists another next-chain $s_{x} s_{x+1} \cdots s_{y-1} s_{y}$ between $s_{x}$ and $s_{y}$, clearly distinct as it contains $s_{y-1} \notin \set(\elemtuptuplea)$.
    This conflicts with \cref{lem:companion-tree-like}, since these two next-chains show that $D_{s_{x},W}(\unravel{A}_{W * n})$ is not a tree.
    Therefore, as claimed, the common elements of $\elemtuptuplea$ and $\elemtuptuples$ form an infix of $\elemtuptuples$.
    This argument is symmetric, hence these common elements are an infix of $\elemtuptuplea$ too, so there are indices $v$ and $w$ such that $\elemtuptuplea_{v\ldots{}w} = \elemtuptuples_{i\ldots{}j}$ and the remaining elements $a_1, \ldots, a_{v-1}, a_{w+1}, \ldots, a_{|\elemtuptuplea|}$ are all distinct from elements of $\elemtuptuples$.
    Now, we iteratively apply~\ref{elemeq:pred} and~\ref{elemeq:succ} from \cref{lem:approx-next} to find elements $b_{1}, \ldots, b_{v-1}$ and $b_{w+1}, \ldots, b_{|\elemtuptuplea|}$ from $\unravel{B}_{W * n}$ such that $\elemtuptupleb$ with $\elemtuptupleb = b_{1}\cdots{}b_{v-1}\elemtuptuplet_{i\ldots{}j}b_{w+1}\cdots{}b_{|a|}$ is a next-chain.
    As $t_{i} \approx_{W * k} s_{i}$ and $t_{j} \approx_{W * k} s_{j}$, we have $b_{v-x} \approx_{W * k - x} a_{v-x}$ for $x < v$ and $b_{w+x} \approx_{W * k - x} a_{w+x}$ for $x \le |a|-w$.
    In both cases, $x < W$, since $|\elemtuptuplea| = |\elemtuptupleb|$ and $|\elemtuptuplea| \le W$ as $\elemtuptuplea$ is live.
    Therefore, we have $a_{x} \approx_{W * (k - 1)} b_{x}$ for $1 \le x$ and $x \le |\elemtuptuplea|$.
    By \cref{cor:tuple-similar-iso}, there exists an isomorphism $\mu_{(\elemtuptuplea,\elemtuptupleb)}$.
    Now, if $\elemtuptuplea$ is live, then by isomorphism $\elemtuptupleb$ must also be live.
    Thus the isomorphism $\mu_{(\elemtuptuplea,\elemtuptupleb)}$ is in $Z_{k-1}$.
    The common domain of $\mu_{(\elemtuptuples,\elemtuptuplet)}$ and $\mu_{(\elemtuptuplea,\elemtuptupleb)}$ is $\set(\elemtuptuplea_{v\ldots{}w})$, on which $\mu_{(\elemtuptuples,\elemtuptuplet)}$ and $\mu_{(\elemtuptuplea,\elemtuptupleb)}$ agree since $\elemtuptuplea_{v\ldots{}w} = \elemtuptuples_{i\ldots{}j}$ and $\elemtuptupleb_{v\ldots{}w} = \elemtuptuplet_{i\ldots{}j}$.
    This satsifies the requirements for~\ref{bisim:forth}.

    \item
    If $\elemtuptuplea$ and $\elemtuptuples$ have no common elements, then we first find an element $b_{1} \in \unraveldom{B}_{W * n}$ that is $(W * k - 1)$-similar to $a_{1}$.
    Let $\seq{a_{1}} = \cdots \elemtuples^{(0)} \cdots (i^{(z)}, j^{(z)}) \elemtuptuples^{(z)}$ for $z = \min(W * k, |\seq{a_{1}}|)$.
    Since $\str{A}$ and $\str{B}$ are $2 * W * n$-$\FGF$-bisimilar, there must be a tuple $\elemtuplet^{(0)}$ such that $\str{A}, \elemtuples^{(0)} \bisimto_{\FGF}^{2 * W * n - 1} \str{B}, \elemtuplet^{(0)}$.
    Next, we apply~\ref{bisim:forth} $z$ times to find tuples $\elemtuplet^{(1)}, \ldots, \elemtuplet^{(z)}$ such that $\sigma$ with $\sigma = \elemtuplet^{(0)}\cdots(i^{(z)}, j^{(z)})\elemtuplet^{(z)}$ is a bisimulation sequence and $\str{A}, \elemtuples^{(x)} \bisimto_{\FGF}^{2 * W * n - 1 - x} \str{B}, \elemtuplet^{(x)}$ for all $x \in [1,z]$.
    As $z \le W * k$ and $W * k \le W * n$, we have $2 * W * n - 1 - z \ge  W * k - 1$ and thus $\str{A}, \elemtuples^{(x)} \bisimto_{\FGF}^{W * k - 1} \str{B}, \elemtuplet^{(x)}$ for all $x \in [1, z]$.
    We construct the element $b_{1} \in \unraveldom{B}_{W * n}$ with $\seq{b_{1}} = \sigma$ and $\ctr{b_{1}} = \ctr{a_{1}}$.
    By construction of $\sigma$, we have $\hist{(W * k - 1)}{b_{1}} = \hist{(W * k - 1)}{a_{1}}$ so $a_{1} \approx_{W * k - 1} b_{1}$.
    Similar to case (i), we now use~\ref{elemeq:succ} to find elements $b_{2}, \ldots, b_{|\elemtuptuplea|} \in \unraveldom{B}_{W * n}$ such that $b_{2} \cdots b_{|\elemtuptuplea|}$ is a next-chain.
    Since $|\elemtuptuplea| \le W$, this requires at most $W - 1$ applications of~\ref{elemeq:succ}.
    Hence, $a_{x} \approx_{W * (k-1)} b_{x}$ for $1 \le x$ and $x \le |\elemtuptuplea|$.
    Now $\mu_{(\elemtuptuplea, \elemtuptupleb)} \in Z_{k-1}$ satisfies the requirements for~\ref{bisim:forth}.
  \end{romanenumerate}
\end{proof}
