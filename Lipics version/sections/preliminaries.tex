%!TEX root = ../main.tex

\section{Preliminaries}\label{sec:preliminaries}
We employ standard terminology from (finite and classical) model theory~\cite[Sec. 1--3]{Libkin04}.
A quick reference of all notation used in this paper is given in \cref{fig:notation-quickref}.

\begin{figure}
  \centering
  \bgroup
  \def\arraystretch{1.1}
  \begin{tabularx}{\textwidth}{c X r}
    notation & meaning & introduced in \\
    \hline
    $\str{A}$, $\str{B}$, \ldots & structures with domains $A$, $B$, \ldots & \cref{sec:preliminaries} \\
    $\Sigma$ & the signature of all structures in this paper (purely relational) & \cref{sec:preliminaries} \\
    $\arity(\Sigma)$ & the maximum arity among all predicates in $\Sigma$ (width of $\Sigma$) & \cref{sec:preliminaries} \\
    $\elemtuplea, \elemtupleb$, \ldots & tuples of elements & \cref{sec:preliminaries} \\
    $\elemtuptuplea, \elemtuptupleb$, \ldots & tuples of tuples / tuples of elements from unravelings & \cref{sec:preliminaries} \\
    $f[\elemtuplea]$ & componentwise-image of $f$ on $\elemtuplea$, i.e.\ $(f(a_{1}), \ldots, f(a_{n}))$ & \cref{sec:preliminaries} \\
    $\tp{\logicL}{\str{A}}{\elemtuplea}$ & $\logicL[\sigma]$-type of $\elemtuplea$ in $\str{A}$ (all $\logicL[\sigma]$-formulaes satisfied by $\elemtuplea$) & \cref{sec:preliminaries} \\
    $\atp{\logicL}{\str{A}}{\elemtuplea}$ & atomic-$\logicL[\sigma]$-type of $\elemtuplea$ in $\str{A}$ & \cref{sec:preliminaries} \\
    $\logicL_{\ell}$ & logic $\logicL$ restricted to formulae with quantifier rank at most $\ell$ & \cref{sec:preliminaries} \\
    $\str{A}, \elemtuplea \bisimto_{\FGF} \str{B}, \elemtupleb$ & an $\FGF$-bisimulation containing $(\elemtuplea, \elemtupleb)$ exists between $\str{A}$ and $\str{B}$ & \cref{sec:preliminaries} \\
    $\str{A}, \elemtuplea \bisimto_{\FGF}^{\ell} \str{B}, \elemtupleb$ & an $\ell$-$\FGF$-bisimulation containing $(\elemtuplea, \elemtupleb)$ exists between $\str{A}$ and $\str{B}$ & \cref{sec:preliminaries} \\
    $\unravel{A}$ & the (possibly infinite) tree unraveling of $\str{A}$ & \cref{sec:unraveling} \\
    $\unravel{A}_{\ell}$ & the finite unraveling of $\str{A}$ for threshold $\ell$ & \cref{sec:finite} \\
    $\relNext$ & the parent-child relation in the tree unraveling & \cref{sec:unraveling} \\
    $\relNext_{\ell}$ & finitary variant of $\relNext$, used for defining $\unravel{A}_{\ell}$ & \cref{sec:finite} \\
    $\seq{a}$ & sequence part of an unraveling element, i.e.\ $\sigma$ for $a = (\sigma, k)$ & \cref{sec:unraveling} \\
    $\ctr{a}$ & counter part of an unraveling element, i.e.\ $k$ for $a = (\sigma, k)$ & \cref{sec:unraveling} \\
    $\bound{\elemtuptuplea}$ & counter of the last component of $\elemtuptuplea$, i.e.\ $\ctr{a_{|\elemtuptuplea|}}$ & \cref{sec:unraveling} \\
  \end{tabularx}
  \egroup
  \caption{Overview of relevant notation. Some of these are only introduced later in this paper.}\label{fig:notation-quickref}
\end{figure}

All the logics considered in this paper are fragments of the first-order logic ($\FO$) over purely-relational equality-free vocabularies, under the usual syntax and semantics. 
We fix a countably infinite set $\V \deff \{x_i \colon\, i \in \N\}$ of variables and a countably infinite set~$\R$ of predicates (we employ $\arity(\relR)$ to denote the arity of $\relR$ from $\R$).
Throughout this paper all formulae use variables from $\V$ and predicates from a fixed, finite signature $\Sigma$ which is a subset of $\R$.
Since $\Sigma$ is finite, there is a maximum arity of any relation in $\Sigma$, which we denote by $\arity(\Sigma)$.
Given a formula $\varphi$ we use $\sig(\varphi)$ to denote the set of predicates appearing in $\varphi$. 
We write $\varphi(\vartuplex)$ to indicate that all free variables from $\varphi$ are members of $\vartuplex$. 
If $\vartuplex$ contains precisely the free
variables of $\varphi$, then we emphasise this fact separately.
A formula without free variables is called a \emph{sentence}.
Given a structure $\str{A}$ and $B \subseteq A$, we use $\restr{\str{A}}{B}$ to denote the \emph{substructure} of $\str{A}$ induced by $B$.\\

\noindent \textbf{Tuples and subsequences.}
An $n$-tuple is a list of $n$ elements.
Given a tuple $\elemtuplea$ we use $\set(\elemtuplea)$ to denote the set of its components. 
The $0$-tuple is denoted with~$\emptytupl$.
We use $\vartuplexfromto{i}{j}$ to denote the $(j{-}i{+}1)$-tuple $\varx_i, \varx_{i+1}, \ldots, \varx_j$.
We say that $\vartuplexfromto{i}{j}$ is an infix of a tuple $\vartuplexfromto{k}{l}$ if $k \leq i \leq j \leq l$ holds.
We use the notation ``$\cdots e$'' for a tuple where the last element is equal to $e$ and preceding elements are not important.
For a function $f$ and tuple $\elemtuplea_{1\ldots\ell}$, we write $f[\elemtuplea_{1\ldots\ell}]$ for the tuple $(f(a_{1}), \ldots, f(a_{\ell}))$.
To improve readability, tuples of tuples are denoted by arrows, for instance with $\elemtuptuplea$.
For a set $S$, we write $\vartuplex \sqin S$ iff $\varx_i \in S$ for all indices $1 \leq i \leq |\vartuplex|$, where $|\vartuplex|$ denotes the length of~$\vartuplex$. 
We say that a tuple $\elemtuplea \sqin A$ is \emph{live} in $\str{A}$ if $|\elemtuplea| \leq 1$ or~$\elemtuplea \in \relR^{\str{A}}$ for some predicate~$\relR \in \Sigma$.\\

\noindent \textbf{Types and logical equivalence.}
Fix a logic $\logicL$ and a finite signature~$\sigma \subseteq \R$. 
We~employ~$\logicL[\sigma]$ in place of~$\{ \varphi \in \logicL \colon\,  \sig(\varphi) \subseteq \sigma \}$.
Since all structures and formulae throughout this paper use a fixed signature $\Sigma$, we omit the signature from now on and write just $\logicL$ instead of $\logicL[\Sigma]$.
Moreover, we use $\logicL_\ell$ to denote the restriction of $\logicL$ to formulae of \emph{quantifier rank} (i.e.\ the maximal number of nested quantifiers) at most~$\ell$.
%
The \emph{$\logicL$-type} of $\elemtuplea$ in $\str{A}$, denoted with~$\tp{\logicL}{\str{A}}{\elemtuplea}$, consists of all $\logicL$-formulae with free variables~$\vartuplexfromto{1}{n}$ that are satisfied by $\elemtuplea$ in $\str{A}$.
If we restrict the type to only atomic and negated atomic $\logicL$-formulae, we obtain the \emph{atomic-$\logicL$-type} of $\elemtuplea$ in $\str{A}$, denoted with~$\atp{\logicL}{\str{A}}{\elemtuplea}$.
The atomic-$\logicL$-type of $\elemtuplea$ is logically equivalent to the $\logicL_{0}$-type of $\elemtuplea$ in $\str{A}$.
We write $\str{A} \equiv_\logicL \str{B}$ if $\str{A}$ and~$\str{B}$ satisfy the same $\logicL$-sentences.
For pointed structures $(\str{A}, \elemtuplea), (\str{B}, \elemtupleb)$ with $n$-tuples $\elemtuplea$ and~$\elemtupleb$ we employ the notation $(\str{A}, \elemtuplea) \equiv_\logicL (\str{B}, \elemtupleb)$ to indicate that for all $\varphi \in \logicL$ with free variables contained in the set $\{ \varx_1, \ldots, \varx_n \}$ we have $\str{A} \models \varphi[\elemtuplea]$ if and only if $\str{B} \models \varphi[\elemtupleb]$ (here variables $\varx_i$ are substituted, respectively, for $\elema_i$ and $b_i$).\\

\noindent \textbf{Guarded fragments.}
We recall the definition of the \emph{guarded fragment}~\cite[Sec. 4.1]{AndrekaNB98}, \ie the fragment of $\FO$ obtained by requiring that blocks of quantifiers are appropriately relativised by atoms.
Formally $\GF$ is the smallest fragment of $\FO$  such that:
\begin{itemize}\itemsep0em
    \item Every atomic formula is in $\GF$;
    \item $\GF$ is closed under boolean connectives $\land, \lor, \neg, \to$;
    \item If $\varphi(\vartuplex, \vartupley)$ is in $\GF$, $\alpha(\vartuplex, \vartupley)$ is an atom containing all free variables of $\varphi$, and $\vartupley$ is a tuple of variables then both $\forall{\vartupley} \; (\alpha(\vartuplex, \vartupley) \to \varphi(\vartuplex, \vartupley))$ and $\exists{\vartupley} \; (\alpha(\vartuplex, \vartupley) \land \varphi(\vartuplex, \vartupley))$ are in $\GF$; 
    \item If $\varphi(\varx)$ has only a single free-variable $\varx$, then $\forall{\varx}\; \varphi(\varx)$ and $\exists{\varx}\; \varphi(\varx)$ are in $\GF$.
\end{itemize}
The predicates $\alpha$ appearing in the 3rd item of the above definition are called \emph{guard} for $\varphi$.
We~stress that in definition of a quantifier rank for $\GF$ we treat quantifiers $\exists{\vartuplex}$ introducing tuples of variables as a single quantifier (not as an abbreviation for a block of quantifiers).

\begin{example}
A formula $\exists{x_{1}x_{2}x_{3}}\; (\relR(x_{1}, x_{2}, x_{3}) \land \forall x_{4}(\relR(x_{4}, x_{4}, x_{1}) \to \relR(x_{1}, x_{4}, x_{1})))$ belongs to $\GF$, but the formula $\exists{x_{1}x_{2}x_{3}}\; (\relE(x_{1}, x_{2}) \land \relE(x_{2}, x_{3}) \land \relE(x_{3}, x_{1}))$ does not.
\end{example}

The \emph{forward guarded fragment}~\cite[Sec. 3.1]{Bednarczyk21} (or $\FGF$) restricts $\GF$ in a way that the allowed sequences of atoms are infixes of the sequence of already-introduced variables (in the order of their quantification).
A formal definition comes next, which will be followed by a bunch of examples.
Let $\FGF(n)$ for $n \in \N$ be the smallest fragment of $\FO$ satisfying:
\begin{itemize}\itemsep0em
    \item An atom $\alpha(\vartuplex)$ belongs to $\FGF(n)$ if $\alpha$ is equality-free and $\vartuplex$ is an infix of $\vartuplexfromto{1}{n}$.
    \item $\FGF(n)$ is closed under boolean connectives $\land, \lor, \neg, \to, \iff$;
    \item If $\alpha$ and $\varphi$ are in $\FGF(n{+}k)$ for a positive $k$ where $\alpha(\vartuplex, \vartupley)$ is an atom containing all free variables of $\varphi$ and $\vartupley$ is a $k$-tuple of variables then $\forall{\vartupley} \; (\alpha(\vartuplex, \vartupley) \to \varphi(\vartuplex, \vartupley))$ and $\exists{\vartupley} \; (\alpha(\vartuplex, \vartupley) \land \varphi(\vartuplex, \vartupley))$ are both in $\FGF(n)$;
    \item If $\varphi(\varx_1) \in \FGF(1)$ has only a single free-variable $\varx_1$, then $\forall{\varx_1}\; \varphi$ and $\exists{\varx_1}\; \varphi$ are in $\FGF(0)$.
\end{itemize}
We use $\FGF$ to denote $\FGF(0)$. Note that $\FGF(0)$ is solely composed of sentences. 

\begin{example}
A formula $\exists{x_1x_{2}x_{3}}\; [\relR(x_{1}, x_{2}, x_{3}) \land \forall{x_{4}x_{5}}(\relS(x_{3}, x_{4}, x_{5}) \to \relT(x_{4})) \land \exists{x_{3}}(\relR(x_{1}, x_{2}, x_{3}))]$ belongs to $\FGF$.
Moreover, first-order translations of (polyadic) modal and description logics are also in $\FGF$.
In contrast, formulae $\exists{x_{1}}\; \relE(x_{1}, x_{1})$ and $\exists{x_{1}}\; (\forall{x_{2}} \relT(x_{1}, x_{2}))$ are not in $\FGF$.
For the former formula the reason is that the sequence $x_1x_1$ is not an infix of the sequence~$x_1$.
The latter formula is not even in $\GF$ as the sequence $x_{1}, x_{2}$ is not guarded.
\end{example}

