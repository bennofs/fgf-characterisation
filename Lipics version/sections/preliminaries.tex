%!TEX root = ../main.tex



\section{Preliminaries}\label{sec:preliminaries}
We employ standard terminology from (finite and classical) model theory~\cite[Sec. 1--3]{Libkin04}.

All the logics considered here will be fragments of the first-order logic ($\FO$) over purely-relational equality-free vocabularies, under the usual syntax and semantics. 
We fix a countably infinite set $\V \deff \{x_i \mid i \in \N\}$ of variables and a countably infinite set~$\R$ of predicates (we use $\arity(\relR)$ to denote the arity of $\relR$ from $\R$). 
Throughout this paper all the formulae will use only variables from $\V$ and predicates from $\R$.
Given a formula $\varphi$ we use $\sig(\varphi)$ we denote the set of predicates appearing in $\varphi$. 
We write $\varphi(\vartuplex)$ to indicate that all free variables from $\varphi$ are members of $\vartuplex$. 
If $\vartuplex$ contains precisely the free
variables of $\varphi$, then we will emphasise this separately.
A formula without free variables will be called a \emph{sentence}.
Given a structure $\str{A}$ and $B \subseteq A$, we will use $\restr{\str{A}}{B}$ to denote the \emph{substructure} of $\str{A}$ induced by $B$.\\

\noindent \textbf{Tuples and subsequences.}
An $n$-tuple is a sequence with $n$ elements. The $0$-tuple is denoted with $\emptytupl$.
We use $\vartuplexfromto{i}{j}$ to denote the $(j{-}i{+}1)$-tuple $\varx_i, \varx_{i+1}, \ldots, \varx_j$.
We say that $\vartuplexfromto{i}{j}$ is an infix of a tuple $\vartuplexfromto{k}{l}$ if $k \leq i \leq j \leq l$ holds. 
To improve readability, tuples of tuples will be denoted by arrows, for instance with $\elemtuptuplea$.
For a set $S$, we write $\vartuplex \sqin S$ iff $\varx_i \in S$ for all indices $1 \leq i \leq |\vartuplex|$, where $|\vartuplex|$ denotes the length of~$\vartuplex$. 
A tuple $\elemtuplea \sqin A$ is \emph{$\sigma$-live} (or simply \emph{live} if a signature $\sigma$ is known from the context) in $\str{A}$ if $|\elemtuplea| \leq 1$ or~$\elemtuplea \in \relR^{\str{A}}$ for some~$\relR \in \sigma$.\\

\noindent \textbf{Types and logical equivalence.}
Fix a logic $\logicL$, a finite signature~$\sigma \subseteq \R$. 
We~employ $\logicL[\sigma]$ in place of~$\{ \varphi \in \logicL \mid \sig(\varphi) \subseteq \sigma \}$.
Moreover, we use $\logicL_\ell[\sigma]$ to denote the restriction of $\logicL[\sigma]$ to formulae of quantifier rank (i.e.\ the maximal number of nested quantifiers) at most~$\ell$.  
%
The $\logicL[\sigma]$-type of $\elemtuplea$ in $\str{A}$, denoted with~$\tp{\logicL[\sigma]}{\str{A}}{\elemtuplea}$, consists of all $\logicL[\sigma]$-formulae with free variables~$\vartuplexfromto{1}{n}$ that are satisfied by $\elemtuplea$ in $\str{A}$.
We write $\str{A} \equiv_\logicL \str{B}$ if $\str{A}$ and $\str{B}$ satisfy the same $\logicL$-sentences.
For pointed structures $(\str{A}, \elemtuplea), (\str{B}, \elemtupleb)$ we use the notation $(\str{A}, \elemtuplea) \equiv_\logicL (\str{B}, \elemtupleb)$ to express that for all $\varphi \in \logicL$ with at most $\vartuplexfromto{1}{|\elemtuplea|}$ we have $\str{A} \models \varphi[\elemtuplea]$ if and only if $\str{B} \models \varphi[\elemtupleb]$ (variable $\varx_i$ is substituted with $\elemtuplea_i$). 
A similar notation is used for $\logicL[\sigma]$ instead of $\logicL$.\\

\noindent \textbf{Guarded fragments.}
We recall the definition of the \emph{guarded fragment}~\cite[Sec. 4.1]{AndrekaNB98}, \ie the fragment of $\FO$ obtained by requiring that blocks of quantifiers are appropriately relativised by atoms.
Formally $\GF$ is the smallest fragment of $\FO$  such that:
\begin{itemize}\itemsep0em
    \item Every atomic formula is in $\GF$;
    \item $\GF$ is closed under boolean connectives $\land, \lor, \neg, \to$;
    \item If $\varphi(\vartuplex, \vartupley)$ is in $\GF$ and $\alpha(\vartuplex, \vartupley)$ is an atom containing all free variables of $\varphi$ and $\vartupley$ is a tuple of variables then both $\forall{\vartupley} \; (\alpha(\vartuplex, \vartupley) \to \varphi(\vartuplex, \vartupley))$ and $\exists{\vartupley} \; (\alpha(\vartuplex, \vartupley) \land \varphi(\vartuplex, \vartupley))$ are in $\GF$; 
    \item If $\varphi(\varx)$ has only a single free-variable $\varx$, then $\forall{\varx}\; \varphi$ and $\exists{\varx}\; \varphi$ are in $\GF$.
\end{itemize}
The atoms $\alpha$ appearing in the 3rd item of the above definition are called \emph{guard}.
We stress that in definition of a quantifier rank for $\GF$ we count treat quantifiers $\exists{\vartuplex}$ introducing tuples of variables as as a single quantifier (not as an abbreviation for a block of quantifiers).

\begin{example}
\bbe{TODO: Benno, please add here an example of a formulae in $\GF$ and an example of a formula that is in $\FO$ but not in $\GF$.}
\end{example}

The \emph{forward guarded fragment} $\FGF$~\cite[Sec. 3.1]{Bednarczyk21} restrict $\GF$ in a way that the allowed sequences of atoms are infixes of the sequence of already-introduced variables in the order of their quantification.
A formal definition comes next, which will be follow by a bunch of examples.
Let $\FGF(n)$ for $n \in \N$ be the smallest fragment of $\FO$ satisfying:
\begin{itemize}\itemsep0em
    \item An atom $\alpha(\vartuplex)$ belongs to $\FGF(n)$ if $\alpha$ is equality-free and $\vartuplex$ is an infix of $\vartuplexfromto{1}{n}$.
    \item $\FGF(n)$ is closed under boolean connectives $\land, \lor, \neg, \to, \iff$;
    \item If $\varphi$ is in $\FGF(n{+}k)$ for a positive $k$ and $\alpha(\vartuplex, \vartupley)$ is an atom containing all free variables of $\varphi$ and $\vartupley$ is a $k$-tuple of variables then  $\forall{\vartupley} \; (\alpha(\vartuplex, \vartupley) \to \varphi(\vartuplex, \vartupley))$ and $\exists{\vartupley} \; (\alpha(\vartuplex, \vartupley) \land \varphi(\vartuplex, \vartupley))$ are both in $\FGF(n)$; 
    \item If $\varphi(\varx_1) \in \FGF(1)$ has only a single free-variable $\varx_1$, then $\forall{\varx_1}\; \varphi$ and $\exists{\varx_1}\; \varphi$ are in $\FGF(0)$.
\end{itemize}
We use $\FGF$ to denote $\FGF(0)$. 
Note that $\FGF(0)$ is solely composed of sentences. 

\begin{example}
\bbe{TODO: Benno, please include examples of formulae that are in $\FGF$ but not in $\GF$.}
\end{example}

\noindent \textbf{Bisimulations for guarded fragments.}
%
We next present notion of bisimulation relations tailored towards $\GF$ and $\FGF$, based on presentations from~\cite[Sec. 2.2.3]{Otto04} and~\cite[Sec. 2]{BednarczykJ22}.

For a signature $\sigma \subseteq \R$ and $\sigma$-structures $\str{A}$ and $\str{B}$ we denote with $\PartIso{\str{A}}{\str{B}}$ the set of all partial isomorphisms between $\str{A}$ and $\str{B}$. 
For $\bisimZ, \bisimZ' \in \PartIso{\str{A}}{\str{B}}$ we say that $\bisimZ'$ satisfies back-and-forth conditions for $\bisimZ$ if for every partial isomorphism $\partisof \in \bisimZ$ we have:
%
\begin{description}\itemsep0em
  \item[\desclabel{(Forth)}{bisim:forth}] For any $\sigma$-live $\elemtuplea \sqin A$, there is $\partisog \in \bisimZ'$ with the domain $\elemtuplea$ such that $\partisof$ and $\partisog$ agree on their common domain. 
  \item[\desclabel{(Back)}{bisim:back}] For any $\sigma$-live $\elemtupleb \sqin B$, there is $\partisog \in \bisimZ'$ with the image $\elemtuplea$ such that $\partisof$ and $\partisog$ agree on their common domain. 
\end{description}
The set $\bisimZ$ is a $\GF[\sigma]$-\emph{bisimulation} between $\str{A}$ and $\str{B}$ if it itself satisfies \ref{bisim:forth} and~\ref{bisim:back} conditions given above and every partial isomorphism $\partisof \in \bisimZ$ maps $\sigma$-live tuples to $\sigma$-live tuples.
An $\ell$-$\GF[\sigma]$-bisimulation between $\str{A}$ and $\str{B}$ is a sequence $\bisimZ_0, \bisimZ_1, \ldots, \bisimZ_\ell$ of partial isomorphisms from $\PartIso{\str{A}}{\str{B}}$ mapping $\sigma$-live tuples to $\sigma$-live tuples  such that for all $i > 0$ we have that $\bisimZ_i$ satisfies~\ref{bisim:forth} and~\ref{bisim:back} conditions for $\bisimZ_{i{-}1}$.
We say that pointed structures $(\str{A}, \elemtuplea)$ and~$(\str{B}, \elemtupleb)$ are $\GF[\sigma]$-\emph{bisimilar}, denoted $(\str{A}, \elemtuplea) \bisimto_{\GF[\sigma]} (\str{B}, \elemtupleb)$ if there exists a $\GF[\sigma]$-bisimulation  between $\str{A}$ and $\str{B}$ containing the partial isomorphism $\elemtuplea \mapsto \elemtupleb$.
We analogously speak about $\ell$-$\GF[\sigma]$-\emph{bisimilarity} and employ the notation $(\str{A}, \elemtuplea) \bisimto_{\GF[\sigma]}^{\ell} (\str{B}, \elemtupleb)$.

The following classical lemma links bisimulations and logical~equivalence.
\begin{lemma}[Thm. 1.12 of~\cite{Gradel014}]\label{lemma:GF-bisimulations-work-well}
For any finite signature $\sigma$ and $\sigma$-structures $(\str{A}, \elemtuplea)$ and~$(\str{B}, \elemtupleb)$:
\begin{enumerate}[(a)]
\item $(\str{A}, \elemtuplea) \bisimto_{\GF[\sigma]} (\str{B}, \elemtupleb)$ implies $(\str{A}, \elemtuplea) \equiv_{\GF[\sigma]} (\str{B}, \elemtupleb)$;
\item $(\str{A}, \elemtuplea) \bisimto_{\GF[\sigma]}^{\ell} (\str{B}, \elemtupleb)$ implies $(\str{A}, \elemtuplea) \equiv_{\GF_\ell[\sigma]} (\str{B}, \elemtupleb)$;
\end{enumerate}
Moreover, the converse holds for $\omega$-saturated $\str{A}$ and $\str{B}$.
\end{lemma}

The following example presents two structures that are $\GF$-bisimilar but can be distinguished by a first-order formula.
\begin{example}
\bbe{TODO: Benno}
\end{example}

Now a similar characterisation can be provided for $\FGF$. 
Due to the ``forwardness'' of the underlying logic, it is convenient to think about maps as tuples.
A \emph{system of forward partial maps} between $\sigma$-structures $\str{A}$ and $\str{B}$ is any non-empty subset of $\bigcup_{i=0}^{\infty} (A^i \times B^i)$ satisfying:
\begin{description}\itemsep0em
  \item[\desclabel{(AtomicEq)}{bisim:atomiceq}] For all $(\elemtuplea, \elemtupleb) \in \bisimZ$ we have that $\elemtuplea$, $\elemtupleb$ are $\sigma$-live and $\tp{\FGF[\sigma]}{\str{A}}{\elemtuplea} = \tp{\FGF[\sigma]}{\str{B}}{\elemtupleb}$~holds.
\end{description}
If $\bisimZ, \bisimZ'$ are systems of forward partial maps between $\str{A}$ and $\str{B}$, we say that $\bisimZ'$ satisfies back-and-forth conditions for $\bisimZ$ if for every $(\elemtuplea, \elemtupleb) \in \bisimZ$ the following conditions hold:
\begin{description}\itemsep0em
  \item[\desclabel{(fForth)}{bisim:fforth}] For a (possibly empty) affix $\elemtuplecfromto{i}{j}$ of $\elemtuplec$ and a $\sigma$-live tuple $\elemtuplee$ in $\str{A}$ such that $\elemtuplecfromto{i}{j} = \elemtupleefromto{1}{j{-}i{+}1}$ there is a $\sigma$-live tuple $\elemtuplef$ with $\elemtupledfromto{i}{j} = \elemtupleffromto{1}{j{-}i{+}1}$ such that $(\elemtuplee, \elemtuplef) \in \bisimZ$ holds.
  %
  \item[\desclabel{(fBack)}{bisim:fback}] For a (possibly empty) affix $\elemtupledfromto{i}{j}$ of $\elemtupled$ and a $\sigma$-live tuple $\elemtuplef$ in $\str{B}$ such that $\elemtupledfromto{i}{j} = \elemtupleffromto{1}{j{-}i{+}1}$ there is a $\sigma$-live tuple $\elemtuplee$ with $\elemtuplecfromto{i}{j} = \elemtupleefromto{1}{j{-}i{+}1}$ such that $(\elemtuplee, \elemtuplef) \in \bisimZ$ holds.
\end{description}
A system of forward partial maps $\bisimZ$ between $\str{A}$ and $\str{B}$ is a $\FGF[\sigma]$-\emph{bisimulation} between $\str{A}$ and $\str{B}$ if it itself satisfies the above conditions. 
An $\ell$-$\FGF[\sigma]$-bisimulation between $\str{A}$ and $\str{B}$ is a sequence $\bisimZ_0, \bisimZ_1, \ldots, \bisimZ_\ell$ of systems of forward partial maps between $\str{A}$ and $\str{B}$ such that for all $i > 0$ we have that $\bisimZ_i$ satisfies~\ref{bisim:fforth} and~\ref{bisim:fback} conditions for $\bisimZ_{i{-}1}$.
We speak about $\FGF[\sigma]$-\emph{bisimilar} and $\ell$-$\FGF[\sigma]$-\emph{bisimilar} (pointed) structures in total analogy to the case of the guarded fragment.
A ``forward'' counterpart of \cref{lemma:GF-bisimulations-work-well} is presented below.
\begin{lemma}\label{lemma:FGF-bisimulations-work-well}
For any finite signature $\sigma$ and $\sigma$-structures $(\str{A}, \elemtuplea)$ and~$(\str{B}, \elemtupleb)$:
\begin{enumerate}[(a)]
\item $(\str{A}, \elemtuplea) \bisimto_{\FGF[\sigma]} (\str{B}, \elemtupleb)$ implies $(\str{A}, \elemtuplea) \equiv_{\FGF[\sigma]} (\str{B}, \elemtupleb)$;
\item $(\str{A}, \elemtuplea) \bisimto_{\FGF[\sigma]}^{\ell} (\str{B}, \elemtupleb)$ implies $(\str{A}, \elemtuplea) \equiv_{\FGF_\ell[\sigma]} (\str{B}, \elemtupleb)$;
\end{enumerate}
Moreover, the converse holds for $\omega$-saturated $\str{A}$ and $\str{B}$.
\end{lemma}
\begin{proof}
\bbe{TODO: Benno, please provide a proof here.}
\end{proof}

The following example presents two structures that are $\FGF$-bisimilar but not $\GF$-bisimilar.
\begin{example}
\bbe{TODO: Benno}
\end{example}
